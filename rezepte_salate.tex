
% created Montag, 10. Dezember 2012 16:22 (C) 2012 by Leander Jedamus
% modifiziert Mittwoch, 11. März 2015 17:13 von Leander Jedamus
% modifiziert Montag, 09. März 2015 14:26 von Leander Jedamus
% modified Freitag, 28. Dezember 2012 14:51 by Leander Jedamus
% modified Montag, 10. Dezember 2012 16:30 by Leander Jedamus

  \mynewchapter{Salate}

    \mynewsection{Endiviensalat}\glossary{Salat>Endivien-}%
              \label{endiviensalat}

      \begin{zutaten}
        1 Kopf & \myindex{Endiviensalat}\index{Salat>Endivien-} \\
      \end{zutaten}
      \begin{zutat}{Essig/Ölsoße}
        etwas & \myindex{Salz}, \myindex{Pfeffer}, \myindex{Zucker} \\
        1 & kleine \myindex{Zwiebel} \\
        ca. \breh{} Teelöffel & \myindex{Senf} \\
        etwas & \myindex{Essigessenz} mit \textbf{heißem} \myindex{Wasser} \\
        1 Eßloffel & Öl\index{Oel=Öl} \\
      \end{zutat}
      \begin{zutat}{Senf/Sahnesoße}
        etwas & \myindex{Salz}, \myindex{Pfeffer}, \myindex{Zucker} \\
        \breh{} Teelöffel & \myindex{Senf} \\
        \breh{} Becher & \myindex{süße Sahne}\index{Sahne>süß} (100g) \\
      \end{zutat}

      \begin{zubereitung}
        Festen knackigen Kopf auswählen, Inneres des Kopfes ansehen. Äußere
	Blätter und alles vergilbte wegschneiden. Kleine Bündel vom Kopf
	abtrennen. Auf Küchenbrett in feine Streifen schneiden. Dabei das
	untere Ende der Blätter wegschneiden (ca. 1,5~cm). Die Streifen in
	reichlich Wasser waschen, mindestens zweimal. Meistens dreimal nötig.
	Gut abtropfen lassen. Am Besten den Salat schon 2~Stunden vor dem Essen
	gewaschen stehen haben. \\
        Essig/Ölsoße : Zwiebel schälen, fein würfeln, Salz, Pfeffer und Zucker
	dazu. Darauf heißes Essigwasser, verrühren. Am Schluß das Öl zugeben.
	Abschmecken und in die warme Soße den relativ trockenen Endiviensalat
	geben (Hat man eine Bratensoße, dann hiervon 2--3~Eßlöffel über den
	Salat geben). Den Salat mindestens 30~Minuten vor dem Essen in der
	Soße ziehen lassen. \\
        Senf-Sahne-Soße: Salz, Pfeffer, Zucker in die Schüssel geben, Senf
	dazu. Alles sorgfältig zu einer Masse verrühren. Dann die Sahne
	dazurühren. Trockenen Salat etwa 5--10~Minuten vor dem Essen dazugeben.
	Gibt man zu nassen Salat in die Soße, schmeckt die Soße zu schwach und
	fad. Also immer gut abtropfen lassen. Wenn möglich in Geschirrtuch
	schleudern oder in Salatschleuder oder über längeren Zeitraum in einem
	Durchschlag. \\
        Variationen (bei Essig/Ölsoße):
        Maiskörner aus der Dose zugeben \\
        ODER    frische Orangenstücke (Haut der Orangenspalten entfernen) \\
        ODER    Radieschenscheiben \\
        ODER    hartgekochte Eiviertel \\
        ODER    Tomatenvierteln \\
        oder eigene Varianten ausprobieren. Man kann auch \frisee{}salat nehmen,
	dann besser die Soße statt mit Essig mit Zitronensaft zubereiten.
	Lollo-Rosso, Eichblattsalat werden ebenso zubereitet. \\
      \end{zubereitung}

    \mynewsection{Kopfsalat (auch Schnittsalat)}\glossary{Salat>Kopf-}%
                                             \glossary{Salat>Schnitt-}

      \begin{zutaten}
        1 Kopf & grüner \myindex{Kopfsalat}\index{Salat>Kopf-} \\
        1 & Soße wie Essigöl (Endiviensalat) \\
      \end{zutaten}

      \begin{zubereitung}
        Möglichst keinen Treibhaussalat kaufen, da nitratbelastet. Schmeckt
	auch nicht so gut. Am besten den schmutzigeren und rauhblättrigeren
	Freilandsalat. Äußere Blätter entfernen, wenn nicht verwendbar. Salat
	vorsichtig waschen in reichlich Wasser. Dabei Salat nicht drücken.
	Locker in Durchschlag geben. Am besten in Salatschleuder trocknen.
	Soße wie die Essig-Öl-Soße bei Endiviensalat zubereiten, nur nimmt
	man statt des Essigs hier Zitronensaft. Den Salat gibt man direkt vor
	dem Auftragen erst in die Soße. Er wird schnell unansehnlich in einer
	Soße. \\
        Varianten: Radieschenscheiben \\
        Maiskörner \\
        Gurkenscheiben \\
        Tomatenscheiben oder -Viertel \\
        Oliven schwarze, grüne \\
      \end{zubereitung}

    \mynewsection{Feldsalat}\glossary{Salat>Feld-}

      \begin{zutaten}
        ca. 200--300 g & \myindex{Feldsalat}\index{Salat>Feld-} \\
        & Soße wie Endiviensalat \\
      \end{zutaten}

      \begin{zubereitung}
        Feldsalat gibt es typischerweise im Winter. Ich ziehe die
	kleinblättrigere und dunklere Sorte vor, da sie nicht so schnell
	,,schlapp`` macht. Die Pflanze sieht büschelig aus und besitzt noch
	ihre Wurzel, teils hängen dort Erdreste, Sand etc. Die Wurzel und welke
	Teile entfernt man Stück für Stück (kann länger dauern, je nach
	Verschmutzung und Abfallgrad) und wäscht dann den Salat sehr vorsichtig
	in viel Wasser. In den Durchschlag geben und ab und zu Wasser
	abschütteln, aber Salat nicht schleudern. Soße wie bei Endivien als
	Essig-Öl-Soße machen, nur statt des Essigs hier Zitrone nehmen. \\
        Varianten: geschälte Walnußkerne \\
        hartgekochte Eischeiben oder Viertel \\
      \end{zubereitung}

    \mynewsection{Gurkensalat}\glossary{Salat>Gurken-}

      \begin{zutaten}
        1 & \myindex{Schlangengurke}\index{Gurke>Schlangen-} \\
        & Essig-Öl-Soße wie beim Endiviensalat
	  (Seite \pageref{endiviensalat}) \\
        1 & \myindex{Knoblauchzehe} \\
      \end{zutaten}

      \begin{zubereitung}
        Gurke vom Blütenansatz zum Stielende hin schälen (da die Gurke sonst
	bitter werden kann, früher war es jedenfalls immer so). Gurke hobeln
	oder in dünne Scheiben schneiden. Soße wie bei Endivien zubereiten,
	Knoblauch klein hacken und dazugeben, pfeffern. Man kann auch statt
	Essig Zitronensaft nehmen. Kurz vor dem Servieren mischen. \\
        Varianten: mit grünem Salat mischen \\
        mit Tomatenstücken mischen \\
        mit Sahne und Dill \\
      \end{zubereitung}

    \mynewsection{Eisbergsalat}\glossary{Salat>Eisberg-}

      \begin{zutaten}
        1 Kopf & \myindex{Eisbergsalat}\index{Salat>Eisberg-} \\
        & Soße wie bei Endivien (Seite \pageref{endiviensalat}) \\
      \end{zutaten}

      \begin{zubereitung}
        Eisbergsalat gibt es meist im Winter. Kann man länger im Kühlschrank
	verwahren und man muß nicht gleich den ganzen Kopf aufbrauchen. Blätter
	ablösen, in Stücke reißen und 1~mal waschen. Abtropfen lassen,
	schleudern. Schmeckt gut mit Sahnesoße. Hält sich darin auch einige
	Zeit. \\
      \end{zubereitung}

    \mynewsection{Tomatensalat}\glossary{Salat>Tomaten-}

      \begin{zutaten}
        ca. 500 g & feste, rote \myindex{Tomate}n \\
        1 Zehe & \myindex{Knoblauch} (oder mehr) \\
        2--3 Eßlöffel & \myindex{Olivenöl}\index{Oel=Öl>Oliven-} \\
        etwas & \myindex{Oregano} \\
        und/oder etwas & \myindex{Basilikum} \\
        \breh{} Teelöffel & \myindex{Salz} \\
        etwas & \myindex{Pfeffer} \\
        ODER & Soße wie Essig-Öl-Soße bei Endivien
	       (Seite \pageref{endiviensalat}) \\
      \end{zutaten}

      \begin{zubereitung}
        Tomaten säubern und in Scheiben oder Viertel/Achtel teilen. mit Soße
	mischen. Kann man auch etwas stehen lassen, also schon länger vor dem
	Servieren mischen. \\
        Varianten: mit grünem Salat \\
        mit Gurken \\
        mit schwarzen Oliven und Schaf/Ziegenkäse \\
        mit Zwiebelringen \\
      \end{zubereitung}

    \mynewsection{Nudelsalat}\glossary{Salat>Nudel-}

      \begin{zutaten}
        ca. 500 g & \myindex{Hörnchennudeln}\index{Nudeln>Hörnchen-} \\
        \breh{} Ring & \myindex{Fleischwurst} mit Knoblauch \\
        mehrere & \myindex{Gewürzgurke}n\index{Gurke>Gewürz-} \\
        einige Eßlöffel & \myindex{Salatmayonnaise}\index{Mayonnaise>Salat-}
	                  (THOMY, NADLER) \\
      \end{zutaten}

      \begin{zubereitung}
        Nudeln in Salzwasser garen, abgießen. Fleischwurst längs in etwa
	\breh{}~cm dicke Scheiben schneiden, dann in Streifen, dann in Würfel.
	Gewürzgurken genauso. Man nimmt etwas weniger Gewürzgurke als
	Fleischwurst. Fleischwurst und Gewürzgurke mit Mayonnaise mischen.
	Ergibt FLEISCHSALAT. Entweder man teilt jetzt Fleischsalat ab
	(meistens) und den Rest mischt man mit den abgekühlten Nudeln oder man
	gibt alles zusammen. \\
      \end{zubereitung}

    \mynewsection{Hühnersalat}\glossary{Salat>Hühner-}

      \begin{zutaten}
        ca. 300 g & gekochtes \myindex{Huhn}/\myindex{Hähnchen} \\
        oder 500 g & \myindex{Hühnerklein} gefroren (billig) \\
        kl. Dose & \myindex{Spargelabschnitte} \\
        oder gr. Dose & ganzer \myindex{Spargel} teilweise \\
        kl. Dose & \myindex{Ananasstücke} \\
        einige Eßlöffel & \myindex{Salatmayonnaise}\index{Mayonnaise>Salat-}
	                  (THOMY, NADLER) \\
      \end{zutaten}

      \begin{zubereitung}
        Huhn kochen, Haut und Knochen entfernen oder fertig gegartes Stück
	Huhn/Hähnchen verwenden. Weißes Hühnerfleisch zerteilen in mundgerechte
	Happen. Spargel in Stücken von etwa 3--4~cm Länge beigeben, ebenso
	Ananas und einige Eßlöffel Salatmayonnaise. Durchmischen und ziehen
	lassen. Kühl stellen. \\
      \end{zubereitung}

    \mynewsection{Brüsseler Weintraubensalat}\glossary{Salat>Weintrauben-}

      \begin{zutaten}
        500 g & blaue \myindex{Weintrauben} \\
        1 Dose & \myindex{Ananas} \\
        100 g & \myindex{Gouda}\index{Käse>Gouda} \\
        & \myindex{Zitrone}nsaft \\
        1 & \myindex{\chicoree{}} \\
        100 g & \myindex{Mandel}stifte \\
        & \myindex{Butter} \\
      \end{zutaten}

      \begin{zubereitung}
        Trauben halbieren und entkernen, Ananas zerkleinern, Gouda in Stifte
	schneiden. \chicoree{} ganz fein schneiden. Butter in Pfanne geben und
	darin die Mandelstifte anrösten. Alles unterheben und abschmecken. \\
      \end{zubereitung}

    \mynewsection{Rettich-Apfel-Salat}\glossary{Salat>Rettich-Apfel-}

      \begin{zutaten}
        1 großer & weißer \myindex{Rettich} \\
        2 große & Äpfel\index{Aepfel=Äpfel} \\
        1 & \myindex{Apfelsine} \\
        1 & \myindex{Zitrone} \\
        1 Prise & \myindex{Salz} \\
        1 Teelöffel & \myindex{Honig} \\
      \end{zutaten}

      \begin{zubereitung}
        Rettich und geschälte Äpfel fein raspeln, Apfelsine schälen und klein
	schneiden, Zitrone pressen. Zutaten würzen mit Salz und Honig. Gut
	mischen und ziehen lassen. \\
      \end{zubereitung}

    \mynewsection{Frischer Krautsalat mit Chinakohl}\glossary{Salat>Kraut-}%
              \label{krautsalat}

      \begin{zutaten}
        ca. 500 g & \myindex{Chinakohl}\index{Kohl>China-}
                    in sehr feinen Streifen \\
        1 Teelöffel & \myindex{Salz} \\
        & \myindex{Pfeffer} \\
        ca. 3--4 Eßlöffel & \myindex{Apfelessig}\index{Essig>Apfel-} \\
        3 Eßlöffel & \myindex{Olivenöl}\index{Oel=Öl>Oliven-} \\
        2 & \myindex{Schalotte}n oder \myindex{Frühlingszwiebel}n \\
        100 g & durchwachsener \myindex{Speck} fein gewürfelt \\
        1 Teelöffel & \myindex{Kümmel} \\
      \end{zutaten}

      \begin{zubereitung}
        Chinakohl in feine Streifen schneiden, In einer großen Schüssel mit
	Salz bestreuen und leicht durchkneten. Den Saft, der dabei entsteht,
	abgießen. Dann Essig und Öl sowie Pfeffer hinzufügen. Zum Schluß nach
	Geschmack die fein geschnittenen Frühlingszwiebeln mit Grün oder die
	fein gewürfelten Schalotten untermischen. Den Speck in winzige Würfel
	schneiden, kurz in einer Pfanne anrösten, den Kümmel hinzufügen und
	über den Salat verteilen. \\
        Dazu: Käsekartoffeln (siehe Seite \pageref{kaesekartoffeln}) \\
      \end{zubereitung}

    \mynewsection{Eisbergsalat mit Honigmelone, Trauben und Gurke}%
              \glossary{Salat>Eisberg-}

      \begin{zutaten}
        1 kleiner & \myindex{Eisbergsalat} \\
        250 g & grüne \myindex{Weintrauben} \\
        1 kleine & \myindex{Honigmelone} \\
        1 & \myindex{Lauchzwiebel}\index{Zwiebel>Lauch-} \\
        \breh{} Bund & \myindex{Schnittlauch} \\
        2 Eßlöffel & \myindex{Zitrone}nsaft \\
        1 Prise & \myindex{Zucker} \\
        5 Eßlöffel & \myindex{Sonnenblumenöl}\index{Oel=Öl>Sonnenblumen-} \\
        4 Eßlöffel & \myindex{Essig} \\
        4 Eßlöffel & \myindex{Beerenkonfitüre}\index{Konfitüre>Beeren-} \\
        2 Eßlöffel & \myindex{Sonnenblumenkerne} \\
        & \myindex{weißer Pfeffer}\index{Pfeffer>weiß} \\
        & \myindex{Salz} \\
      \end{zutaten}

      \personen{4}

      \begin{zubereitung}
        Vom Salat äußere Blätter entfernen, Salatkopf vierteln und in dünne
	Streifen schneiden. Melone halbieren, Kerne entfernen, kleine Kugeln
	ausstechen. Gurke halbieren, entkernen und in dünne Scheiben schneiden.
	Trauben waschen, halbieren, entkernen. Lauchzwiebel und Schnittlauch
	waschen und putzen, danach in Ringe schneiden. Die Zutaten vermischen.
	\\
        Dressing: Salz, Zitronensaft, Pfeffer, Zucker, Öl, Essig,
	Beerenkonfitüre zur glatten Soße verrühren und über den Salat
	verteilen. Mit Sonnenblumenkernen bestreut servieren. \\
        Dazu: Knuspriges Weißbrot. \\
      \end{zubereitung}

    \mynewsection{Chinakohl-Salat}\glossary{Salat>Chinakohl-}

      \begin{zutaten}
        500 g & \myindex{Chinakohl}\index{Kohl>China-} in Streifen \\
        1 Dose & \myindex{rote Bohnen}\index{Bohnen>rot} \\
        1 Dose & \myindex{Mais} \\
        & \myindex{Salz} \\
        & \myindex{Pfeffer} \\
        etwas & \myindex{Zucker} \\
        viel & \myindex{Balsamico-Essig}\index{Essig>Balsamico-} \\
        & Öl\index{Oel=Öl} \\
      \end{zutaten}

      \begin{zubereitung}
        Chinakohl in Streifen schneiden, Topf und Öl erhitzen und Kohl
	andünsten, leicht bräunen. Die Dose Bohnen heiß abspülen, Mais
	abgießen. Kohl, Bohnen und Mais in Schüssel geben, würzen und mischen.
	Darf länger ziehen. \\
      \end{zubereitung}

    \mynewsection{Kartoffelsalat mit Mayonnaise}%
              \glossary{Salat>Kartoffel-}

      \begin{zutaten}
        ca. 1 kg & festkochende \myindex{Kartoffel}n \\
        \breh{} Teelöffel & \myindex{Salz}, etwas \myindex{Pfeffer} \\
        etwas & \myindex{Essig} \\
        1--2 Eßlöffel & Öl\index{Oel=Öl} \\
        1 & \myindex{Zwiebel} \\
        1 kl. & Glas \myindex{Mayonnaise} (THOMY, NADLER) \\
      \end{zutaten}

      \begin{zubereitung}
        Etwa gleich große möglichst unbeschädigte Kartoffeln (ohne Triebe oder
	faulige Stellen !) wenn möglich wählen. Die größeren unten in den Topf,
	weil sie länger garen müssen. Nach 20--25~Minuten probieren, ob gar.
	Abgießen und pellen. Dann in Scheiben schneiden. Zwiebel würfeln, Salz
	und Pfeffer dazu, Essigessenz mit heißem Wasser versetzen. Diese stark
	gewürzte Brühe (man kann auch noch etwas Brühwürfel reingeben und
	auflösen) in noch warmen/heißem Zustand auf die noch warmen/heißen
	Kartoffeln mit den Zwiebelwürfeln gießen und vorsichtig mischen. Jetzt
	Öl untermischen. Zudecken und 1--2~Stunden etwa ziehen lassen.
	Zwischendurch probieren. Darf ruhig etwas überwürzt schmecken. Sonst
	nachwürzen. Alles kalt werden lassen und dann eßlöffelweise Mayonnaise
	dazugeben (es lohnt nicht, hier an der Qualität der Mayonnaise zu
	sparen, verdirbt den ganzen Salat. Darauf achten, daß keine
	Konservierungsstoffe drin sind!). \\
        Zu Schweinebraten, Würstchen, kaltem Schnitzel, Spiegeleiern usw. \\
        Varianten: hartgekochte Eiviertel \\
        Gewürzgurkenscheiben \\
        Fleischwurstwürfel \\
        usw. \\
      \end{zubereitung}

    \mynewsection{Rucolasalat mit Ei und neuen Kartoffeln}%
              \glossary{Salat>Rucola-}

      \begin{zutaten}
        400 g & neue \myindex{Kartoffel}n \\
        3 & \myindex{Tomate}n \\
        1 Paket & \myindex{Feta}\index{Käse>Feta} (200 g) \\
        4 & \myindex{Ei}er \\
        100 g & \myindex{Frühstücksspeck} \\
        2 Bund & \myindex{Rucola} \\
        1 & \myindex{Schalotte} \\
        1 Eßlöffel & \myindex{Weißweinessig}\index{Essig>Weißwein-} \\
        1 Teelöffel & grobkörniger \myindex{Senf} \\
        2 Eßlöffel & \myindex{Olivenöl}\index{Oel=Öl>Oliven-} \\
        & \myindex{Salz} \\
        & \myindex{Pfeffer} \\
        & \myindex{Zucker} \\
      \end{zutaten}

      \personen{4}
      \kalorien{687}
      \garzeit{20 + 25}

      \begin{zubereitung}
        Die Kartoffeln kräftig abbürsten, dann in kochendem Salzwasser ca.
	25~Minuten garen lassen. Das Wasser abschütten und Kartoffeln je nach
	Größe in Viertel oder Achtel schneiden. Die Eier in Wasser hart kochen
	(ca. 6--8~Minuten, je nach Größe des Eies). Mit kaltem Wasser
	abschrecken und abkühlen lassen. Frühstücksspeck in einer heißen Pfanne
	ohne Fett knusprig braten. Dabei mehrmals wenden. Auf Küchenpapier
	abtropfen lassen. Den Rucola putzen, waschen und in mundgerechte Stücke
	schneiden. Dann in einer Salatschleuder trocken schleudern. Die
	Schalotte schälen und in sehr feine Würfel schneiden. Den Essig in
	einer kleinen Schüssel mit je 1 Prise Salz und Zucker und dem
	grobkörnigen Senf vermengen. Das Öl und die Schalottenwürfel dazugeben
	und mit Pfeffer abschmecken. Den Rucola, die Kartoffelwürfel und die
	geviertelten Tomaten mit der Vinaigrette anmachen. die Eier abpellen,
	in dünne Scheiben schneiden und dazugeben. Auf Tellern anrichten und
	die Speckstreifen darübergeben. \\
      \end{zubereitung}

    \mynewsection{Lauchsalat des Chefs}\glossary{Salat>Lauch-}

      \begin{zutaten}
        500 g & \myindex{Lauch} hellgrün und weiß \\
	250 g & gebratenes \myindex{Hähnchenbrustfilet} \\
	1 & \myindex{Zwiebel} \\
	100 g & \myindex{Möhre}n jung \\
	100 g & \myindex{Champignon}\index{Pilze>Champignon}s \\
	100 g & \myindex{Tomate}n \\
      \end{zutaten}
      \begin{zutat}{Marinade}
	2 & \myindex{Eigelb} \\
	4 Eßlöffel & \myindex{Mayonnaise} \\
	1 Becher & \myindex{Joghurt} \\
	& \myindex{Senf} \\
	& \myindex{Salz} \\
	& \myindex{Pfeffer} \\
	& \myindex{Essig} \\
      \end{zutat}

      \personen{4}

      \begin{zubereitung}
        Lauch --- nur die hellgrünen und weißen Teile --- in zarte Ringe
	schneiden. Locker mischen mit gebratenem Hähnchenbrustfilet in
	Streifen, Zwiebel in Ringen, geraspelten jungen Möhren, frischen
	Champignons in Scheibchen und gehäuteten, entkernten Tomaten in
	Streifen. \\
	Marinade aus Eigelbe, Mayonnaise, Joghurt, Senf, Salz, Pfeffer und
	Essig. \\
      \end{zubereitung}

    \mynewsection{Spinat-Matjes-Salat}\glossary{Salat>Spinat-Matjes-}

      \begin{zutaten}
        200 g & \myindex{Spinat} \\
	& \myindex{Salz} \\
	1 & \myindex{Fleischtomate}\index{Tomate>Fleisch-} \\
	1 & \myindex{Zwiebel} \\
	2 Stiele & \myindex{Majoran} \\
	3 Eßlöffel & \myindex{Weinessig}\index{Essig>Wein-} \\
	6 Eßlöffel & \myindex{Olivenöl}\index{Oel=Öl>Oliven-} \\
	4 & \myindex{Matjes}filets \\
	1 Gläschen & \myindex{Kapern} (Einwaage 35 g) \\
      \end{zutaten}

      \personen{4}
      \kalorien{425}

      \begin{zubereitung}
        Spinat putzen, waschen und tropfnaß mit wenig Salz in einen Topf geben.
	Bei mittlerer Hitze 2~Minuten dünsten, mit kaltem Wasser übergießen
	und abtropfen lassen. Fleischtomate entkernen und in kleine Würfel
	schneiden. Zwiebelwürfel mit gehacktem Majoran, Essig und Olivenöl
	verrühren. Spinat mit Matjesfilets, Tomatenwürfeln und abgetropften
	Kapern auf einer Platte oder auf Portionstellern anrichten und mit der
	Soße begießen. \\
      \end{zubereitung}

    \mynewsection{Zucchinisalat (Insalata di Zucchini)}%
      \glossary{Salat>Zucchini-}

      \begin{zutaten}
        2 & \myindex{Zucchini} (\'a 150 g) \\
	200 g & \myindex{Tomate}n \\
	4 Eßlöffel & Öl\index{Oel=Öl} \\
	& \myindex{Pfeffer} \\
	& \myindex{Salz} \\
	& \myindex{Majoran} \\
	1 Bund & \myindex{Petersilie} \\
      \end{zutaten}

      \personen{4}
      \kalorien{150}

      \begin{zubereitung}
        Zucchini waschen, halbieren, in mundgerechte Stücke schneiden. Tomaten
	abziehen, vierteln, entkernen. Zucchini im heißen Öl anbraten. Würzen.
	Tomaten und gehackte Petersilie zufügen. \\
      \end{zubereitung}

    \mynewsection{Weiß-roter Salat}\glossary{Salat>weiß-rot}

      \begin{zutaten}
        1 & \myindex{Radicchio}staude \\
	1 & \myindex{\chicoree{}}staude \\
	2 kleine & grüne \myindex{Radicchio}pflänzchen (eventuell) \\
	1 & \myindex{Radicchio} di Castelfranco \\
	1 & weiße \myindex{Zwiebel} \\
	2 Hand voll & abgezupfter Kräuter:
	  \myindex{glatte Petersilie}\index{Petersilie>glatt},
	  \myindex{Basilikum},
	  \myindex{Melisse},
	  \myindex{Liebstöckel},
	  \myindex{Pimpinelle} \\
        1 Hand voll & gelbe \myindex{\frisee{}}blätter \\
      \end{zutaten}
      \begin{zutat}{Marinade}
        2 Eßlöffel & \myindex{Balsamico-Essig}\index{Essig>Balsamico-} \\
	3 Eßlöffel & \myindex{Zitrone}nsaft \\
	\breh{} Teelöffel & \myindex{Salz} \\
	& frisch gemahlener \myindex{Pfeffer} \\
	1 Teelöffel & \myindex{scharfer Senf}\index{Senf>scharf} \\
	1 & durchgepreßte \myindex{Knoblauchzehe} \\
	3--4 Eßlöffel & erstklassiges \myindex{Olivenöl}\index{Oel=Öl>Oliven-} \\
      \end{zutat}
      
      \personen{4}

      \begin{zubereitung}
        Die Stauden in Blätter zerlegen, diese in Stücke teilen oder auch quer
	in schmale Streifen zupfen --- nach Belieben. Die Zwiebel in feine
	Ringe hobeln, die Kräuter von den Stielen pflücken, auch den \frisee{}
	zerkleinern. Alles mischen und mit der Marinade anmachen, für die
	sämtliche Zutaten mit dem Schneebesen oder mit der Gabel aufgeschlagen
	und cremig vermischt wurden. \\
      \end{zubereitung}

    \mynewsection{Kartoffelsalat mit zweierlei Bohnen}%
              \glossary{Salat>Kartoffel-}

      \begin{zutaten}
        800 g & fest kochende \myindex{Kartoffel}n (zum Beispiel Bamberger 
	        Hörndl, Sieglinde oder die Vitelotte, Annabell) \\
        4--5 & \myindex{Anchovis}\index{Fisch>Anchovis}
	       (Sardellenfilets in Öl) \\
	3 & \myindex{Knoblauchzehe}n \\
	& \myindex{Pfeffer} \\
	3--4 Eßlöffel & \myindex{Olivenöl}\index{Oel=Öl>Oliven-} \\
	2--3 Eßlöffel & \myindex{Sherryessig}\index{Essig>Sherry-} \\
	2--3 Eßlöffel & \myindex{Apfelessig}\index{Essig>Apfel-} \\
	1 Schuß & \myindex{Brühe} \\
	2 Eßlöffel & \myindex{Kapern} \\
	1 & \myindex{rote Zwiebel}\index{Zwiebel>rot} \\
	250 g & feinste \myindex{grüne Bohnen}\index{Bohnen>grün}
	        (blanchiert) \\
        1--2 Tassen & gekochte \myindex{weiße Bohnen}\index{Bohnen>weiß}kerne
	              \\
        1--2 & feste \myindex{Tomate}n (eventuell) \\
	1 Tasse & kleine \myindex{schwarze Oliven}\index{Oliven>schwarz} \\
	& \myindex{Salz} \\
	& Kräuter (\myindex{Schnittlauch}, \myindex{Petersilie}) \\
	& \myindex{Salat}blätter zum Anrichten \\
      \end{zutaten}

      \personen{4}

      \begin{zubereitung}
        Die Kartoffeln in der Schale gar kochen, nur ein wenig auskühlen
	lassen, dann pellen und in Scheibchen schneiden. \\
	Anchovis, Kapern und Knoblauch sehr fein hacken und mit grobem Pfeffer
	mischen. Essig, Brühe und Olivenöl sowie die übrigen Salatzutaten
	unterrühren: rote Zwiebel in feinen Segmenten, Böhnchen im Ganzen, die
	abgetropften Bohnenkerne. Alles in einer Schüssel mit der Marinade und
	den Kartoffelscheibchen mischen. Gewürfelte, entkernte Tomate, die
	ganzen Oliven und die gehackten Kräuter hinzufügen, herzhaft
	abschmecken und salzen. Auf Salatblättern anrichten. Die Vinaigrette
	kann man auch prima zu puren ganzen Kartoffeln servieren, die man dann
	auf dem Teller genüßlich zerdrücken kann. \\
	Beilage: Bauernbrot. \\
	Getränk: Pils. \\
      \end{zubereitung}

    \mynewsection{Griechischer Bauernsalat}%
              \glossary{Salat>griechischer Bauern-}%
	      \glossary{Bauernsalat}

      \begin{zutaten}
        2 & \myindex{Fleischtomate}\index{Tomate>Fleisch-}n \\
	1 & \myindex{Salatgurke}\index{Gurke>Salat-} \\
	2 & \myindex{grüne Paprika}\index{Paprika>grün}schoten \\
	2 & \myindex{Zwiebel}n \\
	250 g & \myindex{Schafkäse}\index{Käse>Schaf-} \\
	100 g & \myindex{Oliven} \\
	& \myindex{Salz} \\
	& \myindex{Pfeffer} aus der Mühle \\
	8 Eßlöffel & \myindex{Olivenöl}\index{Oel=Öl>Oliven-} \\
	& frischer \myindex{Oregano} \\
      \end{zutaten}

      \personen{4}
      \kalorien{550}

      \begin{zubereitung}
        Gemüse putzen und waschen. Tomaten und Gurke in \breh{}~cm dicke
	Scheiben schneiden. Große Scheiben eventuell noch halbieren.
	Paprikaschoten halbieren, entkernen und in Streifen, geschälte Zwiebeln
        in dünne Ringe schneiden. Alle Zutaten vermischen. Zerbröckelten
	Schafkäse und Oliven darübergeben. Salzen und mit Pfeffer übermahlen.
	Mit Olivenöl beträufeln und Oreganoblättchen darüberstreuen. Dann alles
	vorsichtig unterheben. \\
	Dazu Gersten- oder Fladenbrot (gibt's in griechischen Geschäften, siehe
	Seite \pageref{fladenbrot})
	servieren. \\
      \end{zubereitung}

    \mynewsection{Bunter Salat mit Schinken und Käse}%
              \glossary{Salat>bunt}

      \begin{zutaten}
        1 Kopf & \myindex{Batavia}, \myindex{Romana} oder
	         \myindex{Eisbergsalat} \\
        \breh{} Kopf & \myindex{Radicchio} \\
	einige Blätter & \myindex{Eichblattsalat} \\
	3 Stangen & \myindex{Bleichsellerie} \\
	200 g & \myindex{gekochter Schinken}\index{Schinken>gekocht} \\
	150 g & \myindex{mittelalter Gouda}\index{Käse>Gouda>mittelalt}
	        (in Scheiben) \\
	1 Portion & \myindex{Vinaigrette} (siehe Seite \pageref{vinaigrette})
	            z.B. mit Oliven- oder Maiskeimöl und Balsamessig \\
      \end{zutaten}

      \personen{6}
      \kalorien{250}

      \begin{zubereitung}
        Salate putzen, waschen und in mundgerechte Stücke zupfen. Dann gut
	abtropfen lassen. Bleichselleriestangen waschen und in feine Scheiben
	schneiden. Schinken und Käse zuerst in 1~cm breite Streifen, dann in
	Rauten schneiden. Alle Zutaten vermischen und mit der Vinaigrette
	anmachen. \\
	Dazu Laugenbrezeln oder kräftiges Bauernbrot servieren. \\
      \end{zubereitung}

    \mynewsection{Nizzasalat}\glossary{Salat>Nizza-}

      \begin{zutaten}
        1 & \myindex{Kopfsalat}, \myindex{Batavia} oder \myindex{Romana} \\
	350 g & \myindex{Kartoffel}n \\
	250 g & grüne \myindex{Bohnen} \\
	500 g & \myindex{Tomate}n \\
	1 große Dose & \myindex{Thunfisch} in Öl \\
	2 & hartgekochte \myindex{Ei}er \\
	2 Eßlöffel & \myindex{Kapern} \\
	1 Bund & \myindex{Petersilie} oder \myindex{Schnittlauch} \\
	1 Portion & \myindex{Vinaigrette} (siehe Seite \pageref{vinaigrette}),
	            z.B. mit \myindex{Olivenöl}\index{Oel=Öl>Oliven-} und
		    \myindex{Rotweinessig}\index{Essig>Rotwein-} \\
      \end{zutaten}

      \personen{4}
      \kalorien{390}

      \begin{zubereitung}
        Salat putzen, in mundgerechte Stücke zupfen und waschen. Dann gut
	abtropfen lassen. Kartoffeln in der Schale garen, dann schälen und in
	Scheiben schneiden. Bohnen putzen, dabei Enden kappen und waschen.
	Große Bohnen halbieren. In kochendem Salzwasser 8--10~Minuten bißfest
	kochen. Sofort kalt abschrecken, damit sie ihre grüne Farbe behalten.
	Tomaten waschen und achteln. Thunfisch gut abtropfen lassen und mit
	einer Gabel in grobe Stücke zerpflücken. Eier schälen und vierteln.
	Alle Zutaten mit den Kapern vermischen. Feingehackte Petersilie oder
	Schnittlauchröllchen unter die Vinaigrette rühren und Salat damit
	anmachen. Sofort servieren. \\
	Dazu Stangenweißbrot reichen. \\
	Übrigens: Sie können auch zusätzlich noch je 1~rote und grüne
	Paprikaschote (in feine Streifen geschnitten), einige in Öl eingelegte
	Sardellenfilets (am besten feingehackt) oder schwarze Oliven
	untermischen. \\
      \end{zubereitung}

    \mynewsection{Grüner Salat mit Pilzen und Speck}%
              \glossary{Salat>Grün mit Pilzen und Speck}

      \begin{zutaten}
        200 g & junge \myindex{Spinat}blätter \\
	\brev{} Kopf & \myindex{\frisee{}}\index{Salat>\frisee{}-}salat oder
	               \myindex{Sommerendivie}\index{Salat>Endivien-} \\
        100 g & \myindex{Egerlinge}\index{Pilze>Egerling}e oder
	        \myindex{Champignon}\index{Pilze>Champignon}s \\
	& \myindex{Zitrone}nsaft \\
	1 Portion & \myindex{Vinaigrette} (siehe Seite \pageref{vinaigrette}),
	            z.B. mit \myindex{Walnußöl}\index{Oel=Öl>Walnuß-} oder
		    \myindex{Sonnenblumenöl}\index{Oel=Öl>Sonnenblumen-} und
		    \myindex{Apfelessig}\index{Essig>Apfel-} \\
      \end{zutaten}
      \begin{zutat}{Außerdem}
        100 g & durchwachsener \myindex{Speck} (in Scheiben) \\
	2 Scheiben & \myindex{Toastbrot}\index{Brot>Toast-} \\
	2 Eßlöffel & Öl\index{Oel=Öl} \\
      \end{zutat}
      
      \personen{4}
      \kalorien{380}

      \begin{zubereitung}
        Spinat und Salat putzen und waschen. Salat in mundgerechte Stücke
	zupfen. Beides gut abtropfen lassen. Pilze (nur wenn nötig) waschen
	und blättrig schneiden. Sofort mit etwas Zitronensaft beträufeln, damit
	sie nicht braun werden. Dann alle Zutaten vermischen und mit der
	Vinaigrette anmachen. Speck in dünne Streifen, Toastbrot in feine
	Würfel schneiden. Zuerst den Speck im erhitzten Öl braten. Dann
	Brotwürfel hinzufügen und rundum goldbraun rösten. Salat mit Speck und
	Brotwürfeln servieren. \\
	Übrigens: Nur wenn nach dem Waschen das überschüßige Wasser wieder
	entfernt wird, können sich Salatblätter und Soße richtig verbinden.
	Das gelingt am besten mit einer Salatschleuder (gibt's zu verschiedenen
	Preisen im Fachhandel). \\
      \end{zubereitung}

    \mynewsection{Bohnensalat}\glossary{Salat>Bohnen-}

      \begin{zutaten}
        100 g & gekochte \myindex{weiße Bohnen}\index{Bohnen>weiß} (Dose) \\
        1 & \myindex{Lauchzwiebel}\index{Zwiebel>Lauch-} \\
        1 kleine & \myindex{Tomate} \\
        1 Teelöffel & \myindex{Senf} \\
        2 Eßlöffel & \myindex{Zitrone}nsaft \\
        & \myindex{Knoblauch} \\
        & \myindex{Oregano} \\
        1 Scheibe & \myindex{Parmaschinken}\index{Schinken>Parma-} (20 g) \\
      \end{zutaten}

      \kalorien{200}

      \begin{zubereitung}
        Die Bohnen abtropfen lassen und mit der feingeschnittenen Lauchzwiebel
	mischen. Tomate achteln und hinzufügen. Senf mit Zitronensaft,
	Knoblauch und Oregano verrühren und darunterheben. Eine Scheibe
	Parmaschinken auf dem Salat anrichten. \\
      \end{zubereitung}

    \mynewsection{Knackiger Salat mit gebratenen Gemüsen}%
              \glossary{Salat>Knackig mit gebratenen Gemüsen}

      \begin{zutaten}
        1 & \myindex{Kopfsalat}\index{Salat>Kopf-} \\
        2 & \myindex{Zucchini} \\
        2 & \myindex{Paprika} \\
        1 & \myindex{Aubergine} \\
        8 Eßlöffel & \myindex{Olivenöl}\index{Oel=Öl>Oliven-} mit Gewürzeinlage \\
        & \myindex{Rosmarin} \\
        & \myindex{Jodsalz}\index{Salz>Jod-} \\
        & \myindex{Pfeffer} \\
        60 g & \myindex{Pinienkerne} \\
        100 g & \myindex{Parmigiano-Reggiano-Käse} (Parmesan) \\
        6 Eßlöffel & \myindex{Essig-Öl-Dressing} \\
      \end{zutaten}

      \personen{4}
      \garzeit{15}

      \begin{zubereitung}
        Kopfsalat waschen, trockenschleudern, große Blätter klein zupfen.
	Gemüse waschen, abtrocknen. Zucchini in dünne Scheiben schneiden.
	Strunk und Kerne der Paprika entfernen, dann vierteln. Paprikaviertel
	in Streifen schneiden, Aubergine würfeln. \\
	4~Eßlöffel Rosmarin-Olivenöl in einer beschichteten Pfanne erhitzen,
	darin das Gemüse 5~Minuten braten. Ab und an umrühren. Gemüse mit Salz
	und Pfeffer würzen, in eine flache Schüssel legen und mit dem
	restlichen Öl einige Minuten marinieren. \\
	Pinienkerne ohne Fett in der Pfanne braun rösten, abkühlen lassen.
	Mit dem Sparschäler dünne Späne vom Parmesanstück hobeln. Salat mit
	Essig-Öl-Dressing marinieren und mit den Gemüsen anrichten. Parmesan
	und Pinienkerne darüberstreuen. \\
      \end{zubereitung}

    \mynewsection{\chicoree{}salat mit Ananas}\glossary{Salat>\chicoree{}-}

      \begin{zutaten}
        2--3 & \myindex{\chicoree{}} \\
        1 Dose & \myindex{Ananas}stückchen \\
        3 Eßlöffel & geschälte \myindex{Walnüsse} \\
        1 Eßlöffel & \myindex{Mayonnaise} \\
        1 Eßlöffel & \myindex{Joghurt} \\
        1--2 Eßlöffel & \myindex{Ananassaft} \\
        2 Teelöffel & \myindex{Curry} \\
        12 & \myindex{Schnittlauch}halme fein geschnitten \\
        & \myindex{Jodsalz}\index{Salz>Jod-} \\
        & \myindex{weißer Pfeffer}\index{Pfeffer>weiß} \\
      \end{zutaten}

      \personen{2}

      \begin{zubereitung}
        \chicoree{} waschen und putzen, Strünke entfernen und \chicoree{} in
	Ringe schneiden. In eine Schüssel geben, Ananas abtropfen lassen (Saft
	auffangen). Geschälte Walnüsse fein hacken. \chicoree{}, Ananas und
	Nüsse mischen. Dressing (Mayonnaise usw.) anrühren und über den Salat
	gießen. \\
      \end{zubereitung}

    \mynewsection{\chicoree{}salat mit Walnüssen}%
              \glossary{Salat>\chicoree{}-}

      \begin{zutaten}
        500 g & \myindex{\chicoree{}} \\
        1 & \myindex{Radicchio} \\
        100 g & \myindex{Champignon}\index{Pilze>Champignon}s \\
        50 g & \myindex{roher Schinken}\index{Schinken>roh} \\
        50 ml & \myindex{Schlagsahne}\index{Sahne>Schlag-} \\
        \breh{} Teelöffel & \myindex{Salz} \\
        1 Teelöffel & \myindex{Zucker} \\
        etwas & \myindex{Senf} \\
        2 Eßlöffel & \myindex{Himbeeressig}\index{Essig>Himbeer-} \\
        3 Eßlöffel & \myindex{Walnußöl}\index{Oel=Öl>Walnuß-} \\
        & \myindex{schwarzer Pfeffer}\index{Pfeffer>schwarz} \\
        100 g & \myindex{Walnüsse} geschält \\
      \end{zutaten}

      \begin{zubereitung}
        \chicoree{} waschen und grob schneiden (Strunk entfernen), Radicchio
	waschen und zerpflücken, Champignons in Scheiben schneiden, Schinken
	zerpflücken. Für die Salatsoße restliche Zutaten mischen. Soße
	vorsichtig mit Salat mischen und mit gehackten Walnüssen bestreuen.
	Dazu frisches Baguette. \\
      \end{zubereitung}

    \mynewsection{Joghurtdressing}\label{joghurtdressing}

      \begin{zutaten}
        1 Becher & \myindex{Joghurt} \\
        2 Eßlöffel & Öl\index{Oel=Öl} \\
        1 kleine & fein gehackte \myindex{Zwiebel} \\
        & \myindex{weißer Pfeffer}\index{Pfeffer>weiß} \\
        1 Prise & \myindex{Zucker} \\
        1 Eßlöffel & \myindex{Zitrone}nsaft \\
        \breh{} Bund & fein geschnittener \myindex{Schnittlauch} \\
      \end{zutaten}

      \begin{zubereitung}
        Den Joghurt in einer Schüssel mit dem Schneebesen schaumig schlagen.
	Dabei nach und nach das Öl zugeben. Mit der Zwiebel, weißem Pfeffer,
	dem Zucker und dem Zitronensaft würzen. Abschmecken. Zum Schluß die
	Schnittlauchröllchen unterheben. \\
      \end{zubereitung}

    \mynewsection{\chicoree{}salat mit Camembert}%
              \glossary{Salat>\chicoree{}- mit Camembert}

      \begin{zutaten}
        400 g & \myindex{\chicoree{}} \\
        100 g & blaue \myindex{Trauben}\index{Trauben>blau} \\
        50 g & \myindex{Camembert}\index{Käse>Camembert} \\
        1 Portion & \myindex{Joghurtdressing}
	            (siehe Seite \pageref{joghurtdressing}) \\
        1 Eßlöffel & \myindex{Kresse}blättchen \\
      \end{zutaten}

      \personen{4}
      \kalorien{160}

      \begin{zubereitung}
        \chicoree{} putzen, Strunk keilförmig herausschneiden, Blätter lösen,
	waschen, gut abtropfen lassen und quer in ca. 3~cm breite Streifen
	schneiden, Trauben waschen, halbieren und Kerne entfernen. Camembert
	in Streifen schneiden. Joghurtdressing verrühren, die Salatzutaten
	darin mischen und mit Kresseblättchen bestreut servieren. \\
      \end{zubereitung}

    \mynewsection{Kopfsalat mit Sprossen}\glossary{Salat>Kopf- mit Sprossen}

      \begin{zutaten}
        1 kleiner & \myindex{Kopfsalat}\index{Salat>Kopf-} \\
        50g & blaue \myindex{Trauben} \\
        1 Portion & \myindex{Vinaigrette} (siehe Seite \pageref{vinaigrette})
	            \\
        6 & \myindex{Walnüsse} \\
        2 Eßlöffel & \myindex{Sprossen} \\
      \end{zutaten}

      \personen{4}
      \kalorien{150}

      \begin{zubereitung}
        Kopfsalat putzen, waschen, gut abtropfen lassen, blaue Trauben waschen,
	halbieren, Kerne entfernen. Vinaigrette verrühren, die Salatzutaten
	darin mischen und anrichten. Walnüsse und Sprossen darübergeben und
	servieren. \\
      \end{zubereitung}

    \mynewsection{Bunter Feldsalat}\glossary{Salat>Feld- bunt}

      \begin{zutaten}
        75 g & \myindex{Feldsalat}\index{Salat>Feld-} \\
        100 g & frische \myindex{Champignon}\index{Pilze>Champignon}s \\
        2 & Äpfel\index{Aepfel=Äpfel} \\
        & \myindex{Zitrone}nsaft \\
        2 & \myindex{rote Zwiebel}\index{Zwiebel>rot}n \\
        1 Portion & \myindex{Vinaigrette} (siehe Seite \pageref{vinaigrette})
	            \\
      \end{zutaten}

      \personen{4}
      \kalorien{140}

      \begin{zubereitung}
        Feldsalat putzen, waschen und gut abtropfen lassen. Champignons putzen,
	waschen und in Scheiben schneiden, Äpfel schälen, achteln,
	Kerngehäuse entfernen, in dünne Scheiben schneiden und mit Zitronensaft
	beträufeln. Zwiebeln schälen, in dünne Scheiben schneiden, Vinaigrette
	verrühren, die Salatzutaten darin mischen und sofort servieren. \\
      \end{zubereitung}

    \mynewsection{Radicchiosalat}\glossary{Salat>Radicchio-}

      \begin{zutaten}
        200 g & \myindex{Radicchio} \\
        1 Dose & \myindex{Pfirsich}e (370~ml) \\
        75 g & \myindex{Emmentaler}\index{Käse>Emmentaler} \\
        1 Portion & \myindex{Vinaigrette} (siehe Seite \pageref{vinaigrette})
	            \\
        3 Eßlöffel & \myindex{\cremefraiche{}} \\
	3 Eßlöffel & Abtropfflüssigkeit von den Pfirsichen \\
      \end{zutaten}

      \personen{4}
      \kalorien{180}

      \begin{zubereitung}
        Radicchio putzen, waschen, gut abtropfen lassen. Pfirsiche abtropfen
	lassen, Abtropfflüssigkeit für die Soße verwenden. Pfirsiche in Spalten
	schneiden, Käse in Streifen schneiden, Vinaigrette mit
	Abtropfflüssigkeit und \cremefraiche{} verrühren, die Salatzutaten
	darin mischen und servieren. \\
      \end{zubereitung}

    \mynewsection{Pussta-Salat}\glossary{Salat>Pussta-}

      \begin{zutaten}
        2 & \myindex{grüne Paprika}\index{Paprika>grün}schoten \\
        2 & \myindex{rote Paprika}\index{Paprika>rot}schoten \\
        1 & \myindex{Zwiebel} \\
        1 Portion & \myindex{Vinaigrette} (siehe Seite \pageref{vinaigrette})
	            \\
      \end{zutaten}

      \personen{4}
      \kalorien{185}

      \begin{zubereitung}
        Paprikaschoten waschen, halbieren, Kerne und weiße Innenhäute
	entfernen, in Streifen schneiden. Zwiebel schälen und in Scheiben
	schneiden. Vinaigrette verrühren, die Salatzutaten darin mischen, ca.
	30~Minuten durchziehen lassen und servieren. \\
      \end{zubereitung}

    \mynewsection{Gurkensalat mit Schafkäse}%
              \glossary{Salat>Gurken- mit Schafkäse}

      \begin{zutaten}
        1 & \myindex{Salatgurke}\index{Gurke>Salat-} \\
        1 Portion & \myindex{Vinaigrette} (siehe Seite \pageref{vinaigrette})
	            \\
        50 g & \myindex{Schafkäse}\index{Käse>Schaf-} \\
        10 & \myindex{schwarze Oliven}\index{Oliven>schwarz} \\
      \end{zutaten}

      \personen{4}
      \kalorien{150}

      \begin{zubereitung}
        Salatgurke waschen, in Scheiben schneiden, Vinaigrette verrühren, die
	Salatgurke darin mischen. Den Salat anrichten, zerbröckelten Schafkäse
	und schwarze Oliven darübergeben und servieren. \\
      \end{zubereitung}

    \mynewsection{Gurken-Mais-Salat}\glossary{Salat>Gurken-Mais-}

      \begin{zutaten}
        1 & \myindex{Salatgurke}\index{Gurke>Salat-} \\
        \breh{} Dose & \myindex{Mais} \\
        1 Portion & \myindex{Vinaigrette} (siehe Seite \pageref{vinaigrette})
	            \\
        1 Eßlöffel & geröstete \myindex{Sonnenblumenkerne} \\
      \end{zutaten}

      \personen{4}
      \kalorien{140}

      \begin{zubereitung}
        Salatgurke waschen, längs halbieren, Kerne mit einem Löffel entfernen
	und in Scheiben schneiden. Mais abtropfen lassen, Vinaigrette
	verrühren, die Salatzutaten darin mischen und mit Sonnenblumenkernen
	bestreut servieren. \\
      \end{zubereitung}

    \mynewsection{Salat Romana}

      \begin{zutaten}
        1 kleiner & Kopf \myindex{Römischer Salat}\index{Salat>Römisch} \\
        1 kleine & \myindex{Zucchini} \\
        1 & \myindex{Zwiebel} \\
        1 & \myindex{rote Paprika}\index{Paprika>rot}schote \\
        2 & \myindex{Tomate}n \\
        1 Eßlöffel & gefüllte \myindex{grüne Oliven}\index{Oliven>grün} \\
        1 Portion & \myindex{Vinaigrette} (siehe Seite \pageref{vinaigrette})
	            \\
      \end{zutaten}

      \personen{4}
      \kalorien{135}

      \begin{zubereitung}
        Römischen Salat waschen, gut abtropfen lassen und in Streifen
	schneiden, Zucchini waschen und in dünne Scheiben schneiden,
	Paprikaschote halbieren, Kerne und weiße Innenhäute entfernen, waschen
	und in Streifen schneiden, Tomaten waschen und in Scheiben schneiden,
	gefüllte Oliven in Scheiben schneiden, Vinaigrette verrühren, die
	Salatzutaten darin mischen und anrichten. \\
      \end{zubereitung}

    \mynewsection{Tomatensalat mit Mozzarella}

      \begin{zutaten}
        400 g & \myindex{Tomate}n \\
        150 g & \myindex{Mozzarella} \\
        10 & \myindex{schwarze Oliven}\index{Oliven>schwarz} \\
        1 Portion & \myindex{Vinaigrette} (siehe Seite \pageref{vinaigrette})
	            \\
        1 Bund & \myindex{Basilikum} \\
      \end{zutaten}

      \personen{4}
      \kalorien{250}

      \begin{zubereitung}
        Tomaten waschen, mit dem Mozzarella in Scheiben schneiden und auf einer
	Platte anrichten. Oliven darübergeben, Vinaigrette verrühren, über die
	Salatzutaten geben. Basilikum waschen, Blätter von den Stengeln
	abzupfen und über den angerichteten Salat geben. \\
      \end{zubereitung}

    \mynewsection{Kidney-Bohnen-Mais-Salat}\glossary{Salat>Bohnen-Mais-}

      \begin{zutaten}
        1 Dose & \myindex{Kidney-Bohnen}\index{Bohnen>Kidney-} (420 ml)\\
        1 Dose & \myindex{Mais} (425 ml) \\
        1 & \myindex{grüne Paprika}\index{Paprika>grün}schote \\
        1 Portion & \myindex{Vinaigrette} (siehe Seite \pageref{vinaigrette})
	            \\
      \end{zutaten}

      \personen{4}
      \kalorien{265}

      \begin{zubereitung}
        Kidney-Bohnen und Mais abtropfen lassen, Paprikaschote halbieren, Kerne
	und weiße Innenhäute entfernen, waschen, in feine Streifen schneiden.
	Vinaigrette verrühren, die Salatzutaten darin mischen, ca. 30~Minuten
	durchziehen lassen und servieren. \\
      \end{zubereitung}

    \mynewsection{Eichblattsalat mit Käse}\glossary{Salat>Eichblatt- mit Käse}

      \begin{zutaten}
        1 & \myindex{Eichblattsalat}\index{Salat>Eichblatt-} \\
        3 & \myindex{Tomate}n \\
        50 g & \myindex{Roquefort}\index{Käse>Roquefort}-Käse\\
        1 Portion & \myindex{Vinaigrette} (siehe Seite \pageref{vinaigrette})
	            \\
      \end{zutaten}

      \personen{4}
      \kalorien{155}

      \begin{zubereitung}
        Eichblattsalat putzen, waschen, zerpflücken, Tomaten waschen, in Achtel
	schneiden, Roquefort-Käse in Würfel schneiden, Vinaigrette verrühren,
	die Salatzutaten darin mischen und anrichten. \\
      \end{zubereitung}

    \mynewsection{Kopfsalat Berta}\glossary{Salat>Kopf- Berta}

      \begin{zutaten}
        1 & \myindex{Kopfsalat}\index{Salat>Kopf-} \\
        2 & \myindex{gelbe Paprika}\index{Paprika>gelb}schoten \\
        1 & \myindex{Ei} \\
        1 Portion & \myindex{Joghurtdressing}
	            (siehe Seite \pageref{joghurtdressing}) \\
        & \myindex{Schnittlauch} \\
      \end{zutaten}

      \personen{4}
      \kalorien{105}

      \begin{zubereitung}
        Kopfsalat putzen, waschen, gut abtropfen lassen, Paprikaschoten
	halbieren, Kerne und weiße Innenhäute entfernen, waschen und in
	Streifen schneiden, Ei kochen, pellen und in Achtel schneiden.
	Joghurtdressing verrühren, die Salatzutaten darin mischen und mit
	Schnittlauchstengeln garniert servieren. \\
      \end{zubereitung}

    \mynewsection{Brokkolisalat}\glossary{Salat>Brokkoli-}

      \begin{zutaten}
	\brea{} l & \myindex{Wasser} \\
	& \myindex{Gemüsebrühe} \\
	1 Packung & tiefgefrorener \myindex{Brokkoli} (\'a 300g) \\
	50 g & \myindex{Champignon}\index{Pilze>Champignon}s \\
	& \myindex{Zitrone}nsaft \\
        1 Portion & \myindex{Joghurtdressing}
	            (siehe Seite \pageref{joghurtdressing}) \\
	2 Kaffeelöffel & gehackte \myindex{Haselnüsse} \\
      \end{zutaten}

      \personen{2}
      \kalorien{160}

      \begin{zubereitung}
        Wasser und Gemüsebrühe zum Kochen bringen, Brokkoli dazufügen und ca.
	10~Minuten dünsten, abtropfen lassen, Champignons putzen, waschen, in
	Scheiben schneiden, mit Zitronensaft beträufeln, die Salatzutaten auf
	zwei Salattellern anrichten. Joghurtdressing darübergeben und mit
	gehackten Haselnüssen bestreuen. \\
      \end{zubereitung}

    \mynewsection{Eisbergsalat mit Aprikosen}%
              \glossary{Salat>Eisberg- mit Aprikosen}

      \begin{zutaten}
	\breh{} Kopf & \myindex{Eisbergsalat}\index{Salat>Eisberg-} \\
	1 Dose & \myindex{Aprikosen} (236 ml) \\
	1 Eßlöffel & \myindex{Sprossen} \\
        1 Portion & \myindex{Joghurtdressing}
	            (siehe Seite \pageref{joghurtdressing}) \\
      \end{zutaten}

      \personen{2}
      \kalorien{140}

      \begin{zubereitung}
        Eisbergsalat in 4 Teile schneiden, waschen, abtropfen lassen, Aprikosen
	abtropfen lassen, Salatzutaten auf 2~Salattellern anrichten.
	Joghurtdressing darübergeben und die Sprossen darüberstreuen. \\
      \end{zubereitung}

    \mynewsection{Radicchiosalat mit Zuckerschoten}%
              \glossary{Salat>Radicchio- mit Zuckerschoten}

      \begin{zutaten}
	200 g & \myindex{Zuckerschoten} \\
	\brea{} l & \myindex{Wasser} \\
	& \myindex{Gemüsebrühe} \\
	100 g & \myindex{Radicchio}\index{Salat>Radicchio} \\
        1 Portion & \myindex{Vinaigrette} (siehe Seite \pageref{vinaigrette})
	            \\
	1 Eßlöffel & gehackte \myindex{Haselnüsse} \\
      \end{zutaten}

      \personen{2}
      \kalorien{215}

      \begin{zubereitung}
        Zuckerschoten putzen, waschen, Wasser und Gemüsebrühe zum Kochen
	bringen. Zuckerschoten dazufügen und ca. 5~Minuten dünsten. Radicchio
	putzen, waschen, gut abtropfen lassen. Salatzutaten auf 2~Salattellern
	anrichten und Vinaigrette darübergeben, mit gehackten Haselnüssen
	bestreut servieren. \\
      \end{zubereitung}

    \mynewsection{Gemischter Blattsalat mit Austernpilzen}%
              \glossary{Salat>Blatt- gemischt mit Austernpilzen}

      \begin{zutaten}
	100 g & gemischter \myindex{Blattsalat}\index{Salat>Blatt-} \\
	10 g & \myindex{Butter} \\
        100 g & \myindex{Austernpilze}\index{Pilze>Austern-} \\
        1 Portion & \myindex{Vinaigrette} (siehe Seite \pageref{vinaigrette})
	            \\
      \end{zutaten}

      \personen{2}
      \kalorien{185}

      \begin{zubereitung}
        Blattsalat waschen, gut abtropfen lassen, eventuell zerpflücken. Butter
	heiß werden lassen, Austernpilze ca. 5~Minuten darin braten. Blattsalat
	auf 2~Salattellern anrichten, Vinaigrette darübergeben. Die
	gebratenen Austernpilze darauf anrichten. \\
      \end{zubereitung}

    \mynewsection{Bohnensalat}\glossary{Salat>Bohnen-}

      \begin{zutaten}
	\brea{} l & \myindex{Wasser} \\
	& \myindex{Gemüsebrühe} \\
        1 Packung & tiefgefrorene \myindex{Brechbohnen}\index{Bohnen>Brech-}
	            (300~g)\\
        100 g & \myindex{Cocktailtomate}\index{Tomate>Cocktail-}n \\
	4 & \myindex{Artischockenherzen} \\
        1 Portion & \myindex{Vinaigrette} (siehe Seite \pageref{vinaigrette})
	            \\
      \end{zutaten}

      \personen{2}
      \kalorien{245}

      \begin{zubereitung}
        Wasser und Gemüsebrühe zum Kochen bringen, tiefgefrorene Brechbohnen
	dazufügen und ca. 20~Minuten dünsten, abtropfen lassen. Cocktailtomaten
	waschen, halbieren, Artischockenherzen halbieren. Salatzutaten auf
	2~Salattellern anrichten und Vinaigrette darübergeben. \\
      \end{zubereitung}

    \mynewsection{\frisee{}salat mit Fisch}%
              \glossary{Salat>\frisee{}- mit Fisch}

      \begin{zutaten}
	1 kleiner Kopf & \myindex{\frisee{}} \\
	\breh{} Bund & \myindex{Radieschen} \\
	ca. 200 g & \myindex{Forellen}\index{Fisch>Forelle}nfilets (oder
	            anderer gegarter Fisch) \\
        1 Portion & \myindex{Vinaigrette} (siehe Seite \pageref{vinaigrette})
	            \\
      \end{zutaten}

      \personen{2}
      \kalorien{345}

      \begin{zubereitung}
        \frisee{} putzen, waschen, gut abtropfen lassen, eventuell zerpflücken,
	Radieschen putzen, waschen, in Scheiben schneiden. Forellenfilets
	zerpflücken, Salatzutaten auf 2~Salattellern anrichten, Vinaigrette
	dazugeben. \\
      \end{zubereitung}

    \mynewsection{Spinatsalat}\index{Salat>Spinat-}

      \begin{zutaten}
        150 g & \myindex{Blattspinat}\index{Spinat>Blatt-} \\
        1 Portion & \myindex{Vinaigrette} (siehe Seite \pageref{vinaigrette})
	            mit Rotweinessig \\
	1 & \myindex{Zwiebel} \\
	50 g & \myindex{Champignon}\index{Pilze>Champignon}s \\
	1 & \myindex{Tomate} \\
	100 g & gebratenes Fleisch (z.B. \myindex{Roastbeef}) \\
      \end{zutaten}

      \personen{2}
      \kalorien{242}

      \begin{zubereitung}
        Blattspinat putzen, waschen und gut abtropfen lassen. Mit Vinaigrette
	mischen. Zwiebel schälen und in Scheiben schneiden. Champignons und
	Tomate waschen und in Scheiben schneiden. Fleisch in Streifen
	schneiden. Alle Salatzutaten mischen und auf 2~Tellern anrichten. \\
      \end{zubereitung}

    \mynewsection{Bunter \chicoree{}salat}\glossary{Salat>\chicoree{}- bunt}

      \begin{zutaten}
        \breh{} Dose & \myindex{rote Bohnen}\index{Bohnen>rot} (425 ml) \\
	1 & \myindex{Apfel} \\
	1 & \myindex{Birne} \\
        1 Portion & \myindex{Vinaigrette} (siehe Seite \pageref{vinaigrette})
	            mit Rotweinessig \\
	1 & \myindex{\chicoree{}} \\
      \end{zutaten}

      \personen{2}
      \kalorien{346}

      \begin{zubereitung}
        Rote Bohnen abtropfen lassen. Apfel und Birne waschen, Kerngehäuse
	entfernen und die Früchte in dünne Scheiben schneiden. Vinaigrette
	darübergießen und durchziehen lassen. \chicoree{} putzen, waschen, in
	grobe Stücke schneiden, mit den Salatzutaten mischen und den Salat auf
	2~Tellern anrichten. \\
      \end{zubereitung}

    \mynewsection{Spargel-Eier-Salat}\glossary{Salat>Spargel-Eier-}

      \begin{zutaten}
	1 Dose & \myindex{Spargel} (460 ml) \\
	2 & \myindex{Ei}er \\
	50 g & \myindex{Champignon}\index{Pilze>Champignon}s \\
        50 g & \myindex{gekochter Schinken}\index{Schinken>gekocht} \\
	\breh{} Schächtelchen  & \myindex{Kresse} \\
        1 Portion & \myindex{Joghurtdressing}
	            (siehe Seite \pageref{joghurtdressing}) \\
      \end{zutaten}

      \personen{2}
      \kalorien{230}

      \begin{zubereitung}
        Spargel abtropfen lassen, Eier hart kochen, pellen und in Scheiben
	schneiden. Champignons waschen und in Scheiben schneiden. Gekochten
	Schinken in Streifen schneiden. Kresse waschen, die Blätter mit einer
	Schere abschneiden. Alle Salatzutaten auf 2~Tellern anrichten und das
	Joghurtdressing darübergeben. \\
      \end{zubereitung}

    \mynewsection{Salat Waldorf}

      \begin{zutaten}
	150 g & \myindex{Sellerie} \\
	1 & \myindex{Apfel} \\
	15 g & \myindex{Walnüsse} \\
        1 Portion & \myindex{Joghurtdressing}
	            (siehe Seite \pageref{joghurtdressing}) \\
	& \myindex{Salatblätter} vom \myindex{Kopfsalat}\index{Salat>Kopf-} \\
	10 g & \myindex{Walnüsse} \\
      \end{zutaten}

      \personen{2}
      \kalorien{210}

      \begin{zubereitung}
        Sellerie schälen und grob raspeln. Apfel schälen, Kerngehäuse
	entfernen und den Apfel in dünne Scheiben schneiden. Walnüsse hacken.
	Die Salatzutaten mit dem Joghurtdressing mischen und durchziehen
	lassen. 2~Teller mit Salatblättern auslegen, den Salat darauf anrichten
	und mit Walnüssen garnieren. \\
      \end{zubereitung}

    \mynewsection{Italienischer Zucchini-Auberginen-Salat}%
              \glossary{Zucchini-Auberginen-Salat}%
	      \glossary{Auberginen-Zucchini-Salat}

      \begin{zutaten}
        2 & \myindex{Zucchini} \\
	2 & \myindex{Aubergine}n \\
	8 Eßlöffel & \myindex{Olivenöl}\index{Oel=Öl>Oliven-} \\
	& \myindex{Salz} \\
	& \myindex{Pfeffer} aus der Mühle \\
	4 Eßlöffel & \myindex{Rotweinessig}\index{Essig>Rotwein-} \\
	\breh{} Bund & \myindex{Basilikum} \\
	200 g & \myindex{Mozzarella}\index{Käse>Mozzarella} \\
      \end{zutaten}

      \personen{4}
      \kalorien{290}

      \begin{zubereitung}
        Zucchini und Auberginen waschen. Zuerst in fingerdicke Scheiben, dann
	in Würfel schneiden. Portionsweise im erhitzten Öl rundrum goldbraun
	braten. Auf Küchenkrepp abtropfen und abkühlen lassen. Mit Salz und
	Pfeffer würzen und vermischen. Dann mit Rotweinessig beträufeln und
	eventuell noch etwas Olivenöl darübergießen. Mit Basilikumblättchen und
	in Scheiben geschnittenem Mozzarella anrichten. Dazu Stangenweißbrot
	und einen trockenen italienischen Rotwein servieren. \\
	Schmeckt auch gut, wenn Sie den Salat mit Thunfisch anrichten. Dafür
	1~große Dose Thunfisch naturell abtropfen lassen und mit einer Gabel in
	grobe Stücke zerpflücken. Dann statt Basilikum feingehackte Petersilie
	verwenden. \\
      \end{zubereitung}

    \mynewsection{Gebackene Fischstreifen auf buntem Salat}%
              \glossary{Fischstreifen>gebacken}

      \begin{zutaten}
        ca. 500 g & \myindex{Fischfilet}\footnote{Frischer Fisch ist vor
	                                          Feiertagen schwierig
						  zubesorgen}
		    (\myindex{Zander}\index{Fisch>Zander},
		     \myindex{Karpfen}\index{Fisch>Karpfen},
		     \myindex{Seelachs}\index{Fisch>Seelachs}) \\
        & \myindex{Salz} \\
        & \myindex{Pfeffer} \\
        & geriebene \myindex{Zitrone}nschale \\
        & Öl\index{Oel=Öl}\footnote{"Ol ist heiß genug, wenn sich am
	                        Holzkochlöffelstiel Bläschen zeigen}
	  zum Frittieren \\
      \end{zutaten}
      \begin{zutat}{Ausbackteig}
        2 & rohe \myindex{Ei}er \\
        & \myindex{Salz} \\
        120 g & \myindex{Mehl} \\
        1 Messerspitze & \myindex{Backpulver} \\
        \brev{} l & \myindex{Bier} \\
        & \myindex{Pfeffer} \\
        etwas & \myindex{Muskatnuß} \\
      \end{zutat}
      \begin{zutat}{Salat}
        1 Handvoll & \myindex{Feldsalat}\index{Salat>Feld-} \\
        1 Handvoll & \myindex{Radicchio} \\
        1 Handvoll & \myindex{\frisee{}} \\
        1 Handvoll & \myindex{Eichblattsalat}\index{Salat>Eichblatt-} \\
      \end{zutat}
      \begin{zutat}{Vinaigrette}
        \breh{} Teelöffel & \myindex{Salz} \\
        3--4 Eßlöffel & guter \myindex{Essig} (Weinessig, Apfel-Sherry) \\
        4 Eßlöffel & \myindex{Olivenöl}\index{Oel=Öl>Oliven-} \\
        & \myindex{Pfeffer} \\
        1 Teelöffel & scharfer \myindex{Senf} \\
      \end{zutat}
      \begin{zutat}{Remoulade}
        2 & hartgekochte \myindex{Ei}er \\
        & \myindex{Salz} \\
        1 Eßlöffel & scharfer \myindex{Senf} \\
        5 Eßlöffel & \myindex{Olivenöl}\index{Oel=Öl>Oliven-} \\
        100 g & \myindex{saure Sahne}\index{Sahne>sauer} \\
        1 & \myindex{Schalotte} gewürfelt \\
        3--4 Eßlöffel & kleine \myindex{Kapern} \\
        2--3 & \myindex{Cornichons} (kleine Gürkchen) \\
        1 Eßlöffel & \myindex{Essig} \\
        1 Teelöffel & rosa \myindex{Pfeffer}beeren \\
        & \myindex{Petersilie} \\
        & \myindex{Schnittlauch} \\
        1 Messerspitze & \myindex{Honig} \\
      \end{zutat}

      \personen{4--6}

      \begin{zubereitung}
        \emph{Fisch}: Fisch in 3~cm breite Streifen schneiden und würzen mit
	Salz, Pfeffer, Zitronenschale. \\
        \emph{Backteig}: Eier mit dem Mehl verrühren,Bier und Gewürze
	hinzugeben. Ca. 20~Minuten quellen lassen. Kurz vorm Ausbacken die
	Messerspitze Backpulver einrühren. \\
        Ausbacken Fischstreifen: Fischstreifen in Backteig tauchen, abtropfen
	lassen und einzeln im heißen Öl schwimmend golden ausbacken, auf
	Küchenkrepp abtropfen lassen. \\
        Salat\footnote{Zu Weihnachten umständlich zu besorgen}: Sauber waschen
	und zerzupfen, gut abtropfen lassen. Vorm Servieren mit Vinaigrette
	mischen. \\
        \emph{Remoulade}: Eigelbe auslösen, mit einer Gabel in einer Schüssel
	mit dem Senf zerdrücken, würzende Zutaten und saure Sahne unterrühren.
	Abschmecken mit Zitronensaft, Pfeffer, Öl, Essig. Eiweiß würfeln und
	unter die Soße rühren. \\
        Anrichten: Salatblätter mit Vinaigrette als Häufchen auf Mitte des
	Vorspeisentellers setzen. Darauf die knusprigen Fischfiletstreifen
	setzen. Mit der Remouladensoße umklecksen, restliche Soße getrennt
	dazu reichen. \\
        \emph{Beilage}: Knuspriges Baguette. Getränk: kraftvoller säurefrischer
	Weißwein --- Riesling-Kabinett aus der Pfalz oder von der Nahe. \\
        gekocht Weihnachten~2005: Fisch frisch zu bekommen sowie Blattsalat war
	schwierig. \\
      \end{zubereitung}

    \mynewsection{\chicoree{}-Salat mit Hähnchenfiletstreifen}

      \begin{zutaten}
	6 & Stauden \myindex{\chicoree{}} \\
	4 & Äpfel\index{Aepfel=Äpfel} (Granny Smith) \\
	250 g & \myindex{Hähnchenfilet}streifen \\
	20 g & \myindex{Butterschmalz}\index{Schmalz>Butter-} \\
	300 g & \myindex{Joghurt} \\
	2 Eßlöffel & \myindex{Olivenöl}\index{Oel=Öl>Oliven-} extra vergine \\
	2 Teelöffel & \myindex{Curry}pulver \\
	1 & \myindex{Zitrone} \\
	30 ml & \myindex{Weißwein}\index{Wein>weiß} \\
	& \myindex{Fleur de Sel} \\
	& \myindex{Pfeffer} aus der Mühle \\
      \end{zutaten}

      \personen{4}

      \begin{zubereitung}
        \chicoree{} putzen, unteres Ende abschneiden und den Strunk
	herausschneiden. Vier Stauden in einzelne Blätter teilen, restliche
	2~Stauden in feine Streifen schneiden. \\
	Einen Apfel schälen, vierteln, Kerngehäuse entfernen. Den Apfel in ein
	hohes Gefäß geben, Weißwein dazugießen und pürieren. Die 3~verbliebenen
	Äpfel waschen und in kleine Würfel schneiden. Zitrone halbieren, Saft
	auspressen und die Äpfel damit beträufeln. \\
	Joghurt in eine Schüssel geben, den pürierten Apfel sowie das
	Olivenöl hinzufügen und alles glattrühren. Mit Currypulver, Fleur de
	Sel und Pfeffer abschmecken. Die Apfelwürfel und die
	\chicoree{}streifen in die Joghurtsoße geben und vorsichtig
	vermengen. \\
	In einer großen Schale die \chicoree{}blätter auslegen und den
	angemachten Salat in die Mitte füllen. \\
	Butterschmalz in einer Pfanne erhitzen und die Hähnchenfiletstreifen
	darin anbraten, würzen mit Fleur de Sel und Pfeffer. Auf dem
	fertigen Salat anrichten. \\
      \end{zubereitung}

    \mynewsection{Mittelmeersalat}

      % aus ,,Treu bis in den Tod`` (Inspektor Barnaby Roman)

      \begin{zutaten}
        & \myindex{Parmesan} gehobelt \\
        & kleine \myindex{Artischocke}nherzen \\
	& \myindex{schwarze Oliven}\index{Oliven>schwarz} \\
	& \myindex{rote Paprika}\index{Paprika>rot} \\
	& kleingeschnittene \myindex{Tomate}n ,,Ailsa Craig`` \\
	& \myindex{Romana}salat in Stücken \\
	& halbierte \myindex{Anchovis} \\
	& \myindex{Croutons} \\
	& \myindex{Olivenöl}\index{Oel=Öl>Oliven-} \\
	& \myindex{Zitrone}nsaft \\
	& \myindex{Kräuter} \\
      \end{zutaten}

      \begin{zubereitung}
        Croutons in Knoblauch getränkt und warm gestellt in einer Eisenpfanne.
	Vinaigrette aus Olivenöl, Zitronensaft und Kräutern aus dem Garten. \\
      \end{zubereitung}

    \mynewsection{Caesar's Salad}

      % ARD 18.07.2010

      \begin{zutaten}
        3 & \myindex{Knoblauchzehe}n \\
	100 ml & \myindex{Olivenöl}\index{Oel=Öl>Oliven-} \\
	4 Scheiben & \myindex{Toastbrot} \\
	2 Köpfe & \myindex{Romana}salat \\
	8 & \myindex{Sardellen}filets aus dem Glas \\
	& Saft von 1 \myindex{Zitrone} \\
	\breh{} Teelöffel & \myindex{Worcestershiresoße} \\
	\breh{} Teelöffel & \myindex{Salz} \\
	\breh{} Teelöffel & \myindex{Pfeffer} gemahlen \\
	100 g & frisch geriebener \myindex{Parmesan} \\
      \end{zutaten}

      \personen{8}

      \begin{zubereitung}
        Knoblauch abziehen, durchpressen. Mit Olivenöl vermischen. Toastbrot
	würfeln, in 4~Eßlöffel der Knoblauch-Öl-Mischung knusprig braten. Auf
	Küchenpapier abtropfen lassen. \\
	Salat putzen, waschen und zerzupfen. Sardellen abbrausen, klein
	schneiden. \brzd{} der Sardellenstückchen in der übrigen
	Knoblauch-Öl-Mischung zerdrücken. Zitronensaft, Worcestersoße, Salz und
	Pfeffer kräftig drunterschlagen, die Marinade abschmecken. \\
	Salatblätter in der Marinade wenden. Parmesan unterheben. Mit übrigen
	Sardellen und Brotwürfeln anrichten und sofort servieren. \\
      \end{zubereitung}

    % \mynewsection{Text}

      % \begin{zutaten}
      % \end{zutaten}

      % \begin{zubereitung}
      % \end{zubereitung}
