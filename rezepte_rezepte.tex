
% created Montag, 10. Dezember 2012 16:22 (C) 2012 by Leander Jedamus
% modifiziert Mittwoch, 11. März 2015 17:14 von Leander Jedamus
% modifiziert Montag, 09. März 2015 14:25 von Leander Jedamus
% modified Montag, 10. Dezember 2012 16:30 by Leander Jedamus

  \mynewchapter{Rezepte}

    \mynewsection{Grießklößchen}

      \begin{zutaten}
	\brea{} l & \myindex{Milch} \\
	etwas & \myindex{Butter} \\
	1 Prise & \myindex{Salz} \\
	10 g & \myindex{Zucker} (1~Eßlöffel) \\
	60 g & \myindex{Grieß}  \\
	1 & \myindex{Ei} \\
      \end{zutaten}

      \garzeit{5}

      \begin{zubereitung}
	Zutaten bereitstellen, Eigelb vom Eiweiß trennen. Milch, Butter, Salz
	lässt man aufkochen, schüttet nach und nach unter Rühren den Grieß ein
	und brennt ihn zum Kloß ab (aufpassen, brennt gern an). Das Eigelb
	kommt in die heiße Masse. Masse abkühlen lassen. Eiweiß zu Eisschnee
	schlagen und unter die abgekühlte Masse heben. Mit nassem Teelöffel
	Klößchen abstechen und in kochendem Salzwasser garkochen (5~Minuten
	Kochzeit). \\
      \end{zubereitung}

    \mynewsection{Apfelkompott}

      \begin{zutaten}
	1 kg & \myindex{Tafeläpfel}\index{Aepfel=Äpfel} (saure oder \myindex{Boskopp}\index{Apfel>Boskopp}) \\
	\brea{} l & \myindex{Wasser} (eventuell etwas mehr) \\
	100 g & \myindex{Zucker} \\
	1 & abgeriebene Zitronenschale (unbehandelte \myindex{Zitrone}) \\
	1 Päckchen & \myindex{Vanillezucker}\index{Zucker>Vanille-} \\
      \end{zutaten}

      \garzeit{ca. 20}

      \begin{zubereitung}
	Äpfel waschen, zerteilen, Kerngehäuse ausschneiden und mit Wasser und
	übrigen Zutaten aufsetzen, gardünsten bei kleiner Flamme. Ab und zu
	umrühren. Wenn die Apfelstücke zerfallen sind, ist die Garzeit um (ca.
	20~Minuten). Man kann den Kompott durch ein Sieb streichen, um Apfelmus
	zu erhalten. \\
      \end{zubereitung}
      \\\\
      Variante \begin{zutaten}
		& gewaschene ungeschwefelte \myindex{Rosinen} \\
		1 & \myindex{Nelke} \\
		1 & \myindex{Vanillestange} \\
		& \myindex{Zimt} \\
               \end{zutaten}

      \begin{zubereitung}
	Vanillestange mitkochen und mit Zimt würzen. \\
      \end{zubereitung}

    \mynewsection{Holländische Soße}\glossary{Soße>Holländisch}%
      \label{hollandsosse}

      \begin{zutaten}
        40 g & \myindex{Butter} oder \myindex{Margarine} \\
        40 g & \myindex{Mehl} \\
        \breh{} l & \myindex{Milch} oder \myindex{Gemüsebrühe} oder
	            \myindex{Fleischbrühe} (Würfel) \\
        & \myindex{Salz} \\
        1 Prise & \myindex{Zucker} \\
        1 Eßlöffel & \myindex{Zitrone}nsaft \\
        20 g & \myindex{Butter} (eventuell zum Verfeinern) \\
        1 & \myindex{Ei} \\
      \end{zutaten}

      \garzeit{ca. 20}

      \begin{zubereitung}
        Bei Blumenkohl die Brühe zu \breh{} Milch und \breh{} Gemüsebrühe mit
	wenig Brühwürfel herstellen. Zutaten und Schneebesen bereitlegen, d.h.
	auch die Brühe sollte fertig bereitstehen. Topf \textbf{langsam} auf
	Wärme bringen (keine große Hitze!!!), Butter oder Margarine hineingeben,
	flüssig werden lassen. Dann unter ständigem Rühren (bei kleiner bis
	mittlerer Hitze) mit dem Schneebesen nach und nach das Mehl einrühren.
	Mehl und Fett haben sich zu einer zähen Masse verbunden. Jetzt langsam
	nach und nach unter ständigem Rühren wenig Brühe/Flüssigkeit zugeben,
	d.h. wenn Masse zu kompakt ist, dann wieder Flüssigkeit zugeben. Die
	Masse köchelt und gart dabei. Wenn die Flüssigkeit aufgebraucht ist,
	sollte bei ständigem Rühren und köcheln die Soße die richtige
	Beschaffenheit haben. Herdplatte ausschalten und/oder Topf auf die
	Seite ziehen. Salz, Zucker, evtl. wenig zerbröseltes Stück
	Fleischbrühwürfel einrühren, Wenn die Soße nicht mehr köchelt und
	leicht abgekühlt ist, Zitronensaft einrühren, kurz darauf das Ei unter
	schnellem Rühren dazugeben (wenn die Soße zu heiß ist, werden sich
	weiße und unerwünschte Fäden zeigen). Jetzt sollte die Soße einen
	leichten Gelbton haben. Eventuell noch abschmecken, frisch geriebene
	Muskatnuß drüber und sofort über das Gemüse geben. \\
      \end{zubereitung}

    \mynewsection{Senfsoße}\glossary{Soße>Senf-}

      \begin{zutaten}
        40 g & \myindex{Butter} oder \myindex{Margarine} \\
        40 g & \myindex{Mehl} \\
        \breh{} l & \myindex{Fleischbrühe} \\
        1 Teelöffel & \myindex{Löwensenf}\index{Senf>Löwen-} oder \\
        1 Eßlöffel & \myindex{mittelscharfer Senf}\index{Senf>mittelscharf} \\
        etwas & \myindex{Zucker} \\
        etwas & \myindex{Salz} \\
      \end{zutaten}

      \garzeit{15}

      \begin{zubereitung}
        Butter/Margarine auslassen, sofort die gesamte Menge Mehl einrühren,
	Senf dazugeben und dann die lauwarme Fleischbrühe dazurühren, ca.
	15~Minuten köcheln lassen, ab und zu umrühren, dann abschmecken, evtl.
	noch Senf dazugeben. Die Senfsoße gibt man über die Eihälften gekochter
	Eier (10~Minuten kochen lassen). Pro Person rechnet man 2~Eier. Dazu
	paßt gut Endiviensalat (siehe Seite \pageref{endiviensalat}). \\
      \end{zubereitung}

    \mynewsection{Möhrengemüse}

      \begin{zutaten}
        1000--1500 g & \myindex{Möhre}n \\
        40 g & \myindex{Margarine} \\
        1 mittlere & \myindex{Zwiebel} \\
        \breh{} Teelöffel & \myindex{Salz} \\
        1 Prise & \myindex{Zucker} \\
        etwas & \myindex{Pfeffer} \\
        etwas & \myindex{Muskatnuß} \\
        etwas & \myindex{Fleischbrühwürfel} \\
        ca. \brev{} l & heißes \myindex{Wasser} \\
        1 Bund & \myindex{Petersilie} \\
      \end{zutaten}

      \garzeit{20}

      \begin{zubereitung}
        Am besten frische Möhren und etwas dicker als fingerdicke nehmen. Zu
	dicke Möhren schmecken nicht so gut. Zeitung unterlegen. Schadstellen
	und Blattansätze abschneiden. Möhren putzen. Entweder man schabt die
	Möhren oder man schält die Möhren dünn oder aber man nimmt sich einen
	Haushaltshandschuh und einen großen Topf mit Salz und nimmt Salz in die
	behandschuhte Hand und reibt damit die Möhren sauber (Spart auch eine
	eventuell notwendige Wäsche der Möhren wegen Erdklumpen). Jetzt die
	Möhren entweder (je nach Dicke) in dünnere Scheiben, Stifte oder
	Viertelscheiben schneiden. Die Putzarbeit dauert am längsten. \\
        Zwiebel schälen, würfeln und zwischendurch größeren Topf oder Pfanne
	aufstellen und erhitzen (aber nicht sofort auf volle Stufe). Fett
	zerlassen. In zerlassenes und heiß werdendes Fett die Zwiebelwürfel
	geben und glasig werden lassen (rühren, rühren und mittlere Hitze).
	Wenn die Zwiebelwürfel glasig sind, die geputzten Möhren hinzugeben und
	andünsten. Hitze höher stellen und etwas Würfel und lauwarmes/heißes
	Wasser drüber geben, Salz, Pfeffer, Zucker, Würfel dazu. Man kann auch
	das Wasser weglassen und statt dessen ständig Margarine dazugeben (so
	mache ich es meistens). Dann bräunen die Möhren etwas und schmecken
	auch anders, als bei Wasserzugabe. Die Gewürze gibt man in dem Fall
	ziemlich am Schluß hinzu. Kurz vor Garende (20~Minuten etwa)
	abschmecken, geriebene Muskatnuß dazu, Petersilie waschen, gründlich
	ausschütteln und fein hacken und unter die Möhren geben (darf nicht
	mehr kochen). \\
        Varianten: \\
        Man kann auch 100~g geschälte Mandeln in den letzten 5~Minuten der
	Garzeit hinzugeben oder gebratene Mandelstifte, wenn es etwas
	,,feiner`` sein soll, dann nimmt man am Schluß nur \brev{}--\breh{}
	Bund Petersilie. \\
        ODER \\
        man gibt am Schluß \breh{} Becher süße Sahne an die Möhren. \\
      \end{zubereitung}

    \mynewsection{Brokkoli}

      \begin{zutaten}
        ca. 500 g & \myindex{Brokkoli} \\
        etwas & \myindex{Salz} \\
        & \myindex{Butter}, \myindex{Muskatnuß} \\
      \end{zutaten}

      \begin{zubereitung}
        Nur feste und geschlossene Brokkoli-Rosen auswählen. Sobald die
	Oberseite mehr gelblich aussieht oder gar Samen sichtbar sind, nicht
	kaufen!! Schmeckt bitter. Stiele schälen. Wenn man die Brokkoli nicht
	in einen Topf stellen kann mit den Rosen oben, dann besser die Stiele
	abtrennen und diese zuerst in Salzwasser garen (10~Minuten), danach für
	die restlichen 5--7~Minuten die oberen Teile dazugeben. Diese vertragen
	keine vollen Garzeiten, zerfallen zu Mus. \\
        Abgießen, Butter dazu, zerfließen lassen. Evtl. mit frisch geriebener
	Muskatnuß würzen. \\
        Variante: Fast gare Brokkoli in feuerfeste Form oder Pfanne (ohne
	Kunststoffgriffe), die man mit wenig Butter ausgestrichen hat, legen,
	darüber Scheiben Käse (Leerdamer oder Emmentaler) und überbacken. Wenn
	der Käse sich verflüssigt, rausnehmen und servieren. \\
      \end{zubereitung}

    \mynewsection{Blumenkohl}\glossary{Kohl>Blumen-}

      \begin{zutaten}
        1 Kopf & \myindex{Blumenkohl}\index{Kohl>Blumen-}%
                 \footnote{fest, grüne Blätter außen, muß nicht ,,weiß`` sein} \\
        \brdv{} l & \myindex{Wasser} \\
        1 Teelöffel & \myindex{Salz} \\
        etwas & geriebene \myindex{Muskatnuß} \\
        60 g & \myindex{Butter} \\
        oder & holländische \myindex{Soße}
	       (siehe Seite \pageref{hollandsosse}) \\
      \end{zutaten}

      \garzeit{15--20}

      \begin{zubereitung}
        Blumenkohl\footnote{am besten aus eigenem Anbau} von grünen Blättern
	befreien, äußere Haut der Stengel (soweit es möglich ist) entfernen.
	Entweder läßt man den Kopf ganz (passenden Topf wählen) oder teilt ihn
	in einzelne Röschen (kürzere Garzeit). Kohl sauber putzen (braune
	Stellen wegschneiden). Eventuell den Kohl in kaltes Salz- oder
	Essigwasser legen, um Raupen und Insekten zu entfernen. Den sauberen
	Kohl gibt man in kochendes Salzwasser. Blumenkohl mit geriebener
	Muskatnuß und Butter anrichten oder mit holländischer Soße servieren.
	\\
      \end{zubereitung}

    \mynewsection{Rosenkohl}\glossary{Kohl>Rosen-}

      \begin{zutaten}
        1 kg & \myindex{Rosenkohl}\index{Kohl>Rosen-} \\
        \breh{} Teelöffel & \myindex{Salz} \\
        etwas & \myindex{Butter} \\
        etwas & \myindex{Muskatnuß} \\
      \end{zutaten}

      \begin{zubereitung}
        Am besten mittelgroße und gleichgroße Rosenkohlröschen mit fester
	Beschaffenheit (nicht gelb) wählen. Rosenkohl sollte ersten Frost
	abbekommen haben. Rosenkohl von losen Blättern befreien, Strunk
	abschneiden und möglichst kleinen Kegel in den Strunk schneiden. Mit
	Wasser und Salz aufsetzen und 20~Minuten garen. Abgießen, Muskatnuß
	drüber reiben und Butter darüber geben. Schwenken. \\
      \end{zubereitung}

    \mynewsection{Grünkohl}\glossary{Kohl>Grün-}

      \begin{zutaten}
        ca. 1 kg & frischer \myindex{Grünkohl}\index{Kohl>Grün-} \\
        ca. 50 g & durchwachsener \myindex{Speck} \\
        2--3 Eßlöffel & \myindex{Schmalz} (50g) \\ \\
        1 Teelöffel & \myindex{Salz} \\
        1 Prise & \myindex{Zucker} \\
        4--5 & \myindex{Nelke}n \\
        1 & \myindex{Zwiebel} \\
        \brev{} l & \myindex{Wasser} \\
        1 & trockene feste \myindex{Mettwurst} \\
      \end{zutaten}

      \garzeit{30--45}

      \begin{zubereitung}
        Grünkohl frisch muß unbedingt ersten Frost abbekommen haben. Fragen
	beim Einkauf. Sieht man oft auch, daß Eisstücke/Schnee dran hängt.
	Grünkohl von den einzelnen Wedeln abstreifen. Alles gut waschen und
	tropfnaß in einem großen Topf ohne Wasserzugabe dünsten bis der
	Grünkohl zusammenfällt. Grünkohlbrühe aufheben, Kohl abtropfen und
	abkühlen lassen, dann hackt man ihn klein. Fett in Topf zerlassen,
	gewürfelten Speck auslassen bei geringer bis mittlerer Hitze. Speck
	rausnehmen oder drinlassen. Gewürfelte Zwiebeln glasig dünsten,
	Mettwurst dazugeben (vorher einige Male einstechen), Grünkohl mit
	Grünkohlbrühe, Wasser darauf (evtl. mit wenig Brühwürfel), Gewürze dazu
	und läßt alles bei kleiner Flamme 30--45~Minuten köcheln. Abschmecken.
	\\
        Dazu paßt gut Kartoffelbrei. \\
      \end{zubereitung}

    \mynewsection{Grünkohl mit Mettwürstchen}\glossary{Kohl>Grün-}

      \begin{zutaten}
        750 g & \myindex{Grünkohl}\index{Kohl>Grün-} \\
        50 g \myindex{Schmalz} \\
        1 & \myindex{Zwiebel} \\
        250 g & Mettwürste\index{Mettwurst} \\
        \brev{} l & \myindex{Wasser} \\
        1 Teelöffel & \myindex{Salz} \\
        1 Teelöffel & \myindex{Zucker} \\
        4--5 & \myindex{Nelke}n \\
        1 Prise & \myindex{Zimt} \\
        2 Eßlöffel & \myindex{Haferflocken} \\
        1 Teelöffel & \myindex{Senf} \\
      \end{zutaten}

      \garzeit{30--45}

      \begin{zubereitung}
        Der Kohl schmeckt am besten, wenn er Frost bekommen hat. Der
	abgestreifte, gut gewaschene Grünkohl wird ohne Wasserzugabe gedünstet,
	bis er zusammenfällt. Man läßt den Kohl abtropfen, hackt ihn oder gibt
	ihn durch die Maschine. Man zerläßt das Fett, dünstet darin die Zwiebel
	und gibt die Wurst hinein, in die man ein paar Mal mit der Gabel
	gestochen hat. Man gibt den Grünkohl darauf, das Wasser, die
	Grünkohlbrühe und die Gewürze. Die Haferflocken unterrührt man und läßt
	das Gericht 30~Minuten bei kleiner Flamme dünsten. Zuletzt unterrührt
	man den Senf und schmeckt ab. \\
      \end{zubereitung}

    \mynewsection{Wirsinggemüsetopf}

      \begin{zutaten}
        1 & großer \myindex{Wirsing} \\
        & \myindex{Kartoffel}n geschält und in Stückchen \\
        1 Teelöffel & \myindex{Salz} \\
        1 Prise & \myindex{Pfeffer} \\
        1 Teelöffel & \myindex{Gemüsebrühe} \\
        & \myindex{Sonnenblumenöl}\index{Oel=Öl>Sonnenblumen-} \\
        ca. \breh{} l & lauwarmes \myindex{Wasser} \\
      \end{zutaten}

      \garzeit{ca. 40}

      \begin{zubereitung}
        Wirsing von äußeren dunklen Blättern befreien und vierteln. Strunk
	entfernen. Die Viertel in Streifen schneiden. Topf mit reichlich Öl
	erhitzen und Wirsing nach und nach anbraten und wenden. Öl zugeben bei
	Bedarf. Dann Kartoffeln und die Gewürze  zugeben. Mit Wasser aufgießen
	und umrühren. Nach Aufkochen klein stellen. Garzeit ca. 40~Minuten. Man
	kann auch Würstchen, Mettwürstchen oder Hackbällchen mitgaren. \\
      \end{zubereitung}

    \mynewsection{Wirsing gebraten}

      \begin{zutaten}
        1 & großer \myindex{Wirsing}kopf \\
        1 Teelöffel & \myindex{Salz} \\
        1 Prise & \myindex{weißer Pfeffer}\index{Pfeffer>weiß} \\
        & Öl\index{Oel=Öl} zum Braten \\
      \end{zutaten}

      \garzeit{20}

      \begin{zubereitung}
        Der Wirsing wird von äußeren, schadhaften Blättern befreit, geviertelt,
	der Strunk entfernt. Die Viertel werden in schmale Streifen geschnitten
	und in reichlich heißem Öl angebraten. \\
        Ständig wenden, eventuell Öl nachgeben, bis der Wirsing hellbraun
	gebraten ist. Danach würzen. \\
      \end{zubereitung}

    \mynewsection{Kohlrouladen}

      \begin{zutaten}
        500--750 g & \myindex{Wirsing} oder
                     \myindex{Weißkohl}\index{Kohl>Weiß-} \\
        & \myindex{Salzwasser} \\
      \end{zutaten}
      \begin{zutat}{Füllung}
        150--200 g & \myindex{Gehacktes}%
	             \footnote{alternativ:
	                       Hirsefrikadellenbratlingsmischung siehe Seite
			       \pageref{hirsefrikadellen}}
		     \footnote{125 g = 1 Tasse Hirse reicht für die Füllung
		               für 2 Personen} \\
        1 & \myindex{Brötchen} \\
        1 & \myindex{Zwiebel} \\
        & \myindex{Salz} \\
        & \myindex{Pfeffer} \\
        1 & \myindex{Ei} \\
        & \myindex{Paniermehl} \\
        60 g & \myindex{Margarine} \\
        \brev{} l & \myindex{Wasser} oder \myindex{Brühe} \\
        1 Eßlöffel & \myindex{Mehl} \\
        1 & \myindex{Muskatnuß} \\
      \end{zutat}

      \garzeit{ca. 30}

      \begin{zubereitung}
        Der vorbereitete Kohlkopf wird 3--5~Minuten in kochendes Salzwasser
	gegeben. Danach werden die Blätter vosichtig gelöst und die dick
	aufliegenden Rippen flachgeschnitten. \\
        Gehacktes: Fleisch in die Schüssel geben, ein altes Brötchen länger
	einweichen, Zwiebel in Würfel schneiden, 1~Ei, die Zwiebel, Salz und
	Pfeffer, das ausgedrückte Brötchen an das Fleisch geben und gut
	durchkneten. Der Teig sollte gut gewürzt sein. \\
        Man legt 2--3~große Blätter übereinander, gibt das Gehackte
	hinein, rollt die Blätter zusammen und verschnürt diese mit Bindfaden.
	Kohlrollen abtrocknen und danach in das ausgelassene Fett legen. Von
	allen Seiten leicht anbräunen, Brühe hinzugeben und 15--20~Minuten
	schmoren lassen. Die Rollen aus der Brühe heben und die Soße mit Mehl
	sämig machen. \\
      \end{zubereitung}

  \mynewsection{Weißkohl (Kümmelkraut)}

    \begin{zutaten}
      750--1000 g & \myindex{Weißkohl}\index{Kohl>Weiß-} \\
      2 Eßlöffel & \myindex{Mehl} \\
      \brev{}--\breh{} l & \myindex{Brühe} \\
      40 g & \myindex{Fett} \\
      1 Eßlöffel & \myindex{Kümmel} \\
      & \myindex{Salz} \\
      & \myindex{Pfeffer} \\
    \end{zutaten}

    \begin{zubereitung}
      Vom Weißkraut werden die äußeren Blätter und die Strunke entfernt, dann
      wird es gehobelt oder in Streifen geschnitten. Nun gibt man das Kraut mit
      dem Fett und der Brühe in den Topf, streut Salz, etwas Pfeffer und Kümmel
      darüber und dämpft es weich. Mit etwas Mehl abstäuben, nochmals kurz
      nachkochen und abschmecken. Zum Schluß kann man nach Belieben ein Glas
      Weißwein darangießen. \\
    \end{zubereitung}

    \mynewsection{Rotkohl}\glossary{Kohl>Rot-}\label{rotkohl}

      \begin{zutaten}
        1 großen & frischen \myindex{Rotkohl}\index{Kohl>Rot-}
                   (am besten Bio) \\
        50 g & durchwachsener \myindex{Speck} \\
        2--3 Eßlöffel (50 g) & \myindex{Schmalz} \\
        1 Teelöffel & \myindex{Salz} \\
        1 Teelöffel & \myindex{Zucker} \\
        1 Prise & \myindex{Pfeffer} \\
        2 Eßlöffel & \myindex{Johannisbeergelee} \\
        2 Eßlöffel & \myindex{Essigessenz} \\
        2 & \myindex{Boskopp}äpfel\index{Aepfel=Äpfel>Boskopp-} \\
        1 & \myindex{Lorbeer}blatt \\
        5 & \myindex{Nelke}n \\
        1 & \myindex{Zwiebel} \\
        1 Teelöffel & \myindex{Gemüsebrühe} \\
        \brev{} l & heißes \myindex{Wasser} \\
          & \myindex{Himbeeressig} zum abschmecken \\
      \end{zutaten}

      \begin{zutat}{Marinieren}
        & \myindex{Salz} \\
        & \myindex{Zucker} \\
        & \myindex{Essigessenz} \\
      \end{zutat}

      \garzeit{30--45}

      \begin{zubereitung}
        Vom Rotkohl die ersten Blätter entfernen, vierteln, die Strünke
	entfernen. Die Viertel in feine Streifen schneiden oder hobeln.
	Lagenweise (etwa 1\breh{} Finger hoch) in eine sehr große Schüssel
	geben. Darauf Salz und Zucker (\breh{}--\brdv{} Teelöffel) sowie
	einige Spritzer Essigessenz geben. Die Schüssel dicht mit
	Klarsichtfolie verschließen. Zwischendurch die Schüssel öfter auf
	den Kopf stellen. 24h stehen lassen. \\
	Die Äpfel schälen, vierteln, Kerngehäuse entfernen und den Rest
	würfeln. Eine ganze geschälte Zwiebel mit den Nelken spicken. \\
        Fett im Topf zerlassen, gewürfelten Speck auslassen bei geringer bis
	mittlerer Hitze. Speck rausnehmen. Jetzt Rotkohl lagenweise anschmoren.
	Danach den restlichen Rotkohl mit den übrigen Zutaten zugeben.
	Eventuell muß nach 1~Stunde der Rotkohl z.B. mit Himbeeressig 
	nachgewürzt werden. \\
        Rotkohl sollte besser am Vortag oder noch früher zubereitet werden, da
	er dann zum Essen besser schmeckt. \\
      \end{zubereitung}

    \mynewsection{Lauchgemüse (Porree)}\glossary{Porree}

      \begin{zutaten}
        ca. 1 kg & \myindex{Porree}\index{Lauch} \\
        \breh{} Teelöffel & \myindex{Salz} \\
        & \myindex{Wasser} \\
        200 g & \myindex{süße Sahne}\index{Sahne>süß} \\
      \end{zutaten}

      \garzeit{15}

      \begin{zubereitung}
        Porree von nicht verwertbaren Blättern und Teilen befreien. Gemüse in
	feine Ringe schneiden und etwa 3 mal waschen. Noch tropfnaß in einen
	Topf geben, Salz und Wasser dazu ($\frac{2}{3}$ des Porrees etwa sind
	im Wasser). 15~Minuten garen, abgießen. Einfach den Becher Sahne drüber
	geben und servieren. \\
      \end{zubereitung}

    \mynewsection{Hackfleischsoße}\glossary{Soße>Hackfleisch-}

      \begin{zutaten}
        ca. 50 g & gewürfelten durchwachsenen \myindex{Speck} \\
        1 Eßlöffel & \myindex{Fett} \\
        1 & \myindex{Zwiebel} \\
        100 g & \myindex{Gehacktes} (\breh{} Rind, \breh{} Schwein) \\
        1 & \myindex{Bratensoße} (Maggi, Knorr) \\
        \brev{} l & heißes \myindex{Wasser} \\
        1 & kleines \myindex{Lorbeer}blatt \\
      \end{zutaten}

      \garzeit{20--30}

      \begin{zubereitung}
        Pfanne oder Topf heiß werden lassen, Fett und kleingewürfelten Speck
	hineingeben. Speck vorsichtig auslassen (keine große Hitze !!!), ab und
	zu wenden. Ausgebratene Speckwürfel drin lassen oder rausheben.
	Zwiebelwürfel glasig dünsten. Hackfleisch zerkrümeln und dazugeben.
	Hitze höher stellen und Hackfleisch durchrühren/wenden, bis es
	bräunlich ist. Lorbeerblatt dazu und heißes Wasser. Bratensoße
	einrühren, aufkochen lassen und dann klein stellen. Die letzten
	5~Minuten Platte abstellen und Soße ziehen lassen. Abschmecken. \\
        Varianten: geschnittene oder Gewürzgurke dazugeben. \\
        geschnittene Champignons aus der Dose mitdünsten (pfeffern nicht
	vergessen). \\
      \end{zubereitung}

    \mynewsection{Champignons in Sahnesoße}

      \begin{zutaten}
        ca. 200 g & frische \myindex{Champignon}\index{Pilze>Champignon}s \\
        ca. 100 g & \myindex{Margarine} \\
        1 & \myindex{Zwiebel} \\
        & \myindex{Salz}, \myindex{Pfeffer} \\
        200 g & \myindex{süße Sahne}\index{Sahne>süß} (1 Becher) \\
        1 Bund & \myindex{Petersilie} \\
        1 Eßlöffel & frischer \myindex{Zitrone}nsaft \\
      \end{zutaten}

      \garzeit{20}

      \begin{zubereitung}
        Champignons waschen, untere Teile und schadhafte Stellen wegschneiden.
	Dann die Pilze in Scheiben schneiden und falls nötig in sauberem
	Geschirrtuch trocken machen. Pfanne heiß werden lassen. Etwa 1~Eßlöffel
	Fett zerlassen und gewürfelte Zwiebeln glasig dünsten. Dann Champignons
	dazugeben und dünsten, nach Bedarf Fett nachgeben. Meist ziehen die
	Pilze Wasser in der Pfanne, so daß man die Platte auf große Hitze
	stellen muß und die Pilze erst wendet, wenn man riecht, daß die
	Bräunung einsetzt (Vorsicht, verbrennt manchmal). Sind alle Pilze braun
	und die Flüssigkeit ziemlich verschwunden, salzen und pfeffern, süße
	Sahne drüber. Das Ganze kocht auf, (sollte aber nicht richtig
	weiterkochen !) Hitze ganz klein stellen oder ausstellen. Durchrühren,
	abschmecken, evtl. etwas Zitronensaft dazugeben. Gehackte Petersilie
	unterheben und auftragen. \\
        Dazu passen Nudeln und frische Salate oder Gemüse. Die Soße paßt auch
	zu gebratenen Fleischgerichten oder Kartoffelbrei. \\
      \end{zubereitung}

    \mynewsection{Makkaroniauflauf mit Schinken (im Backofen)}

      \begin{zutaten}
        250 g & \myindex{Makkaroni} \\
        1 \breh{} l & \myindex{Wasser} \\
        1 Teelöffel & \myindex{Salz} \\
        200--250 g & Würfel von rohem \myindex{Schinken} \\
        60 g & geriebener \myindex{Käse} \\
        2 & Eier \\
        \brev{} l & \myindex{Milch} oder
	            \myindex{saure Sahne}\index{Sahne>sauer} \\
        1 Prise & \myindex{Salz} \\
        & \myindex{Muskatnuß} \\
        & \myindex{Cayennepfeffer}\index{Pfeffer>Cayenne-} \\
        & \myindex{Worcestershiresoße} \\
        2 Eßlöffel & \myindex{Paniermehl} \\
        2 Eßlöffel & \myindex{Butter}flocken \\
      \end{zutaten}

      \garzeit{10 + 45}

      \begin{zubereitung}
        Makkaroni in etwa 7~cm lange Stück brechen und in kochendes Salzwasser
	geben und garen (ca. 10~Minuten, auf die Packung sehen). Abgießen und
	mit kaltem Wasser abspülen, abtropfen lassen. Auflaufform mit Butter
	ausstreichen und schichtweise Makkaroni und Schinkenwürfel hineingeben.
	Eier, Milch (oder saure Sahne), Salz, Muskat, Cayennepfeffer und
	Worcestershiresoße verquirlen und über die Masse gießen. Obenauf mit
	Paniermehl abdecken. Darauf die Butterflöckchen verteilen und in den
	vorgeheizten Backofen (größte Hitze) stellen und 45~Minuten garen. \\
        Dazu grünen Salat. \\
	Alternativ: Erbsen und gekochter Schinken. \\
      \end{zubereitung}

    \mynewsection{Grillhähnchen (Grill Voraussetzung)}

      \begin{zutaten}
        ca. 900 g & \myindex{Hähnchen} oder \myindex{Poularde} \\
        & \myindex{Salz} zum Einreiben \\
        & \myindex{Curry}-\myindex{Paprika}-Gemisch (\breh{}+\breh{}) \\
        & Öl\index{Oel=Öl} zum Einpinseln \\
      \end{zutaten}

      \bemerkung{Drehregler auf ,,8``, alle Schalter auf ein.} \\

      \begin{zubereitung}
        Hähnchen am Vorabend im Kühlschrank ganz auftauen lassen (vorher Folie
	entfernen und auf tiefen Teller legen) oder frisches Hähnchen kaufen.
	Innereien entfernen. Darauf achten, daß innen keine Galle mehr ist
	(Fleisch verdirbt dann völlig!!). Galle beim Entfernen nicht
	beschädigen. Hähnchen jetzt innen und außen gut ab- und ausspülen und
	mit sauberem Geschirrtuch abtrocknen. Innen das Hähnchen gut mit Salz
	einreiben. Außen mit Curry-Paprika-Öl-Gemisch in einer Tasse/Becher
	bepinseln oder mit Curry-Paprika-Pulvermischung bestreuen und im Grill
	mit Öl bepinseln. Das Hähnchen auf den Grillspieß bringen, die Flügel
	und die Beine mit Reihgarn (das ist so eine Art Heftfaden zum Nähen)
	festmachen am Rumpf. Grill vorheizen (Anleitung für Geflügel beachten),
	Spieß einhängen und ab und zu das Hähnchen mit Öl bepinseln. Garzeit
	ca. 45~Minuten. Garprobe mit Stricknadel in Hühnerbein bis auf den
	Knochen. Tritt noch rote/rosa Flüssigkeit aus, dann noch nicht gar. \\
        Dazu gebratene Kartoffeln, Kartoffelbrei, Salate. \\
        SALMONELLENGEFAHR !!! Sofort nachdem das Hähnchen im Grill ist,
	\textbf{alle Gerätschaften} (Brettchen, Messer, Teller, Flächen, eigene
	Hände usw.) \textbf{gründlich} mit heißem Wasser und Scheuerpulver
	reinigen, \textbf{bevor} irgend eine andere Essenszutat zubereitet
	wird. Benutztes Geschirrtuch (evtl. auch den Spüllappen) in die Wäsche
	geben --- vorher trocknen lassen. \\
        Das ist keine übertriebene Vorsicht, sondern leider schon schmerzhaft
	erlebt worden über einen langen Krankheitszeitraum. \\
        Reste von dem Hahn nie länger aufheben. Auch aufgetaute und nicht
	zubereitete Hähnchen lieber fortwerfen, als eine Salmonellenvergiftung
	zu riskieren. \\
      \end{zubereitung}

    \mynewsection{Schweinekrustenbraten (im Backofen)}

      \begin{zutaten}
        ca. 1 kg & \myindex{Schweinebraten}\footnote{Schinkenstück, vorbestellen mit rautenförmigen Einschnitten, bis in die Fettschicht und besser direkt sagen, daß auf der Schwarte kein blauer Stempel sein soll.} mit Schwarte \\
        reichlich & \myindex{Salz} zum Einreiben \\
        etwas & \myindex{weißer Pfeffer}\index{Pfeffer>weiß} \\
      \end{zutaten}

      \garzeit{ca. 90}

      \begin{zubereitung}
        Schwarte mit reichlich Salz einreiben. Etwas pfeffern. Backofen
	vorheizen. Braten auf den Rost legen oder in gußeisernen Topf (wenig
	Fett/Öl, Schmalz), etwa 1~\breh{}~Stunden garen. \\
        Dazu Semmelknödel oder Kartoffelsalat und Gurkensalat. \\
      \end{zubereitung}

    \mynewsection{Semmelknödeln}\glossary{Knödel>Semmel-}

      \begin{zutaten}
        10 & \myindex{Brötchen} vom Vortag \\
        1 & größere \myindex{Zwiebel} \\
        3 & \myindex{Ei}er \\
        1 Bund & \myindex{Petersilie} \\
        ca. \brev{} l & lauwarme \myindex{Milch} \\
        etwas & \myindex{Salz}, \myindex{weißer Pfeffer}\index{Pfeffer>weiß} \\
      \end{zutaten}

      \garzeit{15--20}

      \begin{zubereitung}
        Brötchen/Semmeln in Scheiben schneiden (etwa \breh{}~cm dick) und in
	eine größere Schüssel geben. Zwiebel würfeln, Petersilie hacken und in
	Fett andünsten. Milch über die Semmeln gießen, dann die Eier, Salz und
	Pfeffer und die gedünstete Zwiebel mit Petersilie samt Fett dazugeben
	und gut (aber nicht zu heftig) durchmischen und abschmecken. Sollte
	jetzt schon sehr würzig schmecken. Mit feuchten Händen Knödel formen
	und in großem Topf (damit alle Knödel genug Platz nebeneinander haben)
	in reichlich kochendes Salzwasser einlegen. Salzwasser sollte danach
	leicht kochen und Knödel müßten dann in 15--20~Minuten gar sein. Am
	Schluß nur noch sieden lassen. \\
        Dazu Schweinsbraten, auch zu gebratenen Pilzen mit Soße, Gurkensalat
	oder grüner Salat. \\
      \end{zubereitung}

    \mynewsection{Radi}

      \begin{zutaten}
        1 mittlerer/großer & weißer \myindex{Rettich} (\myindex{Radi}) \\
        & \myindex{Salz} zum Einreiben \\
        & \myindex{Butterbrote}\index{Brot>Butter-} \\
      \end{zutaten}

      \begin{zubereitung}
        Den festen Rettich evtl. säubern. Spiralförmig in dünnere Scheiben
	schneiden. Tüchtig salzen und Wasser ziehen lassen. Mit Brot und Butter
	und evtl. kräftigem Emmentaler Käse (und eigentlich einem leichtem
	bayerischen Bier) servieren. \\
        Gut als Abendbrot. \\
      \end{zubereitung}

    \mynewsection{Zwiebelkuchen}\glossary{Kuchen>Zwiebel-}

      \begin{zutaten}
      \end{zutaten}
      \begin{zutat}{Teig}
        500 g & \myindex{Mehl} \\
        20 g & \myindex{Hefe} (frisch) \\
        \brev{} l & \myindex{Milch} (lauwarm) \\
        & \myindex{Salz} \\
        & \myindex{Zucker} \\
        1 & \myindex{Ei} \\
      \end{zutat}
      \begin{zutat}{Belag}
        700 g & \myindex{Zwiebel}n \\
        200 g & gewürfelter, roher \myindex{Schinken}/\myindex{Schinkenspeck} \\
        2 Becher & \myindex{Schmand} \\
        2 & \myindex{Ei}er \\
        & \myindex{Salz} \\
        & \myindex{Rosenpaprika}\index{Paprika>Rosen-} \\
        & \myindex{weißer Pfeffer}\index{Pfeffer>weiß} \\
      \end{zutat}

      \garzeit{20 + 40}

      \begin{zubereitung}
        Die Zutaten des Teiges sollten Zimmertemperatur haben. Eventuell sollte
	man den Teig am Tag zuvor bereiten und in den Kühlschrank stellen. \\
        Teig: Mehl sieben und abwiegen. In Schüssel geben und Vertiefung in
	Mitte machen. Hefe in eine Tasse zerbröseln, \breh{}~Teelöffel Zucker
	und wenig lauwarme Milch ({$\frac{1}{16}$}~l ca.) dazugeben, rühren bis
	die Hefe sich aufgelöst hat. \\
        Backofen auf \grad{50} stellen. Hefemischung in Vertiefung des Mehls
	geben und mit wenig Mehl bestäuben. Tuch darüber und in den Backofen
	für 15 Minuten stellen. Danach Backofen abstellen, nur Licht anlassen
	(wegen der Wärme). Dann nachsehen, ob Vorteig Blasen zeigt und sich
	wölbt. Dann seitlich Salz, Ei und Hälfte der lauwarmen Milch zugeben.
	Kneten und bei Bedarf weiter Milch zugeben. Teig sollte glatt sein und
	sich von der Schüssel lösen. Mit Tuch bedecken und für 30~Minuten im
	Backofen stellen. Danach sollte der Teig sich verdoppelt haben. \\
        Backofen auf \grad{220} vorheizen. Teig auf gefettetes Backblech
	ausrollen, Ränder andrücken. Mit Tuch abdecken und an ruhige Stelle
	tun. \\
        Belag: Zwiebeln würfeln, 2~Pfannen anheizen. In Fett Zwiebeln und
	Schinkenwürfel glasig dünsten. Danach 10~Minuten mit Deckel ziehen
	lassen in einer Pfanne. \\
        Schmand in Schüssel geben, 2~Eier dazu, würzen mit Salz, weißem Pfeffer
	und viel Paprika. Mischung sollte sehr würzig schmecken. \\
        In die Schmandmischung die Hälfte der Zwiebeln rühren und auf den Teig
	verteilen. Dann oben drauf restliche Zwiebeln verteilen. Dann Backen
	etwa 40~Minuten bei \grad{220} zweite Schiene von unten. Mit
	Federweißer servieren. \\
      \end{zubereitung}

    \mynewsection{Heringfilets}

      \begin{zutaten}
        3x 120 g & enthäutete, entgrätete
	           \myindex{Hering}\index{Fisch>Hering}filets \\
        3 Becher & \myindex{saure Sahne}\index{Sahne>sauer} \'a 150 g \\
        2 Eßlöffel & \myindex{Salatmayonnaise}\index{Mayonnaise>Salat-}
	             vom Thomy, Nadler \\
        1 & großen \myindex{Boskopp}\index{Apfel>Boskopp}-Apfel \\
        1 & große \myindex{Zwiebel} \\
        & \myindex{Salz} \\
        & \myindex{Pfeffer} \\
        & \myindex{Senf} \\
        & \myindex{Paprika} \\
        & \myindex{Zucker} \\
      \end{zutaten}

      \begin{zubereitung}
        In eine Schüssel Sahne und Mayonnaise, die geschälten und in Stückchen
	(Scheibchen) geschnittenen Äpfel und die dünnen Zwiebelringe geben, ca.
	\breh{}~Teelöffel Salz, Prise Zucker, ewas Pfeffer, etwa 1~Teelöffel
	Nadler-Senf verrühren und Filets einlegen. Mindestens 12~Stunden im
	Kühlschrank ziehen lassen. Dazu Pellkartoffeln. \\
      \end{zubereitung}

    \mynewsection{Stippmilch (süßer Quark)}\glossary{Quark>süß}

      \begin{zutaten}
        250 g & \myindex{Quark} \\
        \brea{} l & \myindex{Milch} oder \myindex{Sahne} \\
        1 Prise & \myindex{Salz} \\
        30 g & \myindex{Zucker} (3~Eßlöffel) \\
        1 & \myindex{Vanillezucker}\index{Zucker>Vanille-} \\
      \end{zutaten}

      \begin{zubereitung}
        Der Quark wird durch ein Sieb gestrichen, mit Milch, Zucker und
	Vanillezucker vermischt und schaumig geschlagen. Man kann ihn beim
	Anrichten mit Geleestückchen verzieren. Man kann den Quark ohne
	Vanillezucker herstellen und mit Zucker und Zimt bestreuen. \\
      \end{zubereitung}

    \mynewsection{Haferflockenauflauf}\glossary{Auflauf>Haferflocken-}

      \begin{zutaten}
        \brdv{} l & \myindex{Milch} \\
        200 g & \myindex{Haferflocken} \\
        1 Prise & \myindex{Salz} \\
        75 g & \myindex{Margarine} \\
        50 g & \myindex{Zucker} (etwa 3~Eßlöffel) \\
        2 & \myindex{Eigelb}e \\
        \breh{} & abgeriebene \myindex{Zitrone}nschale \\
      \end{zutaten}

      \begin{zubereitung}
        Haferflocken in Milch ausquellen lassen. Die anderen Zutaten der Reihe
	nach verrühren, den erkalteten Haferflockenbrei hinzufügen. Die Hälfte
	der Masse in eine gefettete Auflaufform geben, Obst darauf geben, dann
	die restlichen Haferflocken darüber geben. Den Auflauf bei starker
	Hitze backen (30--40~Minuten). Das Eiweiß zu steifem Schnee schlagen
	und den Zucker dazugeben. Die Baisermasse auf den vorgebackenen Auflauf
	füllen und bei kleiner Flamme 5~Minuten goldgelb werden lassen. \\
      \end{zubereitung}

    \mynewsection{Spritzgebäck}

      \begin{zutaten}
        150 g & \myindex{Margarine} \\
        2 & \myindex{Ei}er \\
        200 g & \myindex{Zucker} \\
        2 & \myindex{Vanillezucker}\index{Zucker>Vanille-}
	    oder \myindex{Aroma} \\
        \breh{} Teelöffel & \myindex{Salz} \\
        1 Teelöffel & \myindex{Backpulver} \\
        500 g & \myindex{Mehl} \\
        ein paar & geriebene, bittere \myindex{Mandel}n \\
      \end{zutaten}

      \begin{zubereitung}
        Fett, Zucker und Eier werden schaumig gerührt. Das Mehl wird mit dem
	Backpulver gesiebt und nach und nach darunter gerührt. Der Teig wird in
	eine Kuchenspritze gefüllt und in verschiedenen Formen auf das
	eingefettete Blech gespritzt. \\
      \end{zubereitung}

    \mynewsection{Sauerkrautauflauf}\glossary{Auflauf>Sauerkraut-}

      \begin{zutaten}
        750--1000 g & mehlig kochende \myindex{Kartoffel}n \\
        \brev{} l & \myindex{Milch} \\
        & \myindex{Salz} \\
        1 Prise & \myindex{Muskatnuß} \\
        1 Scheibe & \myindex{Butter} \\
        1 große Dose & \myindex{Sauerkraut} \\
        200--250 g & roher \myindex{Schinken} gewürfelt \\
        & \myindex{Margarine} \\
        & \myindex{Paniermehl} \\
      \end{zutaten}

      \garzeit{30}

      \begin{zubereitung}
        Kartoffeln schälen und in Salzwasser garen, durchpressen und ein
	Kartoffelpüree herstellen. Auflaufform mit Margarine einfetten, mit
	Paniermehl ausbröseln. Sauerkraut verteilen, darauf die Schinkenwürfel
	und zum Schluß das Kartoffelpüree. Bei \grad{200} ca. für 30~Minuten in
	den Backofen schieben. \\
      \end{zubereitung}

    \mynewsection{Mexikanische Gemüsepfanne}\glossary{Gemüsepfanne>mexikanisch}

      \begin{zutaten}
        1 & \myindex{Möhre} \\
        1 & \myindex{Zwiebel} \\
        2 & \myindex{Knoblauchzehe}n \\
        1 & \myindex{Zucchini} \\
        1 & \myindex{rote Paprika}\index{Paprika>rot}schote \\
        1 kleine Dose & \myindex{Mais} \\
        1 kleine Dose & \myindex{Kidney-Bohnen} (rote Bohnen) \\
        1 kleines Fläschchen & \myindex{Pace-Taco-Soße}
	                       (Soße für Tortillas) \\
        & \myindex{Salz} \\
        & \myindex{schwarzer Pfeffer}\index{Pfeffer>schwarz} \\
        1 Teelöffel & \myindex{Gemüsebrühe} \\
      \end{zutaten}

      \garzeit{20}

      \begin{zubereitung}
        Je eine Möhre, eine Zwiebel, 2~Knoblauchzehen schälen, fein hacken.
	Eine Zucchini und eine Paprikaschote klein würfeln. Möhre, Zwiebel und
	Knoblauch in Öl andünsten, danach Zucchini und Paprikaschote hinzugeben
	und wenden. Eine Dose Mais und Bohnen öffnen abgießen bzw. überbrühen
	und in die Pfanne geben. Würzen mit Salz, Pfeffer und Gemüsebrühe,
	abschließend die Taco-Soße unterrühren und mit geschlossenem Deckel zu
	Ende garen lassen. Dazu Nudeln. \\
      \end{zubereitung}

    \mynewsection{Pfannkuchen (Eierkuchen)}\glossary{Kuchen>Pfann-}%
              \glossary{Kuchen>Eier-}

      \begin{zutaten}
        250 g & \myindex{Mehl} \\
        \breh{} & \myindex{Backpulver} \\
        \breh{} Teelöffel & \myindex{Salz} \\
        \breh{} Teelöffel & \myindex{Zucker} \\
        2 & \myindex{Eigelb}e \\
        $\frac{3}{8}$ l & \myindex{Milch} \\
        2 & \myindex{Eischnee} \\
        80 g & \myindex{Fett} (Öl oder Palmin) \\
      \end{zutaten}

      \begin{zubereitung}
        Das Mehl wird mit dem Backpulver in eine Schüssel gesiebt und mit Salz
	und Zucker vermischt. In der Mitte macht man eine Vertiefung und gibt
	das Eigelb hinein. Die Milch rührt man nach und nach darunter. Zuletzt
	unterzieht man den Eischnee. In einer Pfanne erhitzt man etwas Fett und
	gibt eine dünne Teiglage hinein. Man bäckt den Pfannkuchen auf beiden
	Seiten hellbraun. \\
      \end{zubereitung}

    \mynewsection{Frikadellen (Bulletten, Bratklopse)}\glossary{Bulletten}%
              \glossary{Bratklopse}

      \begin{zutaten}
        250 g & \myindex{Gehacktes} (Rind + Schwein) \\
        1 & \myindex{Brötchen} \\
        1 & \myindex{Ei} \\
        1 & \myindex{Zwiebel} \\
        & \myindex{Paniermehl} \\
        \breh{} Teelöffel & \myindex{Salz}, \myindex{Pfeffer} \\
        50 g & \myindex{Fett} \\
        1 Eßlöffel & \myindex{Mehl} \\
        \brev{} l & \myindex{Wasser} \\
      \end{zutaten}

      \garzeit{15--20}

      \begin{zubereitung}
        Man macht aus dem Gehackten, der eingeweichten, gut ausgedrückten
	Semmel, dem Salz, dem Ei und der Zwiebel einen Fleischteig, formt
	daraus Klopse und brät sie bei mittlerer Flamme im heißen Fett braun.
	Das Mehl läßt man im Bratensatz bräunen, gibt langsam die Brühe hinzu
	und kocht die Soße sämig. \\
      \end{zubereitung}

    \mynewsection{Möhren-Schwarzwurzel-Gratin}%
              \glossary{Gratin>Möhren-Schwarzwurzel-}

      \begin{zutaten}
        500 g & \myindex{Schwarzwurzeln} \\
        500 g & kleine \myindex{Möhre}n \\
        1 Bund & \myindex{Lauchzwiebel}n\index{Zwiebel>Lauch-} \\
        3 Eßlöffel & \myindex{Essig} \\
        1 Eßlöffel & \myindex{Dijon-Senf}\index{Senf>Dijon-} \\
        50 ml & \myindex{Schlagsahne}\index{Sahne>Schlag-} \\
        2 & \myindex{Eigelb}e \\
        100 ml & \myindex{Apfelsaft} \\
        1 & \myindex{Knoblauchzehe} \\
        75 g & geriebener \myindex{Emmentaler}\index{Käse>Emmentaler} \\
        & \myindex{Jodsalz}\index{Salz>Jod-} \\
        & \myindex{weißer Pfeffer}\index{Pfeffer>weiß} \\
      \end{zutaten}

      \garzeit{15 + 10 + 10}

      \begin{zubereitung}
        Schwarzwurzeln schälen und in Essigwasser mit Salz ca. 15~Minuten
	kochen lassen. Möhren schälen, in wenig Salzwasser ca. 10~Minuten
	dünsten. Lauchzwiebeln 5~Minuten mitdünsten. Senf, Sahne, Eigelb und
	Apfelsaft auf dem heissen Wasserbad schaumig schlagen. Knoblauch fein
	hacken und mit dem Käse unterrühren. Mit Salz und Pfeffer abschmecken.
	Gemüse abtropfen lassen und in eine feuerfeste Form schichten. Sosse
	darübergeben. Im Backofen bei \grad{225} ca. 10~Minuten goldbraun
	gratinieren. \\
      \end{zubereitung}

    \mynewsection{Knoblauch-Baguette}

      \begin{zutaten}
        & \myindex{Baguette}brot \\
        & weiche \myindex{Butter} \\
        viel & \myindex{Knoblauch} \\
        viel & \myindex{Basilikum} \\
        & \myindex{Salz} \\
        3--4 & \myindex{Sardellen}\index{Fisch>Sardellen}filets in Öl \\
      \end{zutaten}

      \garzeit{10}

      \begin{zubereitung}
        Knoblauch und Basilikum mit Salz zerstoßen im Mörser, weiche Butter und
	Sardellenfilets dazugeben. Baguettebrot von oben einschneiden und die
	Masse in die Schnitte einfüllen. Brot locker in Alufolie legen und
	10~Minuten im Backofen garen. \\
      \end{zubereitung}

    \mynewsection{Nußbraten}

      \begin{zutaten}
        200 g & \myindex{Vollkornbrot}\index{Brot>Vollkorn-} \\
        & \myindex{saure Sahne}\index{Sahne>sauer} \\
        & \myindex{Wasser} \\
        1 Eßlöffel & \myindex{Buchweizenmehl}\index{Mehl>Buchweizen-} \\
        1 Eßlöffel & \myindex{Sojamehl}\index{Mehl>Soja-} \\
        200 g & \myindex{Zwiebel}n gewürfelt \\
        1 Teelöffel & \myindex{Margarine}/\myindex{Butter} \\
        200 g & \myindex{Haselnüsse} gerieben/gemahlen \\
        1 Messerspitze & \myindex{Majoran} \\
        1 Messerspitze & \myindex{Curry} \\
        1 gestrichener Teelöffel & \myindex{Frugola} \\
        1 große Prise & \myindex{Muskatnuß} \\
      \end{zutaten}

      \garzeit{60--80}

      \begin{zubereitung}
        Vollkornbrot zerbröckeln und in saure Sahne und Wasser für 30~Minuten
	einweichen, Buchweizen- und Sojamehl dazugeben, Zwiebeln in Margarine
	dünsten und dazugeben, Nüsse dazugeben und würzen mit Majoran usw.
	Kastenform fetten, mit Paniermehl ausbröseln, Masse einfüllen. In den
	kalten Backofen stellen. Bei \grad{180--200} ca. 60--80~Minuten backen.
	\\
        Dazu: Rohkostsalat. Kann man auch kalt gut essen. \\
      \end{zubereitung}

    \mynewsection{Blätterteigpastete mit Champignons}

      \begin{zutaten}
        & \myindex{Tiefkühl-Blätterteig} \\
        200 g & \myindex{Champignon}\index{Pilze>Champignon}s in Scheiben
	        geschnitten \\
        1 & \myindex{Zwiebel} \\
        & \myindex{Salz} \\
        & \myindex{Pfeffer} \\
        & Öl\index{Oel=Öl} \\
        & \myindex{Petersilie} fein gehackt \\
        & \myindex{Gouda}\index{Käse>Gouda} geraspelt \\
        1 & \myindex{Gervais}\index{Käse>Gervais}stückchen verkneten
	    (Schmelzkäse) \\
        1 & \myindex{Ei} \\
      \end{zutaten}

      \garzeit{10--15}

      \begin{zubereitung}
        Champignons goldgelb braten, Zwiebeln, Salz, Pfeffer dazugeben. Danach
	Petersilie, Gouda usw. dazurühren und in die aufgetauten
	Blätterteigplatten geben. Zu Dreiecken klappen. 10--15~Minuten bei
	\grad{200} backen. \\
      \end{zubereitung}

    \mynewsection{Grünkernbratlinge}

      \begin{zutaten}
        100 g & \myindex{Grünkern} geschrotet \\
        ca. \breh{} l & \myindex{Wasser} \\
        & \myindex{Salz} \\
        1 & \myindex{Zwiebel} \\
        1 Handvoll & gehackte Kräuter (\myindex{Petersilie}, \myindex{Schnittlauch}, \myindex{Porree}) \\
        1--2 & \myindex{Ei}er \\
        30--50 g & \myindex{Paniermehl} \\
        10--20 g & \myindex{Weizenkeime} \\
        & \myindex{Majoran} \\
        & \myindex{Hefewürze} \\
        & \myindex{Pflanzenfett} \\
      \end{zutaten}

      \begin{zubereitung}
        Grünkernschrot mit Gewürzen in Wasser aufkochen lassen, quellen lassen
	und ab und zu umrühren. Restliche Zutaten unterrühren. Teig löffelweise
	zu Bratlingen formen und in heißem Fett ausbacken. \\
      \end{zubereitung}

    \mynewsection{Lauch-Rouladen}

      \begin{zutaten}
        1--2 dicke & \myindex{Lauch}stangen pro Person \\
        1--2 Scheiben & \myindex{roher Schinken}\index{Schinken>roh}
	                pro Person \\
        1--2 Scheiben & \myindex{Käse} \\
        20 g & \myindex{Butter} \\
        20 g & \myindex{Mehl} \\
        \brea{} l & \myindex{Milch} \\
        \brea{} l & \myindex{Lauchbrühe} oder \myindex{Fleischbrühe} \\
        \breh{} Teelöffel & \myindex{Salz} \\
        1 & \myindex{Eigelb} \\
        & \myindex{Zitrone}nsaft \\
        & \myindex{Pfeffer}, \myindex{Muskatnuß},
	  \myindex{weißer Pfeffer}\index{Pfeffer>weiß} \\
      \end{zutaten}

      \garzeit{20--30}

      \begin{zubereitung}
        Lauch putzen, waschen, auf Auflaufformlänge schneiden und vorgaren.
	Jede Stange mit 1~Scheibe rohen Schinken umwickeln und in die Form
	nebeneinander legen. Aus Mehl, Butter usw. eine holländische Soße
	bereiten. Holländische Soße über die Rouladen geben. Auf jede Roulade
	1~Scheibe Käse legen und backen bei 20--30~Minuten mit \grad{200}, bis
	der Käse schmilzt. \\
        Dazu Kartoffelpüree. \\
      \end{zubereitung}

    \mynewsection{Knoblauchbrot}

      \begin{zutaten}
        1 & \myindex{Stangenweißbrot}/\myindex{Baguette} \\
        250 g & \myindex{Butter} \\
        1 TK-Päckchen & \myindex{Petersilie} \\
        1 TK-Päckchen & \myindex{Schnittlauch} \\
        1 TK-Päckchen & \myindex{Dill} \\
        1 & \myindex{Zitrone} gepreßt \\
        2 & \myindex{Knoblauchzehe}n \\
        & \myindex{Salz} \\
        & \myindex{Pfeffer} \\
      \end{zutaten}

      \garzeit{25--30}

      \begin{zubereitung}
        Butter mit allen Zutaten mischen. Stangenweißbrot scheibenweise
	einschneiden, aber nicht durchtrennen. Buttermasse in die Taschen
	verteilen. Brot zusammendrücken und lose in Alufolie wickeln. Im
	vorgeheizten Ofen bei \grad{200} ca. 25--30~Minuten backen. Danach ganz
	aufscheiden und servieren. \\
      \end{zubereitung}

    \mynewsection{Sellerie-Schnitzel}

      \begin{zutaten}
        4 Scheiben & \myindex{Sellerie} ca. 1~cm dick \\
        & \myindex{Mehl} \\
        4 & \myindex{Ei}er \\
        & \myindex{Paniermehl} \\
        & \myindex{Jodsalz}\index{Salz>Jod-} \\
        & \myindex{weißer Pfeffer}\index{Pfeffer>weiß} \\
        & Öl\index{Oel=Öl} \\
      \end{zutaten}

      \personen{2}

      \begin{zubereitung}
        Sellerieknolle sauber schälen. Oberes und unteres Teil abschneiden, um
	genügend große Scheiben zu je 1~cm Dicke zu erhalten. Die Scheiben kurz
	in Wasser vorgaren. Abtropfen lassen, salzen, pfeffern und mehlen.
	Danach in verklepperten Eiern wenden und danach in Paniermehl. In der
	Pfanne mit Öl bei mittlerer bis kleiner Hitze von beiden Seiten
	goldgelb braten. Restlichen Sellerie vom Oberteil/Unterteil einfrieren.
	\\
        Dazu: pikante Tomatensoße. \\
      \end{zubereitung}

    \mynewsection{Holunderblütengetränk}

      \begin{zutaten}
        10 l & \myindex{Wasser} (am besten Brunnenwasser) \\
        1 kg & \myindex{Zucker} \\
        1 & unbehandelte \myindex{Zitrone} \\
        \brev{} l & \myindex{Weinessig}\index{Essig>Wein-} \\
        10 & \myindex{Holunderblüten} \\
      \end{zutaten}

      \begin{zubereitung}
        Alle Zutaten zusammengeben und 3~Tage stehen lassen, jeden Abend
	umrühren. Anfangs am warmen Ort aufbewahren, dann abseihen und auf
	Flaschen abfüllen, Korken zubinden. Stehend verwahren. Sehr
	erfrischend! Muß aber gleich verbraucht werden! \\
      \end{zubereitung}

    \mynewsection{Holunderküchlein (Hollerkiacherl)}

      \begin{zutaten}
        10--12 & schöne \myindex{Holunderdolden} \\
	2--3 & \myindex{Ei}er \\
	200 g & \myindex{Mehl} \\
	\brev{} l & \myindex{Milch} \\
	1 Messerspitze & \myindex{Salz} \\
	& \myindex{Zucker} \\
	& \myindex{Fett} zum Backen \\
      \end{zutaten}

      \begin{zubereitung}
        Die Hollerblüten werden vorsichtig gewaschen und auf ein Sieb zum
	Abtropfen gelegt. \\
	Aus Mehl, Milch, Salz und den Eiern verrührt man einen Pfannkuchenteig,
	taucht die Hollerblüten hinein und bäckt sie im schwimmenden Fett
	schön goldbraun.\\
	Die Hollerkücherl mit Zucker bestreuen und sofort servieren! \\
      \end{zubereitung}

    \mynewsection{Holunderlikör}

      \begin{zutaten}
        2 kg & vollreife \myindex{Holunder}beeren \\
	2 l & \myindex{Branntwein}\index{Wein>Brannt-} \\
	250 g & \myindex{Zucker} \\
	1--2 l & \myindex{Wasser} \\
      \end{zutaten}

      \begin{zubereitung}
        Die Beeren waschen, von den Stielen befreien, zu Mus quetschen, zusammen
	mit dem Branntwein in ein weithalsiges Gefäß füllen, in der Sonne ca.
	8~Wochen ruhen lassen und filtern. Wasser und Zucker so lange kochen,
	bis ein zähflüssiger Sirup entsteht, mit dem Holundersaft mischen und
	in saubere Flaschen füllen. An einem dunklen Ort lagern --- je länger,
	desto besser wird der Holunderlikör. \\
      \end{zubereitung}

    \mynewsection{Holundersekt}

      \begin{zutaten}
        10 & \myindex{Holunder}blütendolden \\
	6 l & \myindex{Wasser} \\
	400 ml & \myindex{Essig} \\
	& \myindex{Zucker} nach Belieben \\
	5 & \myindex{Zitrone}n \\
      \end{zutaten}

      \begin{zubereitung}
        Die frischen Blütendolden abschütteln, waschen und mit den in Scheiben
	geschnittenen Zitronen in große Einmachgläser legen. Wasser, Essig und
	Zucker aufkochen, abkühlen lassen, über die Blüten gießen und 2~Tage
	ruhen lassen. \\
	Den Sud durch ein Haarsieb gießen, in ausgekochte Flaschen füllen und
	verkorken. Nach gut 10~Tagen ist der Sekt trinkfertig. \\
      \end{zubereitung}

    \mynewsection{Schoko-Crossies geeist}

      \begin{zutaten}
        30 g & \myindex{Sonnenblumenkerne} \\
        80 g & \myindex{Cornflakes} \\
        3 Eßlöffel & \myindex{Nuss-Nougat-Creme} (wie Nutella) \\
        \brev{} Teelöffel & \myindex{Koriander}pulver \\
      \end{zutaten}

      Für 20 Stück \\
      \kalorien{39}

      \begin{zubereitung}
        Kühlzeit 120~Minuten, Vorbereitung 15~Minuten. \\
        Sonnenblumenkerne in einer Pfanne ohne Fett rösten und abkühlen lassen.
	Cornflakes, Nuss-Nougat-Creme, Korianderpulver und  Sonnenblumenkerne
	verrühren und mit einem Teelöffel kleine Häufchen auf ein mit
	Backpapier ausgelegtes Blech setzen. 2~Stunden gefrieren und eiskalt
	servieren. \\
      \end{zubereitung}

    \mynewsection{Putenbrust mit Champignons und Rösti}

      \begin{zutaten}
        250 g & \myindex{Putenbrust} \\
        1 & \myindex{Knoblauchzehe} zum Marinieren \\
        1 Bund & \myindex{Lauchzwiebel}n\index{Zwiebel>Lauch-} \\
        1 Eßlöffel & \myindex{Sojasoße} zum Marinieren \\
        400 g & frische \myindex{Champignon}\index{Pilze>Champignon}s \\
        4 Eßlöffel & \myindex{Sonnenblumenöl}\index{Oel=Öl>Sonnenblumen-} \\
        \brea{} l & \myindex{Hühnerbrühe} \\
        1 Eßlöffel & \myindex{Mandelblättchen} \\
        250 g & \myindex{Kartoffel}n \\
        1 & \myindex{Zwiebel} \\
        2 & \myindex{Ei}er \\
        2 Eßlöffel & \myindex{Sonnenblumenkerne} \\
        \brev{} Teelöffel & \myindex{Curry}pulver \\
        1 Teelöffel & \myindex{Salz} \\
        4 cl & \myindex{Sherry} zum Marinieren \\
        & grober \myindex{Pfeffer} zum Marinieren \\
      \end{zutaten}

      \personen{4}
      \garzeit{15}
      \kalorien{310}

      \begin{zubereitung}
        Garzeit: 15~Minuten ca., Vorbereiten 15~Minuten, Marinieren 20~Minuten
	\\
        Putenbrust in schmale Streifen schneiden. Knoblauch zerdrücken mit
	Sojasoße, Sherry und Pfeffer verrühren. Fleisch darin ca. 20~Minuten
	marinieren. \\
        Frühlingszwiebeln und Champignons putzen. Zwiebeln in Stücke,
	Champignons in  Scheiben schneiden. 2~Eßlöffel Öl erhitzen, Fleisch
	unter Rühren scharf anbraten. Zwiebeln und Champignons dazugeben, ca.
	5~Minuten dünsten. Restliche Marinade und die Brühe angießen,
	aufkochen, mit Pfeffer und Soja abschmecken, mit Mandeln bestreuen. \\
        Rösti: Kartoffeln schälen, grob reiben, Zwiebel fein würfeln, mit Eiern
	und Sonnenblumenkernen verrühren, würzen. Rösti in restlichem Öl
	abbacken. \\
      \end{zubereitung}

    \mynewsection{Rindsgulasch}

      \begin{zutaten}
        40 g & \myindex{Fett} oder \myindex{Öl} \\
	800 g & \myindex{Rindfleisch}\footnote{am besten dunkel und marmoriert,
	        aber ohne zähe Sehnen (abgehangen). Wichtig ist ein guter
		Schmortopf und einwandfreies Fleischstück, das man selbst in
		große Würfel schneidet. Rindfleisch aus der Keule oder falsches
		Filet. Beraten lassen, aber einen anständigen Metzger
		aufsuchen.} \\
        1500 g & \myindex{Zwiebel}n, eventuell mehr, in Würfeln \\
	1 gestrichener Teelöffel & \myindex{Salz} \\
	& \myindex{Pfeffer} \\
	& \myindex{Paprikapulver} edelsüß (80\%) und scharf (20\%) (probieren)\\
	2 & \myindex{Lorbeer}blätter \\
	1 Eßlöffel & \myindex{Tomatenmark} \\
	1 großer & \myindex{Brühwürfel} \\
      \end{zutaten}

      \personen{4}

      \begin{zubereitung}
        200 g Fleisch pro Person. \\
	Fleischwürfel trocken machen. Zwiebeln hacken. Topf erhitzen und Fett
	rein, kann auch Öl sein, aber darauf achten, daß es sehr stark erhitzbar
	ist. Fenster auf. \\
	Die Fleischwürfel rasant anbraten (blauer Rauch und viele Spritzer) und
	wenden. Wenn die Würfel deutlich gebräunt sind, gibt man die
	Zwiebelwürfel rein und rührt ständig, bis die Zwiebeln glasig sind.
	Nicht anbrennen lassen. Eine Stelle freischaben und Tomatenmark
	anrösten. Hitze reduzieren, Gewürze rein und mit warmem Wasser
	auffüllen, bis knapp unter die Oberkante des Fleisches. \\
	Ca. 1\breh{}~Stunden leise köcheln lassen. Ab und zu rühren.
	Abschmecken (kein Mehl, keine Sahne rein!). Die Zwiebeln müßten die
	Bindung bringen. Pobieren, ob das Fleich gar ist. Abschmecken.
	Lorbeerblätter rausnehmen. \\
	Dazu gedrehte Nudeln (Spiralnudeln oder ähnliche) und Gurken- oder
	Tomatensalat oder grüner Salat. \\
	Man kann auch 2 oder 3 Gulasch auf einmal kochen. Dann muß man aber
	das Fleisch und später die Zwiebeln partienweise anbraten. Zu voll
	sollte der gußeiserne Topf aber nicht werden. \\
      \end{zubereitung}

    % \mynewsection{Rindsgulasch}

      % \begin{zutaten}
      % \end{zutaten}

      % \begin{zubereitung}
      % \end{zubereitung}
