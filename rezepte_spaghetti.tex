
% created Montag, 10. Dezember 2012 16:22 (C) 2012 by Leander Jedamus
% modifiziert Samstag, 02. Mai 2015 16:07 von Leander Jedamus
% modified Montag, 10. Dezember 2012 16:30 by Leander Jedamus

  \mynewchapter{Spaghetti}

    \mynewsection{Spaghetti Carbonara}

      \begin{zutaten}
        300 g & lange \myindex{Spaghetti} (italienische) \\
        etwas & \myindex{Salz}, \myindex{Pfeffer} \\
        1 Eßlöffel & \myindex{Olivenöl}\index{Oel=Öl>Oliven-} \\
        2 & \myindex{Ei}er \\
        2 Eßlöffel & \myindex{süße Sahne}\index{Sahne>süß} \\
        70 g & geriebener \myindex{Parmesan}\index{Käse>Parmesan}käse oder
	       besser \myindex{Gouda}\index{Käse>Gouda} \\
        125 g & roher \myindex{Schinken}/\myindex{Schinkenspeck} \\
      \end{zutaten}

      \garzeit{10 + 5}

      \begin{zubereitung}
        Spaghetti in reichlich Salzwasser garen (8--10~Minuten). Eier mit der
	Sahne, Käse, Salz und Pfeffer verquirlen. Dann in Pfanne Olivenöl geben
	und zerpflückte Schinkenscheiben bei kleiner bis mittlerer Hitze von
	beiden Seiten braten. Spaghetti abgießen, gebratenen Schinken samt Öl
	und die Eier-Sahnemischung drübergießen und durchmischen, bis die Eier
	nicht mehr so flüssig sind. \\
        Dazu grünen Salat, Tomatensalat usw. \\
      \end{zubereitung}

    \mynewsection{Spaghetti mit Tomatensoße}

      \begin{zutaten}
        400--500 g & lange \myindex{Spaghetti} (italienische) \\
        \breh{}-1 Teelöffel & \myindex{Salz} \\
        1 kg & frische reife \myindex{Tomate}n \\
        oder 1--2 kleine Dosen & \myindex{geschälte Tomate}n
	                         \index{Tomate>geschält} \\
        1 & \myindex{Zwiebel} \\
        1 Zehe & \myindex{Knoblauch} \\
        4 Eßlöffel & \myindex{Olivenöl}\index{Oel=Öl>Oliven-} \\
        \brev{} Teelöffel & getrockneter \myindex{Oregano} (Winter) \\
        oder 1 Handvoll & frische \myindex{Basilikum}blätter (Sommer) \\
        etwas & \myindex{Salz}, \myindex{Pfeffer} \\
        40 g & geriebener \myindex{Parmesan}\index{Käse>Parmesan} \\
      \end{zutaten}

      \begin{zubereitung}
        Tomaten brühen, Haut abziehen und in Stückchen schneiden. Dabei das
	gelbe harte Mark in der Mitte zurücklassen (oder geschälte Tomaten ohne
	die Brühe klein schneiden). Gewürfelte Zwiebel und feingehackten
	Knoblauch in heißem Öl hellgelb rösten, Tomaten dazugeben. Trockenen
	Oregano (oder gehacktes frisches Basilikum) darüber streuen und bei
	kleiner Hitze etwa 15~Minuten köcheln lassen, ab und an umrühren.
	Spaghetti in reichlich kochendem Salzwasser garen (8--10~Minuten).
	Soße mit Salz und frisch gemahlenem Pfeffer abschmecken. Servieren mit
	Parmesan. Dazu grüner Salat. \\
        Variationen: \textbf{Spaghetti Bolognese} (siehe Seite
	\pageref{spaghetti-bolognese}) wird ähnlich zubereitet. Man
	brät allerdings etwa 200~g Hackfleisch (\breh{}~Rind und
	\breh{}~Schwein) bräunlich an und gibt dann die Soßenzutaten hinzu.
	Ich habe meistens noch ein kleines Lorbeerblatt mitgekocht. Bei Bedarf
	mit wenig Brühwürfel mitten in der Kochzeit (hier 5--10~Minuten
	länger) würzen. \\
      \end{zubereitung}

    \mynewsection{Spaghetti al Pesto mit grünen Bohnen und Kartoffeln}

      \begin{zutaten}
      \end{zutaten}
      \begin{zutat}{Pesto}
        50 g & \myindex{Parmesan}\index{Käse>Parmesan} oder
	       \myindex{Peccorino}\index{Käse>Peccorino} (Schafkäse) \\
        50 g & \myindex{Knoblauch} \\
        50 g & \myindex{Pinienkerne} \\
        50 g & \myindex{Basilikum}blätter \\
        3--4 Eßlöffel & \myindex{Olivenöl}\index{Oel=Öl>Oliven-} \\
        \breh{} Teelöffel & \myindex{Salz} \\
      \end{zutat}
      \begin{zutat}{Creme}
        2--3 Eßlöffel & \myindex{süße Sahne}\index{Sahne>süß} \\
        1--2 Eßlöffel & Nudelwasser \\
      \end{zutat}
      \begin{zutat}{Spaghetti}
        250 g & \myindex{Spaghetti} \\
        250 g & \myindex{Kartoffel}scheiben roh \\
        250 g & frische \myindex{Bohnen} \\
        & \myindex{Jodsalz}\index{Salz>Jod-} \\
      \end{zutat}

      \garzeit{5 + 11}

      \begin{zubereitung}
        Kartoffeln schälen, in Scheiben hobeln, Bohnen waschen und putzen.
	Spaghetti 8--11~Minuten kochen --- siehe Packung. Bohnen blanchieren,
	Kartoffeln in Salzwasser blanchieren (ca. 3--5~Minuten
	kochen lassen). Garprobe machen, danach im Eiswasser abschrecken.
	Pesto-Zutaten in Mixer geben (Käse zerbröckeln) und pürieren. Pesto mit
	Creme-Zutaten mischen. Spaghetti abgießen und mit Bohnen und Kartoffeln
	mischen. Pesto und Creme daruntergeben. \\
      \end{zubereitung}

    \mynewsection{Spinat-Spaghetti Carbonara}

      \begin{zutaten}
        250 g & dünne \myindex{Spaghetti} \\
        & \myindex{Salz} \\
        30--50 g & durchwachsener \myindex{Bauchspeck} in dünnen Scheiben \\
        1 & \myindex{rote Zwiebel}\index{Zwiebel>rot}
	    (oder eine junge \myindex{weiße Zwiebel}\index{Zwiebel>weiß}) \\
        1 Eßlöffel & \myindex{Olivenöl} \\
        1--2 Eßlöffel & Butter nach Belieben \\
        1--2 \myindex{Knoblauchzehe}n \\
        300 g & geputzte \myindex{Spinat}blätter \\
        & \myindex{Pfeffer} \\
        & \myindex{Muskatnuß} \\
        etwas & abgeriebene \myindex{Zitrone}nschale \\
        2--3 & \myindex{Ei}er \\
        2 Eßlöffel & frisch geriebener \myindex{Parmesan}\index{Käse>Parmesan}
	             \\
        1 Prise & \myindex{Cayennepfeffer}\index{Pfeffer>Cayenne-} eventuell \\
      \end{zutaten}

      \personen{2--3}

      \garzeit{15}

      \begin{zubereitung}
        Die Spaghetti in reichlich gut gesalzenem Wasser nach
	Packungsaufschrift bißfest kochen. \\
        In der Zwischenzeit in einer tiefen Pfanne den in sehr feine Streifen
	geschnittenen Speck auslassen, die sehr fein gewürfelte Zwiebel
	hinzufügen, etwas Öl und eventuell 1~Eßlöffel Butter auf mildem
	Feuer weich dünsten. Den Knoblauch hinzufügen und kurz mitdünsten. Dann
	die Hitze verstärken und die tropfnassen, frischen, unblanchierten
	Spinatblätter hinzugeben. Salzen und Pfeffern, mit Muskat und
	Zitronenschale würzen. Gut umwenden. \\
        Die Eier zum Spinat in die heiße Pfanne geben, sogleich auch die
	tropfnassen, heißen Spaghetti, den geriebenen Käse sowie die restliche
	Butter. \\
        Im Kontakt mit den heißen Spaghetti stocken die Eier und verbinden sich
	mit Käse und Butter zu einer cremigen Soße, die sich um die Nudeln
	schmiegt. Nicht mehr stehen lassen, sonst werden die Eier fest. Sofor
	in vorgewärmten Tellern servieren. Nach Belieben noch etwas frische
	Zitronenschale darüberreiben. \\
        Getränk: Auch wenn die Oxalsäure hier durch die cremigen Eier gemildert
	wird --- der erdige Geschmack von Spinat braucht einen herzhaftem
	Widerpart: beispielsweise einen klassischen Bordeaux. Es braucht kein
	großes Ch\^ateau zu sein, eine einfache, ,,bürgerliche`` Qualität (Cru
	bourgeois) reicht zu diesem Gericht durchaus --- vielleicht auch von
	den C\^otes de Castillon, oder, und dazu haben wir uns dieses Mal
	entschieden, ein Sangiovese die Romagna ,,Notturno`` von Drei Don\`a
	aus Forli (zwischen Bologna und Rimini gelegen). \\
      \end{zubereitung}

    \mynewsection{Spaghetti al pesto (mit Knoblauch und Kräutern)}

      \begin{zutaten}
        3 Bund & frisches \myindex{Basilikum} \\
	3 Bund & frische \myindex{Petersilie} \\
	4--5 & \myindex{Knoblauchzehe}n \\
	1 Eßlöffel & \myindex{Pinienkerne} \\
	8 Eßlöffel & frisch geriebener
	             \myindex{Parmesan}\index{Käse>Parmesan} \\
        & \myindex{Olivenöl}\index{Oel=Öl>Oliven-} \\
	& \myindex{Salz} \\
	& \myindex{Pfeffer} \\
	300--350 g & \myindex{Spaghetti}
      \end{zutaten}

      \personen{4}
      \garzeit{18}

      \begin{zubereitung}
        Basilikum und Petersilie fein hacken, Knoblauch und Pinienkerne im
	Mörser zerstoßen oder alles zusammen im Mixer eines elektrischen
	Küchengeräts pürieren. Den frisch geriebenen Parmesankäse hinzugeben
	und mit so viel Öl verrühren, bis eine sämige Soße entsteht. Mit Salz
	und Pfeffer abschmecken. Spaghetti kochen und mit der Soße
	vermischen. \\
      \end{zubereitung}

    \mynewsection{Spaghetti al pomodoro (mit Tomaten)}

      \begin{zutaten}
        1 große & feingewürfelte \myindex{Zwiebel} \\
	2 & \myindex{Knoblauchzehe}n \\
	3 Eßlöffel & \myindex{Olivenöl}\index{Oel=Öl>Oliven-} \\
	1 große Dose & \myindex{geschälte Tomate}\index{Tomate>geschält}n \\
	1 Dose & \myindex{Tomatenmark} \\
	1 & \myindex{Lorbeer}blatt \\
	1\breh{} Teelöffel & \myindex{Oregano} \\
	& \myindex{Salz} \\
	& frisch gemahlener \myindex{Pfeffer} \\
	300--350 g & \myindex{Spaghetti} \\
	3 Eßlöffel & geriebener \myindex{Parmesan}\index{Käse>Parmesan} \\
	3 Teelöffel & \myindex{Butter} oder \myindex{Margarine} \\
      \end{zutaten}

      \personen{4}
      \garzeit{12}

      \begin{zubereitung}
        Zwiebelwürfel und feingehackten Knoblauch in Olivenöl andünsten.
	Tomaten und Tomatenmark hinzugeben. Würzen und auf kleiner Flamme
	30~Minuten leicht kochen lassen. Je länger Sie die Soße kochen, desto
	kräftiger wird sie. Sollte die Soße zu dick werden, können Sie sie mit
	etwas Spaghettiwasser verdünnen. Die Spaghetti kochen, abtropfen
	lassen, die Soße daruntermischen und mit Parmesan und Butter
	servieren. \\
      \end{zubereitung}

    \mynewsection{Sizilianische Spaghetti}
      
      \begin{einleitung}       
        Das Lieblingsgericht vieler Sizilianer! Und es ist so schnell gemacht.
	\\
      \end{einleitung}       

      \begin{zutaten}
	400 g & \myindex{Spaghetti} \\
	& \myindex{Salz} \\
	1 & \myindex{Zwiebel} \\
        2--3 Eßlöffel & \myindex{Olivenöl}\index{Oel=Öl>Oliven-} \\
	2--3 & \myindex{Knoblauchzehe}n \\
	3 Eßlöffel & \myindex{Semmelbrösel} \\
	1 Handvoll & \myindex{Petersilie}nblätter \\
	1 Dose & Ölsardinen\index{Oelsardinen=Ölsardinen}
	         (100--150~g Einwaage) \\
	einige & \myindex{Basilikum}blätter \\
	1 Handvoll & \myindex{Pinienkerne} \\
      \end{zutaten}

      \personen{4}

      \begin{zubereitung}
        Zuerst das Nudelwasser aufsetzen, gut salzen. Die Pasta darin nach
	Vorschrift (zwischen 7--9~Minuten je nach Sorte --- die genaue Garzeit
	steht auf der Packung) bißfest kochen. \\
	Inzwischen die Zwiebel fein würfeln, im heißen Öl andünsten. Den
	gehackten Knoblauch und die Pinienkerne hinzufügen, schließlich die
	Semmelbrösel regelrecht mitrösten --- auf kräftigem Feuer. Die
	Petersilie fein gehackt hinzufügen sowie die grob zerkleinerten
	Sardinen hinzugeben. \\
	Schließlich die tropfnassen Spaghetti untermischen, eventuell noch
	einen Schuß Nudelwasser angießen, falls die Pasta trocken wirkt. Das
	gehackte Basilikum darüberstreuen und unverzüglich zu Tisch bringen,
	solange die Semmelbrösel noch schön knusprig sind. \\
	Getränk: ein kräftiger Weißwein, am besten aus Sizilien, ein Grillo
	etwa aus Westsizilien. \\
      \end{zubereitung}

    \mynewsection{Gratinierte Spaghetti mit Brokkoli und Tomaten}

      \begin{zutaten}
	20 g & \myindex{Weizenmehl}\index{Mehl>Weizen-} \\
	125 ml & \myindex{Milch} \\
	1 & \myindex{Lorbeer}blatt \\
	& \myindex{Muskatnuß} \\
	& \myindex{Salz} \\
	& \myindex{Pfeffer} aus der Mühle \\
	400 g & \myindex{Brokkoli} \\
	200 g & \myindex{Tomate}n \\
	120 g & \myindex{Spaghetti} \\
	40 g & \myindex{Zwiebel}n \\
	1 & \myindex{Knoblauchzehe} \\
	1 Teelöffel & Öl\index{Oel=Öl} \\
	20 g & \myindex{Tomatenmark} \\
	& \myindex{Basilikum} \\
	& \myindex{Oregano} \\
	20 g & geriebener \myindex{Emmentaler}\index{Käse>Emmentaler} \\
      \end{zutaten}

      \begin{zubereitung}
        Backofen auf \grad{225} vorheizen. \\
	Mehl mit Milch mischen und mit dem Lorbeerblatt unter ständigem Rühren
	zum Kochen bringen. Die Soße 3~Minuten ausquellen lassen und mit Salz,
	Pfeffer und Muskat abschmecken. \\
	Brokkoli putzen, in Röschen teilen und kurz in kochendem Salzwasser
	blanchieren, gut abtropfen lassen. Tomaten waschen, Stielansatz
	entfernen. Tomaten würfeln. Spaghetti kochen und kalt abschrecken,
	abtropfen lassen. \\
	Zwiebel und Knoblauch würfeln, in heißem Öl glasig dünsten. Tomatenmark
	zugeben, Basilikum, Oregano und zerkleinerte Tomaten zufügen. Mit
	Salz und Pfeffer abschmecken. Dann mischen mit den Spaghetti und in
	eine gefettete Auflaufform füllen. Brokkoli darüber verteilen, mit der
	Soße überziehen und mit Käse bestreuen. \\
	Im Backofen ca. 5~Minuten backen, bis der Käse schön geschmolzen ist. \\
	Dazu nach Belieben Steak oder Maissalat mit roten Zwiebeln. \\
      \end{zubereitung}
