
% created Montag, 10. Dezember 2012 16:21 (C) 2012 by Leander Jedamus
% modifiziert Mittwoch, 24. Juni 2015 22:39 von Leander Jedamus
% modifiziert Samstag, 02. Mai 2015 16:11 von Leander Jedamus
% modifiziert Mittwoch, 22. April 2015 19:52 von Leander Jedamus
% modifiziert Dienstag, 21. April 2015 15:43 von Leander Jedamus
% modifiziert Dienstag, 17. März 2015 17:20 von Leander Jedamus
% modified Donnerstag, 27. Dezember 2012 15:12 by Leander Jedamus
% modified Montag, 10. Dezember 2012 16:30 by Leander Jedamus

  \mynewchapter{Neues II}

    \mynewsection{Grünkernklöße mit Blumenkohl-Bohnen-Gemüse}

      \begin{zutaten}
      \end{zutaten}
      \begin{zutat}{Für die Grünkernklöße}
	200 g & \myindex{Grünkern} (oder \myindex{Grünkern}schrot) \\
	3 & \myindex{Schalotte}n (60~g) \\
	60 g & \myindex{Butter} \\
	1 & zerdrückte \myindex{Knoblauchzehe} \\
	400 ml & \myindex{Gemüse-Hefebrühe} \\
	1 Bund & \myindex{glatte Petersilie}\index{Petersilie>glatt} \\
	1 Bund & \myindex{Schnittlauch} \\
	150 g & \myindex{Sahnequark}\index{Quark>Sahne-} (40\%~F.~i.~Tr.) \\
	1 & \myindex{Ei} \\
	1 Eßlöffel & \myindex{Weizenvollkornmehl}
	             \index{Mehl>Weizenvollkorn-} \\
	3 Eßlöffel & \myindex{Vollkornpaniermehl} \\
	& \myindex{Vollmeersalz}\index{Salz>Vollmeer-} \\
	& \myindex{Pfeffer} \\
      \end{zutat}
      \begin{zutat}{Für das Gemüse}
	1 & \myindex{Blumenkohl}\index{Kohl>Blumen-} (ca. 1 kg) \\
        400 g & \myindex{grüne Bohnen}\index{Bohnen>grün} \\
	400 ml & \myindex{Gemüse-Hefebrühe} \\
	& \myindex{Vollmeersalz}\index{Salz>Vollmeer-} \\
	& \myindex{Pfeffer} \\
	30 g & \myindex{Butter} \\
	1 Eßlöffel & \myindex{Weizenvollkornmehl}
	             \index{Mehl>Weizenvollkorn-} \\
        \brev{} l & \myindex{Schlagsahne}\index{Sahne>Schlag-} \\
        1 Eßlöffel & \myindex{Weißwein}\index{Wein>weiß} \\
	\breh{} & \myindex{Zitrone} ausgepreßt \\
        ca. 5 Stengel & \myindex{Zitronenmelisse} \\
      \end{zutat}

      \personen{4}
      \kalorien{730}

      \begin{zubereitung}
        Die Grünkernklöße: Ganze Grünkernkörner vorher im elektrischen
	Zerkleinerer (oder in der Getreidemühle) grob schroten. Schalotten
	abziehen und mit einem Messer in sehr feine Würfel schneiden. 30~g
	Butter in einem Topf erhitzen. Schalottenwürfel und Knoblauch zufügen,
	glasig dünsten. Grünkernschrot zugeben und kurz mitrösten. Mit der
	Gemüse-Hefebrühe ablöschen. Kurz aufkochen, ca. 10--15~Minuten
	ausquellen lassen. Grünkernmasse in eine Schüssel füllen und abkühlen
	lassen. Kräuter waschen, trockentupfen und fein schneiden. Abgetropften
	Quark, Ei, Vollkornmehl, Vollkornpaniermehl, Vollmeersalz, Pfeffer
	und Kräuter zur Grünkernmasse geben, unterrühren. Kräftig abschmecken.
	Ca. 30~Minuten quellen lassen (inzwischen das Gemüse zubereiten). \\
	Aus der Masse mit angefeuchteten Händen ca. 20~Klöße formen. In einem
	großen breiten Topf gesalzenes Wasser aufkochen. Klöße hineingeben und
	bei geringer Hitze ca. 8--12~Minuten gar ziehen lassen. Herausnehmen,
	auf Küchenpapier abtropfen lassen. Restliche Butter in einer Pfanne
	erhitzen. Klöße darin portionsweise rundherum braun braten. Mit dem
	Gemüse anrichten. Nach Wunsch mit Zitronenmelisse garnieren. Dazu
	schmecken Petersilienkartoffeln. \\
	Das Gemüse: Blumenkohl putzen, in kleine Röschen teilen. Dicke Stiele
	kleinschneiden, Blumenkohl waschen, abtropfen lassen. Bohnen putzen,
	dabei entfädeln. Anschließend waschen. Nach Wunsch halbieren. Gemüse
	in einen Topf geben. Brühe zugießen, ca. 10--12~Minuten dünsten.
	Kräftig würzen. Blumenkohl-Bohnen-Gemüse auf ein Sieb geben, Brühe
	dabei auffangen. Gemüse kurz beseite stellen. Für die Soße Butter
	erhitzen. Vollkornmehl unter Rühren darin anschwitzen. Mit \brda{}~l
	der Gemüsebrühe ablöschen. Schlagsahne, Weißwein und Zitronensaft
	zugießen und kurz aufkochen. Abgetropftes Gemüse zugeben und ca.
	5~Minuten darin erwärmen. Zitronenmelisseblättchen abzupfen, fein
	schneiden. Unter das Gemüse heben. Pikant abschmecken. \\
      \end{zubereitung}

    \mynewsection{Kohlrabi-Möhren-Gratin mit Butterkäse}

      \begin{zutaten}
	1,2 kg & \myindex{Kohlrabi} \\
	600 g & \myindex{Möhre}n \\
        & \myindex{Vollmeersalz}\index{Salz>Vollmeer-} \\
	50 g & \myindex{Cashewkerne} \\
	& \myindex{Fett} für die Form \\
        \brev{} l & \myindex{Schlagsahne}\index{Sahne>Schlag-} \\
	150 g & \myindex{\cremefraiche{}} \\
	2 Meßlöffel & \myindex{Biobin} (pflanzliches Bindemittel) \\
	& \myindex{Pfeffer} \\
	& \myindex{Koriander} \\
        100 g & milder \myindex{Butterkäse}\index{Käse>Butter-} \\
	1 Topf & \myindex{Kerbel} \\
      \end{zutaten}

      \personen{4}
      \garzeit{30}
      \kalorien{550}

      \begin{zubereitung}
        Kohlrabi und Möhren waschen, schälen. Kohlrabi halbieren, nach Wunsch
	vierteln. Kohlrabi und Möhren mit einem Haushaltshobel in dünne
	Scheiben hobeln oder mit einem Messer in ca. \breh{}~cm dünne Scheiben
	schneiden. Vorbereitete Kohlrabi- und Möhrenscheiben in kochendem
	Salzwasser ca. 3--5~Minuten garen. Auf ein Sieb geben, mit kaltem
	Wasser abschrecken und gründlich abtropfen lassen. Cashewkerne mit
	einem großen Messer grob hacken. Anschließend in einer beschichteten
	Pfanne goldbraun rösten, dabei die Cashewkerne mehrmals wenden. Eine
	feuerfeste Form (ca. 30~cm~\durchmesser{}) dünn einfetten. Form mit der
	Hälfte der Cashewkerne ausstreuen. Schlagsahne und \cremefraiche{} in
	einen Topf geben und mit einem Schneebesen glattrühren. Kräftig
	würzen und unter Rühren aufkochen lassen. Binden und pikant
	abschmecken. Butterkäse grob raspeln. Möhren und Kohlrabi in die Form
	schichten. Kerbelblättchen von den Stengeln zupfen und über das Gemüse
	streuen. Soße, geraspelten Käse und restliche Cashewkerne darübergeben.
	Kohlrabi-Möhren-Gratin im vorgeheizten Backofen bei \grad{200} auf der
	mittleren Schiene ca. 20--30~Minuten garen. Eventuell mit
	Pergamentpapier abdecken. Nach Wunsch mit Kerbelblättchen garnieren.
	Dazu schmeckt Kartoffelbrei mit vielen frischen Kräutern. \\
      \end{zubereitung}

    \mynewsection{Umbrische Pizza (Torta rustica)}

      \begin{zutaten}
	375 g & \myindex{Mehl} \\
	\breh{} Päckchen & \myindex{Hefe} \\
	3 Eßlöffel & warmes \myindex{Wasser} \\
	2 & \myindex{Eigelb} \\
	4 & \myindex{Ei}er \\
        100 g & geriebener \myindex{Parmesan}\index{Käse>Parmesan}käse \\
        125 g & geriebener \myindex{Peccorino}\index{Käse>Peccorino}käse \\
	& \myindex{Pfeffer} \\
	ca. \brea{} l & Öl\index{Oel=Öl} \\
      \end{zutaten}

      \begin{zubereitung}
        Mehl in eine Schüssel geben. Hefe mit Wasser anrühren. In eine
	Vertiefung auf das Mehl gießen. Mit etwas Mehl verrühren. Zugedeckt
	15~Minuten gehen lassen. Eier, Eigelb, Käse und etwas Pfeffer verrühren.
	Mit dem Knethaken des Handrührgerätes alles zusammen zu einem Teig
	verarbeiten, dabei nach und nach das Öl hinzufügen. An warmem Ort etwa
	1~Stunde aufgehen lassen. Teig in zwei gefettete Pieformen
	(20~cm~\durchmesser{}) drücken. Bei \grad{200} 20--30~Minuten backen.
	Mit Oliven und Salami servieren. Pizza aufteilen, am Rand
	Salamischeiben und Oliven dekorieren. Dazu trinkt man einen Orvieto,
	am besten ,,Abboccato`` (eher lieblich). \\
      \end{zubereitung}

    \mynewsection{Leber-Toast (Crostini)}

      \begin{zutaten}
	200 g & \myindex{Geflügelleber}\index{Leber>Geflügel-} oder
	        \myindex{Kalbsleber}\index{Leber>Kalbs-} \\
	40 g & \myindex{Butter} \\
	1 mittelgroße & \myindex{Zwiebel} \\
	2 & \myindex{Sardellen}\index{Fisch>Sardellen}filets \\
	2 Eßlöffel & \myindex{Tomatenmark} \\
        6 Eßlöffel & \myindex{Weißwein}\index{Wein>weiß} \\
	1 & zerdrückte \myindex{Knoblauchzehe} \\
	1 Bund & \myindex{Petersilie} \\
	einige & frische \myindex{Salbeiblätter} \\
        1 Eßlöffel & geriebener \myindex{Parmesan}\index{Käse>Parmesan}käse \\
	& \myindex{Salz} \\
	& \myindex{Pfeffer} \\
	6 Scheiben & \myindex{Toastbrot}\index{Brot>Toast-} \\
      \end{zutaten}

      \begin{zubereitung}
        Gewürfelte Leber in heißer Butter braten. Herausnehmen und fein würfeln
	oder hacken. Im selben Fett feingehackte Zwiebel und Sardellenfilets
	andünsten. Leber, Tomatenmark, Wein und Knoblauchzehe zufügen. Alles
	kurz dünsten. Dann feingehackte Kräuter und Parmesankäse unterrühren.
	Würzen und abschmecken. Alles auf die Toastscheiben verteilen. Diagonal
	durchschneiden und kurz übergrillen. \\
	Dazu einen herzhaft-fruchtigen Chianti ,,Classico``. \\
      \end{zubereitung}

    \mynewsection{Ausgebackene Meeresfrüchte (Fritto misto)}%
      \glossary{Meeresfrüchte ausgebacken}

      \begin{zutaten}
	375 g & tiefgefrorene \myindex{Tintenfische} \\
	500 g & frische tiefgefrorene \myindex{Sardinen} \\
        375 g & \myindex{Riesengarnelen}\index{Garnelen>Riesen-} oder
	        \myindex{Scampi} \\
	3--4 Eßlöffel & \myindex{Mehl} \\
	& \myindex{Salz} \\
	& \myindex{Pfeffer} \\
	& \myindex{Plattenfett}\index{Fett>Platten-} zum Fritieren \\
	1 & \myindex{Zitrone} \\
	1 Strauß & \myindex{Petersilie} \\
      \end{zutaten}

      \personen{8}

      \begin{zubereitung}
        Alle Fische putzen, waschen und mit Küchenkrepp trockentupfen.
	Tintenfische in Ringe schneiden. Mehl, Salz und Pfeffer mischen. Fische
	in Mehl wenden. Schwimmend in heißem Fett goldbraun backen. Mit
	Zitronenschnitzen und eventuell Petersiliensträußchen anrichten. \\
	Dazu ein Wein aus Veneto, einem weißen Soave Classico Superiore, D.O.C.
	(Denominazione di Origine Controllata) oder einen Corvo Bianco aus
	Sizilien mit ausgeprägtem tiefgründigen Geschmack. \\
      \end{zubereitung}

    \mynewsection{Champignongratin}

      \begin{zutaten}
	200 g & \myindex{Champignon}\index{Pilze>Champignon}s \\
	3--4 & \myindex{Tomate}n \\
	1 Beutel & \myindex{Mozzarella} \\
        3--4 Scheiben & \myindex{gekochter Schinken}\index{Schinken>gekocht} \\
	2 Eßlöffel & gehackte \myindex{Kräuter} (kann Tiefkühl sein) \\
	& \myindex{Olivenöl}\index{Oel=Öl>Oliven-} \\
      \end{zutaten}

      \begin{zubereitung}
        Die frischen Pilze gibt es nun wirklich in jedem Supermarkt, man kann
	sie sich also rasch auf dem Nachhauseweg noch mitnehmen: Zu Hause
	sollte man als Erstes den Backofen einschalten. \\
	Die Pilze unter fließendem Wasser schnell abbrausen, dann mit dem
	Gurkenhobel oder dem Eierschneider in Scheiben schneiden. In einer
	großen, flachen feuerfesten Form verteilen. Salzen, pfeffern, mit
	gehackten Kräutern (kann auch ein Tiefkühlprodukt sein) vermischen und
	mit etwas Olivenöl beträufeln. Tomatenwürfel darauf verteilen (mit oder
	ohne Haut --- die jedoch schnell entfernt ist, wenn man heißes Wasser
	darübergießt). Alsdann gewürfelten Käse (zum Beispiel Mozzarella) und
	Schinkenstreifen. Alles salzen und pfeffern sowie mit gehackten
	Kräutern bestreuen. \\
	Bei \grad{200} im vorgeheizten Ofen ca. 15~Minuten backen, bis alles
	brodelt. Dazu frisches Brot (Knoblauchbaguette) --- für eine oder zwei
	Personen ein wunderbares, schnelles Essen! Und ein frischer Weißwein,
	ein Glas Buttermilch oder ein Pils: Der Pizzaservice braucht garantiert
	länger! \\
      \end{zubereitung}

    \mynewsection{Blumenkohl italienisch}

      \begin{zutaten}
	1 großer & \myindex{Blumenkohl}\index{Kohl>Blumen-} \\
	2 & \myindex{Zwiebel}n \\
        2 dicke Scheiben & \myindex{gekochter Schinken}
	                   \index{Schinken>gekocht} \\
	125 g & \myindex{Champignon}\index{Pilze>Champignon}s \\
        \brea{} l & \myindex{süße Sahne}\index{Sahne>süß} \\
	& \myindex{Salz} \\
	& \myindex{Pfeffer} \\
	& \myindex{Mehl} \\
	2 Eßlöffel & \myindex{Butter} \\
	& \myindex{Petersilie} \\
	2--3 Eßlöffel & geriebener \myindex{Käse} \\
	2 Eßlöffel & \myindex{Paniermehl} \\
        2 Eßlöffel & \myindex{Olivenöl}\index{Oel=Öl>Oliven-} \\
      \end{zutaten}

      \personen{2--3}

      \begin{zubereitung}
        Den Blumenkohl in Salzwasser knapp gar kochen, abtropfen. Inzwischen
	Zwiebeln hacken, mit den blättrig geschnittenen Champignons in Fett
	andünsten, den gewürfelten Schinken dazugeben, mit etwas Mehl
	überstäuben, noch einmal durchschwitzen und mit der Sahne auffüllen.
	Mit Salz, Pfeffer und gehackter Petersilie abschmecken, über den
	Blumenkohl gießen. Nach Belieben Käse und Paniermehl überstreuen, mit
	Öl beträufeln und kurz im Ofen überbacken. \\
      \end{zubereitung}

    \mynewsection{Gurken süß-sauer}

      \begin{einleitung}       
        Zur Gurkenzeit im Sommer ein ganz billiges und schnelles Essen
        (10~Minuten Kochzeit)! Ich mache gern Reis dazu, körnig gekocht mit ein
        wenig Brühe (klare Fleischsuppe aus der Packung), mit Tomatenmark und
        frisch gemahlenem Pfeffer gewürzt. Ganz Hungrige können noch
        Fleischklößchen dazu nehmen, gekocht oder gebraten. \\
      \end{einleitung}       

      \begin{zutaten}
        2 große & \myindex{Salatgurke}\index{Gurke>Salat-}n (oder etwa 1~kg
	          \myindex{Schmorgurke}\index{Gurke>Schmor-}n) \\
	3 große & \myindex{Zwiebel}n \\
	1 Tasse & \myindex{Brühe} \\
	& Öl\index{Oel=Öl} oder \myindex{Margarine} \\
        ca. \brea{} l & \myindex{süße Sahne}\index{Sahne>süß} oder
	                \myindex{saure Sahne}\index{Sahne>sauer} \\
	etwas & \myindex{Mehl} \\
	& \myindex{Pfeffer} \\
	& \myindex{Salz} \\
	& \myindex{Essig} \\
	& \myindex{Zucker} \\
      \end{zutaten}

      \begin{zubereitung}
        Gurken schälen, der Länge nach durchschneiden (eventuell die Kerne
	etwas herauskratzen, wenn sie schon sehr groß sind) und in große
	Würfel schneiden. Zwiebeln hacken, in Öl oder Margarine (oder
	Rauchspeck) anbraten, Gurkenwürfel dazugeben, ebenfalls kurz
	anschmoren, mit Brühe (klare Fleischsuppe) auffüllen und knapp gar
	kochen. Sahne mit etwas Mehl verquirlen, anrühren, kurz aufwallen
	lassen, mit Salz und frischem Pfeffer abschmecken, zum Schluß mit
	Essig und Zucker würzen. Die Soße muß kräftig süß-sauer schmecken,
	nehmen Sie zuerst etwa 1~Eßlöffel Zucker, verrührt in 1~Eßlöffel Essig,
	und würzen Sie nach. \\
      \end{zubereitung}

    \mynewsection{Dicke Bohnen mit Erdnüssen}

      \begin{einleitung}
        Das Rezept überzeugt auch Leute, die sonst keine dicken Bohnen mögen.
        Ich nehme aus Bequemlichkeit immer Bohnenkerne aus der Dose oder
        tiefgefrorene und bin in 10~Minuten fertig. \\
      \end{einleitung}

      \begin{zutaten}
        1 Dose & \myindex{dicke Bohnen}\index{Bohnen>dick} (oder 2~Pakete
	         tiefgefrorene) \\
	4 & \myindex{Zwiebel}n \\
	1 & \myindex{Knoblauchzehe} \\
	3 Eßlöffel & \myindex{Olivenöl}\index{Oel=Öl>Oliven-} \\
	& \myindex{Pfeffer} \\
	& \myindex{Salz} \\
	1 Päckchen & ungesalzene \myindex{Erdnüsse} \\
      \end{zutaten}

      \personen{2--3}

      \begin{zubereitung}
        Zwiebeln und Knoblauch fein schneiden, in Olivenöl in einer tiefen
	Pfanne leicht anbraten (nicht bräunen), dann die Bohnen dazugeben und
	kurz durchdünsten. Mit Salz und Pfeffer abschmecken. Inzwischen in
	einer anderen Pfanne etwa eine Tasse ungesalzene Erdnüsse mit wenig Öl
	anrösten, unter die Bohnen mischen --- und staunen, wie gut es
	schmeckt! Vielleicht dazu noch Reis? \\
      \end{zubereitung}

    \mynewsection{Schwarzwurzel-Möhren-Gratin}

      \begin{zutaten}
	500 g & \myindex{Schwarzwurzeln} \\
	3 Eßlöffel & \myindex{Essig} \\
	500 g & \myindex{Möhre}n \\
	2 mittlere & \myindex{Zwiebel}n \\
	1--2 & \myindex{Knoblauchzehe}n \\
	& \myindex{Salz} \\
	& Öl\index{Oel=Öl} zum Braten \\
     \end{zutaten}

     \begin{zutat}{Soße}
	1 Eßlöffel & \myindex{Senf} (möglichst
	             \myindex{Dijon-Senf}\index{Senf>Dijon-}) \\
	2 & \myindex{Ei}er \\
	50 ml & \myindex{Sahne} \\
	100 ml & \myindex{Gemüsebrühe} \\
	& \myindex{Salz} \\
        & \myindex{weißer Pfeffer}\index{Pfeffer>weiß} \\
	etwas & \myindex{Muskatnuß} \\
	75 g & \myindex{Emmentaler}\index{Käse>Emmentaler} grob geraffelt \\
      \end{zutat}

      \personen{4}

      \begin{zubereitung}
        Schwarzwurzeln schälen, in Essigwasser legen, Möhren schälen, in dünne
	Scheiben hobeln. Zwiebeln hacken. Schwarzwurzeln in ca. 4~cm dicke
	Stücke schneiden. In Essigwasser und Salz ca. 15~Minuten kochen.
	Möhren in Öl dünsten (10~Minuten), dann Zwiebeln dazugeben und
	mitdünsten. Salzen und pfeffern. Deckel drauf und Platte abschalten. \\
	Soße: Senf, Eier, Sahne, Gemüsebrühe, Salz, Pfeffer, Muskat und Käse
	gründlich vermischen. Schwarzwurzeln abgießen, eventuell vorwürzen und
	in gefettete Auflaufform geben. Möhren dazu und Gemüse etwas
	vermischen. Soße darüber geben, dabei vorsichtig gießen, damit überall
	etwas hinkommt. Dann bei \grad{200} 20--25~Minuten in den Backofen. \\
	Dazu: Pellkartoffeln. \\
      \end{zubereitung}

    \mynewsection{Gemüsemoussaka}

      \begin{zutaten}
	2 kleine & \myindex{Aubergine}n (ca. 250 g) \\
	& \myindex{Salz} \\
	3 & \myindex{Zucchini} \\
	1 & \myindex{Paprika}schote \\
	500 g & \myindex{Tomate}n \\
	250 g & \myindex{Zwiebel}n \\
	2 & \myindex{Knoblauchzehe}n \\
	1 Bund & \myindex{Basilikum} \\
	& \myindex{Fett} für die Form \\
	& \myindex{Pfeffer} \\
	2 Eßlöffel & Öl\index{Oel=Öl} \\
        100 g & \myindex{Schafkäse}\index{Käse>Schaf-} \\
        50 g & \myindex{Emmentaler}\index{Käse>Emmentaler} \\
      \end{zutaten}

      \personen{4}

      \begin{zubereitung}
        Auberginen waschen, in 1~cm dicke Scheiben schneiden, salzen und
	30~Minuten stehen lassen. Abspülen und abtropfen lassen. 
	Zucchini waschen, ebenfalls in Scheiben schneiden. Paprikaschote
	waschen, entkernen und in Streifen schneiden. Tomaten waschen, häuten
	und achteln. Zwiebeln und Knoblauch sowie Basilikumblätter fein hacken.
	Feuerfeste Form fetten. Alle Zutaten mischen und mit Salz, Pfeffer
	bestreuen. Mit Öl beträufeln. Backofen vorheizen auf \grad{200} und
	45~Minuten backen. Den Schafkäse zerkrümeln, mit Emmentaler darüber
	geben und weitere 15~Minuten überbacken. \\
      \end{zubereitung}

    \mynewsection{Grüne Bandnudeln mit Champignons}

      \begin{zutaten}
        300 g & grüne \myindex{Bandnudeln}\index{Nudeln>Band-} \\
	1 Eßlöffel & Öl\index{Oel=Öl} \\
	& \myindex{Salz} \\
	100 g & \myindex{Champignon}\index{Pilze>Champignon}s \\
	30--40 g & \myindex{Parmesan}\index{Käse>Parmesan} \\
	20 g & \myindex{Butter} \\
      \end{zutaten}

      \personen{2}

      \begin{zubereitung}
        Nudeln kochen, Champignons leicht anbraten und salzen, mit Parmesan
	mischen und an die Nudeln geben. Butterflocken darangeben. \\
      \end{zubereitung}

    \mynewsection{Wirsingrouladen mit Bulgurfüllung}

      \begin{zutaten}
	8--10 große & \myindex{Wirsing}blätter \\
	2 Handvoll & \myindex{Wirsing}streifen \\
	2 große & gewürfelte \myindex{Zwiebel}n \\
	100 g & \myindex{Schinkenwürfel} \\
	1\breh{} Tassen & \myindex{Bulgur} \\
	1 & \myindex{Ei} \\
	1 & trockenes \myindex{Brötchen} \\
	& \myindex{Gemüsebrühe} \\
	2--3 Teelöffel & \myindex{Mehl} \\
	& Öl\index{Oel=Öl} zum Braten \\
	& \myindex{Salz} \\
	& \myindex{weißer Pfeffer}\index{Pfeffer>weiß} \\
	& \myindex{Muskatnuß} \\
	1 Prise & \myindex{Zucker} \\
	& Zwirn zum Binden \\
      \end{zutaten}

      \personen{4}

      \begin{zubereitung}
        Von einem großen Wirsingkopf entfernt man die schadhaften und zu rauhen
	Blätter. Dann schneidet man vom Strunk je 1~Blatt los, löst es
	vorsichtig, ohne es zu brechen. Wenn man 10~Blätter hat, schneidet man
	außen die Mittelrippe flach. Die hellen Blätter legt man nach innen,
	die dunkleren bilden die äußere Hülle. \\
	In einem großen Topf mit reichlich Wasser blanchiert man alle
	Wirsingblätter. Das Brötchen weicht man in kaltem Wasser ein. 1\breh{}
	Tassen Bulgur in ein feines Sieb geben und waschen. In einem kleinen
	Topf mit Salz und Wasser ca. 15~Minuten kochen. \\
	Wirsingblätter kalt abspülen, abtropfen lassen und trocknen.
	Wirsingwasser auffangen für die Soße. In einer Pfanne mit wenig Öl die
	Speckwürfel auslassen, 1~Zwiebel dazu und die Wirsingstreifen unter
	Wenden dünsten, Salz und Pfeffer dran. \\
	Bulgur in eine Schüssel geben und abkühlen lassen, dann Inhalt der
	Pfanne dazugeben. Eingeweichtes Brötchen ausdrücken und in die Schüssel
	geben. Mit Salz, Pfeffer und Muskatnuß würzen. Wenn alles abgekühlt
	ist, das Ei dazugeben und alles gut mischen. Die Füllung abschmecken
	und in die hellen Wirsingblätter geben, Ränder einklappen und
	aufrollen, dann in das äußere Blatt wickeln und mit Baumwollfaden
	zuwickeln. Trocknen. \\
	Gußeisernen Topf aufstellen, Öl darin erhitzen und Rouladen ringsum
	braun anbraten. Danach mit Wirsingwasser und Gemüsebrühe auffüllen und
	ca. 20--30~Minuten köcheln lassen. Mehl in kaltem Wasser anrühren. \\
	In der Zwischenzeit geschälte Kartoffeln aufsetzen. Rouladen rausheben
	und Soße mit Mehlwasser verrühren, kochen lassen. Mit Salz, Pfeffer,
	Gemüsebrühe, Zucker, Muskatnuß abschmecken. Zwirn von den Rouladen
	entfernen, restliche Brühe an die Soße geben und durchrühren. \\
	Tip: Wenn man keinen gußeisernen Topf hat, die Rouladen in einer Pfanne
	anbraten, zu Not auch einzeln. Alle Rouladen dann in einen größeren
	normalen Topf geben und wie beschrieben verfahren. \\
      \end{zubereitung}

    \mynewsection{Sauerkraut mit Schinkenknödel}

      \begin{zutaten}
      \end{zutaten}
      \begin{zutat}{Sauerkraut}
	1 kg & \myindex{Sauerkraut} \\
	1 & \myindex{Zwiebel} \\
	1 & \myindex{Apfel} \\
	100 g & \myindex{Speck} \\
	1 Eßlöffel & \myindex{Zucker} \\
	1 Eßlöffel & \myindex{Butterschmalz} \\
	\brev{} l & \myindex{Brühe} \\
	10 & \myindex{Wacholderbeeren} \\
	1 & \myindex{Lorbeer}blatt \\
	& \myindex{Salz} \\
	& \myindex{Pfeffer} \\
      \end{zutat}

      \begin{zubereitung}
        Speck, Zwiebel und Apfel geschält und geschnitten in Butterschmalz
	anbraten --- Kraut dazugeben und gut umrühren. Die Gewürze dazugeben
	und mit der Brühe aufgießen. Zugedeckt ca. 1~Stunde köcheln lassen,
	ab und zu umrühren, damit das Kraut nicht anbrennt. Eventuell
	Flüssigkeit nachgeben, als Alternative kann man Apfelsaft oder
	Weißwein nehmen. \\
      \end{zubereitung}

      \begin{zutaten}
      \end{zutaten}
      \begin{zutat}{Schinkenknödel}
	5 & \myindex{Semmeln} vom Vortag \\
	350 g & \myindex{Schinkenwürfel} \\
	1 & \myindex{Zwiebel} gehackt \\
	\breh{} l & \myindex{Milch} \\
	4 & \myindex{Ei}er \\
	150 g & \myindex{Mehl} \\
	\breh{} Bund & \myindex{Petersilie} gehackt \\
	10 g & \myindex{Butterschmalz}\index{Schmalz>Butter-} \\
	& \myindex{Salz} \\
	& \myindex{Pfeffer} \\
      \end{zutat}

      \begin{zubereitung}
        Die Semmeln kleinschneiden und in warmer Milch einweichen. Zwiebeln und
	Schinkenwürfel in Butterschmalz anbraten und zu den Semmeln geben. Die
	ganzen Eier und das Mehl untermengen und die Masse mit Salz und Pfeffer
	abschmecken --- Vorsicht! Der Schinken ist schon salzig! --- Die
	geschnittene Petersilie unterheben. \\
	1~Probeknödel formen und ins kochende, leicht gesalzene Wasser
	einlegen. Falls er zu weich ist, noch etwas Mehl nachgeben. Die Knödel
	ca. 15~Minuten in leicht kochendem Wasser ziehen lassen. \\
      \end{zubereitung}

    \mynewsection{Seeteufel (Lotte)}

      \begin{zutaten}
        & \myindex{Seeteufel}\index{Fisch>Seeteufel}\index{Fisch>Lotte} \\
	& \myindex{Wirsing} \\
	1 Scheibe & durchwachsener geräucherter \myindex{Speck} \\
      \end{zutaten}

      \begin{zubereitung}
        Der Seeteufel wird in ein blanchiertes Wirsingblatt gewickelt.
	Obendrauf kommt eine Scheibe Speck. Das wird in einer Pfanne angebraten
	und schließlich bei \grad{175} im Backofen gegart. \\
      \end{zubereitung}

    \mynewsection{Lachs-Nudelpfanne mit Senfsoße}

      \begin{zutaten}
	600 g & \myindex{Lachs}\index{Fisch>Lachs}filet \\
	& \myindex{Salz} \\
	1 & \myindex{Salatgurke}\index{Gurke>Salat-} \\
	1 & \myindex{Zwiebel} \\
        1 Bund & \myindex{glatte Petersilie}\index{Petersilie>glatt} \\
	2 Eßlöffel & Pflanzenmargarine\index{Margarine} \\
	300 ml & \myindex{Gemüsebrühe} \\
	& \myindex{Pfeffer} \\
	& \myindex{Zucker} \\
	250 g & \myindex{Bandnudeln}\index{Nudeln>Band-}, gewalzt \\
	150 ml & \myindex{Sahne} \\
	3--4 Teelöffel & \myindex{Senf} \\
	2 Eßlöffel & \myindex{Soßenbinder} für helle Soßen \\
      \end{zutaten}

      \personen{4}

      \begin{zubereitung}
        Lachsfilet waschen, trocken tupfen und in Würfel schneiden. Gurke
	schälen, längs halbieren, entkernen und in Stücke schneiden. \\
	Zwiebel schälen und würfeln. Petersilie waschen und klein hacken. \\
	Wasser für die Nudeln zum Kochen bringen. Einen Eßlöffel
	Pflanzenmargarine in einem Topf erhitzen. Zwiebel und Gurke darin
	andünsten. Mit Gemüsebrühe ablöschen, Salz, Pfeffer und eine Prise
	Zucker zufügen. Weitere 3~Minuten dünsten. \\
	Restliche Pflanzenmargarine erhitzen und die Lachswürfel darin
	anbraten. Nudeln zum kochenden Wasser geben und nach Packungsanweisung
	bißfest kochen. \\
	Lachswürfel, Sahne und Senf zu dem Gemüse geben und weitere 3~Minuten
	dünsten. Petersilie zugeben, die Soße abschmecken und mit Soßenbinder
	binden. \\
	Mit Nudeln auf dem Teller anrichten. \\
      \end{zubereitung}

    \mynewsection{Hirse-Backlinge}

      \begin{zutaten}
        250 g & gekochte \myindex{Hirse} \\
	150 g & \myindex{Haferflocken} \\
	150 ml & \myindex{Tafelwasser}\index{Wasser>Tafel-} \\
	75 g & \myindex{Quark} \\
	50--60 g & \myindex{Pflanzenmargarine}\index{Margarine>Pflanzen-} \\
	200--250 g & fein geriebenes Gemüse (z.B. \myindex{Sellerie},
	             \myindex{Möhre}n, \myindex{Petersilie}nwurzel,
		     \myindex{Lauch}) \\
        & \myindex{Salz} \\
        & \myindex{Pfeffer} \\
        & frische \myindex{Kräuter} \\
      \end{zutaten}

      \personen{4}

      \begin{zubereitung}
        Die Haferflocken in dem Tafelwasser kurz einweichen, währenddessen
	Quark mit der Pflanzenmargarine cremig rühren. Die Masse mit der
	gekochten Hirse, den Haferflocken und dem geriebenen Gemüse
	vermengen, kräftig mit den frisch gehackten Kräutern, Salz und Pfeffer
	abschmecken. Die Masse gut durcharbeiten und zu einer Rolle formen.
	Die Rolle in ca. 1~cm dicke Scheiben schneiden, auf ein gefettetes
	Backblech legen und mit Butterflocken oder geriebenem Käse bestreuen.
	Bei mittlerer Hitze ca. 20--30~Minuten goldgelb backen. \\
      \end{zubereitung}

    \mynewsection{Gemüseauflauf}

      \begin{zutaten}
        3 & mittelgroße \myindex{Möhre}n \\
	1 & große \myindex{Zwiebel} \\
	4--5 & große \myindex{Kartoffel}n \\
	2 & mittelgroße \myindex{Kohlrabi} \\
	1 & größere \myindex{Zucchini} oder \\
	2 & kleine bis mittlere \myindex{Zucchini} \\
	1 & \myindex{rote Paprika}\index{Paprika>rot}schote \\
	1--2 & \myindex{Knoblauchzehe}n \\
	1 Bund & \myindex{glatte Petersilie}\index{Petersilie>glatt} \\
	3--4 Zweige & \myindex{Zitronenmelisse} \\
	10 Blatt & \myindex{Basilikum} \\
	& \myindex{Salz} \\
        & \myindex{weißer Pfeffer}\index{Pfeffer>weiß} und \\
        & \myindex{schwarzer Pfeffer}\index{Pfeffer>schwarz} \\
	& \myindex{Muskatnuß} \\
	1\breh{} Teelöffel & \myindex{Gemüsebrühe} \\
	1 Becher & \myindex{saure Sahne}\index{Sahne>sauer} \\
	ca. \brea{} l & fettarme \myindex{Milch} \\
	etwas & geriebener \myindex{Gouda}\index{Käse>Gouda} \\
        & \myindex{Olivenöl}\index{Oel=Öl>Oliven-} \\
      \end{zutaten}

      \begin{zubereitung}
        Möhren, Kartoffeln, Kohlrabi, Paprika schälen bzw. putzen.
	Möhren, Kartoffeln, Kohlrabi in dünne Scheiben schneiden bzw. hobeln.
	Zwiebel in Ringe hobeln. Paprikaschote erst in Streifen, dann in
	Würfel schneiden. Kräuter getrennt hacken (Melisseblätter abstreifen).
	Große Auflaufform einölen. Möhren auf den Boden verteilen, würzen mit
	Salz und weißem Pfeffer. Darüber die Zwiebelringe und darauf etwas
	Petersilie streuen. Kartoffelscheiben einzeln darüber decken und
	salzen, pfeffern (weiß) und mit Muskat würzen. Darüber noch eine Lage
	Kartoffelscheiben wie vor. Danach Kohlrabischeiben legen und mit Salz,
	weißem Pfeffer und Muskat würzen. Kräuter (Petersilie und Melisse)
	darüberstreuen. Zweite Kohlrabischicht legen wie vor. Darauf
	Zucchini hobeln in kurze Streifen und dazwischen Paprikawürfel streuen.
	Darauf restliche Petersilie und Basilikum geben. Würzen mit Salz und
	schwarzem Pfeffer. \\
	Gouda reiben. Sahne würzen mit einer gepreßten Knoblauchzehe, Salz,
	Pfeffer, Muskat und auffüllen mit Milch, Gemüsebrühe dazu und
	verrühren. Abschmecken. Ringsum am Rand und dann in die Mitte angießen.
	Käse darüberstreuen. Abschließend restliche Kartoffelscheiben auflegen,
	salzen und pfeffern. Olivenöl auf jede Scheibe geben. \\
	Im Backofen bei \grad{200} ca. 60~Minuten garen. \\
      \end{zubereitung}

    \mynewsection{Crostini mit Tomate und Basilikum}

      \begin{zutaten}
        200 g & dunkles \myindex{Brot} \\
	300 g & schnittfeste \myindex{Tomate}n \\
	3 & \myindex{Knoblauchzehe}n \\
	& \myindex{Basilikum} \\
        & \myindex{Olivenöl}\index{Oel=Öl>Oliven-} \\
	& \myindex{Essig} \\
	& \myindex{Salz} \\
	& \myindex{Pfeffer} \\
      \end{zutaten}

      \personen{6}

      \begin{zubereitung}
        Auf das geröstete und mit Knoblauch eingeriebene Brot die geschälten
	und kleingeschnittenen, mit Öl, Essig, Salz und Pfeffer angemachten
	Tomaten geben und mit kleingeschnittenem Basilikum bestreuen. \\
      \end{zubereitung}

    \mynewsection{Gnocchi alla Fiorentina}

      \begin{zutaten}
       1 kg & frischer \myindex{Spinat} \\
       600 g & \myindex{Quark} \\
       90 g & \myindex{Parmesan} \\
       2 & \myindex{Ei}er \\
       60 g & \myindex{Mehl} \\
       60 g & \myindex{Butter} \\
       & \myindex{Pfeffer} \\
       & \myindex{Salz} \\
       & \myindex{Muskatnuß} \\
       & \myindex{Salbei} \\
      \end{zutaten}

      \personen{6}

      \begin{zubereitung}
        Den Spinat kochen, abkühlen lassen, gut ausdrücken, kleinhacken und
	mit dem Quark, den Eiern, dem Mehl und der Hälfte des Parmesan
	vermengen, mit Salz und Muskat abschmecken. \\
	Auf dem bemehlten Tisch den Teig zu einer Schlange formen und in kleine
	Klösse schneiden. Die in Mehl gewendeten Klösse in sprudelndem
	Salzwasser kochen, abtropfen lassen. \\
	In einer feuerfesten Form Butter schmelzen lassen, Salbei in der Butter
	ziehen lassen, Gnocchi beigeben, mit Parmesan bestreuen. \\
      \end{zubereitung}

    \mynewsection{Huhn auf Jägerart}

      \begin{zutaten}
	1,5 kg & \myindex{Huhn} \\
	1 & \myindex{Zwiebel} \\
	3 & \myindex{Knoblauchzehe}n \\
        20 cl & \myindex{Weißwein}\index{Wein>weiß} \\
	300 g & geschnittene \myindex{Tomate}n \\
	300 g & frische \myindex{Pilze} oder
	        \myindex{Champignon}\index{Pilze>Champignon}s \\
	& \myindex{Petersilie} oder \myindex{Dragoncello} \\
        & \myindex{Olivenöl}\index{Oel=Öl>Oliven-} \\
	& \myindex{Salz} \\
	& \myindex{Pfeffer} \\
	1 Bund & aus: \\
	1 Stange & \myindex{Sellerie} \\
	\breh{} & \myindex{Möhre} \\
	1 Zweigchen & \myindex{Salbei} \\
	1 Zweig & \myindex{Rosmarin} \\
	1 & \myindex{Lorbeer}blatt \\
      \end{zutaten}

      \personen{6}

      \begin{zubereitung}
        Die Zwiebel und den Knoblauch kleingeschnitten in Scheibchen in einer
	Pfanne mit etwas Öl andünsten. Die Pilze gut gewaschen und in Scheiben
	geschnitten hinzufügen. Es ist besser, die getrockneten Pilze, zum
	quellen eingeweicht, zuerst separat anzubraten. Die Hälfte des
	Weißweins darübergießen, verdunsten lassen und die Petersilie, die
	kleingeschnittenen Tomaten, Salz und Pfeffer hinzugeben. Das gut
	gesäuberte, gewaschene, in Stücke von mittlerer Grösse geschnittene und
	eingemehlte Huhn separat in einer Pfanne mit heißem Öl anbraten,
	salzen, das Öl abgießen und die andere Hälfte Wein darübergießen,
	verdunsten lassen und in die vorbereitete Soße legen. \\
	Auf niedriger Flamme zum Kochen bringen und dabei falls nötig Brühe
	oder Wasser hinzugeben. Mit feiner Petersilie bestreuen und servieren.
	\\
      \end{zubereitung}
     
    \mynewsection{Süßspeise mit Mascarpone}

      \begin{zutaten}
        350 g & \myindex{Mascarpone} (eine Art Speisequark) \\
        3 & \myindex{Ei}er \\
        150 g & \myindex{Zucker} \\
        10 cl & \myindex{Marsala} \\
        1 Tasse & \myindex{Kaffee} (nicht zu stark) \\
        1 Packung & \myindex{Löffelbiskuit} oder \\
        300 g & Pan di Spagna (siehe Seite \pageref{pandispagna}) \\
        100 g & \myindex{Chantilly}
	        (Schlagsahne mit etwas Zucker geschlagen) \\
      \end{zutaten}

      \personen{6--8}

      \begin{zubereitung}
        Den Mascarpone mit dem Eigelb und 100~g des Zuckers aufschäumen lassen,
	separat das Eiweiß mit 50~g des Zuckers schlagen. Wenn es gut steif
	ist, sie mit der vorherigen Mischung vermengen. \\
	Löffelbiskuit, eingeweicht mit etwas Marsala und etwas bitterem Kaffee
	anordnen. Mit Chantilly dekorieren und in den Kühlschrank stellen, weil
	es leicht gerinnen könnte. \\
      \end{zubereitung}


    \mynewsection{Pan di Spagna}\label{pandispagna}

      \begin{zutaten}
	5 & \myindex{Ei}er \\
	120 g & \myindex{Mehl} \\
	80 g & \myindex{Stärkemehl} \\
	150 g & \myindex{Zucker} \\
      \end{zutaten}

      \personen{6--8}

      \begin{zubereitung}
        Wenn man die Tortenkapsel mit Schokolade macht, tauscht man 20~g Mehl
	gegen 20~g Kakao. \\
	Mit dem elektrischen Schneebesen die Eier und den Zucker solange
	schlagen, bis die Masse schön schaumig und fest ist. Dann vorsichtig
	das Mehl und das Stärkemehl darunterheben. In eine Form geben, die
	vorher gut ausgefettet wurde, und im Ofen bei \grad{180-220} für ca.
	10--15~Minuten backen. Aus dem Ofen nehmen, auf einer Serviette
	umstülpen, damit die restliche, geringe Feuchtigkeit verdampft. Diese
	Masse ist ein Kuchen, der als Basis für viele Torten in Schichten
	dienen kann. \\
      \end{zubereitung}

    \mynewsection{Füllung für Tortellini}

      \begin{zutaten}
	1 & kleine \myindex{Zwiebel} \\
	120 g & \myindex{Schweinefleisch} \\
	120 g & \myindex{Hühnerbrust} \\
	1 & \myindex{Ei} \\
	& \myindex{Muskatnuß} \\
	30 g & \myindex{Parmesan}käse \\
	& \myindex{Salz} \\
        & \myindex{Olivenöl}\index{Oel=Öl>Oliven-} \\
      \end{zutaten}

      \personen{6}

      \begin{zubereitung}
        Die kleingeschnittene Zwiebel andünsten mit ein wenig Öl, das gesamte
	kleingeschnittene Fleisch hinzugeben, mit etwas Wasser (oder Brühe)
	übergießen, bei kleiner Flamme für etwa 10~Minuten ziehen lassen.
	Ganz klein durchdrehen, den Parmesankäse, den Muskat und das Salz
	mit dem Rest gut vermengen. \\
      \end{zubereitung}

    \mynewsection{Tortellini in Sahne}

      \begin{zutaten}
	600 g & frische \myindex{Tortellini} \\
	300 g & \myindex{Sahne} \\
	60 g & \myindex{Butter} \\
	60 g & \myindex{Parmesan}käse \\
	& \myindex{Salz} \\
	& \myindex{Pfeffer} \\
      \end{zutaten}

      \personen{6}

      \begin{zubereitung}
        Die Sahne in einen Topf gießen, die Butter und eine Messerspitze Salz
	und Pfeffer hinzufügen, danach kommen die bereits in Salzwasser
	gekochten Tortellini hinzu. Leicht aufkochen und, wenn die Creme
	andickt, den Parmesankäse hinzufügen. Gut umrühren und servieren. \\
      \end{zubereitung}

    \mynewsection{Schweinsrücken florentinisch}

      \begin{zutaten}
	1 Stück & \myindex{Schweineschulterstück} (Kotelett) (ca. 1,2 kg) \\
	& \myindex{Salbei} \\
	& \myindex{Rosmarin} \\
	& \myindex{Knoblauch} \\
        & \myindex{Olivenöl}\index{Oel=Öl>Oliven-} \\
	& \myindex{Salz} \\
	& \myindex{Pfeffer} \\
	20 cl & \myindex{Weißwein}\index{Wein>weiß} \\
      \end{zutaten}

      \personen{6}

      \begin{zubereitung}
        Das Schulterstück vom Knochen befreien, mit einem aromatischen Salz
	(hergestellt aus Rosmarin, Salbei, fein zerkleinertem Knoblauch, Salz
	und Pfeffer) würzen. Einbinden, damit diese gut zusammenhält. In einem
	Topf mit hohem Rand anbraten. Wenn das Fleisch halb gar ist, das Fett
	abschneiden und den Wein darübergießen. Kochen, bis sie gar ist und,
	falls nötig, ab und zu Bratfett darübergießen. Das Fleisch aus dem Topf
	nehmen. \\
      \end{zubereitung}

    \mynewsection{Weiße und schwarze Crostini}

      \begin{zutaten}
      \end{zutaten}

      \begin{zutat}{weiße Crostini}
        12 & \myindex{Crostini} von hellem Brot \\
	2 & hart gekochte \myindex{Ei}er \\
	ein wenig & \myindex{grüner Salat}\index{Salat>grün} \\
	ein wenig & \myindex{roter Radicchio}\index{Radicchio>rot} \\
	ein Eßlöffel & \myindex{Kapern} \\
	4 & \myindex{Sardellen}\index{Fisch>Sardellen} \\
        & \myindex{Olivenöl}\index{Oel=Öl>Oliven-} \\
	& \myindex{Salz} \\
	& \myindex{Pfeffer} \\
      \end{zutat}

      \begin{zutat}{schwarze Crostini}
        12 & \myindex{Crostini} von dunklem Brot \\
	3 Scheiben & \myindex{Hühnerleber}\index{Leber>Huhn} \\
	3 & \myindex{Sardellen}\index{Fisch>Sardellen}filets \\
	30 g & \myindex{Butter} \\
	1 Stückchen & \myindex{Zwiebel} \\
	& \myindex{Marsala} (trocken) \\
	& \myindex{Salz} \\
	& \myindex{Pfeffer} \\
	& Öl\index{Oel=Öl} \\
	& \myindex{Fleischbrühe} \\
      \end{zutat}

      \personen{6}

      \begin{zubereitung}
        Weiße Crostini: Alles in kleine Stückchen schneiden, mit Öl, Salz und
	Pfeffer anrichten und auf die Crostini streichen. \\
	Schwarze Crostini: Die Zwiebel anbraten, die klein geschnittene, schon
	von beiden Seiten angebratene Leber hinzugeben. Den Marsala
	darübergießen und verdunsten lassen. Kapern, Sardellen, Pfeffer und
	Salz (nur falls nötig) hinzugeben. Mit dem Messer kleinschneiden, mit
	der Butter vermischen und auf die gerösteten, mit Brühe getunkten
	Crostini streichen. \\
      \end{zubereitung}

    \mynewsection{Mandelgebäck}

      \begin{zutaten}
	500 g & \myindex{Mehl} \\
	400 g & \myindex{Zucker} \\
	4 & \myindex{Ei}er \\
	300 g & ungeschälte \myindex{Mandel}n
	        (brühen, abbrausen, häuten, trocknen) \\
	1 Prise & \myindex{Salz} \\
	1 Teelöffel & \myindex{Backpulver} \\
	& abgeriebene Schale einer \myindex{Zitrone} \\
	& \myindex{Vanille} \\
      \end{zutaten}

      \begin{zubereitung}
        Alle Zutaten, außer dem Mehl und dem Backpulver, vermischen. Wenn alles
	gut vermischt ist, die letzten Zutaten unter Rühren hinzufügen. Schnüre
	mit dem Durchmesser eines Fingers formen und sie auf ein eingefettetes
	Backblech legen. Die Oberfläche mit geschlagenem Ei bestreichen und sie
	Backen. Dann nimmt man sie vom Backblech und schneidet sie quer zur
	Länge halbfingerbreit. Das Gebäck mit Vin Santo servieren. \\
      \end{zubereitung}

    \mynewsection{Gnocchi alla Fiorentina II}

      \begin{zutaten}
	500 g & \myindex{Wurzelspinat}\index{Spinat>Wurzel-} \\
	1 kleine & \myindex{Zwiebel} \\
	100 g & \myindex{Butter} \\
	150 g & \myindex{Ricotta}\index{Käse>Ricotta}käse oder
	        \myindex{Mascarpone} ersatzweise abgetropfter Sahnequark \\
	80 g & frisch geriebener \myindex{Parmesan}\index{Käse>Parmesan} \\
	2 & \myindex{Ei}er \\
	1 & \myindex{Eigelb} \\
	2 Teelöffel & \myindex{Salz} \\
	& \myindex{Muskatnuß} \\
	& \myindex{Pfeffer} aus der Mühle \\
        200 g & \myindex{Weizenmehl}\index{Mehl>Weizen-} \\
      \end{zutaten}

      \personen{4}

      \begin{zubereitung}
        Spinat verlesen, waschen, tropfnaß im Topf bei mittlerer Hitze
	zusammenfallen lassen. Abtropfen lassen und grob hacken. Zwiebel
	würfeln, 30 g Butter in einer Pfanne erhitzen und Zwiebel goldgelb
	andünsten. Spinat untermischen, dann abkühlen lassen, bis er nur noch
	lauwarm ist. \\
	Ricotta glatt rühren, Hälfte Parmesan, Eier und Eigelb, Salz, Muskat,
	Pfeffer und Spinat unterrühren. Zuletzt das Mehl unterrühren.
	Masse noch einmal herzhaft mit Salz und Pfeffer abschmecken. \\
	Reichlich Salzwasser aufkochen. Hitze zurückschalten. Mit 2~nassen
	Teelöffeln etwa nußgroße Klößchen aus dem Teig stechen und Gnocchi
	daraus formen. In siedendem Salzwasser in etwa 5~Minuten gar ziehen
	lassen, bis sie an die Oberfläche kommen. Mit der Schaumkelle
	herausheben und in eine flache gebutterte, ofenfeste Form setzen.
	Restliche Butter schmelzen, über die Gnocchi träufeln und mit
	restlichem Parmesan bestreuen. Im vorgeheizten Backofen bei \grad{150}
	ca. 5~Minuten überbacken. \\
	Dazu ein kräftiger, roter Lugana (vom Südufer des Gardasees). \\
      \end{zubereitung}

    \mynewsection{Pan di Spagna II}

      \begin{einleitung}
        Dieser klassische Biskuitteig wird ohne Backpulver zubereitet.
	Wird zur Herstellung delikater Süßspeisen verwendet. \\
      \end{einleitung}

      \begin{zutaten}
	6 & \myindex{Ei}er \\
	180 g & \myindex{Puderzucker}\index{Zucker>Puder-} \\
	1 Päckchen & \myindex{Vanillezucker}\index{Zucker>Vanille-} \\
	100 g & \myindex{Weizenmehl}\index{Mehl>Weizen-} \\
	75 g & \myindex{Speisestärke} \\
      \end{zutaten}

      \begin{zubereitung}
        Den Boden einer Springform ausfetten und mit Backpapier (oder
	Pergamentpapier) auslegen. Backofen auf \grad{180} vorheizen. \\
	Eier trennen und Eigelbe mit dem Zucker und dem Vanillezucker schaumig
	rühren. Eiweiß steif schlagen und unterheben. Mehl und Stärke über die
	Eiermasse sieben und mit dem Rührlöffel unterziehen. Den Teig in die
	Form füllen, glattstreichen und auf mittlerer Schiene 30--35~Minuten
	backen und danach Stäbchenprobe machen. Den Tortenboden erst am
	nächsten Tag verwenden. \\
      \end{zubereitung}

    \mynewsection{Gefüllte Tortellini II}

      \begin{zutaten}
      \end{zutaten}
      \begin{zutat}{Nudelteig}
        300 g & \myindex{Mehl} \\
	3 & \myindex{Ei}er \\
	etwas & \myindex{Salz} \\
      \end{zutat}
      \begin{zutat}{Füllung}
        1 & \myindex{Ei} \\
	100 g & \myindex{Hühnerbrust} \\
	100 g & \myindex{Schweinefilet} \\
	100 g & \myindex{roher Schinken}\index{Schinken>roh} \\
	50 g & echte \myindex{Mortadella} oder \myindex{Salami} \\
	1 Eßlöffel & \myindex{Butter} \\
	100 g & frisch geriebener \myindex{Parmesan}\index{Käse>Parmesan} \\
	& \myindex{Salz} \\
	& \myindex{schwarzer Pfeffer}\index{Pfeffer>schwarz} aus der Mühle \\
	& \myindex{Muskatnuß} \\
      \end{zutat}

      \begin{zubereitung}
        Huhn- und Schweinefleisch in kleine Würfel schneiden. In einer
	Kaserolle die Butter zerlassen und das Fleisch darin gut anbraten.
	Dann mit dem Schinken und der Mortadella feinhacken und durch den
	Fleischwolf drehen. Das Durchgedrehte in einer Schüssel mit dem
	Parmesankäse und dem Ei gut mischen. Mit Salz, Pfeffer und Muskat
	würzen. \\
	Die Nudelteigmasse halbieren und beide Teile möglichst dünn ausrollen.
	Eine Teighälfte mit einem leicht angefeuchteten Küchentuch oder einer
	Plastikfolie zudecken, damit sie geschmeidig bleibt. Auf die andere
	Teigplatte im Abstand von 4~cm mit Hilfe von 2~Teelöffeln haselnußgroße
	Portionen der Füllung verteilen. Die Zwischenräume mit Wasser
	bestreichen. Die zweite Teigplatte locker darüberlegen. Mit der
	Fingerkuppe um die Füllung herum andrücken und mit einem Teigrädchen
	im Schachbrettmuster durchschneiden. Voneinander trennen und auf
	bemehltem Küchentuch kurz antrocknen lassen. \\
	In einem großen Topf reichlich Salzwasser zum Kochen bringen und die
	Teigtäschchen 7--10~Minuten darin ziehen lassen. Dann mit dem
	Schaumlöffel herausheben und anrichten. \\
      \end{zubereitung}

    \mynewsection{Nudelteig Grundrezept}

      \begin{einleitung}
        Auf 100~g Mehl nimmt man 1~Ei, 1~Eßlöffel Öl und 1~Prise Salz. \\
	Pro 100~g Teigwaren bringt man 1~l Wasser und 10~g Salz zum Kochen. \\
      \end{einleitung}


    \mynewsection{Nudelteig mit Ei}\label{nudelteigmitei}

      \begin{zutaten}
      \end{zutaten}
      \begin{zutat}{für 500 g Teig}
        300 g & gesiebtes \myindex{Mehl} \\
	3 & \myindex{Ei}er (Gewichtsklasse 2) \\
	1 Eßlöffel & \myindex{Olivenöl}\index{Oel=Öl>Oliven-} \\
	& \myindex{Mehl} zum Arbeiten \\
      \end{zutat}

      \begin{zubereitung}
        In das Mehl eine Mulde drücken, Eier und Öl hineingeben und von der
	Mitte her zu Teig kneten, so lange durch die Maschine drehen, bis er
	geschmeidig ist, dabei immer wieder mit Mehl bestäuben. \\
	Teig in Klarsichtfolie wickeln. 30--40~Minuten ruhen lassen. Dann in
	4~Stücke teilen und mit der Maschine in Nudeln aufschneiden, auf
	bemehltem Tuch antrocknen lassen. \\
      \end{zubereitung}

    \mynewsection{Nudelteig ohne Ei}

      \begin{zutaten}
      \end{zutaten}
      \begin{zutat}{für 600 g Teig}
        200 g & gesiebtes \myindex{Mehl} \\
	200 g & \myindex{Hartweizengrieß}\index{Grieß>Hartweizen-} \\
	200 ml & kaltes \myindex{Wasser} \\
	& \myindex{Mehl} zum Arbeiten \\
      \end{zutat}

      \begin{zubereitung}
        Mehl und Grieß mischen, Mulde drücken, Wasser zugießen und von der
	Mitte her zu Teig kneten. Dann weiter wie unter \glqq{}Nudelteig mit
	Ei\grqq{} auf Seite \pageref{nudelteigmitei}.
      \end{zubereitung}

    \mynewsection{Schweinsrücken florentinisch II}

      \begin{zutaten}
	1 \breh{} kg & vom Knochen gelöstes \myindex{Kotelett}stück \\
	einige & \myindex{Rosmarin}blätter \\
	1 & \myindex{Knoblauchzehe} \\
	& \myindex{Salz} \\
	& \myindex{Pfeffer} \\
	4 Zweige & \myindex{Rosmarin} \\
	4 & \myindex{Kartoffel}n \\
      \end{zutaten}

      \personen{6}

      \begin{zubereitung}
        Rosmarinblätter und Knoblauch fein wiegen und mit Salz und Pfeffer
	vermischen. Fleisch damit von allen Seiten einreiben. 4~Rosmarinzweige
	der Länge nach an das Fleisch drücken und den Braten mit einem dünnen
	Bindfaden der Länge nach zusammenbinden. \\
	Braten auf einem Spieß stecken oder in eine flache Bratenpfanne geben.
	Kartoffeln schälen, in Viertel schneiden und in die Grillpfanne oder
	den Bratentopf geben, mit Salz bestreuen. \\
	Den Braten auf den rotierenden Spieß oder in der Bratenpfanne bei
	starker Mittelhitze (\grad{225}) in etwa 2~Stunden gar braten, dabei
	immer wieder mit Bratenfett begießen. \\
      \end{zubereitung}

    \mynewsection{Mandelgebäck II}

      \begin{zutaten}
      \end{zutaten}
      \begin{zutat}{für 650 g Gebäck}
        175 g & ungeschälte \myindex{Mandel}n \\
        250 g & gesiebtes \myindex{Mehl} \\
        180 g & \myindex{Zucker} \\
        1 Teelöffel & \myindex{Backpulver} \\
        2 Päckchen & \myindex{Vanillezucker}\index{Zucker>Vanille-} \\
        \breh{} kleine & Flasche \myindex{Bittermandelaroma} \\
        & \myindex{Salz} \\
        25 g & zimmerwarme \myindex{Butter} \\
        2 & \myindex{Ei}er (Gewichtsklasse 2) \\
      \end{zutat}

      \begin{zubereitung}
        Mandeln kurz in kochendes Wasser geben, in ein Sieb schütten, kalt
	abbrausen und häuten. Mandeln auf einem Tuch ausbreiten und über Nacht
	trocknen lassen. \\
	Für den Teig Mehl, Zucker, Backpulver, Vanillezucker, Bittermandelaroma
	und Salz auf die Arbeitsfläche häufeln. In die Mitte eine Mulde
	eindrücken, Butter und Eier hineingeben. Alle Zutaten mit einem Spatel
	zu einem Teig verarbeiten. Die Mandeln unterkneten. Den Teig mit etwas
	Mehl zu einer Kugel formen und 30~Minuten kalt stellen. \\
	Den Teig in 6~Teile schneiden. Aus jedem Teil eine 25~cm lange Rolle
	formen. Ein Backblech mit Backpapier auslegen. Die Rollen im Abstand
	von 8~cm voneinander darauflegen. Im vorgeheizten Backofen (\grad{200})
	10--15~Minuten vorbacken, kalt werden lassen und dann schräg in 1~cm
	dicke Scheiben schneiden. \\
	Gebäck mit einer Schnittfläche auf das Backblech legen und noch einmal
	8--10~Minuten rösten (\grad{200}). Gebäck muß zum Schluß goldbraun
	sein. Ausgekühlte Kekse in einer geschlossenen Blechdose aufbewahren.
	\\
	Dazu Wein: Vin Santo (heiliger Wein), einem schweren Dessertwein. Man
	taucht die Kekse tief in den Wein ein. \\
      \end{zubereitung}

    \mynewsection{Fischcurry}

      \begin{zutaten}
	600 g & \myindex{Rotbarsch}\index{Fisch>Rotbarsch}filet \\
	400 g & \myindex{Ananas} (1 Stück) \\
	1 & \myindex{rote Paprika}\index{Paprika>rot} \\
	1 & \myindex{Banane} \\
	1 & \myindex{Zitrone} \\
	50 g & \myindex{Butter} \\
	125 ml & \myindex{Weißwein}\index{Wein>weiß} \\
	2 Eßlöffel & \myindex{Curry} \\
	& \myindex{Salz} \\
	& \myindex{Zucker} \\
	& \myindex{Chilipfeffer} \\
	& \myindex{Butter} für die Form \\
      \end{zutaten}

      \personen{4}

      \begin{zubereitung}
        Backofen auf \grad{220} vorheizen. Rotbarschfilet in Würfel schneiden
	und in eine Schüssel geben. Zitrone auspressen, 3~Eßlöffel davon
	zum Beträufeln der Fischwürfel nehmen. \\
	Ananas schälen, Strunk entfernen und Ananas würfeln. Paprika waschen,
	entkernen und in Stücke schneiden. Banane schälen, in Scheiben
	schneiden. Paprika, Butter, Weißwein, Curry, Salz, Zucker und
	Chilipfeffer dazu geben und miteinander vermengen. \\
	Alles zusammen in eine gebutterte Auflaufform geben und für ca.
	30~Minuten im Backofen garen. \\
	Dazu Baguettebrot reichen. \\
      \end{zubereitung}

    \mynewsection{Grüner Spargel gebraten}\index{Spargel>grün gebraten}

      \begin{zutaten}
        500 g & \myindex{grüner Spargel}\index{Spargel>grün} \\
	200 g & \myindex{Cocktailtomate}\index{Tomate>Cocktail-}n \\
	150 g & \myindex{Butter} \\
	1 & \myindex{Chilischote} \\
	& \myindex{Pfeffer} aus der Mühle \\
	& \myindex{Salz} \\
        & wenig \myindex{Cayennepfeffer}\index{Pfeffer>Cayenne-} \\
	& wenig \myindex{Himbeeressig}\index{Essig>Himbeer-} \\
      \end{zutaten}

      \personen{2}

      \begin{zubereitung}
        Guten Spargel auswählen, insbesondere frisch sollte er sein. Die
	Spitzen müssen noch fest und geschlossen sein. Die Enden müssen noch
	saftig und nicht rissig oder ausgetrocknet sein. \\
	Spargelstangen einzeln abspülen. Unteres Drittel soll man großzügig
	schälen (bei sehr guter Qualität muß man es nicht unbedingt).
	Dann quer in schräge Stücke schneiden, Köpfe ganz lassen. Länge ca.
	2--3~cm.
	Chilischote entkernen und in kleine Streifen schneiden. 
	Butter zerlassen und Spargel anbraten. Immer wieder Pfanne rütteln
	bzw. wenden.
	Die Chilischote zugeben und würzen mit Salz und Pfeffer,
	Cayennepfeffer und Himbeeressig. Abschmecken und nachwürzen.
	Man kann auch halbierte Cocktailtomaten oder Zuckerschoten
	mitschwenken. \\
	Dazu: 200 g Bandnudeln; das Nudelwasser setzt man zeitgleich mit dem
	Waschen des Spargels auf. Parmesan reiben und Butter an die Nudeln
	geben.
	Man kann auch Schinken (roh oder gekocht) gesondert dazu geben
	(ausprobieren). \\
      \end{zubereitung}

    \mynewsection{Pizza}

      \begin{zutaten}
      \end{zutaten}

      \begin{zutat}{Teig}
        900 g & \myindex{Mehl} \\
	10 g & \myindex{Hefe} \\
	25 g & \myindex{Salz} \\
	\breh{} l & \myindex{Wasser} \\
      \end{zutat}

      \begin{zutat}{Belag 1}
        & \myindex{Olivenöl}\index{Oel=Öl>Oliven-} \\
	2 Eßlöffel & \myindex{Tomatensoße} \\
	& \myindex{Salami} \\
	& \myindex{Mozzarella} \\
	& \myindex{schwarze Oliven}\index{Oliven>schwarz} \\
	& geriebener \myindex{mittelalter Gouda}\index{Gouda>mittelalt} \\
	& \myindex{Pepperoni} \\
	& \myindex{Pilze} \\
      \end{zutat}

      \begin{zutat}{Belag 2 Pizza Hawaii}
        & \myindex{Olivenöl}\index{Oel=Öl>Oliven-} \\
	2 Eßlöffel & \myindex{Tomatensoße} \\
	4 Scheiben & \myindex{gekochter Schinken}\index{Schinken>gekocht} \\
	4 Scheiben & \myindex{Ananas} \\
	& geriebener \myindex{mittelalter Gouda}\index{Gouda>mittelalt} \\
      \end{zutat}

      \begin{zutat}{Belag 3 Pizza Spinachi}
        & \myindex{Olivenöl}\index{Oel=Öl>Oliven-} \\
	2 Eßlöffel & \myindex{Tomatensoße} \\
	4 Scheiben & \myindex{gekochter Schinken}\index{Schinken>gekocht} \\
	& fertig gedünsteter TK-\myindex{Spinat} mit \myindex{Knoblauch} \\
	& geriebener \myindex{mittelalter Gouda}\index{Gouda>mittelalt} \\
      \end{zutat}

      \begin{zutat}{Belag 4 Pizza Funghi Salami}
        & \myindex{Olivenöl}\index{Oel=Öl>Oliven-} \\
	2 Eßlöffel & \myindex{Tomatensoße} \\
	& \myindex{Pilze} \\
	& \myindex{Salami} \\
	& geriebener \myindex{mittelalter Gouda}\index{Gouda>mittelalt} \\
      \end{zutat}

      \begin{zutat}{Pizza Classico mit Tomatensoße und Käse}
        Pizza Magherita: & \myindex{Tomatensoße}, \myindex{Käse} \\
        Pizza Paprika: & \myindex{Paprika}, \myindex{Zwiebel}n,
	                 \myindex{Pepperonisalami}\index{Salami>Pepperoni-} \\
        Pizza Salami: & \myindex{Salami} \\
        Pizza Prosciutto: & \myindex{Schinken} \\
        Pizza Prosciutto Salami: & \myindex{Salami}, \myindex{Schinken} \\
        Pizza Funghi Salami: & \myindex{Salami},
	                       \myindex{Champignon}\index{Pilze>Champignon}s \\
        Pizza Prosciutto Funghi: & \myindex{Schinken},
	                       \myindex{Champignon}\index{Pilze>Champignon}s \\
        Pizza Quattro Stagioni: & \myindex{Salami}, \myindex{Schinken},
	                       \myindex{Champignon}\index{Pilze>Champignon}s,
				  \myindex{Paprika} \\
        Pizza Sole: & \myindex{Schinken}, \myindex{Spiegelei},
	              \myindex{Spargel} \\
        Pizza Serafino: & \myindex{Salami}, \myindex{Schinken},
	                  \myindex{Oliven},
			  \myindex{Sardellen}\index{Fisch>Sardellen},
			  \myindex{Pepperoni}, \myindex{Artischocke}n \\
        Pizza Prosciutto Gorgonzola: & \myindex{Schinken},
	                        \myindex{Gorgonzola}\index{Käse>Gorgonzola} \\
        Pizza Hawaii: & \myindex{Schinken}, \myindex{Ananas} \\
        Pizza Bolognese: & \myindex{Hackfleischsoße}, \myindex{Zwiebel}n \\
        Pizza Capricciosa: & \myindex{Schinken}, \myindex{Artischocke}n,
	                       \myindex{Champignon}\index{Pilze>Champignon}s \\
        Pizza Pazza: & \myindex{Salami},
	               \myindex{Champignon}\index{Pilze>Champignon}s,
	               \myindex{Shrimps}\index{Fisch>Shrimps},
		       \myindex{Schinken}, \myindex{Zwiebel}n,
		       \myindex{Spiegelei} \\
        Pizza Regina: & \myindex{Salami}, \myindex{Schinken},
	                \myindex{Oliven}, \myindex{Thunfisch},
			\myindex{Paprika}, \myindex{Artischocke}n \\
        Pizza Autunno: & \myindex{Champignon}\index{Pilze>Champignon}s,
	                 \myindex{Feta}\index{Käse>Feta},
	                 \myindex{Thunfisch}\index{Fisch>Thunfisch},
			 \myindex{Knoblauch} \\
        Pizza Anna: & \myindex{Schinken}, \myindex{Brokkoli},
	              \myindex{Spiegelei}, \myindex{Hollandaise} \\
        Pizza Dello Chef: & \myindex{Hackfleischsoße}, \myindex{Schinken},
	                    \myindex{Erbsen}, \myindex{Ei} \\
        Pizza Pasta: & \myindex{Spaghetti}, \myindex{Hackfleischsoße} \\
      \end{zutat}

      \begin{zutat}{Pizza di Mare mit Tomatensoße und Käse}
        Pizza Napoli: & \myindex{Sardellen}\index{Fisch>Sardellen},
	                \myindex{Oliven}, \myindex{Kapern} \\
        Pizza Tonno Cipolla: & \myindex{Thunfisch}\index{Fisch>Thunfisch},
	                       \myindex{Zwiebel}n \\
	Pizza Scampi: & \myindex{Shrimps}\index{Fisch>Shrimps},
	                \myindex{Knoblauch} \\
	Pizza Frutti di Mare: & \myindex{Meeresfrüchte},
	                        \myindex{Shrimps}\index{Fisch>Shrimps},
	                        \myindex{Knoblauch} \\
	Pizza Jesolo: & \myindex{Lachs}\index{Fisch>Lachs},
	                \myindex{Shrimps}\index{Fisch>Shrimps}, 
	                \myindex{Spinat} \\
      \end{zutat}

      \begin{zutat}{Pizza Vegetale mit Tomatensoße und Käse}
        Pizza Cipolla: & \myindex{Zwiebel}n \\
        Pizza Funghi: & \myindex{Champignon}\index{Pilze>Champignon}s \\
        Pizza Vegetale: & \myindex{Brokkoli}, \myindex{Spinat},
	                  \myindex{Oliven}, \myindex{Zwiebel}n,
			  \myindex{Champignon}\index{Pilze>Champignon}s,
			  \myindex{Paprika},
			  \myindex{Artischocke}n \\
        Pizza Italia: & \myindex{Tomate}n, \myindex{Mozzarella} \\
        Pizza Primavera: & \myindex{Tomate}n, \myindex{Brokkoli},
	                   \myindex{Mozzarella} \\
        Pizza Millenium: & \myindex{Spinat}, \myindex{Feta}\index{Käse>Feta},
	                   \myindex{Walnüsse},
			   \myindex{Champignon}\index{Champignon}s \\
      \end{zutat}

      \begin{zutat}{Pizza Piccante mit Tomatensoße und Käse}
        Pizza Diavolo: & \myindex{Pepperonisalami}\index{Salami>Pepperoni-},
	                 \myindex{Pepperoni}, \myindex{Jalapenos} \\
        Pizza Mexicano: & \myindex{Putenbrust}filet, \myindex{Mais},
	                  \myindex{Tomate}n, \myindex{Paprika},
			  \myindex{Zwiebel}n, \myindex{Jalapenos} \\
        Pizza Piccante: & \myindex{Thunfisch}\index{Fisch>Thunfisch},
	                  \myindex{Pepperoni},
			  \myindex{Pepperonisalami}\index{Salami>Pepperoni-},
			  \myindex{Oliven} \\
        Pizza Country: & \myindex{Putenbrust}filet, \myindex{Mais},
	                 \myindex{Paprika}, \myindex{Zwiebel}n \\
        Pizza Grecia: & \myindex{Schweinefleisch}, \myindex{Zwiebel}n,
	                \myindex{Spinat}, \myindex{Feta}\index{Käse>Feta},
			\myindex{Oliven} \\
        Pizza Western: & \myindex{Bacon}, \myindex{Zwiebel}n,
	                 \myindex{Mais}, \myindex{Paprika},
			 \myindex{Barbecuesoße} \\
        Pizza Maiale: & \myindex{Schweinefleisch}, \myindex{Zwiebel}n,
	                \myindex{Paprika},
			\myindex{Champignon}\index{Pilze>Champignon}s \\
        Pizza Con Carne: & \myindex{Hackfleischsoße}, \myindex{Bacon},
	                   \myindex{Schweinefleisch} \\
      \end{zutat}

      \begin{zubereitung}
        Teig: Alle Zutaten verkneten und für 6--12~Stunden in den Kühlschrank
	(auch länger, danach gehen lassen). Teile abtrennen zu mittelgroßen
	Kugeln und sanft rollen. Dann in flache runde Fladen formen und nach
	außen größer machen. \\
	Belag: Wenn der Teig fertig ausgerollt ist, etwas Olivenöl draufgeben
	und mit einem Eßlöffel verteilen. Ca. 2~Eßlöffel Tomatensoße mit dem
	Eßlöffel verteilen, darauf den Belag (Käse, Wurst, Fisch etc.). \\
	Im Pizzaofen bei \grad{600} 1\breh{}~Minuten garen, öfter drehen.
	Im Backofen bei \grad{250} etwa 8--10~Minuten backen. \\
      \end{zubereitung}

    \mynewsection{Staudensellerie gebraten}

      \begin{zutaten}
        1 & größerer \myindex{Staudensellerie}\index{Sellerie>Stauden-} \\
	4 & mittlere \myindex{Knoblauchzehe}n \\
	1 Teelöffel & \myindex{Tomatenmark} \\
	4--5 & \myindex{Tomate}n (zerkleinert) \\
	4--6 & große Blätter \myindex{Basilikum} \\
	& \myindex{Salz} \\
	& \myindex{Pfeffer} aus der Mühle \\
	\breh{} Teelöffel & \myindex{Zucker} \\
        & \myindex{Olivenöl}\index{Oel=Öl>Oliven-} zum Braten \\
        & dunkler \myindex{Balsamico-Essig}\index{Essig>Balsamico-} \\
      \end{zutaten}

      \begin{zubereitung}
        Staudensellerie zerlegen, waschen und putzen. Dabei auf beiden
	Seiten die Fäden abziehen. Danach in schmale Ringe schneiden.
	Knoblauch in kleine Stücke schneiden, das Basilikum in Streifen.
	Öl erhitzen, Sellerie anbraten. Knoblauch ebenso. Wenn beides Farbe
	genommen hat und bräunt, das Basilikum anbraten. Das Tomatenmark
	anbraten, dann pfeffern und salzen. Tomaten und eventuell vorhandene
	Tomatenflüssigkeit (zum Beispiel aus ausgehöhlten Tomaten)
	hinzufügen. Weiterbraten. Ablöschen mit Balsamico-Essig. Abschmecken.
	Abkühlen lassen. Durchgezogen oder kalt als Vorspeise oder Beilage. \\
      \end{zubereitung}

    \mynewsection{Kartoffel-Lauch-Küchlein}

      \begin{zutaten}
        200 g & vorwiegend festkochende \myindex{Kartoffel}n \\
	150 g & \myindex{Lauch} \\
	2 & \myindex{Ei}er \\
	100 g & \myindex{Mehl} \\
	& \myindex{Salz} \\
	& \myindex{Pfeffer} \\
	& \myindex{Muskatnuß} \\
	& \myindex{Butterschmalz}\index{Schmalz>Butter-} \\
      \end{zutaten}

      \begin{zubereitung}
        Kartoffeln schälen, reiben und kräftig auspressen (in einem sauberen
	Tuch). Lauch in feine Ringe schneiden und in Butterschmalz dünsten.
	Etwas salzen. Zu den Kartoffeln geben, Eier, Mehl und Gewürze
	dazugeben, vermischen. Kleine Küchlein in die Pfanne mit Butterschmalz
	geben und goldgelb bei kleiner bis mittlerer Hitze ca. 4~Minuten
	backen (sonst bleiben die Kartoffeln roh). \\
      \end{zubereitung}

    \mynewsection{Flammkuchen mit Lauch}

      \begin{zutaten}
        1 Packung & TK-\myindex{Blätterteig}\index{Teig>Blätter-} \\
	2 Stangen & \myindex{Lauch} \\
	1 Becher & \myindex{\cremefraiche{}} (200 g) \\
	3--4 & getrocknete \myindex{Tomate}n (aus dem Glas) \\
	1 Paket & \myindex{Feta}\index{Käse>Feta} (200 g) \\
        & \myindex{Olivenöl}\index{Oel=Öl>Oliven-} zum Braten \\
	ca. 50 g & \myindex{Emmentaler}\index{Käse>Emmentaler} \\
	& \myindex{Salz} \\
	& \myindex{Pfeffer} \\
      \end{zutaten}

      \begin{zubereitung}
        Blätterteig samt Papier auf ein einfaches Backblech legen und mehrfach
	mit der Gabel einstechen. Lauch in dünne Ringe schneiden, in Olivenöl
	andünsten. \cremefraiche{} würzen mit Salz + Pfeffer, dann auf den Teig
	aufstreichen. Darauf den Lauch verteilen. Tomaten in Streifen darüber.
	Feta zerkleinern und dann oben drauf tun. Emmentaler reiben und
	obenauf geben. Bei \grad{250} 20--25~Minuten auf unterster Schiene
	backen. \\
      \end{zubereitung}

    \mynewsection{Sauerbraten}

      \begin{zutaten}
        1200 g & \myindex{falsches Filet}\index{Filet>falsch} vom Rind
	         (Halsstück) \\
        & \myindex{Pflanzenöl}\index{Oel=Öl>Pflanzen-} zum Braten \\
	1 l & \myindex{Wasser} \\
	100 g & \myindex{Zucker} \\
	300 ml & \myindex{Rotwein}\index{Wein>rot} (nicht den sauersten>) \\
	200 ml & \myindex{Rotweinessig}\index{Essig>Rotwein-} \\
	5 & \myindex{Orange}n gepreßt zu Saft \\
	1 & \myindex{Zimtstange} \\
	50 g & Gewürze (\myindex{Zimt}, \myindex{Nelke}, \myindex{Anis},
	                \myindex{Muskatnuß}, \myindex{Ingwer},
			\myindex{Pfeffer}, \myindex{Sternanis},
			\myindex{Piment}, \myindex{Kardamon}) \\
      \end{zutaten}

      \begin{zubereitung}
        Fleisch in einer Pfanne von allen Seiten scharf anbraten. In eine
	Schüssel umfüllen. Gewürze in einem Mörser sehr fein mahlen. \\
	Alle Zutaten in einem Topf aufkochen und heiß über den Braten gießen.
	Den Braten in diesem Fond 3--5~Tage kalt stellen (Kühlschrank),
	zwischendurch wenden. \\
      \end{zubereitung}

      \begin{zutaten}
        50 g & \myindex{Gemüsezwiebel}\index{Zwiebel>Gemüse-} fein gewürfelt \\
	50 g & \myindex{Möhre} fein gewürfelt \\
	50 g & \myindex{Sellerie} fein gewürfelt \\
	50 g & \myindex{Lauch} fein gewürfelt \\
	& \myindex{Pflanzenöl}\index{Oel=Öl>Pflanzen-} zum Braten \\
	30 g & \myindex{Tomatenmark} \\
	100 ml & \myindex{Malzbier} \\
	50 g & \myindex{Rübenkraut} \\
	100 g & \myindex{Rosinen} \\
	& \myindex{Salz} \\
      \end{zutaten}

      \begin{zubereitung}
        Gemüsezwiebeln, Möhre, Sellerie und Lauch in Öl anrösten, Tomatenmark
	dazu, abbrennen und mit dem Malzbier und dem Sauerbratenfond auffüllen.
	Aufkochen lassen und Fleisch reingeben, leicht köcheln lassen. Öfter
	wenden. \\
	Braten ist gar, wenn er beim Einstechen mit der Gabel eine weiche
	Konsistenz hat. Braten herausnehmen und in dünne Scheiben schneiden. \\
	Bratensoße durchpassieren, eventuell einkochen lassen und mit
	Rübenkraut, Salz und Rosinen abschmecken. Bratenscheiben zurück in die
	Soße geben und bis zum Anrichten ziehen lassen. \\
      \end{zubereitung}

    \mynewsection{Zanderfilets aus dem Ofen}

    % ARD 14.05.2010

      \begin{zutaten}
        & \myindex{Olivenöl}\index{Oel=Öl>Oliven-} für die Form \\
	600 g & \myindex{Zander}\index{Fisch>Zander}filets \\
	& \myindex{Salz} \\
	& schwarzer \myindex{Pfeffer} aus der Mühle \\
	400 g & \myindex{Cherrytomaten}\index{Tomate>Cherry-} (sehr kleine
	        Tomaten) \\
        2 & \myindex{Frühlingszwiebel}\index{Zwiebel>Frühlings-}n \\
	2 & \myindex{Knoblauch}zehen \\
	\breh{} Bund & \myindex{Basilikum} \\
	\breh{} Bund & glattblättrige \myindex{Petersilie} \\
	1 Zweig & \myindex{Rosmarin} \\
	1 & \myindex{Peperoncino} (\myindex{Pfefferschote} oder
	    \myindex{Chilischote}) \\
        6 Eßlöffel & \myindex{Olivenöl}\index{Oel=Öl>Oliven-} \\
      \end{zutaten}

      \personen{4}

      \begin{zubereitung}
        Backofen auf \grad{200} vorheizen. Form mit dem Olivenöl ausstreichen.
	Zanderfilets beidseitig mit Salz und Pfeffer würzen und
	nebeneinander in die Form legen.
	Cherrytomaten halbieren, großzügig salzen und pfeffern.
	Frühlingszwiebeln samt Grün in dünne Ringe schneiden, Knoblauch in
	feine Scheiben schneiden.
	Kräuter fein hacken. Den Peperoncino längs halbieren, entkernen und
	ebenfalls hacken.
	Kräuter, Tomaten und Peperoncino über den Fisch verteilen
	(bis hierher kann das Gericht vorbereitet werden).
	Unmittelbar vor dem Einschieben in den Ofen den Fisch und die
	Gemüsezutaten mit dem Olivenöl beträufeln.
	Auf der mittleren Schiene 20~Minuten bei \grad{200} backen. \\
	Dazu reicht man Knoblauchbaguette und grüner oder gemischter Salat. \\
      \end{zubereitung}

    \mynewsection{Frankfurter grüne Soße}

      \begin{zutaten}
        1 großes Bund & gemischte \myindex{Kräuter} (z.B.
	                \myindex{Petersilie}, \myindex{Schnittlauch},
			\myindex{Kerbel}, \myindex{Kresse},
			\myindex{Pimpinelle}, \myindex{Sauerampfer},
			\myindex{Borretsch}) \\
        5 & hart gekochte \myindex{Ei}er \\
	1 & \myindex{Ei}gelb \\
	1 Eßlöffel & mittelscharfer \myindex{Senf} \\
	5 Eßlöffel & \myindex{Olivenöl}\index{Oel=Öl>Oliven-} \\
	1 Eßlöffel & \myindex{Essig} \\
	125 g & \myindex{saure Sahne}\index{Sahne>sauer} \\
	& frisch gemahlenes \myindex{Salz} \\
	& frisch gemahlener \myindex{Pfeffer} \\
      \end{zutaten}

      \begin{zubereitung}
        Kräuter waschen und trockenschütteln. Blättchen abzupfen und sehr fein
	hacken. Ein hart gekochtes Ei pellen, halbieren, das Eigelb
	herauslösen und durch ein Sieb in eine Schüssel streichen. Mit dem
	frischen Eigelb und dem Senf verrühren. \\
	Mit dem Schneebesen das Olivenöl und den Essig tropfenweise unter die
	Eigelbmasse schlagen, bis eine mayonnaiseartige Soße entsteht. Die
	gehackten Kräuter und die Sahne untermischen. Mit Salz und Pfeffer
	abschmecken. Das übrige Eiweiß fein hacken und unterrühren. \\
	Die anderen harten Eier pellen und längs halbieren, auf kleinen Tellern
	anrichten und mit der kalten grünen Soße übergießen. Mit Bauernbrot
	oder heißen Pellkartoffeln servieren. \\
	Dazu paßt ein herber Weißwein, beispielsweise ein Riesling von der
	Bergstraße. \\
      \end{zubereitung}

    \mynewsection{Lachsforelle in der Zeitung}

      % WDR daheim + unterwegs 27.05.2010

      \begin{zutaten}
        1 & ganze \myindex{Lachsforelle}\index{Forelle>Lachs-}
	    \index{Fisch>Forelle>Lachs-} von etwa 1,2~kg, geschuppt und
	    ausgenommen \\
        1 Bund & glatte \myindex{Petersilie}, grob gehackt \\
        1 Bund & \myindex{Basilikum}, grob gehackt \\
        1 Bund & \myindex{Rosmarin}, grob gehackt \\
        1 Bund & \myindex{Thymian}, grob gehackt \\
	1 & \myindex{Zitrone} in Scheiben (etwa 8~Stück) \\
	4 Zehen & \myindex{Knoblauch} in Scheiben \\
	& \myindex{Meersalz}\index{Salz>Meer-} \\
	& \myindex{Pfeffer} aus der Mühle \\
	5 Lagen & große \myindex{Tageszeitung} \\
	& \myindex{Paketband} \\
      \end{zutaten}

      \personen{6--8}

      \begin{zubereitung}
        1 großes Blech (Backblech) als Hilfsmittel nehmen. Die Zeitung in
	5~Lagen aufeinander ausbreiten. Immer wieder mit Wasser leicht
	anfeuchten. Lachsforelle im Eimer oder in der Schüssel sorgfältig
	abspülen. Die Hälfte der Kräuter auf der Mitte der Zeitung geben.
	Darauf die Hälfte der Zitronenscheiben legen. \\
	Die Lachsforelle innen salzen und pfeffern, Knoblauchscheiben hinein,
	außen salzen und pfeffern und auf die Kräuter legen. Dann die andere
	Hälfte der Kräuter und Zitronenscheiben auf den Fisch. \\
	Den Fisch eng in die Zeitung einpacken und wie ein Paket ordentlich
	zuschnüren. Das Paket im Eimer oder in der Schüssel für 20~Minuten
	in Meerwasser (10~Gramm Salz pro Liter) legen, bis es ordentlich
	durchweicht ist, zwischendurch umdrehen. \\
	Der Grill muß schon sehr heiß sein! 30~Minuten auf dem Rost grillen
	wie eine Bratwurst, sobald das Paket dunkel wird und fast brennt,
	wenden. Nicht brennen lassen!! \\
	Zum Servieren schneidet man das Paket auf, klappt die Kräuter weg,
	zieht die Fischhaut auf die Seite, hebt die Gräten ab und verteilt
	die Filets. \\
      \end{zubereitung}

    \mynewsection{Dill-Senf-Soße zum Grillfisch}

      % WDR 27.05.2010

      \begin{zutaten}
        1 Bund & \myindex{Dill} \\
	1 Becher & \myindex{Schmand} \\
	1 Teelöffel & scharfer \myindex{Senf} \\
	4 Eßlöffel & \myindex{Sonnenblumenöl}\index{Oel=Öl>Sonnenblumen-} \\
	1 Teelöffel & \myindex{Zucker} \\
	& \myindex{Salz} \\
	& \myindex{Pfeffer} \\
      \end{zutaten}

      \begin{zubereitung}
        Von den dicken Stengeln des Dills die Spitzen abzupfen und fein hacken.
	Senf, Schmand, Öl, Zucker, Salz und Pfeffer sorgfältig rühren, Dill
	einrühren und abschmecken. \\
	Dazu: grüner Salat, gemischter Salat, eventuell Tomatensoße. \\
      \end{zubereitung}

    \mynewsection{gebratener Reste-Auflauf}

      % gekocht 12.07.2010

      \begin{zutaten}
        4--5 mittlere & gekochte \myindex{Pellkartoffel}n \\
        1 große & \myindex{Zucchini} (oder 2 kleinere) \\
	150--200 g & \myindex{Champignon}s \\
	2 & \myindex{Zwiebel}n \\
	etwas & \myindex{Mehl} \\
	3--4 & \myindex{Knoblauchzehe}n \\
	1 Becher & \myindex{Schmand} (200 g) \\
	& \myindex{Salz} \\
	& \myindex{Pfeffer} aus der Mühle \\
        & \myindex{Cayennepfeffer}\index{Pfeffer>Cayenne-} \\
	& \myindex{Paprika edelsüß} \\
	& \myindex{Olivenöl}\index{Oel=Öl>Oliven-} zum Braten der Zucchini \\
	& \myindex{neutrales Öl}\index{Oel=Öl>neutral} zum Braten der Pilze und
	  der Kartoffeln \\
        wenig & geriebener \myindex{Parmesan}\index{Käse>Parmesan} \\
      \end{zutaten}

      \begin{zubereitung}
        Die Auflaufform mit Olivenöl einfetten, den Rest in die Bratpfanne
	gießen. Kartoffeln pellen und in Scheiben schneiden, eventuell leicht
	mehlen. Zucchini hobeln, Champignons putzen und in Scheiben schneiden.
	Knoblauch pellen, Zwiebeln einzeln in Würfeln schneiden (eine für die
	Champignons, eine für die Kartoffeln). Alles nacheinander braten. \\
	Schmand würzen mit Salz, Pfeffer, Cayennepfeffer und Paprika,
	abschmecken. \\
	1. Lage: Zucchini goldgelb angebraten mit gepreßtem Knoblauch, Salz,
	Pfeffer, dünne Schicht Parmesan. \\
	2. Lage: Champignons goldgelb gebraten mit Zwiebeln, Salz, Pfeffer,
	Schmand. \\
	3. Lage: gebratene Kartoffeln, Salz, Pfeffer. \\
	Die Auflaufform bei \grad{200} für 30~Minuten in den Backofen. \\
      \end{zubereitung}

    \mynewsection{Bunte Paprika geschmort (italienisch)}

      % 20.07.2010 TV

      \begin{zutaten}
        10--12 & \myindex{Paprika}schoten gelb, orange, ... \\
        10--12 & ganze größere \myindex{Knoblauchzehe}n \\
        reichlich & \myindex{Olivenöl}\index{Oel=Öl>Oliven-} \\
        & \myindex{Balsamico-Essig} \\
        & \myindex{Salz} \\
      \end{zutaten}

      \personen{8--10}

      \begin{zubereitung}
        Paprika waschen, putzen und in Streifen oder Stücke schneiden. Gleich
	in eine Bratform geben (oder Pfanne/Bräter). Darauf die geschälten
	Knoblauchzehen geben. Reichlich Olivenöl darüber und braten, bis der
	Knoblauch leicht gebräunt ist, genauso wie die Paprika. \\
	Dann einen ordentlichen Schuß Balsamico-Essig darüber und mit wenig
	Salz würzen (wenig deshalb, weil die Süße der Paprika erhalten bleiben
	soll). In 10--15~Minuten fertig. Servieren, auch lauwarm. \\
	Dazu Brot/Ciabatta oder zu Fisch/Fleisch oder als Vorspeise. \\
      \end{zubereitung}

    \mynewsection{Muskatkürbis-Rahmsoße}

      % WDR daheim & unterwegs 28.09.2010

      \begin{zutaten}
        300 g & \myindex{Kürbis} (Muskatkürbis) gewürfelt \\
	5 & \myindex{Schalotte}n gewürfelt \\
	30 ml & \myindex{Kürbisöl}\index{Oel=Öl>Kürbis-} \\
	etwas & \myindex{Curry}pulver \\
	etwas & \myindex{Zimt} \\
	etwas & \myindex{Muskatnuß} \\
	100 ml & \myindex{Apfelsaft} \\
	200 ml & \myindex{Geflügelbrühe} \\
	100 ml & \myindex{Sahne} \\
	50 g & \myindex{saure Sahne}\index{Sahne>sauer} \\
	& \myindex{Salz} \\
      \end{zutaten}

      \begin{zubereitung}
        In einem Topf Kürbiskernöl erhitzen. Die Schalottenwürfel und die
	Kürbiswürfel dazugeben und etwas anbraten, öfter anwenden. Curry,
	Zimt, Muskat daraufgeben. Ablösen mit Apfelsaft, leicht einkochen.
	Brühe und Sahne dazu. Bei kleiner Hitze einkochen lassen. Soße pürieren
	und durch ein Sieb passieren. Mit Salz und saurer Sahne abschmecken. \\
	Verdünnt auch als Suppe verwendbar. \\
      \end{zubereitung}

    \mynewsection{Kürbisrösti}

      % WDR daheim & unterwegs 28.09.2010

      \begin{zutaten}
        300 g & \myindex{Kartoffel}n, geschält und grob geraffelt (reiben) \\
	100 g & \myindex{Kürbis} (Muskatkürbis) gerieben (fein) \\
	\breh{} & \myindex{Apfel}, geschält und gerieben (fein) \\
	1 & \myindex{Schalotte} fein gewürfelt \\
	50 g & grobe \myindex{Haferflocken} \\
	3 & \myindex{Ei}er \\
	& \myindex{Salz} \\
	& \myindex{Pfeffer} \\
	& \myindex{Muskatnuß} \\
	& \myindex{Traubenkernöl}\index{Oel=Öl>Traubenkern-} \\
      \end{zutaten}

      \begin{zubereitung}
        Alle Zutaten in einer Schüssel gut vermischen. Mit Salz, Pfeffer und
	Muskat abschmecken. \\
	Öl in einer Pfanne erhitzen. Mit einem Löffel kleine Rösti in die
	Pfanne setzen und ausbraten. \\
      \end{zubereitung}

    \mynewsection{Schnitzel von Muskatkürbis}

      % WDR daheim & unterwegs 28.09.2010

      \begin{zutaten}
        4--8 & \myindex{Muskatkürbis}scheiben flach geschnitten, rechteckig \\
	etwas & \myindex{Mehl} \\
	2 & \myindex{Ei}gelb \\
	etwas & \myindex{Sahne} \\
	200 g & \myindex{Weißbrot}\index{Brot>Weiß-}, fein gerieben
	        (Blitzhacker) \\
        20 g & \myindex{Kürbiskernmehl}\index{Mehl>Kürbiskern-} (Kerne im
	       Hacker zerkleinern) \\
	200 ml & \myindex{Kürbisöl}\index{Oel=Öl>Kürbis-} hell \\
	& \myindex{Salz} \\
	& \myindex{Pfeffer} \\
      \end{zutaten}

      \begin{zubereitung}
        Sahne und Eigelb vermischen, mit Salz und Pfeffer würzen. Weißbrot
	mit Kürbiskernmehl mischen. Kürbisscheiben in Mehl wenden, durch das
	Ei ziehen und in der Panade wenden. In einer Pfanne das Öl erhitzen
	und die Schnitzel darin 5--6~Minuten braten. \\
	Es reicht, wenn die Panade Farbe genommen hat und kroß ist. \\
      \end{zubereitung}

    \mynewsection{\chicoree{}-Champignon-Gemüse mit Lachs und legiertem
                  Kartoffelpüree}


    % 17.03.2011 BR Andreas Geitl
    % gekocht 27.03.2011

      \begin{zutaten}
      \end{zutaten}

      \begin{zutat}{Gemüse}
        125 g & \myindex{\chicoree{}}, in breite Streifen geschnitten \\
	100 g & \myindex{Champignon}s, blättrig geschnitten \\
	25 g & getrocknete \myindex{Aprikosen}, klein gewürfelt \\
	25 g & \myindex{Zwiebel}n, gewürfelt \\
	10 g & \myindex{Butter} \\
	60 ml & \myindex{\cremefraiche{}} \\
	60 ml & \myindex{Sahne} \\
	& \myindex{Zucker} \\
	& \myindex{Zitrone}nsaft \\
	& Prise \myindex{Chili} \\
	& Hauch \myindex{Zimt} \\
	& \myindex{Gemüsebrühe} \\
	& \myindex{Salz} \\
	& \myindex{Pfeffer} \\
      \end{zutat}

      \begin{zutat}{Kartoffelpüree}
        300 g & \myindex{Kartoffel}n, geschält \\
	75 ml & \myindex{Sahne}, sehr warm -- nicht heiß \\
	25 g & \myindex{Butter} \\
	2 & \myindex{Ei}gelb (kleine) \\
	& \myindex{Salz} \\
	& \myindex{Muskatnuß}, frisch gerieben \\
      \end{zutat}


      \begin{zutat}{Lachs}
        240 g & \myindex{Lachs}, teilen in 3~Stücke \'a 80 g \\
	15 g & \myindex{Butter} \\
	& \myindex{Salz} \\
	& \myindex{Pfeffer} aus der Mühle \\
	& \myindex{Zitrone}nsaft (zum Straffen des Bindegewebes vom Fisch) \\
      \end{zutat}

      \begin{zutat}{Garnitur}
	25 g & \myindex{Pinienkerne}, geröstet \\
	& frische \myindex{Kräuter} \\
	& \myindex{Melisse}nspitzen als obere Dekoration \\
      \end{zutat}

      \personen{2-3} 

      \begin{zubereitung}
        Gemüse: \chicoree{} putzen, Kern ausschneiden, in 3~cm Stücke schneiden.
	Zwiebel in Butter andünsten, \chicoree{}, Aprikosen und Pilze dazu.
	Andünsten. Dann auffüllen mit \cremefraiche{}, Gemüsebrühe und Sahne,
	ständig rühren, etwas einkochen lassen. Abschmecken mit Salz, Pfeffer,
	Zucker, Zitronensaft. Rühren. Kurz vor dem Servieren einen Hauch Zimt
	und eine kleine Prise Chili darübergeben, rühren. \\
	Kartoffelpüree: Kartoffeln kochen, durch die Presse drücken, Eigelb
	dazu und warme Sahne, Salz und Muskatnuß, Butter nicht vergessen.
	Abschmecken. \\
	Lachs: Lachs in 80~g Portionen aufteilen. Würzen mit Salz, Pfeffer,
	Zitronensaft von beiden Seiten. In Butter braten, jede Seite
	3--4~Minuten. In der Pfanne nachziehen lassen. Lachs soll innen sehr
	glasig bleiben, dann schmeckt er am besten. \\
	Anrichten: Rahmiges Gemüse, 1--2~Eßlöffel Kartoffelpüree, Lachs drauf,
	Melisse, umstreuen mit Pinienkernen, Kräutern. \\
      \end{zubereitung}

    \mynewsection{Flammkuchen mit Pancetta, roten Zwiebeln und Kümmel}

      \begin{zutaten}
      \end{zutaten}

      \begin{zutat}{Teig}
        200 g & \myindex{Mehl} \\
	30 g & \myindex{Hefe} \\
	60 g & \myindex{Wasser} \\
	60 g & \myindex{Milch} \\
	1 Messerspitze & \myindex{Schmalz} \\
	1 Eßlöffel & \myindex{Olivenöl}\index{Oel=Öl>Oliven-}, kaltgepreßt \\
	1 Prise & \myindex{Salz} \\
	1 Prise & \myindex{Zucker} \\
	& \myindex{Mehl} zum Ausrollen \\
      \end{zutat}

      \begin{zutat}{Belag}
        80 g & \myindex{Räucheraal}\index{Fisch>Räucheraal}filet \\
	125 g & \myindex{\cremefraiche{}} \\
	& \myindex{Kümmel} \\
	1 & \myindex{Ei}gelb \\
	1 & \myindex{rote Zwiebel}\index{Zwiebel>rot} \\
	1 & \myindex{Lauchzwiebel}\index{Zwiebel>Lauch-} \\
	80 g & \myindex{Pancetta} (toskanischer Speck) \\
	& \myindex{Salz} \\
	& \myindex{Pfeffer} aus der Mühle \\
      \end{zutat}

      \begin{zubereitung}
        Die Zutaten für den Teig in eine Küchenmaschine geben und zu einem
	glatten Teig verarbeiten. Zugedeckt 30--40~Minuten gehen lassen, bis
	sich das Teigvolumen verdoppelt hat. Aus dem Teig vier gleich große
	Kugeln formen und nochmals 10~Minuten gehen lassen. Den Backstein auf
	\grad{300} vorheizen. Nebenbei die rote Zwiebel und die Lauchzwiebel
	in feine Scheiben sowie den Pancetta in feine Streifen schneiden. \\
	Die Teigkugeln mit Mehl flach ausrollen und jeweils auf eine mit Mehl
	bestreute Holzschaufel geben. Die \cremefraiche{} mit Eigelb
	verrühren und mit Salz und Pfeffer abschmecken, dann jeden Teigfladen
	damit dünn bestreichen. \\
	Die Aalstückchen, Zwiebelringe und die Schinken-Streifen auf die Fladen
	verteilen und mit Kümmel bestreuen. Die Fladen sofort auf den heißen
	Stein geben, damit der Teigboden nicht zu feucht wird. Den Flammkuchen
	etwa 5--6~Minuten auf dem heißen Stein backen und sofort servieren. \\
      \end{zubereitung}

    \mynewsection{Bandnudeln mit Rosenkohl und Walnußpesto}

      % Björn Freitag WDR 21.11.2011

      \begin{zutaten}
        10 Stück & \myindex{Rosenkohl}\index{Kohl>Rosen-} \\
	& kalt gepreßtes \myindex{Olivenöl}\index{Oel=Öl>Oliven-} \\
	& \myindex{Salz} \\
	& \myindex{Pfeffer} \\
	2 & \myindex{Schalotte}n \\
	100 g & \myindex{Walnüsse} (ohne Schale) \\
	3 Zweige & \myindex{Basilikum} (ca. 12--15~Blätter) \\
	2 Zweige & \myindex{Petersilie} \\
	80 g & \myindex{Parmesan} \\
	ca. 350 g & frische \myindex{Bandnudeln}\index{Nudel>Band-} \\
      \end{zutaten}

      \personen{2}

      \begin{zubereitung}
        Rosenkohl putzen und vierteln. In einer Pfanne mit etwas Olivenöl
	anbraten, Salz und Pfeffer dazu. Schalotten würfeln und zum Rosenkohl
	geben, weiterbraten bei geringer Hitze. \\
	In einer Pfanne ohne Fett Walnüsse anrösten. \\
	Topf mit reichlich Wasser aufsetzen, ins kochende Wasser 2--3~Teelöffel
	Salz geben. Nudeln garen (frische brauchen 2~Minuten). \\
	Geröstete Walnüsse im Mörser zerstoßen. Basilikum- und
	Petersilienblätter grob hacken, mit etwas Salz und Walnüssen im Mörser
	zu Brei stampfen, geriebenen Parmesan und 3~Eßlöffel Olivenöl unter
	das Pesto mischen. \\
	Nudeln abgießen und zum Rosenkohl geben. Anrichten, Pesto mit Salz und
	Pfeffer abschmecken und auf den Nudeln verteilen. Sofort servieren. \\
      \end{zubereitung}

    \mynewsection{Erbspüree mit Minze}

      % Jamie Oliver 2012

      \begin{zutaten}
        1 großer Bund & \myindex{Frühlingszwiebel}\index{Zwiebel>Frühlings-}n
	                in Ringe \\
	2 Eßlöffel & \myindex{Olivenöl}\index{Öl>Oliven-} \\
	1 Bund & \myindex{Pfefferminze} \\
	500 g & TK-\myindex{Erbsen}, jung und grün \\
	50 g & \myindex{Butter} \\
      \end{zutaten}

      \begin{zubereitung}
        Frühlingszwiebeln mit Olivenöl im Topf andünsten für 2 Minuten. Minze
	hacken, in den Topf geben, andünsten. Erbsen rein. Deckel drauf,
	12~Minuten garen. Butter rein. Alles in der Küchenmaschine pürieren. In
	eine Schüssel geben und mit 3--4~Minzeblättern garnieren und einen
	Klacks Butter zugeben. \\
      \end{zubereitung}

    \mynewsection{Bärlauch-Pesto}\label{baerlauchpesto}

      \begin{zutaten}
        1 großer Bund & \myindex{Bärlauch} \\
	50 g & \myindex{Parmesan} \\
	50 g & \myindex{Mandeln} \\
	3--4 Eßlöffel & \myindex{Olivenöl} \\
	etwas & \myindex{Salz} \\
      \end{zutaten}

      \begin{zubereitung}
        Alles mixen. \\
      \end{zubereitung}

    \mynewsection{gegrillte Zanderfilets mit Bärlauchpesto}

      \begin{zutaten}
        & \myindex{Bärlauchpesto} (siehe Seite \pageref{baerlauchpesto}) \\
	4 & \myindex{Zander}\index{Fisch>Zander}filets \\
      \end{zutaten}

      \begin{zubereitung}
        Auf den vorgeheizten Grill die Zanderfilets mit aufgestrichenem Pesto
	ca. 10~Minuten garen. \\
      \end{zubereitung}

    \mynewsection{Tofu}

      % 30.10.2012

      \begin{zutaten}
        & \myindex{Tofu} \\
	& \myindex{Worcestershiresoße} \\
	& \myindex{Sojasoße} \\
      \end{zutaten}

      \begin{zubereitung}
	Tofu in kleine Würfel schneiden. Worcestershiresoße und Sojasoße dazu
	und in der Pfanne anbraten.
      \end{zubereitung}

    \mynewsection{Espresso-Kaffee}

      % 30.10.2012

      \begin{zubereitung}
	615 g ergeben die zwei oberen Teile mit Wasser. Dazu kommt 10--12 g
	Kaffeepulver. Alle Teile zusammenschrauben und erhitzen auf der
	Herdplatte.
      \end{zubereitung}

    \mynewsection{Ofenkartoffeln mit Lorbeer und Bacon}

      % 09.12.2012

      \begin{zutaten}
        8 & mittelgroße \myindex{Kartoffel}n (vorwiegend festkochend),
	    gesäubert und ohne Keime \\
	& \myindex{Olivenöl}\index{Oel=Öl>Oliven-} \\
	& \myindex{Salz} \\
	& \myindex{Pfeffer} \\
	1 Pack & \myindex{Bacon} 100 g, 14 Scheiben \\
	16 & \myindex{Lorbeer}blätter \\
	8--10 & \myindex{Knoblauchzehe}n, teils halbiert \\
	& \myindex{\cremefraiche{}} \\
      \end{zutaten}

      \begin{zubereitung}
	Kartoffeln angaren, 5--7~Minuten kochen, abkühlen lassen. Halbieren.
	Oben einen Keil rausschneiden, dort Knoblauch und Lorbeerblatt
	einstecken.  Reine oder Ofenblech mit Öl versehen, darauf ordentlich
	Salz und Peffer.  Bacon auslegen, darauf Kartoffelhälften. Nochmals
	würzen und ölen. Im vorgeheizten Backofen (\grad{190}) ca.
	20--25~Minuten garen. \\
	Nächstes Mal: Kartoffelschnittflächen gesondert vorwürzen. Schnitte
	tiefer. Kleinere Kartoffeln. \\
	Dazu gebratener \chicoree{} (siehe Seite \pageref{chicoreegebraten})
	und Gurkensalat. \\
      \end{zubereitung}

    \mynewsection{Birnen, Bohnen und Speck}
    % 14.09.14 NDR "Mein Schönes Land"

      \begin{zutaten}
        375 g & \myindex{Speck} in \breh{}~cm dicke Scheiben \\
	750 g & frische \myindex{grüne Bohnen}\index{Bohnen>grün} in 4~cm
	        Stücke brechen \\
	2 & \myindex{Zwiebel}n, gewürfelt \\
	2 Zweiglein & \myindex{Bohnenkraut} \\
	500 g & festkochende \myindex{Kartoffel}n, am Stück, nur geschält \\
	500 g & \myindex{Birne}n, Stiel und Blütenansatz entfernen \\
	2 Eßlöffel & körnige \myindex{Brühe} \\
	700 ml & \myindex{Wasser} \\
	50 ml & \myindex{Weißweinessig}\index{Essig>Weißwein-} \\
	& \myindex{Salz} \\
	& \myindex{Pfeffer} \\
	2 Eßlöffel & \myindex{Mehl} zum Andicken
	             (mit kaltem Wasser verrühren) \\
      \end{zutaten}

      \personen{4}

      \begin{zubereitung}
        Wasser und Brühe in einen Topf geben, aufkochen lassen, den Speck
	zugeben, etwa 15~Minuten kochen lassen, aus dem Topf nehmen.
	Die Bohnen, Zwiebeln und Weißweinessig in den Topf geben, 15~Minuten
	kochen lassen. Geschälte, ganze Kartoffeln und geputzte Birnen mit
	Schale in den Eintopf geben. \\
	Nach Belieben salzen und pfeffern, Bohnenkraut dazu. Etwa 30~Minuten
	bei mittlerer Hitze kochen lassen, dann mit angerührtem Mehl andicken.
	\\
	Den Speck in kleinere Stücke schneiden, dazugeben und nochmal kurz
	aufkochen lassen. Eventuell nochmal nachwürzen mit Salz, Pfeffer. \\
	Heiß anrichten. \\
      \end{zubereitung}

    \mynewsection{Wiener Heringssalat}
      % aus Wiener Schmankerln

      \begin{zutat}{Grundzutaten (muss)}
        4 & \myindex{Hering}filets (Salzhering), eventuell \myindex{Matjes},
	                                         gewürfelt \\
        3 mittlere & \myindex{Kartoffel}n, geschält, gewürfelt, gegart \\
        3--4 Eßlöffel & kleine \myindex{weiße Bohnen}\index{Bohnen>weiß},
	                gekocht \\
        3--4 Eßlöffel & \myindex{Linsen}, gekocht \\
        3 & \myindex{Möhren}, geschält, gewürfelt, in Öl gedünstet und
	                     gewürzt \\
        1 & \myindex{weiße Zwiebel}\index{Zwiebel>weiß}, gewürfelt \\
        1 & \myindex{rote Zwiebel}\index{Zwiebel>rot}, in Segmente, teils
	                                               Ringe geschnitten \\
        1 & \myindex{Rote Beete}-Knolle, geschält, gewürfelt, angegart \\
      \end{zutat}

      \begin{zutat}{Weitere Zutaten (kann)}
        \brea{} & \myindex{Sellerie}knolle, geschält, gewürfelt, gegart
	                                    (ca. 150 g) \\
        1 & \myindex{Boskopp}\index{Apfel>Boskopp}-Apfel, geschält, in Scheiben
	                                                  oder Würfel \\
        2 & hartgekochte \myindex{Ei}er, gewürfelt \\
        3 & mittelgroße \myindex{Gewürzgurke}\index{Gurke>Gewürz-}n,
	    gewürfelt \\
        3 Teelöffel & \myindex{Kapern} \\
      \end{zutat}

      \begin{zutat}{Zum Abschmecken}
        \brev{} Teelöffel & \myindex{Zucker} \\
        wenig & \myindex{Salz} \\
        & \myindex{weißer Pfeffer}\index{Pfeffer>weiß} aus der Mühle \\
        2 Eßlöffel & \myindex{Apfelessig}\index{Essig>Apfel-} \\
        3 Eßlöffel & \myindex{Öl} \\
        & \myindex{Mayonnaise} eventuell \\
      \end{zutat}

      \personen{4}

      \begin{zubereitung}
        Ergibt eine Salatschüssel voll und sättigt gut. \\
	Am 09.03.2013 gemacht mit 1 Packung Matjesfilets ohne Mayonnaise. \\
      \end{zubereitung}

    \mynewsection{überbackener Kabeljau}

      \begin{zutaten}
        2 große & \myindex{Kabeljau}filets je 600 g, frisch mit Haut \\
      \end{zutaten}

      \begin{zutat}{für die Auflage (Paste)}
        \breh{} & \myindex{Chilischote} \\
        10 & getrocknete \myindex{Tomate}n und etwas Öl aus dem Glas \\
        2 & \myindex{Knoblauch}zehen, gepreßt \\
        \breh{} & Bio-\myindex{Zitrone}nsaft und -\myindex{Zitrone}nabrieb \\
        1 Handvoll & \myindex{Basilikum}blätter \\
        1 Eßlöffel & \myindex{Balsamico} \\
        40 g & \myindex{Parmesan} \\
      \end{zutat}

      \begin{zutat}{für die Auflage (Brösel)}
        200 g & zerrissenes \myindex{Brot} \\
        4 & \myindex{Sardellen}filets \\
        2 & \myindex{Knoblauch}zehen, gepreßt \\
        & \myindex{Salz} \\
        & \myindex{Pfeffer} \\
        etwas & \myindex{Öl} aus der Fischdose \\
      \end{zutat}

      \begin{zutat}{Außerdem}
        2--3 & \myindex{Rosmarin}zweige \\
        2--3 & \myindex{Thymian}zweige \\
        & \myindex{Olivenöl}\index{Oel=Öl>Oliven-} \\
        & \myindex{Salz} \\
        & \myindex{Pfeffer} \\
        & \myindex{Fenchel}samen \\
      \end{zutat}

      \personen{6--8}

      \begin{zubereitung}
        Alles für die Auflage (Paste) in die Küchenmaschine und dann
	zerkleinern (wird auf den Fisch gestrichen). \\
	Die Brösel kommen auch in die Küchenmaschine und werden danach über den
	Fisch gestrichen. \\
	In eine Aufflaufform reichlich Olivenöl, Salz, Pfeffer, Fenchelsamen
	geben. Den Fisch dazu und gründlich wenden und einreiben. Mit der Haut
	nach unten in den Ofen (maximale Hitze, also \grad{250}). \\
	Dann stellt man die Paste her, abschmecken. Den Fisch aus dem Ofen
	nehmen, mit einem Löffel die Paste auf dem Fisch verteilen. Den Fisch
	wieder in den Ofen. \\
	Anschließend die Brösel herstellen. Das Brot, Sardellenfilets, Salz,
	Pfeffer, Knoblauchzehen in die Küchenmaschine und einschalten.
	Etwas Öl aus der Fischdose dazu und wieder einschalten. Fisch
	rausnehmen und aus dem Behälter alles auf den Fisch schütten.
	Obenauf Zweige von Rosmarin und Thymian (zerzupfen). und wieder in den
	Ofen, aber nicht zu weit oben. \\
	Die Paste und die Auflage sollte man besser vorher herstellen und
	abschmecken. Den Fenchelsamen hat Leander im Teebeutel vom Fencheltee
	vorgeschlagen. 1 Teebeutel (etwa die Hälfte) genommen. Backofen auf
	\grad{250} vorgeheizt. 2. Schiene von oben, aber am Schluß
	räucherte es beinahe. Man sollte die Schiene darunter nach der Hälfte
	der Zeit nehmen. \\
	Schmeckt einfach toll, keines der Gewürze hat hervorgestochen! \\
      \end{zubereitung}

    \mynewsection{Vegetarische Spaghetti Bolognese}\label{spaghetti-bolognese}

      \begin{zutaten}
        3 Eßlöffel & \myindex{Olivenöl}\index{Oel=Öl>Oliven-} \\
	1 Eßlöffel & \myindex{Tomatenmark} \\
	1 & gehackte \myindex{Zwiebel} \\
	1 kleine Dose & gehackte \myindex{Tomate}n \\
	200 g & \myindex{Veggie-Hackfleisch} \\
	1 & \myindex{Lorbeer}blatt \\
	2 Teelöffel & \myindex{Oregano} \\
	& \myindex{Salz} \\
	& \myindex{weißer Pfeffer}\index{Pfeffer>weiß} \\
	2 Zehen & \myindex{Knoblauch} gepreßt \\
	6--8 Blätter  & geschnittener \myindex{Basilikum} (ersatzweise
	                getrockneter Oregano) \\
      \end{zutaten}

      \begin{zubereitung}
        Etwas Olivenöl und Tomatenmark in einem ausreichend großen Topf
	anrösten. Dann etwas mehr Olivenöl und die Zwiebel dazu und andünsten,
	bis sie glasig ist. Jetzt die gehackten Tomaten, das
	Veggie-Hackfleisch, ein Lorbeerblatt, Oregano, Salz, Pfeffer und
	Knoblauch dazu und etwa 20~Minuten köcheln und eindicken lassen. Zum
	Schluß den Basilikum schneiden, zu der Soße geben und abschmecken. \\
	Daneben Vollkornspaghetti kochen, bis sie bißfest sind. \\
	Dazu grüner Salat aus einem Salatherz und einer gewürfelten Gurke,
	sowie 4--6~geviertelte Tomaten (extra) und Parmesan. \\
	Gekocht am 02.05.2015. War sehr lecker. \\
      \end{zubereitung}

    \mynewsection{Panzanella (Brotsalat)}

      % aus Für Sie 28.01.2015

      \begin{zutaten}
        400 g & \myindex{Ciabattabrot}, grob gewürfelt \\
        5 Eßlöffel & \myindex{Olivenöl}\index{Oel=Öl>Oliven-} \\
        500 g & \myindex{Tomate}n, in Achtel geschnitten \\
        2 & \myindex{Knoblauch}zehen, fein gehackt \\
        4 Eßlöffel & \myindex{Weißweinessig}\index{Essig>Weißwein-} \\
        100 ml & \myindex{Gemüsebrühe} \\
        3 Eßlöffel & \myindex{grünes Pesto}\index{Pesto>grün} \\
        80 ml & \myindex{Olivenöl}\index{Oel=Öl>Oliven-} \\
        & \myindex{Salz} \\
        & \myindex{Pfeffer} \\
        250 g & \myindex{Rucola}, grob gehackt \\
        1 Bund & \myindex{Basilikum} (Blätter abzupfen, grob zerteilen oder
	         ganz) \\
      \end{zutaten}

      \personen{4}

      \kalorien{395}

      \begin{zubereitung}
        In einer Pfanne Öl erhitzen und Brotwürfel von allen Seiten knusprig
	anbraten. \\
	Knoblauch und Essig, Brühe, Pesto und Öl, Salz und Pfeffer zur
	Vinaigrette mischen. \\
	Brot, Rucola, Basilikum und Tomaten mischen und mit der Vinaigrette
	beträufeln. Abschmecken mit Salz und Pfeffer. Kühl stellen bis zum
	Servieren. \\
      \end{zubereitung}

    \mynewsection{Brotsalat mit Tomaten}

      % 30.09.2011 WDR Björn Freitag Erntefestrezepte

      \begin{zutaten}
        3 Scheiben & süßliches \myindex{Hefebrot}\index{Brot>Hefe-} \\
        & \myindex{Olivenöl}\index{Oel=Öl>Oliven-} zum Beträufeln (sparsam!) \\
        & \myindex{Salz} \\
        & \myindex{Pfeffer} \\
        & \myindex{Balsamessig}\index{Essig>Balsam-} \\
        \breh{} & \myindex{rote Zwiebel}\index{Zwiebel>rot}, gewürfelt \\
        2 & \myindex{Tomate}n, geviertelt \\
      \end{zutaten}

      \personen{2}

      \begin{zubereitung}
        Ofen auf \grad{100} vorheizen. Auf einen Ofenrost die Brotscheiben
	(gleichmäßig mit Olivenöl beträufelt) legen. Circa 20~Minuten rösten,
	das Brot sollte kross sein. Danach in grobe Würfel schneiden. \\
	Tomaten mit Essig, Öl, Salz und Pfeffer anmachen, Zwiebel dazu.
	Circa 20~Minuten ziehen lassen. Kurz vor dem Servieren die Brotwürfel
	dazu geben und alles gut vermischen.\\
      \end{zubereitung}

    % \mynewsection{Text}

      % \begin{zutaten}
      %   & \myindex{} \\
      % \end{zutaten}

      % \begin{zubereitung}
      % \end{zubereitung}
