
% created Montag, 10. Dezember 2012 16:24 (C) 2012 by Leander Jedamus
% modifiziert Mittwoch, 11. März 2015 17:09 von Leander Jedamus
% modifiziert Montag, 09. März 2015 14:27 von Leander Jedamus
% modified Montag, 10. Dezember 2012 16:30 by Leander Jedamus

  \mynewchapter{Zucchini}

    \mynewsection{Kartoffel-Zucchini-Auflauf (im Backofen)}%
              \glossary{Zucchini-Kartoffel-Auflauf}%
	      \glossary{Auflauf>Kartoffel-Zucchini-}

      \begin{zutaten}
        750 g & \myindex{Zucchini} (grün, klein und fest) \\
        1 mittlerer Topf & \myindex{Kartoffel}n (klein, eiförmig, gleich groß)
	                   \\
        & \myindex{Butter} oder \myindex{Margarine} \\
        & \myindex{Paniermehl} \\
        & \myindex{weißer Pfeffer}\index{Pfeffer>weiß} \\
        & \myindex{schwarzer Pfeffer}\index{Pfeffer>schwarz} \\
        & \myindex{Muskatnuß} \\
        & \myindex{Gemüsebrühe} \\
        4--6 & \myindex{Käse}scheiben bzw.
	       100 g geriebener \myindex{Emmentaler} \\
        2 & \myindex{Ei}er \\
        \brev{} l & \myindex{süße Sahne}\index{Sahne>süß} \\
        & \myindex{Salz} \\
        1 Eßlöffel & \myindex{Parmesan}\index{Käse>Parmesan} \\
        & \myindex{Margarine}flöckchen \\
      \end{zutaten}

      \garzeit{30--40}

      \begin{zubereitung}
        Pellkartoffeln im Topf kochen. Noch im warmen Zustand werden sie von
	der Schale befreit und zusammen mit den Zucchini in Scheiben
	geschnitten. Die Auflaufform mit Margarine ausstreichen und eine dünne
	Schicht Paniermehl auf den Boden streußeln. Die darauf kommende Lage
	Kartoffeln mit wenig weißem Pfeffer, Muskatnuß und Gemüsebrühe
	bestäuben. Die nächste Lage aus Zucchini wird mit schwarzem Pfeffer und
	Gemüsebrühe bestäubt. Wenn die Form voll ist, kommen oben drauf
	Käsescheiben. Parmesan, Eier, Sahne, Gemüsebrühe und Salz verrühren und
	über die Scheiben gießen. Als oberste Schicht Paniermehl auftragen,
	darauf die Margarineflöckchen. \\
        Auf mittlerer Schiene im auf \grad{220} vorgeheizten Backofen etwa
	30~Minuten überbacken. \\
      \end{zubereitung}

    \mynewsection{Zucchini gebraten}

      \begin{zutaten}
        ca. 500 g & bis 4~cm dicke \myindex{Zucchini} (grün, fest) \\
        1--3 Zehen & \myindex{Knoblauch} \\
        2--4 Eßlöffel & Öl\index{Oel=Öl} \\
        & frische Kräuter (\myindex{Basilikum} oder \myindex{Oregano}) \\
        etwas & \myindex{Salz}, \myindex{Pfeffer} \\
      \end{zutaten}

      \garzeit{25--30}

      \begin{zubereitung}
        Zucchini möglichst kleine wählen (die dicken haben Kerne und schmecken
	nicht so gut), säubern falls nötig und in etwa \bred{}~cm starke
	Scheiben schneiden, evtl. mit sauberem Geschirrtuch trocken machen.
	Knoblauch ebenfalls in Scheiben. Öl (man kann auch Olivenöl nehmen,
	dann nicht so stark erhitzen) in Pfanne erhitzen, trockene
	Zucchinischeiben leicht bräunen, Knoblauch und Kräuter dazugeben. Öfter
        wenden, bis die Scheiben weich sind. Eventuell nachwürzen. Dazu Nudeln
	mit Soße. \\
      \end{zubereitung}

    \mynewsection{Zucchini-Auflauf (im Backofen)}

      \begin{zutaten}
        750 g & \myindex{Zucchini} (grün, klein und fest) \\
        2 Zehen & \myindex{Knoblauch} \\
        & \myindex{Butter}/\myindex{Margarine} \\
        Käsesoße:& \\
        60 g & \myindex{Butter} \\
        40 g & \myindex{Mehl} \\
        \brda{} l & \myindex{Milch} (knapp, d.h. etwas weniger) \\
        etwas & \myindex{Salz}, \myindex{Pfeffer},
	        geriebene \myindex{Muskatnuß} \\
        2--3 & Eier \\
        75 g & geriebener \myindex{Parmesan}\index{Käse>Parmesan} \\
        1 Bund & \myindex{Basilikum} \\
      \end{zutaten}

      \garzeit{ca. 30}

      \begin{zubereitung}
        Zucchini säubern, in Scheiben schneiden. Geschälten Knoblauch pressen.
	Beides in heißen Fett ca. 10~Minuten dünsten, bis die sich bildende
	Flüssigkeit fast verdampft ist (die Scheiben sollten jedoch noch als
	Scheiben erkennbar bleiben !!). Beiseite stellen. Jetzt die Käsesoße
	machen. Butter erhitzen (nicht braun werden lassen !!), Mehl dazu geben
	und hellgelb anschwitzen, Milch unter Rühren zugießen und einmal
	aufkochen lassen. Mit Salz, Pfeffer, Muskat kräftig würzen, dann bei
	kleiner Hitze etwa 10~Minuten köcheln lassen, ab und zu umrühren.
	Abkühlen lassen. Dann Eier, geriebenen Parmesankäse und feingehacktes
	Basilikum dazugeben. Die Auflaufform mit Butter ausstreichen und
	abwechselnd Zucchinischeiben und Käsesoße einfüllen. Soße zum Schluß
	als Deckschicht nehmen. Auf der Mittelschiene im auf \grad{200}
	vorgeheizten Backofen etwa 30~Minuten überbacken. Dazu grünen Salat.
	Kann man auch mit Brokkoli (ohne Knoblauch, ohne Basilikum) probieren.
	\\
      \end{zubereitung}

    \mynewsection{Marinierte Zucchini}\glossary{Zucchini>mariniert}

      \begin{zutaten}
        750 g & kleine feste grüne \myindex{Zucchini} \\
        2 Zehen & \myindex{Knoblauch} \\
        etwas & \myindex{Salz}, \myindex{Pfeffer} \\
        \brea{} l & italienischer \myindex{Weinessig}\index{Essig>Wein-} \\
        \breh{} Teelöffel & frischer \myindex{Oregano} oder
	                    \myindex{Pfefferminze} \\
        reichlich & \myindex{Olivenöl} zum Ausbacken \\
      \end{zutaten}

      \begin{zubereitung}
        Zucchini in Scheiben schneiden, trocken machen mit sauberem
	Geschirrtuch. Reichlich Olivenöl in der Pfanne erhitzen, nach und nach
	partienweise die Zucchinischeiben hellbraun braten. Gebratene Zucchini
	beiseite stellen, bis alle Scheiben gebräunt sind. Flachere Schüssel
	bereitstellen. Knoblauch in feine Streifen schneiden, Minzeblättchen
	(oder Oregano) hacken. Die Zucchinischeiben in der Schüssel auslegen,
	dazwischen Kräuter, Salz, Pfeffer und Knoblauchstreifen. Den Essig
	erhitzen und über die eingelegten Scheiben gießen. Mindestens einen Tag
	ziehen lassen. Kühl stellen. \\
        Man kann anstelle der Zucchini auch AUBERGINEN nehmen (die Scheiben muß
	man vorher kräftig salzen, etwa \breh{}~Stunde Wasser ziehen lassen,
	dann wie Zucchini verfahren). \\
        Haben wir vor Jahren zu Weihnachten gegessen. Sehr lecker! \\
      \end{zubereitung}

    \mynewsection{Zucchini in Essig und Öl}

      \begin{zutaten}
        750 g & \myindex{Zucchini} \\
        & \myindex{Salz} \\
        & frisch gemahlener \myindex{Pfeffer} \\
        5 Eßlöffel & \myindex{Olivenöl}\index{Oel=Öl>Oliven-} \\
        2 & \myindex{Knoblauchzehe}n \\
        \brea{} l & \myindex{Weißweinessig}\index{Essig>Weißwein-} \\
        2 Zweige & frische \myindex{Minze} \\
      \end{zutaten}

      \begin{zubereitung}
        Zucchini waschen, trocknen und in Scheiben schneiden. Mit Salz und
	Pfeffer bestreuen. Pfanne erhitzen, Olivenöl erhitzen. Die Zucchini
	portionsweise im Öl goldbraun braten. Knoblauchzehen schälen,
	zerdrücken und kurz mitbraten lassen. Das Gemüse mit Essig aufgießen,
	etwas einkochen und erkalten lassen. Die Minzeblätter vom Stiel lösen,
	fein hacken und darüber streuen. \\
        Statt Minzeblätter kann man auch \breh{} Teelöffel Oregano nehmen und
	die gebratenen Zucchini auf eine Schüssel legen, dann heißer Essig
	darüber. 1~Tag ziehen lassen. \\
      \end{zubereitung}

    \mynewsection{Gefüllte Zucchini}\glossary{Zucchini>gefüllt}

      \begin{zutaten}
        6 mittelgroße & \myindex{Zucchini} \\
        2 Eßlöffel & \myindex{Olivenöl}\index{Oel=Öl>Oliven-} \\
        1 & \myindex{Tomate} \\
        250 g & \myindex{Hackfleisch}%
	        \footnote{alternativ: Hirsefrikadellenmischung
		          siehe Seite \pageref{hirsefrikadellen}} \\
        1 & \myindex{Ei} \\
        & \myindex{Salz} \\
        & \myindex{Pfeffer} \\
        & \myindex{Oregano} oder frisches \myindex{Basilikum} \\
        2 Eßlöffel & \myindex{Butter} \\
        2 Eßlöffel & \myindex{Mehl} \\
        \brev{} l & \myindex{Milch} \\
        \brea{} l & \myindex{Sahne} \\
        2 & \myindex{Eigelb}e \\
      \end{zutaten}

      \garzeit{30}

      \begin{zubereitung}
        Die Haut der Zucchini leicht abschaben, am Stielende ein kleines Stück
	abschneiden, waschen und der Länge nach durchschneiden. Das kernige
	Mark mit einem Löffel entfernen. Das Mark fein hacken und in dem Öl mit
	einer gebrühten, abgezogenen und in Würfel geschnittenen Tomate weich
	dünsten. Zucchini in kochendem Salzwasser 8~Minuten kochen und gut
	abtropfen lassen. \\
        Das Mark mit Hackfleisch, Ei, Salz und Pfeffer sowie Oregano (oder
	kleingehacktes Basilikum) mischen. Die Masse in die Zucchini-Hälften
	füllen. Aus Butter und Mehl eine Mehlschwitze herstellen und mit Milch
	und Sahne aufgießen, durchkochen lassen und vom Herd nehmen, abkühlen
	lassen und mit dem Eigelb und dem Käse verrühren. \\
        Die gefüllten Zucchini in eine feuerfeste, gebutterte Backofen bei
	mittlerer Hitze (\grad{200}) 30~Minuten überbacken. \\
        Variante: mit Tomaten und Speck als Füllung. \\
      \end{zubereitung}

    \mynewsection{Zucchini-Eier-Pfanne}

      \begin{zutaten}
      \end{zutaten}
      \begin{zutat}{Für den Teig}
        125 g & \myindex{Mehl} \\
        3 & \myindex{Ei}er \\
        1 Prise & \myindex{Salz} \\
        1 dl & \myindex{Sahne} \\
        50 g & flüssige \myindex{Butter} \\
      \end{zutat}
      \begin{zutat}{Für die Füllung}
        3--4 & möglichst dicke \myindex{Zucchini} \\
        & \myindex{Salz} \\
        & frisch gemahlener \myindex{Pfeffer} \\
        & Saft von einer \myindex{Zitrone} \\
        & \myindex{Butter} zum Einfetten und Belegen \\
        \brev{} l & \myindex{Sahne} \\
        1 & \myindex{Ei} \\
        30 g & \myindex{Parmesan}\index{Käse>Parmesan} (echten italienischen)
	       \\
      \end{zutat}

      \garzeit{ca. 30}

      \begin{zubereitung}
        Mehl sieben, Eier hineinschlagen, salzen und mit Butter und Sahne
	verrühren. Gut mit dem Schneebesen durchschlagen und zugedeckt zum
	quellen für 1~Stunde kühlstellen. In einer flachen Pfanne mit Butter
        dünne Pfannkuchen backen. Diese übereinanderliegend im Backofen
	warmhalten. \\
        Zucchini waschen, trocknen und in dünne Scheiben schneiden. Ausbreiten
	und mit Salz, Pfeffer und Zitronensaft würzen. Feuerfeste Form mit
	Butter einfetten. Pfannkuchen mit der hellen Seite nach oben
	ausstechen, möglichst im Durchmesser wie die Zucchini. Pfannkuchen und
	Zucchini schuppenartig oder zur Rosette schichten. Sahne, Salz und Ei
	gut verquirlen und darübergießen. Parmesan darüber streuen und mit
	Butterflöckchen abschließen. \\
        Form in \grad{220--230} heißen Ofen schieben, ca. 25~Minuten goldgelb
	und knusprig backen. \\
        Paßt gut zu gebratenem und gekochtem Fleisch. Auch mit einem knackigen
	Salat gut als eigenständiges Essen. \\
      \end{zubereitung}

    \mynewsection{Süß-saure Zucchini}\glossary{Zucchini>süß-sauer}

      \begin{zutaten}
        1 kg & kleine \myindex{Zucchini} \\
        2 Eßlöffel & \myindex{Rosinen} \\
        5 Eßlöffel & \myindex{Olivenöl}\index{Oel=Öl>Oliven-} \\
        1 & \myindex{Knoblauchzehe} \\
        & \myindex{Salz} \\
        2 Eßlöffel & \myindex{Weinessig}\index{Essig>Wein-} \\
        3 & \myindex{Sardellen}\index{Fisch>Sardellen}filets \\
        2 Eßlöffel & \myindex{Pinienkerne} \\
        1 Stück & \myindex{Würfelzucker}\index{Zucker>Würfel-} \\
        einige & Löffel \myindex{Fleischbrühe} \\
      \end{zutaten}

      \garzeit{20--30}

      \begin{zubereitung}
        Die Haut der Zucchini leicht abschaben, am Stielende ein kleines Stück
	abschneiden, waschen und in Streifen schneiden. Rosinen in lauwarmen
	Wasser einweichen. \\
        Öl erhitzen und die Knoblauchzehe darin hellgelb anrösten, aus dem Öl
	nehmen und die Zucchini-Streifen hineingeben. Unter Umrühren mit einem
	Holzlöffel hellgelb rösten, salzen und mit Essig aufgießen. Die
	gehackten Sardellenfilets, Rosinen, Pinienkerne und Zucker dazu. Bei
	leichter Hitze in 20~Minuten gar kochen, ab und zu mit Fleischbrühe
	aufgießen und dann heiß servieren. \\
      \end{zubereitung}

    \mynewsection{Ausgebackene Zucchini-Streifen}%
      \glossary{Zucchini-Streifen>ausgebacken}

      \begin{zutaten}
        8 & kleine \myindex{Zucchini} \\
        & \myindex{Salz} \\
        & \myindex{Mehl} \\
        & \myindex{Olivenöl}\index{Oel=Öl>Oliven-} zum Ausbacken \\
      \end{zutaten}

      \begin{zubereitung}
        Zucchini waschen, am Stielende ein kleines Stück abschneiden und der
	Länge nach durchschneiden. Jede Hälfte nochmal in 4~Teile schneiden, in
	eine Schüssel geben, mit Salz bestreuen und etwas liegen lassen. \\
        Öl erhitzen. Zucchini in Mehl wenden und in dem Öl goldgelb backen und
	sofort servieren. \\
        Man kann die Zucchini auch noch nach dem Wenden in Mehl noch in Eier,
	die mit geriebenem Parmesan verquirlt wurden, wenden und dann erst
	ausbacken. \\
        Nur in Mehl = besonders knusprig \\
        +Ei+Parmesan = besonders saftig \\
      \end{zubereitung}

    \mynewsection{Zucchini in Weißwein}

      \begin{zutaten}
        750 g mittelgroße & \myindex{Zucchini} \\
        2 & \myindex{Knoblauchzehe}n \\
        1 Bund & \myindex{Petersilie} \\
        2 Tassen & \myindex{Wasser} \\
        2 Tassen & \myindex{Weißwein} \\
        6 Eßlöffel & \myindex{Olivenöl}\index{Oel=Öl>Oliven-} \\
        1 & \myindex{Lorbeer}blatt \\
        1 Prise & \myindex{Thymian} \\
        1 Prise & \myindex{Salz} \\
        1 Prise & \myindex{Zucker} \\
        1 Bund & \myindex{Basilikum} \\
      \end{zutaten}

      \garzeit{15--25}

      \begin{zubereitung}
        Knoblauchzehen schälen und zerdrücken, Petersilie waschen und trocknen.
	Zucchini waschen, Blüten- und Stielansatz entfernen. Zucchini quer in 3
	Stücke teilen und der Länge nach vierteln. In eine Kasserole geben und
	Wasser, Wein, Öl, Knoblauch, Lorbeerblatt, Petersilie und Thymian dazu.
	Würzen mit Salz, Pfeffer, Zucker. Zugedeckt bei schwacher Hitze in
	15--25~Minuten knapp ,,al dente`` kochen. \\
        Basilikum waschen und kleinschneiden. Zucchini mit Schaumlöffel
	herausnehmen und gut abgetropft in eine Servierschüssel geben. Die
	Kochflüssigkeit bei starker Hitze zur Hälfte eindampfen lassen, dann
	abseihen. Mit 3--4~Eßlöffel davon die Zucchini beträufeln, das gehackte
	Basilikum darüber streuen. Gut durchmischen und 3--4~Stunden im
	Kühlschank ziehen lassen. Zimmerwarm servieren (Vorspeise). \\
        Kann man auch mit Blumenkohlröschen, Lauch und kleinen Artischocken auf
	diese Art zubereiten. \\
      \end{zubereitung}

    \mynewsection{Überbackene Zucchini}\glossary{Zucchini>überbacken}

      \begin{zutaten}
        750 g mittelgroße & \myindex{Zucchini} \\
        1 kleine & \myindex{Zwiebel} \\
        1 & \myindex{Knoblauchzehe} \\
        1 kleine Dose & \myindex{Tomate}n (500 g) \\
        & \myindex{weißer Pfeffer}\index{Pfeffer>weiß} \\
        & neutrales Öl\index{Oel=Öl} zum Braten \\
        1 Bund & \myindex{Basilikum} \\
        200 g & \myindex{Mozzarella}\index{Käse>Mozzarella} \\
        2 & \myindex{Ei}er \\
        & \myindex{Parmesan}\index{Käse>Parmesan} \\
      \end{zutaten}

      \garzeit{20}

      \begin{zubereitung}
        Zwiebel und Knoblauch schälen und fein hacken. In einer Kasserole das
	Olivenöl erhitzen, Zwiebel und Knoblauch leicht anbraten, aber nicht
	braun werden lassen. Dann Tomaten mit Saft dazugeben. Mit Salz und
	frisch gemahlenem Pfeffer kräftig würzen, umrühren und die Tomaten
	zerdrücken. Zugedeckt schmoren lassen. \\
        Zucchini waschen, trocknen und Blütenansatz und Stiel abschneiden.
	Zucchini der Länge nach in 1~cm dicke Scheiben schneiden. Diese
	portionsweise in ausreichend Öl beidseitig goldbraun braten. Die
	gebratenen Scheiben auf Doppel-Lage Küchenkrepp ausbreiten. \\
        Basilikum waschen, trocknen und klein schneiden. Backofen auf
	\grad{200} vorheizen. Eier mit Salz und frisch gemahlenem Pfeffer
	verquirlen. Feuerfeste Form fetten, 3--4~Eßlöffel Tomatensoße
	gleichmäßig darin verteilen. Darüber gleichmäßig Zucchinischeiben,
	darauf Mozzarella-Scheiben. Mit Salz und Pfeffer würzen. Parmesankäse
	und Basilikum darüber streuen. Wieder löffelweise Tomatensoße darüber.
	Lagenweise wiederholen. Verquirlte Eier darüber, gut mit einer Gabel
	verteilen. Form im Backofen etwa 20~Minuten gratinieren. \\
        Vor dem Servieren auskühlen lassen (nicht in den Kühlschank). Kann man
	auch als Vorspeise (in kleineren Portionen) nehmen. \\
      \end{zubereitung}

    \mynewsection{Zucchinigemüse mit Speck und Tomaten}

      \begin{zutaten}
        600 g & 8 \myindex{Zucchini}, klein und fest \\
        300 g & frische, reife \myindex{Tomate}n oder 1 kleine Dose (500 g) \\
        200 g & durchwachsener \myindex{Speck} \\
        1 große & \myindex{Zwiebel} \\
        2 & \myindex{Knoblauchzehe}n \\
        1 Bund & \myindex{Petersilie} \\
        1 Eßlöffel & \myindex{Olivenöl}\index{Oel=Öl>Oliven-} \\
        & \myindex{Salz} \\
        & \myindex{schwarzer Pfeffer}\index{Pfeffer>schwarz} \\
      \end{zutaten}

      \garzeit{10 + 20}

      \begin{zubereitung}
        Zucchini waschen, trocknen, Blütenansatz und Stiel abschneiden.
	Zucchini der Länge nach in Viertel und diese quer in 3~Stücke
	schneiden. Die Tomaten kurz in kochendes Wasser tauchen, abziehen,
	halbieren, Stengelansatz entfernen und Fruchtfleisch in Stücke teilen.
	Speck in Würfel schneiden. Zwiebel schälen und in Ringe schneiden.
	Knoblauchzehen schälen und mit gewaschener, getrockneter Petersilie
	fein hacken. \\
        In einer Kasserole das Öl erhitzen, Speckwürfel darin gut anbraten,
	Zwiebeln, Knoblauch, Petersilie dazu und alles 10~Minuten braten. Dann
	Tomaten und Zucchini untermischen, mit Salz und frisch gemahlenem
	Pfeffer würzen. Alles gut mischen. Zugedeckt bei schwacher Hitze etwa
	20~Minuten dünsten und ab und zu umrühren. \\
      \end{zubereitung}

    \mynewsection{Zucchini-Tomaten-Gemüse}

      \begin{zutaten}
        600 g & kleine feste \myindex{Zucchini} \\
        300 g & reife \myindex{Tomate}n \\
        1 & \myindex{Zwiebel} \\
        1 & \myindex{Knoblauchzehe} \\
        4 & \myindex{Sardellen}\index{Fisch>Sardellen}filets \\
        50 g & \myindex{Speck} \\
        2 Eßlöffel & \myindex{Butter} \\
        & \myindex{Salz} \\
        & \myindex{schwarzer Pfeffer}\index{Pfeffer>schwarz} \\
        1 Bund & \myindex{Petersilie} \\
        2 Eßlöffel & \myindex{Parmesan}\index{Käse>Parmesan}, frisch gerieben
	             \\
      \end{zutaten}

      \garzeit{20}

      \begin{zubereitung}
        Zwiebel und Knoblauch schälen und mit den Sardellen und dem Speck ganz
	fein hacken. In einer Kasserole die Butter erhitzen und das
	Kleingehackte darin bei schwacher Hitze anbraten. Zucchini waschen,
	trocknen, Blütenansatz und Stiel abschneiden, der Länge nach teilen,
	vierteln und quer in 3~Stücke schneiden. Die Stücke in die Kasserole
	geben und kurz mitbraten. Tomaten in kochendem Wasser häuten, Kerne und
	Stielansatz entfernen. Tomaten grob zerkleinern und mit Salz und
	Pfeffer würzen. Zugedeckt bei schwacher Hitze gar dünsten
	(,,al dente``). Petersilie waschen, trocknen, hacken und mit Käse vor
	dem Servieren unterrühren. \\
      \end{zubereitung}

    \mynewsection{Zucchini-Champignons in Oregano-Vinaigrette}

      \begin{zutaten}
        400 g & \myindex{Zucchini}-Stifte \\
        400 g & \myindex{Champignon}-Scheiben \\
        8 Eßlöffel & \myindex{Olivenöl}\index{Oel=Öl>Oliven-} \\
        & \myindex{Salz} \\
        & \myindex{schwarzer Pfeffer}\index{Pfeffer>schwarz} (Mühle) \\
        2 & \myindex{Knoblauchzehe}n (Scheiben) \\
        \brea{} l & trockener \myindex{Weißwein} \\
        2--3 Eßlöffel & \myindex{Zitrone}nsaft \\
        1--2 Bund & \myindex{Oregano} \\
      \end{zutaten}

      \garzeit{40--50}

      \begin{zubereitung}
        Zucchini und Champignons nacheinander in heißem Öl braten, salzen. Auf
	einer Platte anrichten und mit Pfeffer bestreuen. Knoblauch in Öl
	anbraten, mit Wein ablöschen und mit Zitronensaft abschmecken.
	Oreganoblätter auf das Gemüse streuen, Soße darüber gießen. Etwas
	durchziehen lassen. \\
      \end{zubereitung}

    \mynewsection{Süß-saure Zucchini eingelegt für 3 1-l-Gläser}%
              \glossary{Zucchini>süß-sauer}

      \begin{zutaten}
        3 kg kleine bis mittelgroße & \myindex{Zucchini} \\
        60 g & \myindex{Salz} \\
        500 g & \myindex{Zucker} \\
        \breh{} l & \myindex{Weißweinessig}\index{Essig>Weißwein-} \\
        50 g & frische \myindex{Ingwer}wurzel \\
        100 g & frischer \myindex{Meerrettich} \\
        3 Bund & \myindex{Dill} (oder mehrere Stiele Dillblüten) \\
        100 g & \myindex{Schalotte}n \\
        2 Eßlöffel & \myindex{Senfkörner} \\
        1 Eßlöffel & schwarze Pfeffer\index{schwarzer Pfeffer}körner \\
        4 kleine & getrocknete rote \myindex{Chilischote}n \\
        2 Eßlöffel & \myindex{Wacholderbeeren} \\
        8 & \myindex{Lorbeer}blätter \\
      \end{zutaten}

      \garzeit{10}

      \begin{zubereitung}
        Zucchini waschen und putzen. Größere Früchte längs vierteln und in
	fingerlange Stücke schneiden. Zucchini mit Salz, Zucker und Essig in
	einer großen Schüssel mischen und zugedeckt über Nacht durchziehen
	lassen. \\
        Am nächsten Tag in einen Durchschlag geben und Flüssigkeit auffangen.
	Ingwerwurzel und Meerrettich schälen, in Scheiben schneiden und in die
	Flüssigkeit geben. Dill abzupfen, Schalotten pellen. Beides mit Senf-
	und Pfefferkörnern, Chili, Wacholder und Lorbeerblättern in die
	Flüssigkeit geben und alles 8--10~Minuten kochen lassen. Sud vom Herd
	nehmen und kalt werden lassen. Zucchini und Gewürze lagenweise in
	sterile Gläser (mit Bügelverschluß) einschichten und mit dem Sud
	begießen. Gläser sofort schließen. \\
        Bis zum Servieren mindestens 2~Tage kühl stellen. \\
      \end{zubereitung}

    \mynewsection{Zucchini in roher Tomatensoße}

      \begin{zutaten}
        1,5 kg & nicht zu große \myindex{Zucchini} \\
        & \myindex{Salz} \\
        \brda{} l & \myindex{Tomatensaft} \\
        \brev{} l & \myindex{Olivenöl}\index{Oel=Öl>Oliven-} \\
        3 mittelgroße & \myindex{Knoblauchzehe}n \\
        & \myindex{schwarzer Pfeffer}\index{Pfeffer>schwarz} \\
        4 Bund & \myindex{Basilikum} \\
      \end{zutaten}

      \garzeit{30}

      \begin{zubereitung}
        Zucchini putzen, der Länge nach \breh{}~cm dicke Scheiben schneiden.
	Die Scheiben auf Küchenkrepp ausbreiten, mit Salz bestreuen und 15
	Minuten Wasser ziehen lassen. Tomatensaft mit Öl (\brea{}~l)
	verquirlen. Knoblauch pellen, in den Tomatensaft pressen, herzhaft
	salzen und pfeffern. Zucchinischeiben trocken tupfen. Basilikumblätter
	von den Stielen zupfen. Öl in einer Pfanne erhitzen, Zucchini darin
	portionsweise von allen Seiten braun werden lassen, dann wieder auf
	Küchenkrepp legen. Zucchini mit dem Basilikum in ein Glas schichten,
	mit der Tomatensoße übergießen und zugedeckt über Nacht ziehen lassen
	(auch als Vorspeise). \\
      \end{zubereitung}

    \mynewsection{Zucchini gefüllt mit Thunfisch}

      \begin{zutaten}
        4 kleine & \myindex{Zucchini} (bis 15~cm) \\
        & \myindex{Salz} \\
        250 g & \myindex{Thunfisch} (Dose) \\
        1 Eßlöffel & \myindex{\cremefraiche{}} \\
        1 & \myindex{Schalotte} gewürfelt \\
        1 Eßlöffel & \myindex{Zitrone}nsaft \\
        & \myindex{Pfeffer} (Mühle) \\
        1 Eßlöffel & \myindex{Petersilie} gehackt \\
      \end{zutaten}

      \begin{zubereitung}
        Zucchini waschen, in kochendem Salzwasser blanchieren. Eiskalt
	abschrecken und der Länge nach halbieren. Fruchtfleisch bis auf einen
	kleinen Rand auskratzen, beiseite stellen. Ausgehölte Zucchini auf eine
	Platte geben. \\
        Thunfisch abtropfen lassen. Zucchini-Fruchtfleisch, Thunfisch und
	\cremefraiche{} im Mixer pürieren. Schalotte und Zitronensaft
	dazugeben, mit Salz und Pfeffer abschmecken. Zucchinihälften mit der
	Masse füllen. Petersilie überstreuen. Gut gekühlt servieren. \\
      \end{zubereitung}

    \mynewsection{Zucchini mit Quarkfüllung}

      \begin{zutaten}
        4 kleine & \myindex{Zucchini} \\
        & \myindex{Salz} \\
        3 & \myindex{Schalotte}n, gewürfelt \\
        40 g & \myindex{Butter} \\
        1 Teelöffel & \myindex{Mehl} \\
        50 ml & \myindex{Milch} \\
        50 ml & \myindex{Schlagsahne}\index{Sahne>Schlag-} \\
        & \myindex{Pfeffer} (Mühle) \\
        80 g & \myindex{Magerquark} \\
        2 & \myindex{Eiweiß} \\
        1 Eßlöffel & \myindex{Semmelbrösel} \\
        1 Eßlöffel & \myindex{Olivenöl}\index{Oel=Öl>Oliven-} \\
      \end{zutaten}

      \garzeit{20}

      \begin{zubereitung}
        Zucchini waschen, längs halbieren, aushöhlen, 5~Minuten blanchieren in
	Salzwasser. Abtropfen lassen. Schalotten in Butter mit Fruchtfleisch
	bei milder Hitze 5~Minuten dünsten. Mehl darüberstäuben, Milch und
	Sahne unterrühren mit Salz + Pfeffer. Weitere 10~Minuten garen. Vom
	Herd nehmen, etwas abkühlen lassen, Quark und Eischnee zufügen.
	Zucchini mit der Masse füllen. Gebutterte Form mit Zucchini auslegen,
	Semmelbrösel darüber, mit Olivenöl beträufeln. 20~Minuten im
	vorgeheizten Ofen bei \grad{175--200} goldgelb backen. \\
      \end{zubereitung}

    \mynewsection{Zucchinipuffer mit Paprikasahne (Vollwert)}

      \begin{zutaten}
        1 & \myindex{Ei} \\
        1 & \myindex{Eigelb} \\
        3 Eßlöffel & \myindex{Sahne} \\
        4 Eßlöffel & trockener \myindex{Weißwein} (oder \myindex{Wasser}) \\
        30 g & \myindex{Parmesan}\index{Käse>Parmesan} oder
	       abgelagerter Allgäuer
	       \myindex{Emmentaler}\index{Käse>Emmentaler} \\
        110 g & \myindex{Weizen}, frisch gemahlen \\
        1 Eßlöffel & \myindex{Petersilie} gehackt \\
        \brdv{} Teelöffel & \myindex{Kräutersalz}\index{Salz>Kräuter-} \\
        \breh{} Teelöffel & \myindex{Delikata} (wie Fondor) \\
        250 g & \myindex{Zucchini} \\
        1 große & \myindex{Knoblauchzehe} \\
        3 Eßlöffel & \myindex{Olivenöl}\index{Oel=Öl>Oliven-} \\
        3 Eßlöffel & \myindex{Sesam-Samen} \\
      \end{zutaten}
      \begin{zutat}{Paprikasahne}
        1 mittelgroße & hell\myindex{grüne Paprika}\index{Paprika>grün}schote \\
        2 kleine & reife \myindex{Tomate}n \\
        1 Bund & \myindex{Schnittlauch} \\
        1 Sträußchen & \myindex{Dill} \\
        200 g & \myindex{saure Sahne}\index{Sahne>sauer} \\
        1 & \myindex{Knoblauchzehe} \\
        & \myindex{Kräutersalz}\index{Salz>Kräuter-} \\
        & \myindex{Cayennepfeffer}\index{Pfeffer>Cayenne-} \\
      \end{zutat}

      \personen{2}

      \begin{zubereitung}
        Alle Zutaten für den Teig von Ei bis Petersilie zu einem dickflüssigen
	Teig verrühren. Mit dem Kräutersalz und Delikata kräftig abschmecken.
	Teig zugedeckt 1~Stunde quellen lassen. \\
        Für die Paprikasahne die Schote waschen, halbieren, entkernen. Tomaten
	halbieren, Stielansatz und das Innere entfernen. Paprika und Tomaten
	würfeln. Schnittlauch und Dill fein hacken (1~Dillzweig zum Garnieren
	lassen). Paprika- und Tomatenwürfel sowie Kräuter unter die saure Sahne
	rühren. Knoblauch pellen, pressen und an die Soße geben. Abschmecken
	mit Kräutersalz und Cayennepfeffer. Mit Dill garnieren und kühl
	stellen, bis die Puffer fertig sind. \\
        Zucchini waschen, Stiel und Blütenansatz abschneiden und grob raspeln.
	Knoblauchzehe an die Zucchini pressen, beides in den Teig rühren.
	Olivenöl in einer Pfanne erhitzen, je 2~Eßlöffel Teig zu einem flachen
	Puffer ausstreichen, mit Sesam bestreuen und leicht andrücken. Auf
	beiden Seiten goldgelb backen. Den Backofen auf \grad{60} stellen zum
	Warmhalten der Puffer. Dazu kühle Paprikasoße. \\
      \end{zubereitung}

    \mynewsection{Zucchini-Kartoffel-Auflauf mit Kräutern (Vollwert)}

      \begin{zutaten}
        250 g & \myindex{Zucchini} (am besten große) \\
        100 g & \myindex{Zwiebel}n, gewürfelt \\
        2 & \myindex{Knoblauchzehe}n, gehackt \\
        1\breh{} Eßlöffel & \myindex{Sonnenblumenöl}\index{Oel=Öl>Sonnenblumen-} \\
        300 g & mehlig kochende \myindex{Kartoffel}n \\
        \breh{} Eßlöffel & \myindex{Butter} \\
        50 g & mittelalter \myindex{Gouda}\index{Käse>Gouda}-Käse \\
        50 g & \myindex{Tilsiter}\index{Käse>Tilsiter} \\
        1 Tasse & Kräuter (\myindex{Petersilie}, \myindex{Liebstöckel},
	          \myindex{Basilikum}, \myindex{Thymian}, 
                  \myindex{Majoran} oder \myindex{Oregano}), frisch gehackt \\
        & \myindex{Kräutersalz}\index{Salz>Kräuter-} \\
        & \myindex{schwarzer Pfeffer}\index{Pfeffer>schwarz} \\
        \breh{} Teelöffel & \myindex{Rosenpaprika}\index{Paprika>Rosen-} \\
        6 Eßlöffel & \myindex{Sahne} \\
        1 & \myindex{Ei} \\
        2 Eßlöffel & Allgäuer \myindex{Emmentaler}\index{Käse>Emmentaler}
	             Käse, frisch gerieben (zum Bestreuen) \\
        2 Eßlöffel & \myindex{Sonnenblumenkerne}, geröstet \\
      \end{zutaten}

      \personen{2}

      \garzeit{30}

      \begin{zubereitung}
        Zwiebel und Knoblauch in der Pfanne goldgelb braten. Kartoffeln und
	Zucchini waschen, putzen und eventuell teilen. Butter auf den Zwiebeln
	schmelzen lassen und Kartoffeln und Zucchini in die Pfanne raspeln.
	Backofen auf \grad{200} vorheizen. Käse im Elektrohacker zerkleinern
	und unter die Kartoffeln mit den Kräutern mischen. Abschmecken mit
	1~Teelöffel Kräutersalz und Gewürzen. Die Masse in der Pfanne
	gleichmäßig verteilen und andrücken. \\
        Die Sahne mit dem Ei, der durchgepreßten 2.~Knoblauchzehe und etwas
	Salz verrühren, über den Auflauf gießen. Käse und Sonnenblumenkerne
	darüber streuen. Im Ofen auf mittlerer Schiene etwa 30~Minuten bei
	\grad{200} goldgelb backen. \\
      \end{zubereitung}

    \mynewsection{Zucchini-Auflauf mit Makkaroni}

      \begin{zutaten}
        600 g & \myindex{Zucchini} in Scheiben \\
        400 g & \myindex{Tomate}n, gehäutet, in Vierteln \\
        3 mittelgroße & \myindex{Zwiebel}n, gehackt \\
        4 Eßlöffel & \myindex{Olivenöl}\index{Oel=Öl>Oliven-} \\
        150 g & \myindex{Emmentaler}\index{Käse>Emmentaler} Käse, gerieben \\
        200 g & \myindex{Makkaroni}, gekocht \\
        1 Eßlöffel & Kräuter (\myindex{Basilikum},
	                      \myindex{Rosmarin}), gehackt \\
        & \myindex{Salz} \\
        & \myindex{Pfeffer} \\
        2 & \myindex{Ei}er \\
        100 g & \myindex{Sahne} \\
        1 Eßlöffel & \myindex{Petersilie}, gehackt \\
      \end{zutaten}

      \garzeit{45}

      \begin{zubereitung}
        Zwiebeln goldgelb in Öl anbraten, in eine feuerfeste Form schütten.
	Die Ränder gut mit dem Bratöl einreiben. Auf die Zwiebeln die
	Zucchinischeiben legen, dazwischen die Tomatenviertel. Geriebener Käse
	zur Hälfte darüber und verteilen. Makkaroni darüber, Pfeffer dazu. Eier
	mit Sahne und restlichem Käse quirlen, über den Auflauf geben. Im
	vorgeheizten Ofen bei \grad{180} ca. 45~Minuten garen (goldgelb). Mit
	Petersilie bestreuen. \\
      \end{zubereitung}

    \mynewsection{Zucchini Frikadellen Peleponnes}

      \begin{zutaten}
        & \myindex{Zucchini}, geraspelt und ausgepreßt \\
        & \myindex{Zwiebel}n, gehackt \\
        & \myindex{Feta}\index{Käse>Feta} \\
        4 & \myindex{Ei}er \\
        & \myindex{Mehl} \\
        & \myindex{Salz} \\
        & \myindex{Pfeffer} \\
        & \myindex{Oregano} \\
        & \myindex{Petersilie} \\
      \end{zutaten}

      \begin{zubereitung}
        Aus den Zutaten Teig machen, Frikadellen formen und in Mehl wenden.
	Die Pfanne mit viel Olivenöl erhitzen und Frikadellen braten. \\
      \end{zubereitung}

    \mynewsection{Zucchinischeiben mit Mozzarella überbacken}

      \begin{zutaten}
        1 kg & \myindex{Zucchini} (größere) \\
        1 Dose & passierte \myindex{Tomate}n (400 g) \\
        2--3 große & \myindex{Knoblauchzehe}n \\
        2 & kleine \myindex{Chilischote}n (rot oder grün) \\
        3 Beutel & \myindex{Mozzarella}\index{Käse>Mozzarella} \'a 125 g \\
        60 g & frisch geriebener \myindex{Parmesan}\index{Käse>Parmesan} \\
        & \myindex{Olivenöl}\index{Oel=Öl>Oliven-} \\
        & \myindex{Salz} \\
        & \myindex{Pfeffer} aus der Mühle \\
        etwas & \myindex{Zucker} \\
      \end{zutaten}

      \garzeit{35}

      \begin{zubereitung}
        Zucchini der Länge nach \breh{}~cm dick hobeln und in der Pfanne
	hellgelb braten. Salzen und Pfeffern von beiden Seiten. Auflaufform
	fetten mit Öl. Zuerst Zucchini-Scheiben nebeneinander in die Form
	legen. 1~cm dicke Scheiben Mozzarella darauflegen, dann wieder
	gebratene Zucchini usw. \\
        In dem Bratfett entkernte und fein geschnittene Chilischoten sowie fein
	gehackten Knoblauch anbraten, dann Dose Tomaten drüber und würzen mit
	etwas Zucker, Salz und Pfeffer. Kurz köcheln lassen. Samt Bratfett über
	die geschichteten, mit Mozzarella belegten Zucchinischeiben gießen und
	verteilen. Oben drüber eine Handvoll geriebener Parmesan streuen. Im
	Backofen bei \grad{200} wird das Ganze für 35~Minuten gebacken. \\
        Dazu Nudeln und restlicher Parmesan. \\
      \end{zubereitung}

    \mynewsection{Zucchini geschichtet und überbacken}

      \begin{zutaten}
        ca. 1 kg & mittlere bis kleine \myindex{Zucchini} \\
        4 & \myindex{Knoblauchzehe}n \\
        2 & \myindex{Chilischote}n (rot oder grün) \\
        1 Beutel & \myindex{Mozzarella}\index{Käse>Mozzarella} (1 Kugel) \\
        & \myindex{Olivenöl}\index{Oel=Öl>Oliven-} zum Braten \\
        & \myindex{Pfeffer} aus der Mühle \\
        & \myindex{Salz} \\
        ca. 70 g & \myindex{Parmesan}\index{Käse>Parmesan} oder
	           \myindex{Peccorino}\index{Käse>Peccorino} \\
        2--3 & \myindex{Tomate}n \\
      \end{zutaten}

      \garzeit{20}

      \begin{zubereitung}
        Zucchini waschen, Enden abschneiden und der Länge nach in ca. 1~cm
	dicke Scheiben hobeln. Große Pfanne mit Olivenöl erhitzen, Zucchini
	braten, wenden, mit Salz und Pfeffer würzen. Fertiggebratene Scheiben
	in eine Auflaufform legen, die zweite Seite salzen und pfeffern. In der
	Pfanne weitere Scheiben Zucchini braten wie gehabt. Dann auf die
	Zucchinischeiben in der Auflaufform gehackte Chilischote, gehackten
	Knoblauch und dünn Parmesan über jede Lage streuen. Wenn alle
	gebratenen Scheiben in der Auflaufform sind, Tomaten in Scheiben
	schneiden und auslegen. Abschließend mit Mozzarellascheiben abdecken.
	\\
        Bei \grad{180--200} für 20~Minuten im Backofen garen. \\
        Dazu Nudeln, z.B. Farfalle (250 g). \\
      \end{zubereitung}

    \mynewsection{Zucchini-Nudelauflauf}

      \begin{zutaten}
        1 große & \myindex{Zucchini} (ca. 1 kg) \\
        3--4 & \myindex{Knoblauchzehe}n \\
        2 kleine & \myindex{Chilischote}n \\
        1 kleine Dose & \myindex{geschälte Tomate}n\index{Tomate>geschält} \\
        1 kleine Dose & \myindex{passierte Tomate}n\index{Tomate>passiert} \\
        1 kleiner Becher & \myindex{Ricotta}\index{Käse>Ricotta}
	                   (ital. Frischkäse) \\
        2 Beutel & \myindex{Mozzarella}\index{Käse>Mozzarella} \'a 125 g \\
        60 g & frisch geriebener \myindex{Parmesan}\index{Käse>Parmesan}
	       (und eventuell entrindete Parmesanreste) \\
        & \myindex{Salz} \\
        & \myindex{Pfeffer} aus der Mühle \\
        & \myindex{Olivenöl}\index{Oel=Öl>Oliven-} zum Braten \\
        250 g & kleine \myindex{Nudeln}
      \end{zutaten}

      \garzeit{35}

      \begin{zubereitung}
        Nudeln garen, Zucchini längs in ca. 1~cm dicke Scheiben hobeln. In der
	Pfanne in Olivenöl braten, salzen und pfeffern nach dem Wenden. Danach
	in geölte Auflaufform nebeneinander schichten. Wenn die Hälfte der
	Scheiben gebraten ist, Ricotta und eventuell entrindete Parmesanreste
	darüber geben. \\
        Tomatensoße bereiten: Schältomaten abgießen und mit passierten Tomaten
	in einen Topf geben, gehackte Chilischoten, gepreßten Knoblauch, Salz,
	Pfeffer dazu. Abschmecken nach dem Kochen. \\
        Über die erste Hälfte der Zucchinischeiben die gegarten Nudeln
	verteilen. Darauf den geriebenen Parmesan und abschließend restliche
	Zucchinischeiben. Tomatensoße darüber und Mozzarella in Scheiben dazu.
	Im Backofen bei \grad{180} ca. 35~Minuten garen. \\
      \end{zubereitung}

    \mynewsection{Zucchiniröllchen mit Feta}

      \begin{zutaten}
	4 & kleine bis mittlere \myindex{Zucchini} (etwa 17~cm lang) \\
	1 & große \myindex{Zwiebel} \\
	2 & größere \myindex{Knoblauchzehe}n \\
        & \myindex{Olivenöl}\index{Oel=Öl>Oliven-} \\
        & \myindex{schwarzer Pfeffer}\index{Pfeffer>schwarz} \\
        & \myindex{Salz} \\
        & \myindex{Oregano} (getrocknet) \\
	1 Packung & griechischer \myindex{Feta}\index{Käse>Feta} (200 g) \\
      \end{zutaten}

      \personen{2}

      \begin{zubereitung}
        Zucchini abreiben, Enden abschneiden und dünn auf dem Gemüsehobel
	der Länge nach abhobeln. In der Pfanne die Zucchini nach und nach
	auf jeder Seite kurz anbräunen. Nach dem Wenden die gebratene Seite
	mit Salz und Pfeffer würzen, die zweite Seite nach dem Herausheben auf
	Küchenkrepp zusätzlich mit wenig Oregano bestreuen. \\
	Scheiben abkühlen lassen. Aufflaufform mit Öl einreiben. Die 
	Zucchini-Reste klein hacken, Zwiebeln hacken. In der Pfanne Zwiebel
	dünsten, Zucchini und gepreßte Knoblauchzehen dazu, dann salzen und
	pfeffern. Die Masse in die Auflaufform geben. \\
	Feta in Würfel schneiden. Zucchinischeiben auslegen, 1~Fetawürfel
	reingeben und eng aufrollen und senkrecht in die Auflaufform stellen. \\
        Bei \grad{200} im Backofen 15--20~Minuten garen. Dazu passen heiße
	Bandnudeln mit etwas Butter. \\
	Oder als Vorspeise: für jeden 3--4~Röllchen. \\
      \end{zubereitung}

    \mynewsection{Gefüllte Zucchinitürmchen}

      \begin{einleitung}
        Dafür dürfen die Zucchini ruhig ein bisschen dicker und größer sein,
	also etwa so, wie man sie meistens im Gemüseladen findet. Sie werden
	oben und unten gekappt, dann quer in Stücke von circa 5~Zentimetern
	Länge geschnitten. \\
	Mit einem Löffel oder Kugelbohrer aushöhlen, dabei darauf achten, daß
	sie an einem Ende nicht durchbohrt werden, sondern einen Boden
	behalten. Jetzt haben wir kleine Förmchen, die geradezu danach
	schreien, gefüllt zu werden. \\
      \end{einleitung}

      \begin{zutaten}
        3-4 & mittelgroße \myindex{Zucchini} \\
        2 & junge \myindex{Zwiebel}n \\
        2 & \myindex{Kartoffel}n \\
        1 & milde \myindex{Chilischote} \\
        4 Eßlöffel & \myindex{Olivenöl}\index{Oel=Öl>Oliven-} \\
        2--3 & \myindex{Knoblauchzehe}n \\
        100 g & \myindex{gekochter Schinken}\index{Schinken>gekocht} in zwei Scheiben \\
        2 gehäufte Eßlöffel & \myindex{Semmelbrösel} \\
        2 Eßlöffel & \myindex{Pinienkerne} \\
        & \myindex{Curry}pulver \\
        & \myindex{Salz} \\
        & \myindex{Pfeffer} \\
        & \myindex{glatte Petersilie}\index{Petersilie>glatt} oder \myindex{Basilikum} \\
        1 Glas & \myindex{Weißwein}\index{Wein>weiß} \\
        ca. \brev{} l & \myindex{Brühe} \\
        1 Tasse & gewürfeltes \myindex{Tomate}nfleisch \\
      \end{zutaten}

      \personen{4--6}

      \begin{zubereitung}
        Die Zucchini wie oben beschrieben vorbereiten. Für die Füllung die
	Zwiebeln und Kartoffeln schälen und möglichst akkurat fein (circa
	3~Millimeter) würfeln. In einer Pfanne im heißen Öl sanft andünsten und
	nur ganz wenig bräunen. Dabei salzen und pfeffern. Chili entkernen und
        winzig würfeln, fein gehackten Knoblauch zufügen. Einen Teil vom
	Basilikum oder der Petersilie fein hacken und untermischen. Alles in
	eine Schüssel geben. Jetzt noch einmal zwei Löffel Olivenöl in der
	Pfanne erhitzen. \\
        Die Pinienkerne und die Brösel darin anrösten und mit Currypulver
	würzen. Anschließend den fein gehackten Schinken hinzufügen und mit den
	anderen Zutaten in der Schüssel vermischen. In die ausgehöhlten
	Zucchinitürmchen füllen. Den Bratensatz mit etwas Brühe und Wein
	loskochen. \\
	Die Türmchen nebeneinander in eine feuerfeste Form setzen, mit Olivenöl
	beträufeln und im \grad{200} heißen Ofen zunächst 10~Minuten braten,
	erst dann den Sud aus der Pfanne angießen. Insgesamt eine gute halbe
	Stunde gar schmurgeln lassen. \\
        Tip: Wenn die Kartoffeln am Pfannenboden festsetzen, einfach etwas
	Brühe und Weißwein zum Lösen dazugeben. \\
        Am Ende Tomatenwürfel unter diesen Schmorsaft rühren, abschmecken und
	mit den restlichen Kräutern bestreut servieren. \\
        Beilage: Entweder pur als Vorspeise servieren, also nur mit etwas
	frischem Brot. Oder mit Kartoffelpüree als ganzes Essen. \\
        Getränk: Dazu passt ein sommerleichter aber dennoch herzhafter
	Weißwein. Zum Beispiel ein aromatischer Zlatan Otok von der Insel Hvar
	in Kroatien. \\
      \end{zubereitung}

    \mynewsection{Zucchinitaler}

      \begin{einleitung}
        Hübsch als Fingerfood, zum Aus-der-Hand-Essen oder auch prima fürs
	Partybuffet und zum Grillfest. \\
      \end{einleitung}

      \begin{zutaten}
        2-3 & dickere \myindex{Zucchini} \\
        2 & \myindex{Ei}er \\
        100 g & geriebener \myindex{Peccorino}\index{Käse>Peccorino}
	        (Schafmilchkäse) \\
        50 g & \myindex{Mehl} \\
        & \myindex{Salz} \\
        & \myindex{Pfeffer} \\
        & \myindex{Curry}pulver oder \myindex{Ras el Hanout}
	                             (eine marokkanische Gewürzmischung) \\
        & \myindex{Olivenöl}\index{Oel=Öl>Oliven-} zum Braten \\
      \end{zutaten}

      \personen{6--8}

      \begin{zubereitung}
        Die Zucchini schräg in gleichmäßig starke, nicht zu dicke Scheiben
	schneiden. Eier, geriebenen Pecorino und Mehl glatt quirlen, mit Salz,
	Pfeffer und Currypulver oder Ras el Hanout würzen. Zucchinischeiben mit
	diesem Teig bestreichen, mit einer zweiten Scheibe bedecken und auch
	deren Oberseite mit Teig bestreichen. \\
        Dann in Olivenöl auf beiden Seiten langsam golden braten. \\
        Tip: Die Taler schmecken warm aber auch kalt, dazu ein
	Joghurt-Basilikum-Dip, der rasch angerührt ist: Vollfetten Joghurt ---
	ruhig sogar den festen griechischen oder türkischen Joghurt ---
	mit Salz, Pfeffer, durchgepresstem Knoblauch, Zitronenschale und -saft
	glatt rühren. Das Basilikum sehr fein schneiden und am Ende
	unterrühren. Übrigens: statt Basilikum paßt auch sehr gut Dill! \\
        Getränk: Dazu empfehlen wir einen leichten, sommerlichen Wein, zum
	Beispiel einen Rivaner aus Württemberg. Der schmeckt frisch und leicht
	fruchtig. \\
      \end{zubereitung}

    \mynewsection{Zucchinibrot}

      \begin{einleitung}
        Das klingt vielleicht verrückt --- aber es schmeckt prima, ist herrlich
	saftig und hält sich lange frisch. \\
      \end{einleitung}

      \begin{zutaten}
        800 g & \myindex{Zucchini} \\
        1 Eßlöffel & \myindex{Salz} \\
        & \myindex{Pfeffer} \\
        1--2 & \myindex{Knoblauchzehe}n \\
        1--2 & \myindex{Chili}s (zum Beispiel 1 rote und 1 grüne) \\
        2 Eßlöffel & \myindex{Olivenöl}\index{Oel=Öl>Oliven-} \\
        1 kg & \myindex{Mehl} (am besten Type 812) \\
        20 g & \myindex{Hefe} \\
        1 Tütchen & \myindex{Sauerteigansatz}
	            (oder ein Händchen voll fertiger \myindex{Sauerteig}) \\
        1 gestrichener Eßlöffel & \myindex{Salz} \\
      \end{zutaten}

      \begin{zubereitung}
        Wir rühren den Brotteig statt mit Wasser mit püriertem Zucchinifleisch
	an. Die Zucchini dafür würfeln, mit Salz, Pfeffer, Knoblauch, Chili und
	Olivenöl im Mixer zerkleinern. Wenn es zu trocken wird, kann ruhig mit
	ein wenig Wasser nachgeholfen werden. \\
        Das Mehl in die große Rührschüssel der Küchenmaschine geben, die in
	etwas lauwarmem Wasser aufgelöste Hefe in die Mitte gießen und
	zugedeckt zehn Minuten gehen lassen. Dann den nach Vorschrift
	angerührten Sauerteig zufügen --- der macht das Brot besonders luftig
	und saftig, und gibt ihm schöne Poren. \\
        Sauerteigansatz gibt es in Tütchen im Reformhaus oder im Supermarkt bei
	den Mehlen. Wenn man ein Händchen von diesem Teig in einem Schraubglas
	in den Kühlschrank stellt, dann hat man den fertigen Sauerteig beim
	nächsten Mal direkt zur Hand. \\
        Die Maschine auf langsame Stufe einschalten, jetzt langsam das
	Zucchinipüree zufügen, soviel, bis der Teig sich glatt vom Schüsselrand
	löst. (Auf keinen Fall gleich alles! Wie viel hängt vom Mehl ab, seinem
	Alter, Trockenheit, der Luftfeuchtigkeit.) Den Teig dann aber noch gut
	8--10~Minuten von der Maschine durcharbeiten lassen. Danach mit
	bemehlten Händen herausheben und in einer bemehlten Schüssel zugedeckt
	mindestens 1, ruhig aber auch 2~Stunden an einem warmen Ort gehen
	lassen. \\
        Zwei oder drei Brote formen, nochmal 20--30~Minuten gehen lassen, damit
	es die Form behält, und schließlich backen: bei \grad{200} insgesamt
	eine knappe Stunde. Nach dem Reinschieben ein Glas Wasser auf dem
	Ofenboden verzischen lassen. Nach 15~Minuten die Hitze auf \grad{180}
	herunterschalten. \\
        Wir haben das Brot, sowie die Zucchinitürmchen, in unserem Steinofen
	(Pizzaofen) im Garten gebacken. So ein Ofen ist zwar keine preiswerte
	Angelegenheit, lohnt sich aber unserer Meinung nach auf jeden Fall.
	Siehe auch den Küchentipp. \\
        Tip: Aus dem Teig lassen sich auch sehr gut Focaccia machen,
	ligurische Fladenbrote. Dafür den Teig fingerdick etwa tellergroß
	ausrollen, mit Olivenöl bestreichen, salzen und beliebig mit Kräutern
	und Gemüse belegen. Im Ofen knusprig ausbacken. \\
      \end{zubereitung}

    \mynewsection{Zucchiniblüten}

      \begin{einleitung}
        Wer einen Garten hat, kann sie einfach ernten. Den Gemüsehändler wird
	man überreden müssen, sie zu besorgen, denn die Blüten sind
	empfindlich. Manchmal findet man sie als essbare Blüten verpackt. Sie
	kommen aus Italien oder Frankreich. In Ausbackteig getaucht und
	frittiert --- wie für die Zucchinitaler --- sind sie ein entzückender
        Appetithappen. Man kann sie jedoch auch füllen: eine elegante und
	ungewöhnliche Vorspeise. \\
      \end{einleitung}

      \begin{zutaten}
        1 & Portion Zucchiniaufstrich (siehe unten) \\
        4--8 & \myindex{Zucchiniblüte}n
	       (am schönsten natürlich die weiblichen,
	        mit den kleinen Früchten) \\
        1 & \myindex{Eiweiß} \\
      \end{zutaten}

      \begin{zutat}{Zucchiniaufstrich}
        2 & kleine \myindex{Zucchini} \\
        & \myindex{Olivenöl}\index{Oel=Öl>Oliven-} \\
        2 & \myindex{Knoblauchzehe}n \\
        & \myindex{Salz} \\
        & \myindex{Pfeffer} \\
        200 g & \myindex{Frischkäse}\index{Käse>Frisch-} \\
        & \myindex{Zitrone}nsaft \\
        & \myindex{Basilikum}blättchen oder \myindex{Schnittlauch} \\
      \end{zutat}

      \begin{zutat}{Tomatenvinaigrette}
        2-3 & reife \myindex{Tomate}n \\
        & \myindex{Salz} \\
        & \myindex{Pfeffer} \\
        3 Eßlöffel & \myindex{Olivenöl}\index{Oel=Öl>Oliven-} \\
        & \myindex{Balsamico-Essig}\index{Essig>Balsamico-} \\
        & \myindex{Sherryessig} \\
        1 & kleine \myindex{Zwiebel} \\
        2 & kleine \myindex{Knoblauchzehe}n \\
        & \myindex{Basilikum} \\
      \end{zutat}

      \personen{4}

      \begin{zubereitung}
        Die Blüten sehr behutsam behandeln, damit sie nicht abbrechen.
	Vorsichtig öffnen, den dicken, wattigen Stempel in der Mitte vorsichtig
	abknipsen. Damit Blüten und Fruchtansätze gleichmäßig garen, die
	Früchte zwei bis drei mal längs einschneiden. \\
        Aufstrich aus Zucchini: Dazu zwei kleine Zucchini in winzige Würfelchen
	hobeln, in zwei Esslöffel Olivenöl andünsten, zwei Knoblauchzehen
	gehackt oder gepresst hinzugeben, salzen und pfeffern. Am Ende einige
	Basilikumblättchen oder Schnittlauch sehr fein schneiden und
	untermischen. Etwas abgekühlt unter den Frischkäse rühren und mit etwas
	Zitronensaft abschmecken. \\
        Für die Füllung den Zucchiniaufstrich mit dem Eiweiß glatt rühren (gibt
	eine gute Bindung) und nochmals abschmecken. Die Masse mit einem
	Plastik- oder Spritzbeutel in die Blüten füllen. Die Blütenblätter über
	der Füllung schließen. \\
        Die Früchte mit ihren Blüten auf einen Bambuskorb (vorher mit Öl
	bestreichen) oder in einen Dämpftopf betten. Circa 8--10~Minuten über
	Dampf garen. \\
        Dazu schmeckt eine Tomatenvinaigrette: Dafür die Tomaten häuten,
	entkernen. Die Kerne in einem Mixbecher sammeln, mit Salz, Pfeffer,
	Olivenöl, Balsamico und Sherryessig zu einer cremigen Emulsion mixen.
	Das Tomatenfleisch würfeln, mit Salz und Pfeffer würzen und in einem
	Sieb abtropfen lassen (den Saft auffangen und später trinken ---
	schmeckt köstlich!). In eine Schüssel geben, mit Olivenöl beträufeln
	und die feingehackte Zwiebel und Knoblauch untermischen. Mit gehacktem
	Basilikum bestreuen. \\
        Zum Servieren die gefüllten Blüten auf Vorspeisentellern anrichten. Das
	Tomatenfleisch drum herum verteilen und mit der Tomatenemulsion
	dekorativ beträufeln. \\
        Beilage: Baguette. \\
        Getränk: Ein würziger Weißwein, zum Beispiel ein weißgekelterter
	Spätburgunder vom Bodensee. \\
      \end{zubereitung}

    \mynewsection{Gefüllte Zucchini mit Lauch + Curry}
      
      \begin{zutaten}
        ca. 1 kg & kleine bis mittelgroße \myindex{Zucchini}%
	           \footnote{\underline{Niemals} dicke große
		             Zucchini nehmen. Sie schmecken
			     gar nicht.} (16--17~cm) \\
	500 g & \myindex{Lauch} \\
	2 & \myindex{Knoblauchzehe}n \\
	50 g & \myindex{Butter} \\
	\brea{} l & \myindex{Gemüsebrühe} \\
	& \myindex{Curry} \\
	& \myindex{Pfeffer} \\
	& \myindex{Muskatnuß} \\
	& \myindex{Salz} \\
	& \myindex{Kräuter der Provence} \\
	50 g & \myindex{Sonnenblumenkerne} \\
	150 g & \myindex{Schafkäse}\index{Käse>Schaf-} \\
	\breh{} Becher & \myindex{Sahne} \\
      \end{zutaten}

      \personen{4}

      \begin{zubereitung}
        Zucchini entkernen, Fruchtfleisch bis auf einen schmalen Rand
	(\breh{}~cm) herausnehmen und würfeln. \\
	Lauch putzen und waschen, in feine Ringe schneiden und mit
	feingehackten Knoblauchzehen in der Butter andünsten. Mit Gemüsebrühe
	ablöschen und ca.~10~Minuten bei kleiner Hitze garen. Abschmecken mit
	Curry, Salz, Kräutern, Pfeffer und Muskat. \\
	Das Fruchtfleisch, Sonnenblumenkerne, zerbrökelten Schafkäse mit den
	Lauchringen mischen und die Sahne untermischen. \\
	Ausgehöhlte Zucchini salzen und pfeffern und mit der Masse füllen.
	Die Auflaufform bei \grad{200} ca.~30~Minuten im Ofen lassen und
	danach abkühlen lassen. \\
      \end{zubereitung}

    \mynewsection{Zucchinipuffer mit Tsatsiki}

      \begin{zutaten}
      \end{zutaten}
      \begin{zutat}{Zucchinipuffer}
        1 & \myindex{Zucchini} \\
        1 & \myindex{Lauchzwiebel}\index{Zwiebel>Lauch-} \\
	1 & \myindex{Ei} \\
	1 Eßlöffel & \myindex{Dill} \\
	1 Eßlöffel & \myindex{Petersilie} \\
	3 Eßlöffel & \myindex{Mehl} \\
	etwas & \myindex{Schafkäse}\index{Käse>Schaf-} (nach Belieben) \\
        & \myindex{Olivenöl}\index{Oel=Öl>Oliven-} zum Braten \\
	& \myindex{Salz} \\
	& \myindex{Pfeffer} \\
      \end{zutat}
      \begin{zutat}{Tsatsiki}
        500 g & \myindex{griechischer Joghurt}\index{Joghurt>griechisch} oder
	        \myindex{Quark} \\
	3 kleine & \myindex{Knoblauchzehe}n \\
	1 & \myindex{Schlangengurke}\index{Gurke>Schlangen-} \\
        1 Eßlöffel & \myindex{Olivenöl}\index{Oel=Öl>Oliven-} \\
	1 Eßlöffel & \myindex{Dill} \\
	& \myindex{Salz} \\
	& \myindex{Pfeffer} aus der Mühle \\
      \end{zutat}

      \personen{4}

      \begin{zubereitung}
        Puffer: Zucchini grob raspeln, Zwiebel in feine Ringe, Kräuter fein
	hacken, Schafkäse zerbröseln. Ei und Mehl dazu, Salz + Pfeffer. Alles
	gut mischen. Kräftig würzen. Nach Bedarf Mehl zugeben. \\
	Eßlöffelgroße Portionen auf großer Flamme in reichlich Olivenöl
	goldgelb ausbacken. \\
	\\
	Tsatsiki: Joghurt im Küchentuch abtropfen lassen. Gurke waschen und
	ungeschält grob raspeln. Knoblauch fein reiben, Dill fein hacken. Alle
	Zutaten mit Olivenöl mischen, kräftig mit Salz + Pfeffer würzen.\\
	\\
	Puffer mit Tsatsiki anrichten. Dazu warmes Baguette und ein würziger
	Weißwein. \\
	\\
	Tip: Griechischer Joghurt schmeckt am besten. Es gibt ihn in
	griechischen Läden, Feinkostladen oder gut sortiertem Supermarkt.\\
      \end{zubereitung}

    \mynewsection{Bandnudeln mit Zucchini}

      \begin{zutaten}
        50 g & \myindex{Sahnejoghurt}\index{Joghurt>Sahne-} \\
	1 Eßlöffel & gehacktes \myindex{Basilikum} \\
        2 Eßlöffel & \myindex{saure Sahne}\index{Sahne>sauer} \\
	1 & gepreßte \myindex{Knoblauchzehe} \\
	& \myindex{Salz} \\
	& \myindex{Pfeffer} aus der Mühle \\
	400 g & kleine \myindex{Zucchini} oder 3 mittelgroße \\
	200 g & \myindex{Bandnudeln}\index{Nudeln>Band-} \\
        1 Eßlöffel & \myindex{Olivenöl}\index{Oel=Öl>Oliven-} \\
      \end{zutaten}

      \personen{2}

      \begin{zubereitung}
        Joghurt, saure Sahne, Basilikum, Knoblauch, Salz und Pfeffer verrühren.
	Zucchini putzen, in dünne Scheiben schneiden, in sehr heißem Öl braten,
	auf Küchenkrepp abtropfen lassen. Mit Salz und Pfeffer würzen.
	Bandnudeln kochen. Nudeln mit Soße und Zucchini vermengen. \\
      \end{zubereitung}

    \mynewsection{Zucchinifrittata}

      \begin{einleitung}       
        Jetzt, zur Zucchinizeit, sind sie die angesagte Zutat. Man kann sonst
        natürlich eine Frittata mit allem Möglichen zubereiten. Schnell geht es
        auch mit gekochten Kartoffeln oder einem Nudelrest. Im Prinzip läuft
	das Grundrezept wie folgt. \\
      \end{einleitung}       

      \begin{zutaten}
	4--6 kleine bis mittlere & \myindex{Zucchini} \\
        2 Eßlöffel & \myindex{Olivenöl}\index{Oel=Öl>Oliven-} \\
        1 kleine & \myindex{weiße Zwiebel}\index{Zwiebel>weiß} \\
	1--2 & \myindex{Knoblauchzehe}n \\
	& \myindex{Salz} \\
	& \myindex{Pfeffer} \\
	3--4 & \myindex{Ei}er \\
	100 g & geriebener Käse (\myindex{Parmesan}\index{Käse>Parmesan} oder
	        \myindex{alter Gouda}\index{Gouda>alt}\index{Käse>Gouda>alt})
		\\
	& \myindex{Rauke} oder anderes Kraut \\
	1 & \myindex{Tomate} eventuell \\
      \end{zutaten}

      \personen{2}

      \begin{zubereitung}
        Die Zucchini auf einem Stiftehobel in nicht zu feine Stifte schneiden
	(notfalls eine grobe Raffel nehmen). In einer kleinen Pfanne (nicht
	mehr als 20, höchstens 22~cm~\durchmesser{}, sonst wird die Frittata
	zu flach) im heißen Olivenöl andünsten, dabei salzen und pfeffern. \\
	Die gehackte Zwiebel sowie den zerquetschten Knoblauch hinzufügen und
	mitdünsten. Zum Schluß noch fein geschnittene Rauke unterrühren (oder
	ein anderes Kraut nach Wahl). Die Tomate häuten, entkernen und in
	kleine Stücke schneiden. \\
	In einer Schüssel die Eier verquirlen, den Käse und die
	Tomatenstückchen unterrühren. Ebenfalls salzen und pfeffern. Die
	angedünsteten Zucchini hineinschütten, dabei umrühren, damit durch
	deren Hitze das Ei nicht gerinnt. \\
	Ein Löffelchen Öl in der Pfanne erhitzen, den gesamten Inhalt aus der
	Schüssel hineinkippen und glatt schütteln. Auf mittlerem Feuer stocken
	lassen. \\
	Erst wenn die gesamte Masse nahezu fest geworden ist, einen Topfdeckel
	oder eine Tortenplatte auflegen, das Ganze umdrehen und auf diese Weise
	stürzen, dann die Frittata in die Pfanne zurückgleiten lassen. Auch
	auf dieser Seite bräunen, dafür die Hitze ein wenig verstärken und
	womöglich noch ein wenig Öl darunterfließen lassen. \\
	Die Frittata wird wie ein Kuchen in Stücke geschnitten. \\
	Beilage: Dazu paßt am besten ein knackiger grüner Salat. \\
      \end{zubereitung}

    \mynewsection{Mit Tomaten und Mozzarella gefüllte Zucchini}%
               \glossary{Zucchini gefüllt mit Tomaten und Mozzarella}

      \begin{zutaten}
        8 & \myindex{Zucchini} \\
        4 & \myindex{Tomate}n \\
        250 g & \myindex{Mozzarella}\index{Käse>Mozzarella} \\
        & \myindex{Jodsalz}\index{Salz>Jod-} \\
        & \myindex{Pfeffer} \\
        \breh{} Würfel & \myindex{klare Gemüsebrühe}\index{Gemüsebrühe>klar} \\
        8 Eßlöffel & \myindex{Olivenöl}\index{Oel=Öl>Oliven-}
	             mit Gewürzeinlage \\
        & \myindex{Knoblauch} \\
        1 Topf & frisches \myindex{Basilikum} \\
      \end{zutaten}

      \personen{4}
      \garzeit{50}

      \begin{zubereitung}
        Zucchini und Tomaten waschen, abtrocknen. Zucchini der Länge nach
	halbieren. Fruchtfleisch mit einem Teelöffel herauskratzen, grob
	hacken, auf ein Küchentuch legen, das Wasser auspressen. Tomaten und
	Mozzarella in kleine Würfel schneiden, dann mit dem Fruchtfleisch
	mischen. Die Gemüse-Mozzarella-Masse mit Salz und Pfeffer würzen und
	auf die Zucchinihälften streichen. Backofen auf \grad{200} vorheizen.
	\\
	Brühwürfel in 250~ml kochendem Wasser auflösen. Zucchinihälften in eine
	ofenfeste Form legen, heiße Brühe angießen, jeweils mit 1~Eßlöffel
	Knoblauch-Olivenöl beträufeln. 35--40~Minuten auf mittlerer Schiene
	backen. Basilikumblättchen abzupfen, in feine Streifen schneiden und
	über die Zucchini streuen. \\
	Variante: Zucchini in gleich große, zirka 6~cm lange Stücke teilen.
	Bis auf einen dünnen Boden auslöffeln. Zucchini-Fruchtfleisch grob
	hacken. Tomaten und Mozzarella erst in Scheiben, dann in Streifen,
	zuletzt in Miniwürfel schneiden. Alle 3~Zutaten mischen, würzen und
	in die Zucchinistücke füllen. Im Ofen (\grad{200}) zirka 30~Minuten
	gratinieren. \\
      \end{zubereitung}

    \mynewsection{Zucchini-Omelett}

      \begin{zutaten}
        1 kleine & \myindex{Zucchini} (75 g) \\
        1 kleine & \myindex{Zwiebel} \\
        1 & \myindex{Ei} \\
        & \myindex{Salz} \\
        & \myindex{Pfeffer} aus der Mühle \\
        30 g & \myindex{Schafkäse}\index{Käse>Schaf-} \\
        & frischer \myindex{Rosmarin} \\
        \breh{} Teelöffel & Öl\index{Oel=Öl} \\
      \end{zutaten}

      \kalorien{200}

      \begin{zubereitung}
        Die Zucchini in sehr dünne Scheiben, die Zwiebel in feine Ringe
	schneiden. Das Ei mit Salz und Pfeffer verquirlen, den Käse zerbröseln
	und darunterrühren. Eine beschichtete Pfanne mit Öl auspinseln.
	Rosmarinnadeln, Zwiebel und Zucchini darin leicht anbraten. Das
	verquirlte Ei darübergießen und alles zugedeckt auf niedrigster
	Schaltstufe etwa 3~Minuten lang stocken lassen. \\
      \end{zubereitung}

    \mynewsection{Zucchini mit Mozzarella in Tomatensoße}

      \begin{zutaten}
        3 mittelgroße & \myindex{Zucchini} \\
	2 & \myindex{Knoblauchzehe}n \\
	& \myindex{Salz} \\
	& \myindex{Pfeffer} \\
	& Öl\index{Oel=Öl} \\
	3 & Kugeln \myindex{Mozzarella}\index{Käse>Mozzarella} \\
	1 Handvoll & \myindex{Basilikum}, zerteilt \\
      \end{zutaten}
      \begin{zutat}{Soße}
        2--3 kleine & \myindex{Möhre}n, gewürfelt oder in Scheiben \\
	1 große & \myindex{Zwiebel} \\
	1 & \myindex{Chilischote} \\
	1 Eßlöffel & \myindex{Tomatenmark} \\
	& \myindex{Olivenöl}\index{Oel=Öl>Oliven-} \\
	1 Dose & \myindex{Schältomate}\index{Tomate>Schäl-}n \\
	1 Dose & \myindex{Pizzatomate}\index{Tomate>Pizza-}n \\
	1 Teelöffel & \myindex{Salz} \\
	& \myindex{Pfeffer} \\
	& \myindex{Oregano} \\
	1 Prise & \myindex{Zucker} \\
      \end{zutat}

      \garzeit{35}

      \begin{zubereitung}
        Soße: Im Topf Olivenöl erhitzen, Möhren andünsten, dann Zwiebel und
	kleingeschnittene Chilischote zugeben. Tomatenmark zugeben, würzen und
	Dosen Tomaten zugeben. 20--30~Minuten köcheln lassen, dann pürieren
	und über die Zucchinischeiben geben. \\
        Auflauf: Zucchini waschen und längs in Scheiben schneiden (Gemüsehobel
	größte Stufe). In der Pfanne sehr kurz von beiden Seiten anbräunen,
	dann würzen mit Salz und Pfeffer und in Auflaufform geben. Dünne
	Knoblauchscheiben (2~je Scheibe Zucchini) draufgeben, darüber dünne
	Mozzarellascheiben. Neue Scheiben anbraten. Obenauf Basilikum streuen.
	 \\
	Im Backofen 35~Minuten bei \grad{200}. Dazu 250~g Vollkornnudeln. \\
      \end{zubereitung}

    \mynewsection{Zucchini-Kartoffel-Curry-Auflauf}

      \begin{zutaten}
	3--4 & mittelgroße \myindex{Zucchini} \\
	ca. 1 kg & \myindex{Kartoffel}n \\
	2--3 & mittlere \myindex{Möhre}n \\
	4--5 & \myindex{Knoblauch}zehen \\
	4--5 & rote oder grüne \myindex{Pepperoni} \\
	1 Becher & \myindex{Sahne} (200~g) \\
	& \myindex{Salz} \\
	& \myindex{schwarzer Pfeffer}\index{Pfeffer>schwarz} \\
	& \myindex{Curry}pulver \\
	ca. 60 g & \myindex{Parmesan}\index{Käse>Parmesan}käse gerieben \\
	& \myindex{Olivenöl}\index{Oel=Öl>Oliven-} \\
	& \myindex{Margarine} zum Fetten der Auflaufform \\
      \end{zutaten}

      \begin{zubereitung}
        Kartoffeln in der Schale kochen. Möhren putzen und in Scheiben
	schneiden. Pepperoni in Scheiben schneiden (ohne Kerne innen),
	Knoblauch in Scheiben schneiden. Pfanne mit Olivenöl erhitzen und
	Möhren, Pepperoni und Knoblauch anbraten. Dann pfeffern und salzen.
	Etwas Zucker zugeben. Aus der Pfanne nehmen und mit neuem Olivenöl
	die Zucchini (in Scheiben gehobelt) anbraten. Schwenken, würzen mit
	Salz und Pfeffer. Kartoffeln pellen und in Scheiben schneiden. In die
	gefettete Auflaufform geben, salzen und pfeffern. Darüber die
	Möhrenmischung, dann Zucchini, obenauf zum Schluß den Käse. Sahne
	salzen und pfeffern und reichlich Curry dazu, umrühren und über das
	Gemüse geben. \\
        Bei \grad{200} 30~Minuten backen. \\
      \end{zubereitung}
      
    \mynewsection{Zucchini trifolate}

      \begin{zutaten}
        1 kg & \myindex{Zucchini} \\
	6 & \myindex{Knoblauchzehe}n \\
	1 Bund & \myindex{Petersilie} \\
	& \myindex{Salz} \\
	& \myindex{Pfeffer} \\
        & \myindex{Olivenöl}\index{Oel=Öl>Oliven-} \\
      \end{zutaten}

      \personen{6}

      \begin{zubereitung}
        In einer Pfanne auf kleiner Flamme die kleingeschnittenen
	Knoblauchzehen andünsten, die ebenfalls kleingeschnittene Petersilie
	zugeben, danach die in Scheiben geschnittenen Zucchini und Pfeffer
	kurz vor Ende der Garzeit mit etwas Petersilie bestreuen. \\
      \end{zubereitung}
