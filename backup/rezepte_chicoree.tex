
% created Montag, 10. Dezember 2012 16:19 (C) 2012 by Leander Jedamus
% modifiziert Mittwoch, 11. März 2015 17:18 von Leander Jedamus
% modifiziert Montag, 09. März 2015 14:18 von Leander Jedamus
% modified Montag, 10. Dezember 2012 16:29 by Leander Jedamus

 \mynewchapter{\chicoree{}}

    \mynewsection{\chicoree{} überbacken}

      \begin{zutaten}
        500 g & \myindex{\chicoree{}} (4 Stücke, möglichst gleich groß) \\
        8 Scheiben & roher \myindex{Schinken} \\
        60--100 g & geriebener \myindex{Käse} (\myindex{Emmentaler}) \\
        200 g & \myindex{süße Sahne}\index{Sahne>süß} (Schlagsahne) \\
        2 & \myindex{Ei}er \\
        1 Teelöffel & \myindex{Gemüsebrühe} \\
        & \myindex{schwarzer Pfeffer}\index{Pfeffer>schwarz} \\
        & \myindex{Margarine} \\
        & \myindex{Paniermehl} \\
      \end{zutaten}

      \garzeit{45}

      \begin{zubereitung}
        Die \chicoree{}-Stauden putzen, halbieren und den bitteren Strunk
	herausschneiden. Die Hälften zusammenlegen und überlappend mit rohem
	Schinken umwickeln. Die Form mit Margarine fetten und mit Paniermehl
	ausbröseln. Die \chicoree{} nebeneinander in die Form legen und die
	Eier-Sahnemischung darübergießen. Abschließend mit Käse bestreuen und
	bei \grad{220} für 45~Minuten in den Backofen geben. Dazu reicht man
	kleine Pellkartoffeln. \\
      \end{zubereitung}

    \mynewsection{\chicoree{} in gekochtem Schinken}

      \begin{zutaten}
        1 mitteldicke & \myindex{\chicoree{}} pro Person \\
        1 Scheibe & \myindex{gekochter Schinken}\index{Schinken>gekocht}
	            pro Person \\
        1 & holländische Soße (siehe Seite \pageref{hollandsosse}) \\
        60 g & geriebener \myindex{Käse} \\
      \end{zutaten}

      \garzeit{20}

      \begin{zubereitung}
        \chicoree{} in Salzwasser 8~Minuten garen, in Scheibe Schinken
	einrollen und in gefettete Auflaufform schichten. Holländische Soße
	darüber geben und obenauf geriebenen Käse. Im Backofen bei
	\grad{175--200} ca. 20~Minuten überbacken. \\
      \end{zubereitung}

    \mynewsection{\chicoree{} Canapes}

      \begin{zutaten}
        1 mittlere & \myindex{\chicoree{}} pro Person \\
        & \myindex{roher Schinken}\index{Schinken>roh} \\
        & geräucherter \myindex{Lachs}\index{Fisch>Lachs} \\
        & \myindex{Forellenkaviar}\index{Kaviar>Forellen-} \\
        & \myindex{Käsecreme} \\
        & \myindex{Krabbencocktail} \\
        & gekochte \myindex{Ei}er \\
      \end{zutaten}

      \begin{zubereitung}
        \chicoree{} waschen und die großen Blätter auf einer Platte anordnen.
	Verzieren mit rohem Schinken oder geräuchertem Lachs usw. (siehe oben).
	\\
      \end{zubereitung}

    \mynewsection{\chicoree{} mit Eimasse überbacken}

      \begin{zutaten}
        500 g & \myindex{\chicoree{}} \\
        30 g & \myindex{Fett} \\
        2--3 & \myindex{Ei}er \\
        3 Eßlöffel & \myindex{Butter}flöckchen \\
        60 g & geriebener \myindex{Emmentaler}\index{Käse>Emmentaler} \\
        & \myindex{Jodsalz}\index{Salz>Jod-} \\
        & \myindex{weißer Pfeffer}\index{Pfeffer>weiß} \\
      \end{zutaten}

      \garzeit{35}

      \begin{zubereitung}
        \chicoree{} ganz oder in Stücke geschnitten in Fett dünsten und salzen.
	In Auflaufform geben und mit Eierguß übergießen, obenauf
	Butterflöckchen und Käse. Backen 35~Minuten bei \grad{200}. \\
        Dazu: Kartoffeln und Feldsalat. \\
      \end{zubereitung}

    \mynewsection{\chicoree{}-Nachspeise}

      \begin{zutaten}
        2 Stück & \myindex{\chicoree{}} \\
        2 & rotschalige Äpfel\index{Aepfel=Äpfel} \\
        1 kleine Dose & \myindex{Mandarinen} \\
        2 Eßlöffel & \myindex{Grand Marnier} (Likör) \\
        3 Eßlöffel & \myindex{Zitrone}nsaft \\
        3 Eßlöffel & \myindex{Zucker} \\
        & \myindex{Schokoladeneis}\index{Eis>Schokoladen-} \\
      \end{zutaten}

      \begin{zubereitung}
        Eßlöffel Zitronensaft mit Zucker verrühren, 2~Äpfel mit der Schale
	raspeln und sofort mit dem Zitronensaft mischen. \chicoree{} waschen,
	die ganzen Blätter fein schneiden und mit den Äpfeln mischen.
	Mandarinen dazugeben und Grand Marnier darüber. Mischen und auf
	Dessertschalen verteilen und mit Schokoladeneis servieren. \\
      \end{zubereitung}

    \mynewsection{Warmer \chicoree{} mit Birne und Blauschimmelkäse}%
              \glossary{\chicoree{}>warm}

      \begin{zutaten}
        3--4 & \myindex{\chicoree{}} \\
        1 & reife \myindex{Tafelbirne}\index{Birne>Tafel-} oder
	    \myindex{Williamsbirne}\index{Birne>Williams-} \\
        1 Eßlöffel & \myindex{Zitrone}nsaft \\
        100 g & \myindex{Blauschimmelkäse} (z.B.
	        \myindex{Gorgonzola}\index{Käse>Gorgonzola}) \\
        1 Eßlöffel & \myindex{Olivenöl}\index{Oel=Öl>Oliven-} \\
        & \myindex{Jodsalz}\index{Salz>Jod-} \\
        & \myindex{weißer Pfeffer}\index{Pfeffer>weiß} \\
      \end{zutaten}

      \begin{zubereitung}
        \chicoree{} putzen, äußere Blätter abtrennen. Birne schälen, würfeln
	und mit Zitronensaft beträufeln. Käse in Würfel schneiden. Öl im Wok
	erhitzen und \chicoree{}blätter mit der Birne ca. 3~Minuten andünsten.
	Käse dazugeben und eine Minute miterhitzen, bis der Käse anfängt zu
	schmelzen. Abschmecken mit Salz und Pfeffer. \\
        Paßt gut zu Rindersteak und French Fries. \\
      \end{zubereitung}

    \mynewsection{\chicoree{} mit Pasta und Lachs}

      \begin{zutaten}
        2--3 & \myindex{\chicoree{}} \\
        200g & grüne \myindex{Tagliatelle} (oder andere Pasta) \\
        2 Eßlöffel & \myindex{Olivenöl}\index{Oel=Öl>Oliven-} \\
        2--3 Eßlöffel & \myindex{Senf} \\
        100 ml & trockener \myindex{Weißwein}\index{Wein>weiß} oder \myindex{Gemüsebouillon} \\
        100 ml & \myindex{süße Sahne}\index{Sahne>süß} \\
        200 g & \myindex{geräucherter Lachs}\index{Lachs>geräuchert}
	        \index{Fisch>Lachs>geräuchert} \\
        & \myindex{Dill} zum Garnieren \\
        & \myindex{Jodsalz}\index{Salz>Jod-} \\
        & \myindex{weißer Pfeffer}\index{Pfeffer>weiß} \\
      \end{zutaten}

      \personen{2}

      \garzeit{10--15}

      \begin{zubereitung}
        Nudeln al dente kochen und abgießen. \chicoree{} waschen und der Länge
	nach zweimal durchschneiden. Öl im Wok erhitzen und den \chicoree{}
	unter ständigem Rühren 3--4~Minuten garen. Aus dem Wok nehmen und
	beiseite stellen. Senf, Wein und Sahne in die Pfanne geben und zur Soße
	einkochen lassen, abschmecken. Lachs in Stücke schneiden und in die
	Soße geben, ebenso den \chicoree{}. Kurz aufwärmen. Pasta auf Teller
	geben und \chicoree{} mit der Lachssoße darübergeben. Mit Dill
	garnieren. \\
      \end{zubereitung}

    \mynewsection{\chicoree{}-Kartoffel-Auflauf}

      \begin{zutaten}
        4 & \myindex{\chicoree{}} \\
	7--8 große & \myindex{Kartoffel}n \\
	2 Becher & \myindex{saure Sahne}\index{Sahne>sauer} \'a 200 g \\
	100 g & \myindex{Schinkenwürfel} \\
	3 & \myindex{Ei}er \\
	& geriebener \myindex{Käse} \\
	& \myindex{Salz} \\
	& \myindex{Pfeffer} \\
	& \myindex{Cayennepfeffer}\index{Pfeffer>Cayenne-} \\
	& \myindex{Worcestershiresoße} \\
	& geriebene \myindex{Muskatnuß} \\
      \end{zutaten}

      \begin{zubereitung}
        Kartoffeln schälen, dicke Scheiben hobeln und angaren. Schinkenwürfel
	anbraten. \chicoree{} vom Strunk befreien. \\
	Auflaufform fetten, Kartoffeln einschichten, salzen und pfeffern.
	Darüber Schinkenwürfel. Darauf \chicoree{}hälften legen, mit Käse
	bestreuen. \\
	Soße aus saurer Sahne und Gewürzen bereiten, Eier einrühren und über
	das Gemüse geben. Bei \grad{200} 40~Minuten in den Backofen. \\
      \end{zubereitung}

    \mynewsection{Geschmorter \chicoree{} und Radicchio}%
               \glossary{\chicoree{} und Radicchio geschmort}%
               \glossary{Radicchio und \chicoree{} geschmort}

      \begin{einleitung}
        Bei dieser Garmethode tritt die typische Bitterkeit besonders hervor.
        Das Gemüse schmeckt auch wunderbar kalt. In diesem Fall ist der
	längliche Radicchio di Treviso besser geeignet als die rundköpfige Rose
	von Chioggia. \\
      \end{einleitung}

      \begin{zutaten}
        4 & \myindex{Radicchio}sprossen \\
        4 & \myindex{\chicoree{}}sprossen \\
	& \myindex{Salz} \\
	& \myindex{Pfeffer} \\
	& \myindex{Olivenöl}\index{Oel=Öl>Oliven-} \\
	& frisch geriebener \myindex{Parmesan}\index{Käse>Parmesan}
	  (nach Belieben) \\
      \end{zutaten}

      \personen{4--6}

      \begin{zubereitung}
        Die Kolben putzen, Wurzelansatz und welke Außenblätter entfernen. Die
	Sprossen längs halbieren. Wenn man sie nicht sofort ins heiße Rohr
	schieben will, sie kurz in kochendes Salzwasser tauchen, damit sie sich
	nicht verfärben. In eine feuerfeste Schale betten, mit Salz und Pfeffer
	würzen, großzügig mit Olivenöl beträufeln. Im \grad{250} heißen
	Backofen etwa 15~Minuten schmurgeln. Direkt aus dem Ofen sofort in der
	Form servieren. Wer mag, bestreut alles noch heiß mit frisch
	geriebenem Parmesan. \\
	Tip: Man kann die mit Olivenöl marinierten Gemüsekolben auch wunderbar
	grillen: Zuerst eine Weile in Olivenöl drehen und wenden, dabei ruhig
	schon mit Salz und Pfeffer würzen. Schließlich auf dem Gartengrill oder
	auf dem Rost unter dem Ofengrill rösten. Dabei drehen, damit das Gemüse
	rundum knusprig wird. \\
      \end{zubereitung}

    \mynewsection{Panierte \chicoree{}}\glossary{\chicoree{} paniert}

      \begin{zutaten}
	12 kleine & \myindex{\chicoree{}} \\
	4 & \myindex{Ei}er \\
	100 g & \myindex{Mehl} \\
	150 g & \myindex{Paniermehl} \\
	& \myindex{Erdnußöl}\index{Oel=Öl>Erdnuß-} \\
	& \myindex{Salz} \\
	& \myindex{Pfeffer} \\
      \end{zutaten}

      \personen{6}

      \begin{zubereitung}
        Die \chicoree{} waschen und einige Minuten blanchieren, abtropfen
	lassen und mit einem Tuch trocknen. \\
	Im Mehl, dann in den verrührten Eiern und anschließend in Paniermehl
	wälzen. \\
	Erdnußöl in einer Pfanne erhitzen und \chicoree{} darin backen. \\
	Auf saugfähigem Küchenpapier abtropfen lassen. \\
	Ideal als Beilage für Innereien, Braten und Grillspezialitäten. \\
      \end{zubereitung}

    \mynewsection{Geschmorter \chicoree{} mit Orangen}
      
      \begin{zutaten}
        4 & \myindex{\chicoree{}} \\
	2 & \myindex{Orange}n (Schale\footnote{Die Schale der Orange sollte
	    nicht zu bitter sein! Schmeckt sonst nicht.} und Saft) \\
	0,1 l & trockener \myindex{Weißwein}\index{Wein>weiß} oder
	        \myindex{Wermut} \\
        3 Eßlöffel & \myindex{glatte Petersilie}\index{Petersilie>glatt} \\
	2 Eßlöffel & gehackte \myindex{Minze} \\
	1 & \myindex{Schalotte} \\
	1 & gehackte \myindex{Knoblauchzehe} \\
	& \myindex{Salz} \\
	& \myindex{Pfeffer} \\
      \end{zutaten}
      
      \begin{zubereitung}
        \chicoree{} der Länge nach halbieren. Bittere Strünke entfernen. In
	einer Pfanne mit Butter anbraten. Übrige Zutaten darüber verteilen,
	würzen. Zugedeckt 20~Minuten schmoren. Dann ohne Deckel in der Mitte
	des auf \grad{200} vorgeheizten Backofens weitere 10 bis 15~Minuten
	schmoren. Sofort servieren. \\
      \end{zubereitung}

    \mynewsection{\chicoree{}dip mit verschiedenen Soßen}

      \begin{einleitung}
        \chicoree{} waschen und die einzelnen Blätter lösen. \\
      \end{einleitung}
      
      \begin{zutaten}
      \end{zutaten}

      \begin{zutat}{Knoblauchsoße}
        4 Eßlöffel & \myindex{Mayonnaise} \\
	1 & \myindex{Joghurt} \\
	1 & \myindex{Knoblauchzehe} \\
	1 & \myindex{Ei} \\
	& \myindex{Salz} \\
	& \myindex{Pfeffer} \\
	& \myindex{Petersilie} \\
      \end{zutat}

      \begin{zutat}{Cocktailsoße}
        4 Eßlöffel & \myindex{Mayonnaise} \\
	4 Eßlöffel & \myindex{Joghurt} \\
	3 Eßlöffel & \myindex{Ketchup} \\
	1 Eßlöffel & \myindex{Weinbrand} \\
	& \myindex{Salz} \\
	& \myindex{Pfeffer} \\
      \end{zutat}

      \begin{zutat}{Curry-Soße}
        4 Eßlöffel & \myindex{Mayonnaise} \\
	1 & \myindex{Joghurt} \\
	2 Teelöffel & \myindex{Curry} \\
	1 & \myindex{Apfel} \\
	& \myindex{Zitrone} \\
	& \myindex{Salz} \\
	& \myindex{Pfeffer} \\
      \end{zutat}

      \begin{zubereitung}
        Knoblauchsoße: Mayonnaise mit Joghurt und 1 durchgepreßten
	Knoblauchzehe verrühren. 1 hartgekochtes Ei hacken und unterheben. Mit
	Salz und Pfeffer abschmecken und mit gehackter Petersilie garnieren. \\
	Cocktailsoße: Mayonnaise mit Joghurt, Ketchup und Weinbrand verrühren.
	Mit Salz und Pfeffer abschmecken. \\
	Curry-Soße: Mayonnaise mit Joghurt und Curry verrühren. 1 Apfel fein
	schneiden, mit Zitrone beträufeln und untermischen. Die Soße mit Salz
	und Pfeffer abschmecken. \\
      \end{zubereitung}

    \mynewsection{\chicoree{} in Butter gebraten}\label{chicoreegebraten}

      \begin{zutaten}
        4 & \myindex{\chicoree{}}stauden \\
	100 -- 150 g & \myindex{Butter} \\
	& \myindex{Salz} \\
	& \myindex{Pfeffer} aus der Mühle \\
      \end{zutaten}

      \begin{zubereitung}
        \chicoree{} der Länge nach halbieren, Strunk entfernen. Große
	Auflaufform buttern. Beschichtete Pfanne erhitzen, Butter zerlassen
	und kräftig pfeffern. Dann \chicoree{} mit Schnittfläche nach unten
	anrösten/anbraten. Hitze auf \glqq{}2\grqq{} schalten. Vorsicht!
	Butter verbrennt gern, darf keinesfalls braun werden. \\
	\chicoree{} wenden und auf der anderen Seite anbraten. Salzen.
	Hitze ausstellen. Ofen auf \grad{150} heizen. Dann \chicoree{} in der
	Auflaufform nebeneinander legen, Butter aus der Pfanne darüber gießen.
	15--20~Minuten im Ofen lassen (es sollte aber schon gebruzzelt haben).
	\\
	Schmeckt lauwarm und abgekühlt. Dazu Weißbrot zum Butter aufnehmen.
	Gut als Vorspeise oder Imbiß. \\
      \end{zubereitung}
