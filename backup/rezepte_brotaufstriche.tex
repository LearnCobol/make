
% created Donnerstag, 27. Dezember 2012 08:35 (C) 2012 by Leander Jedamus
% modifiziert Mittwoch, 22. April 2015 13:43 von Leander Jedamus
% modified Freitag, 28. Dezember 2012 15:04 by Leander Jedamus
% modified Freitag, 28. Dezember 2012 12:39 by Leander Jedamus
% modified Donnerstag, 27. Dezember 2012 15:34 by Leander Jedamus
% modified Donnerstag, 27. Dezember 2012 11:15 by Leander Jedamus
% modified Donnerstag, 27. Dezember 2012 08:52 by Leander Jedamus

% Cornelia Trischberger
% Vegetarische Brotaufstriche, 20 auf einen Streich
% ISBN: 978-3-8338-0670-4

  \mynewchapter{Brotaufstriche}

    \mynewsection{Gorgonzolacreme mit Rucola und Apfel}

      % Seite 9

      \begin{zutaten}
        60 g & \myindex{Gorgonzola}\index{Käse>Gorgonzola} (italienischer
	       Blauschimmelkäse) \\
	1 kleiner & süßsaurer \myindex{Apfel} (ca. 100 g) \\
	1 Bund & \myindex{Rucola} \\
	50 g & \myindex{Pinienkerne} \\
	3 Eßlöffel & \myindex{Mascarpone} (italienischer Frischkäse) \\
	2 Eßlöffel & \myindex{Apfelessig}\index{Essig>Apfel-} \\
	& \myindex{Salz} \\
	& \myindex{Cayennepfeffer}\index{Pfeffer>Cayenne-} \\
      \end{zutaten}

      \portion{30 g}{60}{2 g}{5 g}{3 g}

      \begin{zubereitung}
	Vom Gorgonzola die Rinde entfernen, dann den Käse grob würfeln. Apfel
	schälen, halbieren, Stielansatz und Kerngehäuse entfernen und das
	Apfelfruchtfleisch ebenfalls grob würfeln. Rucola putzen, waschen und
	trockenschütteln. Gorgonzola, Apfel und Rucola mit Pinienkernen,
	Mascarpone und Apfelessig in den elektrischen Zerkleinerer (oder den
	Mixer) geben und alles fein pürieren. Gorgonzolacreme in eine Schüssel
	füllen, mit Salz und Cayennepfeffer kräftig abschmecken. \\
	Schmeckt besonders gut mit: Ciabatta-Brot, Nussbrot, Grissini. \\
      \end{zubereitung}

    \mynewsection{Frischkäsecreme mit Chufas-Nüssli}

      % Seite 10

      \begin{zutaten}
        200 g & fettreduzierter \myindex{Frischkäse}\index{Käse>Frisch-} \\
	4 Eßlöffel & \myindex{Chufas-Nüssli}%
	             \footnote{Chufas-Nüssli sind gemahlene oder geflockte
		               Erdmandeln aus Spanien und Portugal. Sie sind
			       natursüß (und dementsprechend lecker) ---
			       und außerdem auch noch super-gesund! Wer sie
			       regelmäßig ißt, bekommt nicht nur eine
			       Extra-Portion Kalium, Kalzium, Magnesium,
			       Eisen und Linolsäure, sondern auch noch
			       reichlich Ballaststoffe.}
		     (aus dem Reformhaus) \\
	2 Eßlöffel & \myindex{Ahornsirup} \\
	2 Eßlöffel & \myindex{Orange}nsaft \\
	2 Eßlöffel & \myindex{Limette}nsaft \\
      \end{zutaten}

      \portion{30 g}{75}{3 g}{5 g}{5 g}

      \begin{zubereitung}
	Frischkäse mit Chufas-Nüssli, Ahornsirup, Orangen- und Limettensaft im
	elektrischen Zerkleinerer (oder im Mixer) fein pürieren.
	Frischkäsecreme in ein Schraubglas füllen, im Kühlschrank aufbewahren.
	\\
	Schmeckt besonders gut zu Weißbrot, Pumpernickel, Hefezopf, Brioche,
	Croissants und Reiswaffeln. \\
      \end{zubereitung}

    \mynewsection{Dattel-Curry-Paste}

      % Seite 13

      \begin{zutaten}
        2 & \myindex{Frühlingszwiebel}\index{Zwiebel>Frühlings-}n \\
	60 g & entsteinte getrocknete \myindex{Datteln} \\
	1 Stück & frischer \myindex{Ingwer} (ca. 2 cm) \\
	50 g & gesalzene \myindex{Cashewnüsse} \\
	\breh{} Teelöffel & gelbe oder grüne \myindex{Thai-Currypaste} (aus
	        dem Asialaden) \\
	3 Eßlöffel & \myindex{Orange}nsaft \\
	3 Eßlöffel & \myindex{Limette}nsaft \\
	1 Eßlöffel & \myindex{Pflaumenmus} (Fertigprodukt, aus dem Glas) \\
	2 Eßlöffel & \myindex{Sojasoße}
      \end{zutaten}

      \portion{30 g}{60}{1 g}{3 g}{8 g}

      \begin{zubereitung}
	Frühlingszwiebeln putzen, waschen und grob hacken. Datteln ebenfalls
	grob hacken. Ingwer schälen und fein hacken. Frühlingszwiebeln und
	Datteln mit Ingwer, Cashewnüssen, Currypaste, Orangensaft,
	Limettensaft, Pflaumenmus und Sojasoße im elektrischen Zerkleinerer
	(oder im Mixer) fein pürieren. \\
	Schmeckt besonders gut mit Vollkornbrot, Reiswaffeln und Krupuk. \\
      \end{zubereitung}

    \mynewsection{Brokkolicreme mit Kapern und Oliven}

      % Seite 14

      \begin{zutaten}
        80 g & \myindex{Brokkoli} \\
	& \myindex{Salz} \\
	1 & hart gekochtes \myindex{Ei} \\
	1 kleines Glas & \myindex{Kapern} (20 g) \\
	10 & entsteinte \myindex{grüne Oliven}\index{Oliven>grün} \\
	2 Eßlöffel & \myindex{Kürbiskerne} \\
	2 Eßlöffel & \myindex{Kürbiskernöl}\index{Oel=Öl>Kürbiskern-} \\
	2 Eßlöffel & \myindex{Frischkäse}\index{Käse>Frisch-} \\
	2 Eßlöffel & \myindex{Zitrone}nsaft \\
	\breh{} Bund & \myindex{Petersilie} \\
	& \myindex{Salz} \\
	& \myindex{Cayennepfeffer}\index{Pfeffer>Cayenne-} \\
      \end{zutaten}

      \portion{30 g}{50}{2 g}{4 g}{1 g}

      \begin{zubereitung}
	Brokkoli putzen, in Röschen zerteilen, in kochendem Salzwasser
	3--4~Minuten garen. In ein Sieb abgießen, kalt abschrecken, abtropfen
	lassen. Das hart gekochte Ei pellen und halbieren. Kapern abtropfen
	lassen. Brokkoli mit Ei, Kapern, Kürbiskernen, Kürbiskernöl, Frischkäse
	und Zitronensaft im elektrischen Zerkleinerer (oder im Mixer) fein
	pürieren. Petersilie waschen, trockenschütteln, Blättchen von den
	Stielen zupfen und fein hacken. Brokkolicreme in eine Schüssel füllen,
	Petersilie daruntermischen, kräftig mit Salz und Cayennepfeffer
	abschmecken. \\
	Schmeckt besonders gut mit kräftigem Vollkornbrot aus Roggen oder
	Dinkel. \\
      \end{zubereitung}

    \mynewsection{Schoko-Bananencreme}

      % Seite 17

      \begin{zutaten}
        1 & \myindex{Banane} (ca. 100g) \\
	3 Eßlöffel & \myindex{Sahnequark} \\
	5 Eßlöffel & \myindex{Zartbitter-Schokoraspel} \\
	2 Eßlöffel & \myindex{Zitrone}nsaft \\
      \end{zutaten}

      \portion{30 g}{60}{1 g}{3 g}{6 g}

      \begin{zubereitung}
	Die Banane schälen und quer in Scheiben schneiden. Bananenstücke mit
	Sahnequark, 3 Eßlöffel Schokoraspeln und Zitronensaft in den
	elektrischen Zerkleinerer (oder in den Mixer) geben, alles fein
	pürieren. Bananenpüree in eine Schüssel füllen, die restlichen
	Schokoraspel untermischen. \\
	Schmeckt besonders gut mit Waffeln, Brioche, Hefezopf oder auch
	Pumpernickel. \\
	Tip: Die Bananencreme mit 2--3~Eßlöffel gerösteten Sesamsamen
	bestreuen. \\
      \end{zubereitung}

    \mynewsection{Avocado-Acerola-Aufstrich}

      % Seite 18

      \begin{zutaten}
        1 & reife \myindex{Avocado} (ca. 300g) \\
	3 & \myindex{Frühlingszwiebel}\index{Zwiebel>Frühlings-}n \\
	\breh{} Bund & \myindex{Basilikum} \\
	150 g & \myindex{Seiden-Tofu}\index{Tofu>Seiden-} (weicher Tofu,
	        aus dem Bio- oder Asialaden) \\
	3 Eßlöffel & \myindex{Acerolasaft} (aus dem Reformhaus) \\
	1 Eßlöffel & \myindex{Limette}nsaft \\
	& \myindex{Salz} \\
	& \myindex{Cayennepfeffer} \\
      \end{zutaten}

      \portion{30 g}{40}{1 g}{4 g}{0 g}

      \begin{zubereitung}

	Avocado halbieren und den Kern entfernen. Die Haut von der Avocado
	abziehen oder die Hälfte schälen und das Fruchtfleisch fein würfeln.
	Frühlingszwiebeln putzen, waschen und in feine Ringe schneiden.
	Basilikum waschen, trockenschütteln, Blättchen von den Stielen zupfen,
	dann in feine Streifen schneiden. \brdv{} der Avocadowürfel, die Hälfte
	der Frühlingszwiebeln und der Basilikumstreifen mit Seiden-Tofu,
	Acerola- und Limettensaft im elektrischen Zerkleinerer (oder im Mixer)
	fein pürieren. Avocado-Püree in eine Schüssel füllen, restliche
	Avocadowürfel, Frühlingszwiebelringe und Basilikumstreifen
	untermischen, mit Salz und Cayennepfeffer kräftig abschmecken. \\
	Schmeckt besonders gut zu Kürbiskernbrot, Nussbrot und dunklem
	Vollkornbrot. \\
      \end{zubereitung}

    \mynewsection{Rote-Bete-Aufstrich mit Aprikosen}

      % Seite 21

      \begin{zutaten}
        100 g & vorgegarte \myindex{Rote Bete} (Fertigprodukt) \\
	1 & \myindex{Aprikose} (ca. 50 g) oder \\
	50 g & getrocknete \myindex{Aprikose}n \\
	50 g & \myindex{Feta}\index{Käse>Feta}-Käse \\
	4 Eßlöffel & gemahlene \myindex{Mandeln} \\
	2 Eßlöffel & \myindex{Himbeeressig} \\
	2 Eßlöffel & \myindex{Orange}nsaft \\
	2 Eßlöffel & \myindex{Zitrone}nsaft \\
	& \myindex{Salz} \\
	& \myindex{Harissa}%
	           \footnote{ Harissa ist eine scharfe, nordafrikanische
		              Chilipaste --- gibt's im Supermarkt bei
			      exotischen Lebensmitteln. Beim Urlaub in
			      Südfrankreich unbedingt mal auf den Wochenmarkt
			      gehen und frisches Harissa kaufen! } \\
      \end{zutaten}

      \portion{30 g}{35}{1 g}{3 g}{1 g}

      \begin{zubereitung}
	Rote Bete grob würfeln. Aprikose waschen, mit einem Sparschäler dünn
	abschälen, halbieren und entkernen. Rote Bete und Aprikose mit
	Feta-Käse, Mandeln, Himbeeressig und Orangen- und Zitronensaft im
	elektrischen Zerkleinerer (oder im Mixer) fein pürieren.
	Rote-Bete-Aufstrich in eine Schüssel füllen, mit Salz und nach
	Geschmack mit \breh{}--1~Teelöffel Harissa kräftig abschmecken. \\
      \end{zubereitung}

    \mynewsection{Tomatenaufstrich All'Arrabbiata}

      % Seite 23

      \begin{zutaten}
        100 g & getrocknete, in Öl eingelegte Tomaten\index{getrocknete Tomate}%
	        \index{Tomate>getrocknet} (aus dem Glas) \\
	3 & \myindex{Frühlingszwiebel}\index{Zwiebel>Frühlings-}n \\
	1 & \myindex{Tomate} (ca. 100g) \\
	\breh{} & Bund \myindex{Basilikum} \\
	1 Eßlöffel & \myindex{Ricotta}\index{Käse>Ricotta} (italienischer
	             Frischkäse) \\
	2 Eßlöffel & \myindex{Aceto balsamico} \\
	2 Eßlöffel & \myindex{Gemüsebrühe} \\
	& \myindex{Salz} \\
	& getrocknete \myindex{Chiliflocken} (im Gewürzregal vom Supermarkt) \\
      \end{zutaten}

      \portion{30 g}{30}{2 g}{1 g}{6 g}

      \begin{zubereitung}
	Die eingelegten Tomaten auf Küchenpapier abtropfen lassen.
	Frühlingszwiebeln putzen, waschen, grob hacken. Die frische Tomate
	waschen, vierteln, Stielansatz und Kerne entfernen. Basilikum waschen,
	trockenschütteln, Blättchen von den Stielen zupfen, streifig schneiden.
	Eingelegte Tomaten mit Frühlingszwiebeln, Basilikum, Tomatenvierteln,
	Ricotta, Balsamico und Gemüsebrühe im elektrischen Zerkleinerer (oder
	im Mixer) fein pürieren. Tomatenaufstrich in eine Schüssel füllen, mit
	Salz und Chiliflocken kräftig abschmecken. \\
	Schmeckt besonders gut mit Pizzabrot, Grissini, Tramezzinibrot,
	geröstetem Bauernbrot. \\
      \end{zubereitung}

    \mynewsection{Obatzda italiano}

      % Seite 24

      \begin{zutaten}
        250 g & \myindex{Mascarpone} (italienischer Frischkäse) \\
	1 kleiner & \myindex{Camembert} (125 g) \\
	1 & \myindex{Knoblauchzehe} \\
	2 & \myindex{Frühlingszwiebel}\index{Zwiebel>Frühlings-}n%
	    \footnote{ Statt Frühlingszwiebeln mal eine fein gehackte
	               \textbf{rote Zwiebel} oder 2 \textbf{Schalotten}
		       nehmen. } \\
	\breh{} Bund & \myindex{Basilikum} \\
	2 Eßlöffel & \myindex{Balsamico bianco} \\
	2 Eßlöffel & \myindex{Zitrone}nsaft \\
	& \myindex{Salz} \\
	& \myindex{Pfeffer} \\
      \end{zutaten}

      \portion{30 g}{65}{2 g}{6 g}{1 g}

      \begin{zubereitung}
	Mascarpone in eine Schüssel geben. Vom Camembert die weiße Rinde
	rundherum dünn abschneiden. Den entrindeten Camembert würfeln,
	Camembertwürfel zum Mascarpone geben. Die Käsewürfel mit einer Gabel
	zerdrücken und gut vermischen. Knoblauchzehe schälen, ganz fein hacken
	oder durch die Knoblauchpresse drücken und zum Mascarpone geben.
	Frühlingszwiebeln putzen, waschen, in feine Ringe schneiden. Basilikum
	waschen, trockenschütteln, Blättchen von den Stielen zupfen, dann in
	feine Streifen schneiden. Frühlingszwiebeln und Basilikumstreifen mit
	Balsamico bianco und Zitronensaft zur Mascarpone-Camembert-Masse geben,
	alles gut vermischen und kräftig mit Salz und Pfeffer abschmecken. \\
	Schmeckt besonders gut mit Bauernbrot, geröstetem Weißbrot, Grissini
	und Laugenbrezeln. \\
      \end{zubereitung}

    \mynewsection{Tiramisu-Creme}

      % Seite 27

      \begin{zutaten}
        30 g & \myindex{Löffelbiskuits} \\
	60 g & \myindex{Espresso-Schokolade}\index{Schokolade>Espresso-} \\
	3 Eßlöffel & \myindex{Mascarpone} \\
	1 Päckchen & \myindex{Bourbon-Vanillezucker}%
	             \index{Zucker>Vanille->Bourbon-} \\
        2 Eßlöffel & \myindex{Eierlikör}\index{Likör>Eier-}, erstatzweise 
	             2--3~Eßlöffel \myindex{Orange}nsaft \\
	6 Eßlöffel & \myindex{Orange}nsaft \\
      \end{zutaten}

      \portion{30 g}{75}{1 g}{4 g}{7 g}

      \begin{zubereitung}
	Löffelbiskuits und 50 g Schokolade in große Stücke brechen. Biskuits
	und Schokolade mit Mascarpone, Vanillezucker, Eierlikör und Orangensaft
	im elektrischen Zerkleinerer (oder Mixer) fein pürieren und in eine
	Schüssel füllen. Restliche Schokolade grob raspeln und die Creme damit
	bestreuen. \\
	Schmeckt besonders gut mit Brioche, Waffeln, Hefezopf und
	Schoko-Zwieback. \\
      \end{zubereitung}

    \mynewsection{Tortillacreme}

      % Seite 28

      \begin{zutaten}
        1 Dose & \myindex{rote Bohnen}\index{Bohnen>rot} (Abtropfgewicht
	         250 g)\\
	2 & \myindex{Frühlingszwiebeln}\index{Zwiebel>Frühlings-} \\
	80 g & \myindex{Cheddar}\index{Käse>Cheddar} (irischer Käse) \\
	3 Eßlöffel & \myindex{saure Sahne}\index{Sahne>sauer} \\
	2 Eßlöffel & \myindex{Orange}nsaft \\
	2 Eßlöffel & \myindex{Limette}nsaft \\
	\breh{} Teelöffel & gemahlener \myindex{Koriander} \\
	\breh{} Teelöffel & gemahlener \myindex{Kreuzkümmel} \\
	& \myindex{Salz} \\
	& \myindex{Tabasco} \\
      \end{zutaten}

      \portion{30 g}{40}{2 g}{2 g}{3 g}

      \begin{zubereitung}
	Bohnen aus der Dose in ein Sieb abgießen, kurz abspülen und abtropfen
	lassen. Frühlingszwiebeln putzen, waschen und grob hacken. Cheddar grob
	würfeln. Beides mit den abgetropften Bohnen, der sauren Sahne, Orangen-
	und Limettensaft im elektrischen Zerkleinerer (oder im Mixer) fein
	pürieren. Tortillacreme in eine Schüssel füllen, mit Koriander und
	Kreuzkümmel vermischen, mit Salz und Tabasco kräftig abschmecken. \\
	Schmeckt besonders gut mit Weißbrot, Knäckebrot und natürlich
	Taco-Chips. \\
      \end{zubereitung}

    \mynewsection{Griechische Salatcreme}

      % Seite 31

      \begin{zutaten}
        \brev{} & \myindex{Salatgurke}\index{Gurke>Salat-} (ca. 80 g)\\
	\breh{} & \myindex{rote Paprika}\index{Paprika>rot}schote (ca. 80 g) \\
	1 kleine & \myindex{Tomate} (ca. 80 g) \\
	2 & \myindex{Frühlingszwiebel}\index{Zwiebel>Frühlings-}n \\
	1 & \myindex{Knoblauchzehe} \\
	150 g & \myindex{Feta}\index{Käse>Feta}-Käse \\
	8 & \myindex{grüne Oliven}\index{Oliven>grün} \\
	1 Eßlöffel & \myindex{Sahnejoghurt}\index{Joghurt>Sahne-} \\
	2 Eßlöffel & \myindex{Rotweinessig}\index{Essig>Wein->Rot-} \\
	\breh{} Teelöffel & getrockneter \myindex{Oregano} \\
	& \myindex{Salz} \\
	& \myindex{Pfeffer} \\
      \end{zutaten}

      \portion{30 g}{25}{2 g}{2 g}{1 g}

      \begin{zubereitung}
	Gurke waschen, halbieren, Kerne mit einem kleinen Löffel aus den
	Hälften schaben, dann das Fruchtfleisch quer in Stücke schneiden.
	Paprikaschote putzen, waschen, Fruchtfleisch grob würfeln. Tomate
	waschen, vierteln, Kerne und Stielansatz entfernen. Frühlingszwiebeln
	putzen, waschen und grob hacken. Knoblauchzehe schälen. Feta-Käse in
	Stücke schneiden. Gurke, Paprika, Tomate, Zwiebeln und die
	Knoblauchzehe mit Oliven, Feta-Käse, Sahnejoghurt und Rotweinessig im
	elektrischen Zerkleinerer (oder im Mixer) fein pürieren. Griechische
	Salatcreme in eine Schüssel füllen, mit Oregano vermischen und mit Salz
	und Pfeffer kräftig abschmecken. \\
	Schmeckt besonders gut mit Fladenbrot, gerösteten Baguette-Scheiben
	und Grissini. \\
      \end{zubereitung}

    \mynewsection{Bircher-Müesli-Aufstrich}

      \begin{zutaten}
        100 g & \myindex{Studentenfutter} (Fertigmischung mit Rosinen und
	                                   Nüssen) \\
	2 Eßlöffel & zarte \myindex{Haferflocken} \\
	2 Eßlöffel & \myindex{Sonnenblumenkerne} \\
	2 Eßlöffel & gehackte \myindex{Pistazien} \\
	2 Eßlöffel & \myindex{Orange}nsaft \\
	2 Eßlöffel & \myindex{Zitrone}nsaft \\
	2 Eßlöffel & \myindex{Honig} \\
	1 Eßlöffel & \myindex{Sahnequark} \\
	1 kleiner & \myindex{Apfel} (ca. 100~g) \\
	80 g & \myindex{Himbeeren} (frisch oder TK) \\
	& \myindex{Zimt}pulver \\
      \end{zutaten}

      \portion{30 g}{55}{1 g}{4 g}{4 g}

      \begin{zubereitung}
        Studentenfutter mit Haferflocken, Sonnenblumenkernen, Pistazien,
	Orangensaft, Zitronensaft, Honig und Sahnequark in den elektrischen
	Zerkleinerer (oder Mixer) geben. \\
	Apfel waschen, halbieren, Stielansatz und Kerne entfernen, Fruchtfleisch
	(mit Schale) grob würfeln. Himbeeren verlesen (oder TK-Himbeeren
	antauen lassen). Apfel und Himbeeren zur Haferflockenmischung geben,
	alles fein pürieren. \\
	Bircher-Müesli-Aufstrich in eine Schüssel füllen, mit 2--3~Prisen
	Zimtpulver abschmecken. Schmeckt besonders gut mit Dinkel-Ciabatta,
	Nussbrot und Hirsebrot. \\
      \end{zubereitung}

    \mynewsection{Asia-Ratatouille-Creme}

      \begin{zutaten}
        \breh{} & \myindex{Zucchino} (ca. 100 g) \\
	\breh{} & \myindex{Aubergine} (ca. 100 g) \\
	\breh{} & \myindex{rote Paprika}\index{Paprika>rot}schote (ca. 100 g) \\
	\breh{} & \myindex{gelbe Paprika}\index{Paprika>gelb}schote
	          (ca. 100 g) \\
	2 & \myindex{Schalotte}n (ca. 80 g) \\
	2 Eßlöffel & \myindex{Olivenöl}\index{Oel=Öl>Oliven-} \\
	1 & \myindex{Knoblauch}zehe \\
	1 Stück & frischer \myindex{Ingwer} (ca. 2~cm) \\
	3 Eßlöffel & \myindex{Reisessig}\index{Essig>Reis-} \\
	1 Teelöffel & \myindex{Ahornsirup} \\
	& \myindex{Salz} \\
	& \myindex{Chili-Chicken-Sauce} (süß-scharfe Asia-Würzsoße, aus dem
	                                 Supermarkt oder Asialaden) \\
      \end{zutaten}

      \portion{30 g}{25}{0 g}{2 g}{2 g}

      \begin{zubereitung}
        Zucchino, Aubergine und Paprikaschoten putzen, waschen und fein würfeln.
	Schalotten schälen und fein hacken. Olivenöl in einer Pfanne erhitzen,
	Gemüse- und Schalottenwürfel darin unter Rühren bei mittlerer Hitze
	in 5--6~Minuten hellbraun braten. \\
	Knoblauchzehe und Ingwer schälen, mit \brdv{} der gebratenen
	Gemüsemischung, Reisessig und Ahornsirup im elektrischen Zerkleinerer
	(oder im Mixer) fein pürieren. \\
	Asia-Ratatouille-Creme in eine Schüssel füllen, restliche
	Gemüsewürfelchen untermischen, alles mit Salz und
	Chili-Chicken-Soße kräftig abschmecken. \\
	Schmeckt besonders gut zu Baguette, Ciabatta-Brot, Reiswaffeln und
	Krupuk. \\
      \end{zubereitung}

    \mynewsection{Ricottacreme India}

      \begin{zutaten}
        1 & \myindex{Knoblauch}zehe \\
        1 Stück & frischer \myindex{Ingwer} (ca. 2~cm) \\
        8 & \myindex{Minze}blättchen \\
        250 g & \myindex{Ricotta} (italienischer Frischkäse) \\
        100 g & gehackte \myindex{Pistazien} \\
        2 Eßlöffel & \myindex{Orange}nsaft \\
        2 Eßlöffel & \myindex{Limette}nsaft \\
        \brev{} Teelöffel & gemahlener \myindex{Koriander} \\
        \brev{} Teelöffel & gemahlenes \myindex{Kreuzkümmel} \\
        & \myindex{Paprikapulver} \\
        & \myindex{Salz} \\
        & \myindex{Sambal oelek} \\
      \end{zutaten}

      \portion{30 g}{70}{3 g}{5 g}{2 g}

      \begin{zubereitung}
        Knoblauchzehe und Ingwer schälen, Minzblättchen waschen und
	trockenschütteln. Alles mit Ricotta, Pistazien, Orangen- und
	Limettensaft im elektrischen Zerkleinerer (oder im Mixer) fein
	pürieren. \\
	Ricottacreme in eine Schüssel füllen, mit Koriander, Kreuzkümmel und
	Parikapulver vermischen, mit Salz und nach Geschmack mit
	\breh{}--1~Teelöffel Sambal oelek kräftig abschmecken. \\
	Schmeckt besonders gut mit Vollkornbrot, Pappadams und Krupuk. \\
      \end{zubereitung}

    \mynewsection{Feldsalatcreme mit Ziegenfrischkäse und Wasabi}

      \begin{zutaten}
        100 g & \myindex{Feldsalat}\index{Salat>Feld-} \\
        2 & \myindex{Frühlingszwiebel}\index{Zwiebel>Frühlings-}n \\
        50 g & \myindex{Sonnenblumenkerne} \\
        80 g & \myindex{Ziegenfrischkäse} \\
        1 Eßlöffel & \myindex{Mascarpone} (italienischer Frischkäse) \\
        1 Eßlöffel & \myindex{Weißweinessig}\index{Essig>Weißwein-} \\
        1 Eßlöffel & \myindex{Orange}nsaft \\
        & \myindex{Salz} \\
        & \myindex{Wasabi} (grüne japanische Meerrettichpaste) \\
      \end{zutaten}

      \portion{30 g}{40}{2 g}{3 g}{1 g}

      \begin{zubereitung}
        Feldsalat putzen, waschen und gut abtropfen lassen. Dann die
	Frühlingszwiebeln putzen, waschen und grob hacken. \\
	Die Feldsalatblättchen zusammen mit Frühlingszwiebeln,
	Sonnenblumenkernen, Ziegenfrischkäse, Mascarpone, Weißweinessig und
	Orangensaft im elektrischen Zerkleinerer (oder im Mixer) fein pürieren.
	\\
	Feldsalatcreme in eine Schüssel geben, mit Salz und nach Geschmack
	mit \breh{}--1~Teelöffel Wasabi abschmecken. \\
	Schmeckt besonders gut mit Vollkornbrot, Baguette und japanischen
	Reis-Crackern. \\
      \end{zubereitung}

    \mynewsection{Erdnuss-Buttercreme}

      \begin{zutaten}
        \breh{} & \myindex{rote Papika}\index{Paprika>rot}schote \\
        3 Eßlöffel & \myindex{Erdnussbutter} (aus dem Glas) \\
        100 g & \myindex{Pinienkerne} \\
        3 Eßlöffel & \myindex{Orange}nsaft \\
        5 Eßlöffel & \myindex{Limette}nsaft \\
        2 Eßlöffel & \myindex{Sojasoße} \\
        2 Eßlöffel & \myindex{Chili-Chicken-Soße} (süßscharfe Asia-Würzsoße,
	             aus dem Supermarkt oder Asialaden) \\
        & \myindex{Salz} \\
        & \myindex{Sambal oelek} \\
      \end{zutaten}

      \portion{30 g}{40}{2 g}{3 g}{1 g}

      \begin{zubereitung}
        Paprikaschote putzen, waschen und grob würfeln. Paprikawürfel mit
	Erdnussbutter, Pinienkernen, Orangensaft, Limettensaft, Soja- und
	Chili-Chicken-Soße im elektrischen Zerkleinerer (oder im Mixer) fein
	pürieren. \\
	Erdnussbuttercreme in eine Schüssel füllen, mit Salz und nach Geschmack
	1--2~Teelöffel Sambal oelek kräftig abschmecken. \\
	Schmeckt besonders gut zu getoastetem Weißbrot, Fladenbrot, knusprigen
	Reis-Crackern und Krupuk. \\
      \end{zubereitung}

    \mynewsection{Guacamole mit Mango}

      \begin{zutaten}
        1 & reife \myindex{Avocado} \\
        \breh{} & \myindex{Mango} \\
        3 & \myindex{Frühlingszwiebel}\index{Zwiebel>Frühlings-}n \\
        \breh{} Teelöffel & abgeriebene Bio-\myindex{Limette}nschale \\
        4 Eßlöffel & \myindex{Limette}nsaft \\
        1 Eßlöffel & \myindex{Mascarpone} (italienischer Frischkäse) \\
        & \myindex{Salz} \\
        & \myindex{Sambal oelek} \\
        & \myindex{Zucker} \\
      \end{zutaten}

      \portion{30 g}{30}{0 g}{3 g}{2 g}

      \begin{zubereitung}
        Avocado längs halbieren, Hälften ausdeinander drehen und den Kern
	entfernen. Die Schale von den Avocadohälften abziehen oder die Hälften
	schälen. Mango schälen und das Fruchtfleisch vom Kern schneiden.
	Avocado- und Mangofruchtfleisch fein würfeln. Frühlingszwiebeln
	putzen, waschen, in feine Ringe schneiden. \\
	\brdv{}~von Avocado, Mango und Frühlingszwiebeln mit Limettenschale,
	Limettensaft und Mascarpone im elektrischen Zerkleinerer (oder im Mixer)
	fein pürieren. \\
	Guacamole in eine Schüssel füllen, restliche Avocado- und
	Mangowürfelchen und Frühlingszwiebelringe untermischen, alles kräftig
	mit Salz, etwas Sambal oelek und einer Prise Zucker abschmecken. \\
	Schmeckt besonders gut zu Nussbrot, Weißbrot, Taco-Chips und Krupuk. \\
      \end{zubereitung}

    \mynewsection{Champignonpaste mit Kichererbsen}

      \begin{zutaten}
        150 g & \myindex{Champignon}s \\
        1 & \myindex{Knoblauch}zehe \\
        2 & \myindex{Schalotte}n \\
        2 Eßlöffel & \myindex{Olivenöl}\index{Oel=Öl>Oliven-} \\
        150 g & \myindex{Kichererbsen} (aus der Dose) \\
        2 Eßlöffel & \myindex{Orange}nsaft \\
        2 Eßlöffel & \myindex{Limette}nsaft \\
        2 Eßlöffel & \myindex{Sahnejoghurt}\index{Joghurt>Sahne-} \\
        1 Teelöffel & \myindex{Honig} \\
        \breh{} Bund & \myindex{Koriander}grün \\
        & \myindex{Salz} \\
        & \myindex{Harissa} (nordafrikanische Chili-Würzpaste, aus dem Glas
	                     oder aus der Tube) \\
      \end{zutaten}

      \portion{30 g}{45}{0 g}{2 g}{1 g}

      \begin{zubereitung}
        Champignons putzen, längs in feine Scheibchen schneiden. Knoblauchzehe
	schälen, ebenfalls in feine Scheiben schneiden. Schalotten schälen und
	fein hacken. \\
	Olivenöl in einer Pfanne erhitzen. Champignons, Knoblauch und
	Schalotten darin unter Rühren 4--5~Minuten braten. \\
	Kichererbsen aus der Dose in einem Sieb abspülen und abtropfen lassen.
	Kichererbsen mit dem gebratenen Champignon-Mix, Orangen- und
	Limettensaft, Joghurt und Honig im elektrischen Zerkleinerer (oder im
	Mixer) fein pürieren. \\
	Koriandergrün waschen, trockenschütteln, Blättchen von den Stielen
	zupfen und fein hacken. Champignonpaste in eine Schüssel geben,
	Koriander untermischen, mit Salz und nach Geschmack mit
	\breh{}--1~Teelöffel Harissa abschmecken. \\
	Schmeckt besonders gut mit geröstetem Weißbrot, Fladenbrot und Grissini.
	\\
      \end{zubereitung}

    % \mynewsection{}

      % \begin{zutaten}
      % \end{zutaten}

      % \portion{30 g}{60}{2 g}{5 g}{3 g}

      % \begin{zubereitung}
      % \end{zubereitung}

% vim:ai sw=2

