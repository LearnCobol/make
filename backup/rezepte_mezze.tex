
% created Montag, 10. Dezember 2012 16:21 (C) 2012 by Leander Jedamus
% modifiziert Mittwoch, 22. April 2015 12:27 von Leander Jedamus
% modified Donnerstag, 27. Dezember 2012 15:24 by Leander Jedamus
% modified Donnerstag, 27. Dezember 2012 15:11 by Leander Jedamus
% modified Montag, 10. Dezember 2012 16:30 by Leander Jedamus

  \mynewchapter{Mezze}

    \mynewsection{Aromen und Gewürze}

      \begin{zutaten}
        Piment & vor allem zu Fleisch \\
	Harrissa & scharfe Chilipaste (nordafrikanisch) in Tuben und Gläsern \\
	Granatapfelkonzentrat & für syrische oder libanesische \\
	Sumale & für gegrilltes Fleisch oder Fisch; luftdicht verwahren \\
	Tahini-Paste & aus gemahlenem Sesam, zusammen mit Olivenöl und
	               Zitronensaft wichtig für Hummus und Sesamsoße; wird in
		       Gläsern verkauft \\
      \end{zutaten}

    \mynewsection{Hummus (Dip)}\label{hummus}

      \begin{zutaten}
        1 Dose & \myindex{Kichererbsen}\index{Erbsen>Kicher-} (400 g),
	         abgespült, abgetropft \\
	1 & \myindex{Knoblauchzehe} mit \brev{} Teelöffel \myindex{Salz} zur
	    Paste zerdrückt \\
	3--4 Eßlöffel & \myindex{Tahini} \\
	& Saft einer \myindex{Zitrone} \\
	\brev{} Teelöffel & gemahlener \myindex{Kreuzkümmel} \\
	& \myindex{Salz} \\
	& natives \myindex{Olivenöl}\index{Oel=Öl>Oliven-} zum Verdünnen (nach
	  Belieben)\\
      \end{zutaten}

      \personen{6--8}

      \begin{zubereitung}
        Kichererbsen in den Mixer geben, aber 1~Eßlöffel zurückbehalten.
	Knoblauch zugeben, 3~Eßlöffel Tahini, 2~Eßlöffel Zitronensaft,
	Kreuzkümmel und Salz nach Geschmack zu einer Creme verarbeiten.
	Eventuell mit Wasser oder Öl verdünnen (bei laufendem Mixer reingeben).
	\\
	Servieren: auf Teller geben, mit Löffel in der Mitte eine Vertiefung
	machen und die ganzen Kichererbsen und etwas Olivenöl geben. Mit
	Paprika und Petersilie bestreuen. \\
	Im Kühlschrank bis zu 3~Tage haltbar, gefroren bis zu 1~Monat. \\
        Dazu: Warmes Brot = Pita Brot. \\
      \end{zubereitung}

    \mynewsection{Gegrillter Auberginendip mit Tahini}

      \begin{einleitung}
        Um den echten Geschmack über Holzkohle gegrillter Auberginen
	nachzuahmen, sollte man die Auberginen in einer stark erhitzten
	gußeisernen Grillpfanne braten (vorher alle Fenster aufmachen!). \\
      \end{einleitung}

      \begin{zutaten}
        & \myindex{Olivenöl}\index{Oel=Öl>Oliven-} \\
	1 große & \myindex{Aubergine} (700~g) \\
	& Saft von einer halben \myindex{Zitrone} \\
	2\breh{} Eßlöffel & \myindex{Tahini} \\
	2 & \myindex{Knoblauchzehe}n, zerdrückt \\
	& \myindex{Salz} \\
	& \myindex{Pfeffer} \\
      \end{zutaten}

      \personen{4--6}

      \begin{zubereitung}
        Gußeiserne Grillpfanne oder schwere Bratpfanne erhitzen, bis ein
	Wassertropfen darin abperlt. Pfanne mit Öl auspinseln. Ganze
	Aubergine mit Öl einreiben. Von jeder Seite 10~Minuten unter
	vorsichtigem Wenden braten (dabei die Haut nicht verletzen, auch wenn
	sie irgendwann aufplatzt), bis die Frucht zusammenfällt und die Haut
	schwarz ist. \\
	Aus der Pfanne nehmen und abkühlen lassen. \\
	Fruchtfleisch aus halbierter Aubergine mit dem Löffel entfernen und im
	Mixer glatt pürieren. Salz, Pfeffer, Tahini, Knoblauch bei laufendem
	Gerät 2 Eßlöffel Olivenöl zugeben. Abschmecken. Eventuell verdünnen mit
	Wasser oder Öl. \\
	Servieren: Auf Teller geben, eine Vertiefung machen, Olivenöl,
	schwarze Oliven, Koriander darüberstreuen. (hält sich im Kühlschrank
	bis zu 3~Tage). \\
	Wenn der Geschmack zu kräftig ist: 1--2~Eßlöffel Naturjoghurt
	unterrühren. \\
	Alternativ zum Braten: Aubergine rundum mit der Gabel einstechen und
	im vorgeheizten Backofen bei \grad{225} 30--40~Minuten rösten. \\
        Dazu: warmes Pita-Brot. \\
      \end{zubereitung}

    \mynewsection{Sesamsoße}\label{sesamsosse}

      \begin{einleitung}
        Paßt zu Fisch, gegrilltem Fleisch, gegrilltem oder rohem Gemüse, auch
	als Dressing für einen Hirtensalat oder eine Schale Kichererbsen. \\
      \end{einleitung}

      \begin{zutaten}
        1 große & \myindex{Knoblauchzehe} mit \brev{}~Teelöffel Salz zur Paste
	          zerdrückt \\
	4--6 Eßlöffel & \myindex{Wasser} \\
	1\breh{} & \myindex{Zitrone}n entsaftet (4~Eßlöffel) \\
	150 g & \myindex{Tahini} \\
	& \myindex{Salz} \\
	1 Eßlöffel & frische, sehr fein gehackte, glatte \myindex{Petersilie} \\
      \end{zutaten}

      \personen{4--6}

      \begin{zubereitung}
        Knoblauch, Wasser, Zitronensaft verrühren, Tahini einrühren, dann mit
	Salz würzen, abschmecken. Petersilie unterziehen. \\
	Soße ist zunächst sehr dünnflüssig, dickt aber im Kühlschrank nach.
	Verdünnbar mit Wasser oder Zitronensaft. \\
        Dazu: warmes Pita-Brot. \\
      \end{zubereitung}

    \mynewsection{Joghurt-Gurken-Dip (Tzatziki)}\label{tzatziki}

      \begin{zutaten}
        1 & \myindex{Knoblauchzehe} mit \breh{}~Teelöffel Salz zur Paste
	    zerdrücken \\
	300 g & \myindex{griechischer Naturjoghurt}
	        \index{Naturjoghurt>griechisch}
		\index{Joghurt>Natur>griechisch} \\
	2 Eßlöffel & natives \myindex{Olivenöl}\index{Oel=Öl>Oliven-} \\
	2 Eßlöffel & getrocknete \myindex{Minze} \\
	\breh{} & \myindex{Zitrone}, entsaftet \\
	& \myindex{Salz} \\
	& \myindex{weißer Pfeffer}\index{Pfeffer>weiß} \\
	1 kleine & \myindex{Gurke} (200 g) \\
	& \myindex{Cayennepfeffer}\index{Pfeffer>Cayenne-} zum Garnieren \\
      \end{zutaten}

      \personen{4--6}

      \begin{zubereitung}
        Knoblauch, Joghurt, Öl, Minze und 1~Eßlöffel Zitronensaft sowie Salz
	und Pfeffer in einer Schüssel verrühren. Abdecken und ziehen lassen. \\
	Gurke grob reiben, in ein Sieb zum Abtropfen geben und mit den Händen
	ausdrücken. Gurke in Soße geben, mischen. Mindestens 2~Stunden in den
	Kühlschrank stellen. \\
	Zum sofort servieren: Gurke fein würfeln (saftet aber). \\
        Dazu: warmes Pita-Brot. \\
      \end{zubereitung}

    \mynewsection{Knoblauchsoße}\label{knoblauchsosse} 

      \begin{einleitung}
        Diese scharfe Soße wird oft mit gemahlenen Mandeln oder Walnüssen
	zubereitet, doch die Kartoffeln mildern das Aroma ab. Paßt sehr gut
	zu Rote-Beete-Salat oder gegrillten Sardinen. Auch als Dip. \\
      \end{einleitung}

      \begin{zutaten}
        2 große & mehlig kochende \myindex{Kartoffel}n (\'a 250 g) \\
	4--5 & \myindex{Knoblauchzehe}n mit 1~Teelöffel Salz zur Paste
	       zerdrückt \\
	\breh{} & \myindex{Zitrone}, entsaftet oder 2~Eßlöffel
	          \myindex{Weißweinessig}\index{Essig>Weißwein-} \\
	\breh{} Teelöffel & \myindex{Salz} \\
	& \myindex{weißer Pfeffer}\index{Pfeffer>weiß} \\
	175 ml & natives \myindex{Olivenöl}\index{Oel=Öl>Oliven-} \\
      \end{zutaten}

      \personen{4--6}

      \begin{zubereitung}
        Kartoffeln in kochendes Wasser geben (bedeckt) und wieder zum Kochen
	bringen, sehr weich kochen, abgießen und abkühlen lassen. Pressen
	(\underline{nicht} mixen). Knoblauch, Zitronensaft, Salz, Pfeffer und
	Öl eßlöffelweise dazu, immer gut umrühren, bis die Creme kein Öl mehr
	annimmt. Abschmecken. Für mindestens 2~Stunden in den Kühlschrank
	stellen. \\
	\begin{description}
	  \item[Servieren] 2 Kartoffeln in Scheiben geschnitten \\
	  1 große Zucchini, längs halbiert, in Stücke \\
	  1 rote Paprika in Streifen geschnitten \\
	  1 grüne Paprika in Streifen geschnitten \\
	\end{description}
	Dazu: 1 warmes Pita-Brot. \\
	Nach einem Tag im Kühlschrank wird der Knoblauch milder. Eventuell
	1--2~Eßlöffel Naturjoghurt unterrühren. \\
      \end{zubereitung}

    \mynewsection{Nuß-Paprika-Mus}

      \begin{zutaten}
        2 große & \myindex{rote Paprika}\index{Paprika>rot} mit
	          Olivenöl eingerieben \\
        100 ml & \myindex{Olivenöl}\index{Oel=Öl>Oliven-} + etwas zum Bedecken \\
	1 & gehackte \myindex{Zwiebel} \\
	\breh{} Teelöffel & gemahlenener \myindex{Kreuzkümmel} \\
	2 Eßlöffel & Walnuß\index{Walnüsse}stücke, geröstet, gehackt und etwas
	             zum Garnieren \\
	2 Eßlöffel & \myindex{Pinienkerne}, geröstet, gehackt und etwas zum
	              Garnieren \\
	2 Eßlöffel & \myindex{Cashewkerne}, geröstet, gehackt und etwas zum
	              Garnieren \\
	2 Eßlöffel & frische weiße \myindex{Semmelbrösel} \\
	\brev{} Teelöffel & \myindex{Chiliflocken} \\
	\brev{}--\breh{} Teelöffel & \myindex{Granatapfelkonzentrat} \\
	\breh{} & \myindex{Zitrone}, entsaftet \\
	& \myindex{Salz} \\
	\breh{} & \myindex{Granatapfel} zum Garnieren \\
	& \myindex{Pita-Brot}-Dreiecke zum Servieren \\
      \end{zutaten}

      \personen{4--6}

      \begin{zubereitung}
        Pita-Dreiecke: \grad{180} 4--6~Pita-Brote in Dreiecke schneiden,
	Innenseite nach oben, dünn mit Öl bestrichen 20~Minuten backen. \\
	Backofen auf \grad{225} bzw. auf höchster Stufe vorheizen und ein
	kleines Backblech mit Alufolie auslegen. Paprika aufs Blech legen und
	15--20~Minuten rösten, bis die Haut schwarz ist. Aus dem Ofen nehmen
	und mit einem sauberen Geschirrtuch abdecken. Haut abziehen. Früchte
	längs halbieren, entkernen und in dünne Streifen schneiden. \\
	3 Eßlöffel Öl in einer Pfanne erhitzen. Zwiebel auf mittlerer Stufe
	unter Rühren 2~Minuten andünsten. Kreuzkümmel zufügen, weitere
	2--3~Minuten dünsten, bis die Zwiebel weich ist. \\
	Zwiebelmischung samt Öl in den Mixer geben, Paprika, Walnüsse,
	Pinienkerne, Cashewkerne, Semmelbrösel, Chiliflocken,
	Granatapfelkonzentrat, 1~Eßlöffel Zitronensaft und Salz nach Geschmack
	zufügen, alles gut mischen. bei laufendem Gerät soviel Öl zufügen, bis
	ein dickes Mus entsteht. Abschmecken mit Zitronensaft und/oder Salz. In
	eine Schale geben. Wenn nicht sofort serviert wird, Oberfläche mit
	dünner Ölschicht bedecken. Bis zu 3~Tage im Kühlschrank haltbar. Kurz
	vor dem Servieren Granatapfelhälften mit der Schnittfläche nach unten
	über das Mus halten und auf die Frucht klopfen, bis die Kerne auf das
	Mus fallen. \\
      \end{zubereitung}

    \mynewsection{Falafel (Bohnenkugeln frittiert)}

      \begin{zutaten}
        40 g & mittelfein geschroteter \myindex{Bulgur} \\
	1 & \myindex{Pita-Brot} vom Vortag, in kleine Stücke gezupft \\
	1 Eßlöffel & kochendes \myindex{Wasser} \\
	1 & \myindex{Zwiebel}, geviertelt \\
	4 & \myindex{Knoblauchzehe}n \\
	2 Eßlöffel & grob gehackte, \myindex{glatte Petersilie}
	             \index{Petersilie>glatt} oder \myindex{Koriander} \\
	125 g & getrocknete \myindex{dicke Bohnen}\index{Bohnen>dick},
	        abgespült und mindestens 12~Stunden in kaltem Wasser
		eingeweicht \\
	1\breh{} Eßlöffel & gemahlener \myindex{Kreuzkümmel} \\
	1\breh{} Eßlöffel & gemahlener \myindex{Koriander} \\
	1 Teelöffel & gemahlener \myindex{Kurkuma} \\
	1 Teelöffel & \myindex{Backpulver} \\
	2 Teelöffel & \myindex{Salz} \\
	& \myindex{Pfeffer} \\
	& Erdnuß-\index{Erdnußöl}\index{Oel=Öl>Erdnuß-} oder
	         \myindex{Sonnenblumenöl}\index{Oel=Öl>Sonnenblumen-} \\
      \end{zutaten}

      \stueck{30}

      \begin{zubereitung}
        Servieren mit Zitronenspalten, warmes Fladenbrot oder Pita-Brot,
	Sesamsoße (siehe Seite \pageref{sesamsosse}), Hummus (siehe Seite
	\pageref{hummus}) oder Joghurt-Gurken-Dip (siehe Seite
	\pageref{tzatziki}). \\
	Bulgur mit kochendem Wasser bedecken, abdecken und mindestens
	20~Minuten quellen lassen. Pitabrot in einer anderen Schüssel
	einweichen. Zwiebel und Knoblauch im Mixer fein hacken, Petersilie
	zugeben, Brot ausdrücken, Bohnen abgießen, abspülen, kurz trocknen und
	in den Mixer geben. Zu einer leicht körnigen Masse verarbeiten. In eine
	große Schüssel füllen. Bulgur ausdrücken und in die Schüssel geben,
	Gewürze dazu, ebenso Backpulver. Gut durchkneten und mit den Händen
	ca 30~walnußgroße Kugeln formen und auf 1~Zentimeter flachdrücken.
	Mit Frischhaltefolie bedecken und für mindestens 30~Minuten in den
	Kühlschrank stellen. \\
	In einer Friteuse oder schweren Pfanne ausreichend Öl zum Frittieren
	auf \grad{180} erhitzen (ein Brotwürfel sollte darin in 30~Sekunden
	bräunen). Die Falafel in kleinen Portionen 6--8~Minuten goldbraun
	frittieren. Mit einem Schaumlöffel auf ein Küchenpapier geben.
	Felafel heiß oder zimmerwarm servieren. \\
      \end{zubereitung}

    \mynewsection{Spinat-Feta-Ecken}

      \begin{einleitung}
        In traditionell griechischen Rezepten werden keine Rosinen- oder
	Pinienkerne verwendet, das ist ein arabischer Einfluß. Paßt aber
	wunderbar zu Feta und Spinat. \\
      \end{einleitung}

      \begin{zutaten}
        12 Blätter & \myindex{Filo-Teig} (ca. 30~cm x 23~cm) Tiefkühlware \\
	200 g & \myindex{Butter}, zerlassen und abgekühlt \\
      \end{zutaten}

      \begin{zutat}{Füllung}
        250 g & junger \myindex{Blattspinat}\index{Spinat>Blatt-} \\
	2 Eßlöffel & \myindex{Olivenöl}\index{Oel=Öl>Oliven-} + etwas zum
	             Einfetten \\
	4 & \myindex{Frühlingszwiebel}\index{Zwiebel>Frühlings-}n, fein
	    gehackt \\
	1 kleine & \myindex{Knoblauchzehe}, zerdrückt \\
	2 Eßlöffel & frisch gehackter \myindex{Dill} \\
	125 g & \myindex{Feta}\index{Käse>Feta}, zerbröckelt \\
	1 großes & \myindex{Ei}, verquirlt \\
	\brev{} Teelöffel & frisch geriebene \myindex{Muskatnuß} \\
	2 Eßlöffel & \myindex{Pinienkerne}, geröstet \\
	2 Eßlöffel & \myindex{Rosinen} \\
	& \myindex{Salz} \\
	& \myindex{Pfeffer} \\
      \end{zutat}

      \stueck{12}

      \begin{zubereitung}
        Spinat 10~Minuten im Topf garen, abkühlen lassen und auspressen.
	Öl in der Pfanne erhitzen, Frühlingszwiebeln andünsten, zum Spinat in
	die Schüssel geben. Dill, Ei, Feta, Muskat, eventuell Rosinen,
	Pinienkerne, Salz und Pfeffer dazu. Abschmecken. Die Füllung ist recht
	flüssig. \\
	Backofen auf \grad{180} vorheizen, ein oder zwei Backbleche dünn mit Öl
	bepinseln. 1~Blatt Filo-Teig auf die Arbeitsfläche legen und ganz mit
	zerlassener, abgekühlter Butter bestreichen. Zweites Filo-Blatt
	auflegen und mit Butter bestreichen, 3. Blatt ebenso. Diesen 3-lagigen
	Teig in 3~lange Streifen zu je 7,5~Zentimeter breit schneiden. Insgesamt
	gibt das 12 Streifen. Kurze Seite des Streifens nach vorn zeigen lassen.
	Restliche Streifen mit sauberem feuchten Tuch oder Küchenpapier vor dem
	Austrocknen bewahren. \\
	Füllung umrühren, 1~Eßlöffel auf die untere linke Ecke des Teigstreifens
	setzen, ca. 5~Millimeter von der Ecke entfernt. Ecke vorsichtig über
	die Füllung falten, so daß ein Dreieck entsteht und die ehemalige
	Unterkante des Teigstreifens jetzt auf dem rechten Rand liegt, Dreieck
	nach oben falten und dann nach links, so daß der offene Rand jetzt oben
	ist. Erneut nach oben umfalten, um die Füllung zu verschließen. So
	weiter machen, bis der obere Rand erreicht ist. Rand mit Wasser
	anfeuchten, Dreieck umfalten und andrücken. Mit der Naht nach unten auf
	das vorbereitete Blech legen und mit Butter bestreichen. Alle Streifen
	fertigstellen. \\
	In 12~Minuten goldbraun und knusprig backen, heiß oder zimmerwarm
	servieren. \\
      \end{zubereitung}

    \mynewsection{Käsetaschen}

      \begin{einleitung}
        Lassen sich gut vorher zubereiten und vor dem Servieren frisch
	backen. \\
      \end{einleitung}

      \begin{zutaten}
      \end{zutaten}

      \begin{zutat}{Teig}
        125 g & \myindex{Mehl} + etwas zum Bestäuben \\
	75 g & weißes \myindex{Pflanzenfett} (z.B. Kokosfett) \\
	\breh{} & \myindex{Zitrone}, entsaftet \\
	5 Eßlöffel & \myindex{Wasser}, gekühlt mit 1~Eiswürfel \\
	1 & \myindex{Ei}gelb verquirlt, zum Bestreichen \\
      \end{zutat}

      \begin{zutat}{Füllung}
        75 g & \myindex{Halloumi}, gewürfelt \\
	75 g & \myindex{Ricotta}\index{Käse>Ricotta} \\
	25 g & \myindex{Feta}\index{Käse>Feta}, zerbröckelt \\
	1 & \myindex{Ei}gelb \\
	1\breh{} Eßlöffel & frische, sehr fein gehackte \myindex{Minze} oder
	                    \myindex{Dill} \\
	\breh{} & \myindex{Zitrone}, Schale fein gerieben \\
	& \myindex{Salz} \\
      \end{zutat}

      \stueck{22--24}

      \begin{zubereitung}
        Mehl in eine Schüssel geben, Fett untermischen. Zitronensaft und nach
	und nach Wasser zugeben, zu einem weichen Teig verarbeiten. Auf
	bemehlter Arbeitsfläche zur Kugel kneten und in Frischhaltefolie
	wickeln, 30~Minuten mindestens in den Kühlschrank stellen. \\
	Füllungszutaten gut mischen, Minze und Zitronenschale kurz einarbeiten,
	abgedeckt in den Kühlschrank stellen (kann bis zu 24~Stunden darin
	bleiben). \\
	Backofen auf \grad{200} vorheizen, Backblech dünn einfetten. Teig
	halbieren und eine Hälfte auf der bemehlten Fläche 3~Millimeter dick
	ausrollen. Mit einem bemehlten Teigausstecher \durchmesser{}
	7,5~Zentimeter 12~Kreise austechen und eventuell Teigreste erneut
	ausrollen. Mit der 2.~Teighälfte ebenso verfahren. \\
	1~Teelöffel Füllung auf eine Hälfte eines Teigkreises setzen. Rand
	rundum mit Wasser befeuchten, Teig über die Füllung schlagen, sodaß
	ein Halbkreis entsteht, Ränder zusammendrücken. Zinken einer Gabel in
	Mehl tauchen und damit den Rand rundum andrücken. Tasche auf vorbereites
	Blech legen. Restliche Taschen fertigstellen, eventuell mit Folie
	bedecken und kalt stellen, um später zu backen. \\
	Tasche dünn mit Eigelb bestreichen. 15--18~Minuten goldbraun backen.
	Heiß oder lauwarm servieren. \\
      \end{zubereitung}


    \mynewsection{Eingemachte Zitronen}\label{eingemachtezitronen}

      \begin{einleitung}
        Werden viel in der arabischen Küche verwendet zum Aromatisieren. \\
      \end{einleitung}

      \begin{zutaten}
        6 & \myindex{Zitrone}n, fast geviertelt, Fruchtansatz bleibt
	    \underline{ganz}! \\
	2 & \myindex{Zitrone}n, entsaftet
      \end{zutaten}

      \begin{zubereitung}
        Schnittflächen mit grobem Meersalz einreiben. 2~Eßlöffel Meersalz in ein
	Schraubglas geben, das für 6~Zitronen gerade groß genug ist. 3~Zitronen
	hineingeben, zusammendrücken und mit einer dünnen Salzschicht bedecken.
	Die restlichen Zitronen zufügen und erneut mit Salz bestreuen.
	2~weitere Zitronen auspressen und den Saft in das Glas geben. Glas
	verschließen und 2--4~Wochen kalt stellen, dabei täglich 1x umdrehen.
	Danach können die Zitronen abgespült und verwendet werden. Im
	verschlossenen Glas halten sie sich im Kühlschrank mehrere Monate. \\
      \end{zubereitung}

    \mynewsection{Zitronenoliven}

      \begin{zutaten}
        250 g & gemischte, dicke grüne und schwarze \myindex{Oliven},
	        abgespült und trocken getupft \\
	& natives \myindex{Olivenöl}\index{Oel=Öl>Oliven-} extra \\
	\breh{} Teelöffel & gemahlener \myindex{Koriander} \\
	1 Teelöffel & getrockneter \myindex{Oregano} oder \myindex{Thymian} \\
	1 Prise & getrocknete \myindex{Chiliflocken} \\
	1 & eingemachte Zitrone (siehe Seite \pageref{eingemachtezitronen}),
	    abgespült und in dünne Scheiben geschnitten \\
	2 Teelöffel & frisch gehackter \myindex{Dill} oder \myindex{Koriander},
	              zum Garnieren \\
      \end{zutaten}

      \personen{4--6}

      \begin{zubereitung}
        Oliven in eine Schale geben und knapp mit Olivenöl bedecken, Koriander,
	Oregano und Chiliflocken unterrühren. Zitronenscheiben untermischen. Bis
	1~Woche im Kühlschrank haltbar. Kurz vor dem Servieren mit Dill
	bestreuen. \\
        Dazu: warmes Pita-Brot oder Fladenbrot. \\
      \end{zubereitung}

    \mynewsection{Paprika mit Fetafüllung}

      \begin{einleitung}
        Läßt sich gut vorher zubereiten. Man braucht lange, schlanke rote oder
	gelbe Paprika oder kurze dicke Chilis. Alternativ geröstete Paprika aus
	dem Glas. \\
      \end{einleitung}

      \begin{zutaten}
        75 g & \myindex{Feta}\index{Käse>Feta} \\
	12 & lange, schlanke rote oder gelbe \myindex{Paprika}, mit Olivenöl
	     eingerieben \\
	& natives \myindex{Olivenöl}\index{Oel=Öl>Oliven-} extra, zum Beträufeln \\
	& \myindex{Pfeffer} \\
	& \myindex{Rucola} zum Garnieren \\
      \end{zutaten}

      \stueck{12}

      \begin{zubereitung}
        Feta in einer Schüssel mit warmem Wasser bedecken. 1~Stunde wässern,
	Wasser dabei zwei- bis dreimal wechseln. Backofengrill vorheizen.
	Paprika auf ein Blech legen und auf der oberen Schiene unter
	einmaligem Wenden 10~Minuten grillen, bis die Haut gerade schwarz ist.
	In eine Schüssel geben und mit einem sauberen, zusammengefalteten
	Geschirrtuch abgedeckt abkühlen lassen. Haut abziehen, Stielenden
	abschneiden, sodaß die Spitzen 4~Zentimeter lang sind. Mit einem
	Teelöffel vorsichtig die Kerne entfernen, ohne das Fruchtfleisch zu
	verletzen. \\
	Feta gut abtropfen lassen und in einer Schüssel mit einer Gabel zu einer
	dicken Paste zerdrücken. Je 1~Teelöffel Feta in die Paprikaspitzen geben
	und vorsichtig mit den Fingern hineindrücken. \\
	Auf einem Teller anrichten, mit Öl beträufeln und nach Geschmack mit
	Pfeffer würzen. Bis zum Servieren abgedeckt in den Kühlschrank stellen.
	Zum Servieren mit Rucola garnieren. \\
	Aus den Paprika-/Chilies-Resten einen Salat bereiten oder in feine
	Streifen zum Kichererbsensalat oder in Stücke roh mit Sesamsoße. \\
      \end{zubereitung}

    \mynewsection{Ägyptische braune Bohnen}

      \begin{einleitung}
        Leicht erdig schmeckende Bohnen sind Nationalgericht in Ägypten, werden
	schon zum Frühstück gegessen. \\
      \end{einleitung}

      \begin{zutaten}
	300 g & getrocknete dicke \myindex{braune Bohnen}\index{Bohnen>braun},
	        abgespült und mindestens 12~Stunden mit 1~Eßlöffel Natron in
		kaltem Wasser eingeweicht \\
	2 Eßlöffel & \myindex{Olivenöl}\index{Oel=Öl>Oliven-} \\
	1 & \myindex{Zwiebel}, fein gehackt \\
	1 große & \myindex{Knoblauchzehe} mit 1~Teelöffel Salz zerdrückt \\
	1 große & \myindex{Tomate}, entkernt und fein gewürfelt \\
	& \myindex{Salz} \\
	& \myindex{Pfeffer} \\
      \end{zutaten}

      \begin{zutat}{Servieren}
	1 & frische \myindex{rote Chili}\index{Chili>rot}, eventuell entkernt,
	    fein gehackt ODER \\
	\breh{} Teelöffel & getrocknete \myindex{Chiliflocken} \\
	1 & \myindex{Zitrone}, halbiert \\
	& natives \myindex{Olivenöl}\index{Oel=Öl>Oliven-} extra \\
	& warmes Pita-Brot/Fladenbrot \\
      \end{zutat}

      \personen{4--6}

      \begin{zubereitung}
        Eingeweichte Bohnen abgießen, gut abspülen, in einen Topf geben, mit
	frischem, kaltem Wasser bedecken und zum Kochen bringen. 10~Minuten
	sprudelnd aufkochen, dabei gegebenenfalls Schaum abschöpfen. Auf
	schwache Hitze runterschalten und mindestens 2~Stunden köcheln lassen
	(dabei bei Bedarf Wasser nachfüllen), bis die Bohnen so weich sind, daß
	sie sich zwischen den Fingern zerdrücken lassen. Die Garzeit hängt vom
	Alter der Bohnen ab, je älter, desto länger brauchen sie. \\
	Bohnen abgießen, Kochwasser auffangen. Perfektionisten schälen jetzt
	die Bohnen, die meisten Hobbyköche nicht. \\
	Olivenöl in einer großen Pfanne erhitzen, Zwiebel auf mittlerer Stufe
	unter Rühren 3~Minuten andünsten. Knoblauch zufügen und weitere
	2~Minuten unter häufigem Rühren, bis die Zwiebel weicht ist
	(glasig/goldgelb). Die Hälfte der Bohnen mit dem Schaumlöffel in die
	Pfanne geben, zerdrücken und mit der Zwiebel mischen. Restliche Bohnen
	und Tomate zufügen und erhitzen. Salzen, pfeffern, abschmecken. \\
	Zu dicke Mischungen mit Bohnenkochwasser verdünnen. Bohnen in eine
	Schale füllen und nach Wunsch mit Chili bestreuen, eventuell den Saft
	aus Zitronenhälften dazu, Olivenöl darüber. Dazu Brot reichen. \\
      \end{zubereitung}

    \mynewsection{Kichererbsensalat}

      \begin{zutaten}
	\breh{} Eßlöffel & \myindex{Butter} \\
	1 Eßlöffel & \myindex{Olivenöl}\index{Oel=Öl>Oliven-} \\
	2 & \myindex{rote Zwiebel}\index{Zwiebel>rot}n, in dünne Ringe
	    geschnitten \\
	3 Eßlöffel & natives \myindex{Olivenöl}\index{Oel=Öl>Oliven-} extra \\
	\breh{} & \myindex{Zitrone}, entsaftet \\
	\breh{} Eßlöffel & frische \myindex{Thymian}blätter \\
	\breh{} Teelöffel & getrocknete \myindex{Thymian}blätter \\
	\brev{} Teelöffel & gemahlener \myindex{Kreuzkümmel} \\
	\brev{} Teelöffel & gemahlener \myindex{Koriander} \\
	1 Prise & gemahlener \myindex{Kurkuma} \\
	& \myindex{Salz} \\
	& \myindex{Pfeffer} \\
	1 Dose & \myindex{Kichererbsen} (400 g), abgespült und gut abgetropft \\
      \end{zutaten}

      \begin{zutat}{Servieren}
	1 & \myindex{Granatapfel}, halbiert \\
	2 Eßlöffel & frisch gehackte Kräuter (\myindex{Dill},
	                                      \myindex{Petersilie},
					      \myindex{Minze} oder
					      \myindex{Koriander}) \\
      \end{zutat}

      \personen{4--6}

      \begin{zubereitung}
        Butter und Öl in einer schweren Pfanne mit gut schließendem Deckel
	erhitzen. Zwiebeln auf mittlerer Stufe unter Rühren andünsten. Auf
	schwache Hitze runterschalten, ein feuchtes, leicht zerknülltes
	Backpapier auf die Zwiebeln legen und die Pfanne verschließen. Zwiebeln
	20~Minuten weiter schwach dünsten, bis sie sehr weich und zartrosa, aber
	nicht braun sind. \\
	Inzwischen Öl, Zitronensaft, Thymian, Kreuzkümmel, Koriander, Kurkuma,
	Salz und Pfeffer in einer Schüssel verrühren, Kichererbsen untermischen.
	\\
	Zwiebel auf Küchenkrepp geben und abtupfen. Zu den Kichererbsen geben,
	mischen. Abkühlen, dann abgedeckt für mindestens 1~Stunde in den
	Kühlschrank stellen. \\
	15~Minuten vor dem Servieren aus dem Kühlschrank nehmen. Abschmecken mit
	Salz, Pfeffer, eventuell Zitronensaft und Öl. Kräuter frisch gehackt
	unterrühren. Auf eine Granatapfelhälfte klopfen, damit die Kerne auf die
	Kichererbsen fallen. Salat umrühren und servieren. \\
      \end{zubereitung}

    \mynewsection{Rotweinkartoffeln}

      \begin{einleitung}
        Zyprioten genießen diese Kartoffeln meist heiß direkt aus der Pfanne,
	sie schmecken aber auch warm oder kalt sehr lecker. Man braucht eine
	große Pfanne mit einem dicht schließendem Deckel, damit die Kartoffeln
	nebeneinander Platz finden --- oder in 2~Portionen teilen. \\
      \end{einleitung}

      \begin{zutaten}
	12--16 & neue, fest kochende \myindex{Kartoffel}n, gründlich gesäubert
	         und trocken getupft (werden nicht geschält) \\
	& \myindex{Olivenöl}\index{Oel=Öl>Oliven-} \\
	125 ml & \myindex{Rotwein} \\
	2 Eßlöffel & \myindex{Koriander}samen, zerstoßen \\
	& \myindex{Salz} \\
	& \myindex{Pfeffer} \\
	& frischer \myindex{Thymian} ODER frisch gehackte,
	  \myindex{glatte Petersilie}\index{Petersilie>glatt} zum Garnieren \\
      \end{zutaten}

      \personen{4--6}

      \begin{zubereitung}
        Kartoffeln in eine sehr große Pfanne geben (oder 2~Pfannen nehmen), so
	daß alle Kartoffeln nebeneinander liegen. 1~Zentimeter hoch mit Olivenöl
	bedecken und das Öl auf mittlerer Stufe erhitzen, bis die ersten Blasen
	aufsteigen, es aber noch nicht kocht. Auf sehr schwache Hitze
	herunterschalten und abgedeckt 10--12~Minuten köcheln lassen, bis die
	Kartoffeln sehr weich sind und eine Messerspitze oder ein Spieß
	mühelos hineingleitet. Kartoffeln abgießen, dann zurück ind die Pfanne
	geben. Wein und Koriander sowie Salz und Pfeffer nach Geschmack
	zugeben und bei starker Hitze zum Kochen bringen. Ohne Deckel unter
	gelegentlichem Rütteln der Pfanne 3--4~Minuten kochen lassen, bis der
	ganze Wein verdampft ist. \\
	Kartoffeln in eine Servierschale geben und mit Thymian oder Petersilie
	bestreuen. Heiß oder zimmerwarm servieren. \\
      \end{zubereitung}

    \mynewsection{Gebratener Käse mit rotem Paprikasalat}

      \begin{einleitung}
        Meistens wird zypriotischer Halloumi gebraten oder gegrillt. Die
	Griechen bevorzugen Kasseri oder Kefalotiri. Alternativ kann man auch
	einen Provolone nehmen. \\
      \end{einleitung}

      \begin{zutaten}
	8 Scheiben & je 1~Zentimeter dick, trocken getupfter \myindex{Käse}
	             (siehe oben) \\
	2 Eßlöffel & \myindex{Mehl} zum Betäuben \\
	& \myindex{Olivenöl}\index{Oel=Öl>Oliven-} zum Braten \\
	& \myindex{Salz} \\
	& \myindex{Pfeffer} \\
	& eventuell \myindex{Zitrone}nspalten zum Servieren \\
      \end{zutaten}

      \begin{zutat}{Roter Paprikasalat}
	2 große & \myindex{rote Paprika}\index{Paprika>rot}, mit Olivenöl
	          eingerieben \\
	75 g & dicke \myindex{schwarze Oliven}\index{Oliven>schwarz}
	       (eventuell abspülen), entsteint und in Scheiben geschnitten \\
	4 & \myindex{Frühlingszwiebel}\index{Zwiebel>Frühlings-}n, fein
	    gehackt \\
	2 Eßlöffel & \myindex{Olivenöl}\index{Oel=Öl>Oliven-} nativ, extra \\
	2 Teelöffel & \myindex{Rotweinessig} \\
	& \myindex{Salz} \\
	& \myindex{Pfeffer} \\
	1 große & Handvoll \myindex{Rucola} zum Servieren \\
      \end{zutat}

      \personen{4}

      \begin{zubereitung}
        Für die Paprika Backofengrill vorheizen. Paprika auf ein Blech legen
	und unter einmaligem Wenden 15~Minuten grillen, bis die Paprika
	zusammenfällt und die Haut schwarz ist. Abkühlen lassen, Haut abziehen.
	Längs halbieren, entkernen. Fleisch in dünne Streifen schneiden. Mit
	den restlichen Zutaten vermischen, abschmecken mit Salz und Pfeffer.
	Beiseite stellen. \\
	Käsescheiben leicht mit Mehl bestäuben, abschütteln. Eine dünne Schicht
	Olivenöl in einer großen beschichteten Pfanne erhitzen. Auf mittlerer
	Hitze (Stufe 2) goldbraun braten, wenden. Auf großen Servierteller
	geben, restliche Scheiben ebenso braten. Rucola unter den Paprikasalat
	geben, eventuell nachwürzen. Eventuell Zitronenspalten zum Käse und
	Salat geben. \\
      \end{zubereitung}

    \mynewsection{Spinatomelett}

      \begin{einleitung}
        Arabischer Eierkuchen im Stil der italienischen Frittata oder
	spanischen Tortilla, kompakt, fest und gut schneidbar. \\
      \end{einleitung}

      \begin{zutaten}
	300 g & junger \myindex{Blattspinat}\index{Spinat>Blatt-},
	        schleudertrocken \\
	6 große & \myindex{Ei}er \\
	6 & \myindex{Frühlingszwiebel}\index{Zwiebel>Frühlings-}n, fein
	    gehackt \\
	4 Eßlöffel & frisch gehackte \myindex{Kräuter}, z.B. Schnittlauch,
	             Koriander, Dill, Minze und/oder glatte Petersilie \\
	100 g & \myindex{Kichererbsen}\index{Erbsen>Kicher-} aus der Dose,
	        abgespült, gut abgetropft \\
	& frisch geriebene \myindex{Muskatnuß} \\
	& \myindex{Salz} \\
	& \myindex{Pfeffer} \\
	3 Eßlöffel & \myindex{Olivenöl}\index{Oel=Öl>Oliven-} \\
	15 g & \myindex{Butter} \\
      \end{zutaten}

      \personen{4--6}

      \begin{zubereitung}
        Spinat zirka 10~Minuten garen, in ein Sieb geben, abkühlen lassen,
	überschüssige Feuchtigkeit ausdrücken. \\
	Eier in einer großen Schüssel verquirlen, Spinat, Frühlingszwiebeln,
	Kräuter, Kichererbsen, Muskat, Salz, Pfeffer zugeben und beiseite
	stellen. \\
	Backofengrill vorheizen. Beschichtete Pfanne mit feuerfestem Griff auf
	mittlerer Stufe erhitzen, Öl hineingeben und Pfanne etwas schwenken,
	Butter zufügen und erhitzen, bis es zischt. Die Eimischung in die
	Pfanne geben und sofort mit der Rückseite eines Holzlöffels glatt
	streichen. \\
	Auf mittlere Hitze runterschalten, 5~Minuten backen, bis der Boden
	goldbraun ist und zwischendurch das ungekochte Ei an der Oberfläche mit
	einer Gabel zur Mitte ziehen. Wenn der Boden und die Ränder gestockt
	sind, die Oberseite des Omelettes aber noch flüssig ist, die Pfanne
	unter den Backofengrill schieben und 5~Minuten überbacken, bis die
	Oberfläche ganz gestockt ist. \\
	Das Omelett auf einen Teller stürzen und mit Küchenpapier das Öl von
	der Oberfläche abtupfen. Auf Zimmertemperatur abkühlen lassen, dann in
	Tortenstücke schneiden und servieren. \\
      \end{zubereitung}

    \mynewsection{Auberginen mit Joghurt}

      \begin{einleitung}
        Klassisches Beispiel für die Vorliebe der Türken, heiße Zutaten mit
	gekühlten zu kombinieren. Da Auberginenscheiben beim Frittieren viel
	Öl aufsaugen, muß das Öl unbedingt die richtige Temperatur haben!
	Dadurch werden die Scheiben versiegelt und weniger fettig. \\
      \end{einleitung}

      \begin{zutaten}
	1 große & \myindex{Aubergine} (700 g) \\
	& \myindex{Salz} \\
	300 g & griechischer \myindex{Naturjoghurt}\index{Joghurt>Natur-} \\
	2 & \myindex{Knoblauchzehe}n, mit \breh{}~Teelöffel grobem Meersalz
	    zerdrückt \\
	& \myindex{Olivenöl}\index{Oel=Öl>Oliven-} zum Frittieren \\
	& getrocknete \myindex{Minze} zum Garnieren \\
      \end{zutaten}

      \personen{4--6}

      \begin{zubereitung}
        Die Aubergine in 1~Zentimeter dicke Scheiben schneiden. Eine große
	Schüssel, in der alle Auberginenscheiben Platz haben, mit kaltem
	Wasser füllen. 2~Eßlöffel Salz zufügen, unter Rühren auflösen.
	Auberginenscheiben in das Salzwasser geben, mit einem Teller
	beschweren, damit sie unter Wasser bleiben und 45~Minuten ziehen
	lassen. \\
	Unterdessen Joghurt und Knoblauch in einer Schüssel verrühren.
	Abschmecken und nach Wunsch noch zusätzlich Knoblauch und/oder Salz
	zugeben. Abgedeckt in den Kühlschrank stellen. \\
	Auberginenscheiben abgießen, abspülen und trocken tupfen. 5~Zentimeter
	Öl in einer großen Pfanne auf \grad{180} erhitzen (ein Brotwürfel sollte
	darin in 30~Sekunden bräunen). So viele Auberginenscheiben ins Öl
	geben, wie in einer Schicht nebeneinander passen und 1~Minute goldbraun
	frittieren. Scheiben wenden und 30--60~Sekunden auf der anderen Seite
	bräunen. \\
	Fertig gebratene Auberginenscheiben auf mit Küchenkrepp ausgelegten
	Teller setzen und trocken tupfen. Heiße Auberginen auf einem
	Servierteller anrichten und die gut gekühlte Joghurtsoße darübergeben.
	Mit Minze bestreut sofort servieren. \\
      \end{zubereitung}

    \mynewsection{Kräuter-Bulgur-Salat}

      \begin{einleitung}
        Libanon und Syrien behaupten, Heimat dieses leuchtend grünen Salats zu
	sein. Dort liegt die Bedeutung aber mehr auf den Kräutern, im Westen
	meist mehr auf dem Bulgur. \\
      \end{einleitung}

      \begin{zutaten}
	100 g & fein oder mittelfein geschroteter \myindex{Bulgur} \\
	1 & \myindex{Zitrone}, enstaftet \\
	& natives \myindex{Olivenöl}\index{Oel=Öl>Oliven-} extra, zum Bedecken \\
	1 großer Bund & \myindex{glatte Petersilie}\index{Petersilie>glatt},
	           Stiele entfernt, Blätter in Streifen geschnitten \\
	1 kleiner Bund & \myindex{Minze}, Blätter abzupfen, schneiden \\
	6 & \myindex{Frühlingszwiebel}\index{Zwiebel>Frühlings-}n, fein gehackt
	    \\
	2 & \myindex{Tomate}n, gehäutet, entkernt und fein gewürfelt \\
	& \myindex{Salz} \\
	& \myindex{Pfeffer} \\
      \end{zutaten}

      \begin{zutat}{Zum Servieren}
	& \myindex{Weinblätter} in Salzlake, gut abgespült und trocken
	  getupft \\
	& \myindex{Romana-Salat}blätter, gewaschen und trocken getupft ODER \\
	& warmes \myindex{Fladenbrot} oder \myindex{Pita-Brot} \\
      \end{zutat}

      \personen{4--6}

      \begin{zubereitung}
        Bulgur in ein Sieb geben und unter sehr heißem Wasser abspülen. Mit den
	Händen ausdrücken und in eine große Schüssel geben. 3~Eßlöffel
	Zitronensaft dazu. Bulgur knapp mit Öl bedecken. Mindestens 1~Stunde
	beiseitestellen, bis die Körner die Flüssigkeit aufgenommen haben. \\
	Petersilie, Minze, Frühlingszwiebeln, Tomaten, Salz, Pfeffer zugeben
	und mischen. Abschmecken. Sofort servieren oder abgedeckt in den
	Kühlschrank stellen. \\
	Kurz vor dem Servieren eine Servierschüssel mit Weinblättern auslegen
	und den Salat darauf anrichten. Alternativ mit Fladenbrot oder Pita-Brot
	reichen. \\
	Tip: Je länger Salat zieht, umsomehr mehr Öl und Zitronensaft nimmt er
	auf. Deshalb vorm Servieren gründlich umrühren und prüfen, ob noch
	Zitronensaft und/oder Öl nötig ist. \\
      \end{zubereitung}

    \mynewsection{Hirtensalat}

      \begin{einleitung}
        Dieser Salat ist im ganzen Mittelmeerraum verbreitet und wird auf seine
	eigene Weise zubereitet. Allen Rezepten ist jedoch gemeinsam, daß die
	Zutaten sehr sorgfältig in einheitlich große Würfel geschnitten werden.
	Paßt besonders gut zu gegrilltem Fleisch. \\
      \end{einleitung}

      \begin{zutaten}
	\breh{} & \myindex{Zitrone}, entsaftet \\
        1\breh{} Eßlöffel & natives \myindex{Olivenöl}\index{Oel=Öl>Oliven-} extra\\
	1 & \myindex{Knoblauchzehe} mit \brev{}~Teelöffel Salz zur Paste
	    zerdrückt \\
	3 große & reife \myindex{Tomate}n \\
	2 & rote oder grüne \myindex{Paprika} \\
	1 kleine & \myindex{Gurke} (200 g) \\
	50 g & frische \myindex{glatte Petersilie}\index{Petersilie>glatt},
	       fein gehackt \\
	30 g & frische \myindex{Minze}, fein gehackt \\
	& \myindex{Salz} \\
	& \myindex{Pfeffer} \\
      \end{zutaten}

      \begin{zutat}{Zum Servieren}
	& \myindex{Romana-Salat}blätter, gewaschen und getrocknet \\
	& warmes \myindex{Pita-Brot} \\
      \end{zutat}

      \personen{4--6}

      \begin{zubereitung}
        Zitronensaft, Öl, Knoblauch verrühren. Tomaten halbieren, entkernen,
	Fleisch in 5~Millimeter große Würfel schneiden und in eine Schüssel
	geben. Paprika halbieren, entkernen und in ebenso große Würfel
	schneiden. Gurke halbieren, Kerne entfernen, Fleisch ebenfalls in
	5~Millimeter Würfel schneiden. Petersilie, Minze, Salz und Pfeffer
	zugeben, mischen. Sofort servieren. \\
	Tip: Wenn der Salat vorher bereitet werden soll, werden die
	Gurkenwürfel mit Salz bestreut, im Sieb 20~Minuten ziehen gelassen,
	abgespült, getrocknet und zu den übrigen Zutaten gegeben. \\
      \end{zubereitung}

    \mynewsection{Brotsalat}

      \begin{einleitung}
        Dieser Salat ist im Nahen Osten allgegenwärtig und wurde zur
	Verwertung von Brot vom Vortag erfunden --- alle übrigen Zutaten
	sollten ganz frisch sein. \\
      \end{einleitung}

      \begin{zutaten}
	1 & \myindex{Gurke}, entkernt und in mundgerechte Stücke zerteilt \\
	2 & \myindex{Romana-Salat}herzen, zerzupft \\
	6 & \myindex{Radieschen}, halbiert und in dünne Halbkreise
	    geschnitten \\
	2 & \myindex{Tomate}n, halbiert, entkernt und gewürfelt \\
	4 & \myindex{Frühlingszwiebel}n, fein gehackt \\
	1 & grüne \myindex{Paprika}, entkernt, in Würfel geschnitten \\
	1 große Handvoll & frische glatte \myindex{Petersilie}, fein
	                   gehackt \\
	1 Eßlöffel & getrocknete \myindex{Minze} \\
	2 & \myindex{Pita-Brot}e, aufgeschnitten \\
      \end{zutaten}

      \begin{zutat}{Dressing}
	6 Eßlöffel & natives \myindex{Olivenöl}\index{Oel=Öl>Oliven-} extra \\
	1 & \myindex{Zitrone}, entsaftet \\
	& \myindex{Salz} \\
	1 Prise & \myindex{Sumach} (nach Belieben) \\
      \end{zutat}

      \personen{4--6}

      \begin{zubereitung}
        Gurkenstücke in ein Sieb geben, mit Salz bestreuen. 20~Minuten ziehen
	lassen. \\
	Backofengrill vorheizen. Für das Dressing Zutaten verrühren.
	Gurkenstücke abspülen, trocknen. In das Dressing geben. Salatblätter,
	Radieschen, Tomaten, Frühlingszwiebeln, Paprika, Petersilie, Minze
	zugeben, mischen. \\
	Pita-Brothälften unter dem Backofengrill von beiden Seiten knusprig
	und zartbraun rösten, dann in mundgerechte Stücke brechen, noch
	heiß zum Salat geben und umrühren. Sofort servieren. \\
	Tip: Gemüse kann mehrere Stunden im voraus mit dem Dressing gemischt
	und abgedeckt in den Kühlschrank gestellt werden. Das Brot sollte erst
	kurz vor dem Servieren zugegeben werden. \\
      \end{zubereitung}

    \mynewsection{Karottensalat}

      \begin{einleitung}
        Traditioneller marokkanischer Salat, bei dem die Karotten erst
	gekocht, dann gerieben und mit einem pikant--scharfen Dressing
	angemacht und dann kalt gestellt wird. Die Aromen können sich über
	Nacht besonders gut entfalten. \\
      \end{einleitung}

      \begin{zutaten}
	500 g & \myindex{Möhren}, geschält \\
	2 Eßlöffel & \myindex{Olivenöl}\index{Oel=Öl>Oliven-} oder
	             \myindex{Sonnenblumenöl}\index{Oel=Öl>Sonnenblumen} \\
	1 große & \myindex{Knoblauchzehe}, fein gehackt \\
	1\breh{} Teelöffel & gemahlener \myindex{Kreuzkümmel} \\
	1 Teelöffel & \myindex{Salz} \\
	1 Teelöffel & \myindex{Zucker} \\
	\breh{} Teelöffel & gemahlene \myindex{Kurkuma} \\
	\brev{} & \myindex{Harissa} (nach Geschmack) \\
	1 & \myindex{Zitrone}, entsaftet \\
	1 große & \myindex{Zitrone}, Schale fein gerieben \\
	& \myindex{Pfeffer} \\
	2 Eßlöffel & frisch gehackte \myindex{glatte Petersilie}
	             \index{Petersilie>glatt} zum Garnieren \\
      \end{zutaten}

      \begin{zutat}{Zum Servieren}
	& zerbröckelter \myindex{Feta}\index{Käse>Feta} \\
	& entsteinte \myindex{schwarze Oliven}\index{Oliven>schwarz} \\
      \end{zutat}

      \personen{4--6}

      \begin{zubereitung}
        Einen großen Topf mit Wasser bei starker Hitze zum sprudelnden Kochen
	bringen. Die ganzen Karotten hineingeben, die Hitze leicht mindern und
	die Karotten 10~Minuten weich kochen. Abgießen, dabei 200~Milliliter
	vom Kochsud auffangen und abkühlen lassen. Abgekühlte Karotten mit
	grobem Reibaufsatz der Küchenmaschine oder mit einer groben Handreibe
	reiben. In eine Schüssel geben. \\
	Öl in einer Pfanne erhitzen. Knoblauch auf mittlerer Stufe unter Rühren
	1--2~Minuten weichdünsten. Kreuzkümmel, Salz, Zucker, Kurkuma und
	Harissa zufügen und unter Rühren 30~Sekunden anschwitzen. 2~Eßlöffel
	Zitronensaft, die Zitronenschale und 150~Milliliter Kochsud zugeben und
	unter Rühren zum Kochen bringen. Herunterschalten und alles 5~Minuten
	köcheln lassen, dann über die Karotten geben und gründlich mischen.
	Vollständig abkühlen lassen. Abgedeckt über Nacht in den Kühlschrank
	stellen. \\
	Den Salat kurz vor dem Servieren umrühren, den restlichen Zitronensaft
	zugeben und eventuell mit Salz und Pfeffer abschmecken. Mit Petersilie
	bestreuen, dann Feta und Oliven unterheben und servieren. \\
      \end{zubereitung}

    \mynewsection{Eingelegte Rübchen}

      \begin{einleitung}
        Im Nahen Osten wird zu Mezze-Speisen immer auch sauer eingelegtes
	Gemüse gereicht. Rübchen sind dabei der Klassiker, aber auch anderes
	blasses Gemüse, wie z.B. Weißkohl und Blumenkohl kann derselben
	Farbbehandlung unterzogen werden. \\
      \end{einleitung}

      \begin{zutaten}
	500 g & Teltower \myindex{Rübchen}, geputzt \\
	1 kleine & gekochte \myindex{Rote Beete}, geschält und in dünne
	           Scheiben geschnitten \\
	1 & frisches \myindex{Lorbeer}blatt \\
      \end{zutaten}

      \begin{zutat}{Essigmarinade}
	300 ml & \myindex{Wasser} \\
	150 ml & \myindex{Weißwein}\index{Wein>weiß} oder
	         \myindex{Apfelessig}\index{Essig>Apfel} \\
	2 Eßlöffel & grobes \myindex{Meersalz} \\
	4 & \myindex{Knoblauchzehe}n, in Scheiben geschnitten \\
	1 Teelöffel & \myindex{Koriander}samen, zerstoßen \\
	\breh{} Teelöffel & getrocknete \myindex{Chiliflocken} (nach Belieben)
	                    \\
      \end{zutat}

      \ergibt{500 g}

      \begin{zubereitung}
        Für die Marinade Wasser und Essig in einem Topf zum Kochen bringen. Das
	Salz zufügen und unter Rühren auflösen. Vom Herd nehmen. Knoblauch,
	Koriander und nach Wunsch Chili unterrühren und ziehen lassen. \\
	Unterdessen einen Topf mit gesalzenem Wasser zum Kochen bringen.
	Rübchen darin 5~Minuten blanchieren, dann abgießen und abkühlen lassen.
	Abgekühlte Rübchen schälen und in 5~Millimeter dicke Scheiben schneiden.
	Rübchen und Rote Beete in ein 700~Milliliter Schraubglas schichten. \\
	Die abgekühlte Mariande durch ein Sieb über die Rübchen gießen. Das
	Gemüse muß komplett bedeckt sein. Das Lorbeerblatt an den Rand neben die
	Rübchen schieben. \\
	Das Glas fest verschlossen an einem kühlen, dunklen Ort 2~Tage ziehen
	lassen, dabei einmal am Tag umdrehen. Dann in den Kühlschrank stellen
	und vor Gebrauch weitere 12~Tage ziehen lassen. \\
	Variation: Eingelegte grüne Chilies sind ebenfalls häufig auf
	Mezze-Tafeln anzutreffen. Dazu ersetzt man die Rübchen durch frische
	grüne Chilies und verwendet von allen Zutaten nur die halbe Menge
	(einschließlich der Chilies). \\
      \end{zubereitung}

    \mynewsection{Rote-Beete-Salat}

      \begin{zutaten}
	900 g & \myindex{rote Beete} (roh) \\
	4 Eßlöffel & natives \myindex{Olivenöl}\index{Oel=Öl>Oliven-} extra \\
	1\breh{} Eßlöffel & \myindex{Rotweinessig}\index{Essig>Rotwein-} \\
	2 & \myindex{Knoblauchzehe}n, fein gehackt \\
	2 & \myindex{Frühlingszwiebel}n, sehr fein gehackt \\
	& grobes \myindex{Meersalz}\index{Salz>Meer-} \\
	& \myindex{Knoblauchsoße} zum Servieren (siehe Seite
	  \pageref{knoblauchsosse}, nach Belieben) \\
      \end{zutaten}

      \personen{4--6}

      \begin{zubereitung}
        Die Wurzeln der Roten Beete entfernen, ohne die Schale einzuschneiden,
	die Stiele auf 2\breh{}~Zentimeter kürzen. Die Knollen vorsichtig
	unter kaltem Wasser waschen, ohne die Schale zu verletzen. In einen
	Topf mit Wasser (bedeckt) geben, kochen, herunterschalten,
	25--40~Minuten kochen, bis die dickste Knolle weich einzustechen ist. \\
	Unterdessen Öl, Essig, Knoblauch, Frühlingszwiebeln und Salz in ein
	Schraubglas geben und gut schütteln. \\
	Rote Beete abgießen, unter kaltem Wasser abspülen. Abgekühlte Knollen
	schälen und in dicke Würfel oder Scheiben schneiden. In eine Schüssel
	geben und das Dressing darübergeben. Abgedeckt mindestens 1~Stunde im
	Kühlschrank ziehen lassen. \\
	Zum Servieren den Salat vorsichtig umrühren und auf einer Servierplatte
	anrichten. Nach Wunsch dazu eine Schale mit Knoblauchsoße reichen. \\
      \end{zubereitung}

    \mynewsection{Frittierte Tintenfischringe}

      \begin{einleitung}
        Jedes Mittelmeerland bereitet dieses Gericht ein wenig anders zu. Dieses
	ist eines der schlichtesten Rezepte, bei dem der Tintenfisch vor dem
	Frittieren nur leicht mit gewürztem Mehl bestäubt wird. Die Ringe
	müssen bei der richtigen Temperatur frittiert werden, da sie sonst zäh
	werden. \\
      \end{einleitung}

      \begin{zutaten}
	& \myindex{Olivenöl}\index{Oel=Öl>Oliven-} zum Frittieren \\
	100 g & \myindex{Mehl} \\
	1 Prise & scharfer \myindex{Paprika} oder \myindex{Chilipulver} \\
	& \myindex{Salz} \\
	& \myindex{weißer Pfeffer}\index{Pfeffer>weiß} \\
	600 g & \myindex{Tintenfisch}ringe und -tentakel (sofern erhältlich),
	        abgespült und trocken getupft\\
	& \myindex{Koriander}zweige zum Garnieren \\
      \end{zutaten}

      \begin{zutat}{Zum Servieren}
	& grobes \myindex{Meersalz}\index{Salz>Meer} zum Bestäuben \\
	& \myindex{Zitrone}nspalten zum Beträufeln nach Belieben \\
	& \myindex{Sesamsoße} (siehe Seite \pageref{sesamsosse}, nach Belieben)
	  zum Dippen \\
      \end{zutat}

      \personen{4--6}

      \begin{zubereitung}
        In der Fritteuse oder einer schweren Pfanne ausreichend Öl zum
	Frittieren auf \grad{180} erhitzen (ein Brotwürfel sollte darin in
	30~Sekunden bräunen). \\
	Mehl und Paprika sowie Salz und Pfeffer in einen Gefrierbeutel geben
	und gut schütteln. Den Tintenfisch hineingeben und durch Schütteln
	rundum mit dem Mehl bestäuben. Die Tintenfischstücke mit einer Zange
	aus dem Beutel nehmen und überschüssiges Mehl abschütteln. \\
	Die Tintenfischringe und -stücke in kleinen Portionen in das heiße Öl
	geben und 3~Minuten goldbraun frittieren, dabei gelegentlich wenden.
	Mit einem Schaumlöffel herausnehmen und auf einen Teller mit
	Küchenpapier abtropfen lassen. Leicht mit Meersalz bestreuen und im
	Backofen warm halten, während der restliche Tintenfisch frittiert wird.
	Darauf achten, daß das Öl immer wieder die richtige, sehr heiße
	Temperatur annimmt, alte Panadereste stets entfernen. \\
	Tintenfisch garnieren und heiß servieren, eventuell mit Zitronenspalten
	oder Semsamsoße zum Dippen. Dazu Brot. \\
      \end{zubereitung}

    \mynewsection{Chiligarnelen}

      \begin{einleitung}
        Diese pikanten Garnelen sind in den Restaurants, Tavernen und Bars
	entlang der gesamten Mittelmeerküste sehr populär. Reichlich Brot dazu
	Servieren zum Auftunken der köstlichen Soße. \\
      \end{einleitung}

      \begin{zutaten}
	2 große & \myindex{Knoblauchzehe}n, fein gehackt \\
	1 & frische \myindex{rote Chili}\index{Chili>rot}, gehackt oder
	    1~Teelöffel getrocknete \myindex{Chiliflocken} \\
	5 Eßlöffel & \myindex{Olivenöl}\index{Oel=Öl>Oliven-} \\
	& \myindex{Salz} \\
	600 g große & rohe \myindex{Garnelen} \\
      \end{zutaten}

      \begin{zutat}{Zum Servieren}
	& \myindex{Zitrone}nspalten (nach Belieben) \\
	& \myindex{Brot} \\
      \end{zutat}

      \personen{4--6}

      \begin{zubereitung}
        Knoblauch, Chili, Öl und Salz nach Geschmack in eine Auflaufform oder
	in eine kleine feuerfeste Schale geben. Die Köpfe der Garnelen
	abtrennen und wegwerfen und die Schwänze auslösen. Mit einem kleinen
	scharfen Messer den Darmfaden entfernen. Dazu die Garnelen am Rücken
	entlang leicht einschneiden und den dunklen Faden mit der Messerspitze
	herausheben. \\
	Die Garnelen ins Öl geben und mit den Händen mit der Marinade einreiben.
	Abgedeckt im Kühlschrank zirka 1~Stunde marinieren. Den Backofen auf
	\grad{225} beziehungsweise auf höchster Stufe vorheizen. Garnelen gut
	umrühren, dann im vorgeheizten Backofen 6~Minuten backen, bis sie rosa
	werden und innen gar sind (zur Garprobe eine Garnele durchschneiden).
	Direkt aus dem Ofen in der Auflaufform servieren. \\
	Tip: Rohe Garnelen sollten immer am Tag des Einkaufs zubereitet und zu
	Hause sofort in den Kühlschrank gelegt werden. \\
      \end{zubereitung}

    \mynewsection{Miesmuscheln im Bierteig}

      \begin{einleitung}
        Diese direkt aus der Fritteuse servierten Muscheln erfreuen sich in
	zahlreichen Tavernen am Bosporus in Instanbul höchster Beliebtheit. Die
	Hefe im Bier sorgt für einen lockeren Teig, der beim Ausbacken knusprig
	und goldbraun wird. Dazu eine oder zwei gekühlte Soßen reichen. \\
      \end{einleitung}

      \begin{zutaten}
	24--36 große & \myindex{Miesmuscheln}\index{Muscheln>Mies-} \\
	& \myindex{Olivenöl}\index{Oel=Öl>Oliven-} zum Frittieren \\
	& \myindex{Mehl} zum Bestäuben \\
      \end{zutaten}

      \begin{zutat}{Bierteig}
	50 g & \myindex{Mehl} \\
	\brev{} Teelöffel & \myindex{Salz} \\
	1 Eßlöffel & \myindex{Olivenöl}\index{Oel=Öl>Oliven-} \\
	1 & \myindex{Ei}, getrennt \\
	100 ml & \myindex{Bier} \\
      \end{zutat}

      \begin{zutat}{Servieren}
	& grobes \myindex{Meersalz}\index{Salz>Meer-} \\
	& \myindex{Zitrone}nspalten \\
	& \myindex{Sesamsoße} (siehe Seite \pageref{sesamsosse}) \\
	& \myindex{Knoblauchsoße} (siehe Seite \pageref{knoblauchsosse}) \\
	& \myindex{Walnußsoße} \\
      \end{zutat}

      \personen{4--6}

      \begin{zubereitung}
        Für den Bierteig Mehl und Salz in eine größere Schüssel sieben und in
	der Mitte eine Vertiefung machen. Öl und Eigelb in einer zweiten
	Schüssel verrühren und in die Vertiefung geben. Mit der Gabel oder einem
	Schneebesen nach und nach die flüssigen Zutaten mit dem Mehl vom Rand
	her verrühren. Langsam das Bier zugießen und glatt rühren (eventuell
	braucht man weniger oder mehr Bier). Abgedeckt bei Zimmertemperatur
	mindestens 1~Stunde quellen lassen. \\
	Unterdessen die Miesmuscheln abbürsten, Bärte entfernen. Miesmuscheln
	aussortieren, wenn sie sich auf Klopfen nicht schließen. Die gesäuberten
	Muscheln nur mit dem anhaftenden Wasser in einen großen, schweren Topf
	geben und fest verschlossen bei mittlerer Hitze unter häufigem Rütteln
	des Topfes 3--4~Minuten garen, bis die Muscheln sich geöffnet haben.
	Nicht geöffnete Muscheln aussortieren. Abgießen und abkühlen lassen,
	dann das Muschelfleisch aus den Schalen lösen. Abgedeckt in den
	Kühlschrank stellen. \\
	In der Fritteuse oder einer schweren Pfanne ausreichend Öl zum
	Frittieren auf \grad{180} erhitzen (ein Brotwürfel sollte darin in
	30~Sekunden bräunen). Teig gut verrühren, Eiweiß in einer anderen
	Schüssel steif schlagen, unter den Teig heben. Muscheln trocken
	tupfen, leicht mit Mehl bestäuben, Überschuß abschütteln. Muscheln in
	den Teig geben. Mit einer Zange oder 2~Gabeln unter einmaligem Wenden
	1--2~Minuten goldbraun ausbacken. Mit einem Schaumlöffel rausnehmen,
	auf dem Küchenkrepp abtropfen lassen, im Ofen warm halten. Dann den
	Rest frittieren, darauf achten, daß die Temperatur wieder hochgeheizt
	wird und Reste der Panade aus dem Öl holen. \\
	Muscheln mit Meersalz bestreuen, Zitronenspalten, Sesamsoße und
	Walnuß-Knoblauchsoße dazu reichen. \\
      \end{zubereitung}

    \mynewsection{Schwertfischspieße}

      \begin{einleitung}
        Schwertfisch ist vor den Küsten der Türkei weit verbreitet, und diese
	Spieße werden daher in vielen Hafenrestaurants serviert. \\
      \end{einleitung}

      \begin{zutaten}
	20 & frische \myindex{Lorbeer}blätter \\
	& \myindex{Olivenöl}\index{Oel=Öl>Oliven-} zum Einölen \\
	600 g & grätenfreie \myindex{Schwertfisch}steaks, ca. 2\breh{}~cm
	        dick, in 2\breh{}~cm große Würfel geschnitten \\
      \end{zutaten}

      \begin{zutat}{Marinade}
	4 Eßlöffel & natives \myindex{Olivenöl}\index{Oel=Öl>Oliven-} extra \\
	\breh{} & \myindex{Zitrone}, entsaftet \\
	1 & \myindex{Knoblauchzehe}, zerdrückt mit \brev{}~Teelöffel Salz \\
	\brev{}~Teelöffel & \myindex{weißer Pfeffer}\index{Pfeffer>weiß} \\
	1 Prise & scharfe oder geräucherte \myindex{Paprika} \\
	1 & \myindex{Zwiebel}, halbiert und in Spalten geschnitten \\
	4 & frische \myindex{Lorbeer}blätter in der Hälfte durchgerissen \\
      \end{zutat}

      \begin{zutat}{Dressing}
	5 Eßlöffel & natives \myindex{Olivenöl}\index{Oel=Öl>Oliven-} extra \\
	1\breh{} & \myindex{Zitrone}n, entsaftet \\
	2 Eßlöffel & frisch gehackter \myindex{Dill} \\
      \end{zutat}

      \personen{4--6}

      \begin{zubereitung}
        Für die Marinade Öl, Zitronensaft, Knoblauch, Pfeffer und Paprika in
	einer Schüssel verrühren. Die Fischwürfel hineingeben und mit den Händen
	vorsichtig in der Marinade wenden. Die Zwiebel und die halbierten
	Lorbeerblätter darüberstreuen. Abgedeckt im Kühlschrank mindestens
	4~Stunden marinieren. \\
	Dressing in einer kleinen Schüssel verrühren, abdecken. \\
	Die ganzen Lorbeerblätter in eine Schüssel geben, mit kochendem Wasser
	knapp bedecken und 1~Stunde einweichen. Abgießen und trocken tupfen.
	Wenn Sie Holzspieße verwenden, sollten diese zeitgleich mit den
	Lorbeerblättern in kaltem Wasser eingeweicht werden. \\
	Eine gußeiserne Grillpfanne auf höchster Stufe erhitzen oder den
	Backofen- bzw. Holzkohlegrill vorheizen. 4~lange flache Metallspieße
	bzw. die eingeweichten Holzspieße dünn mit Öl bestreichen. Die
	marinierten Fischwürfel in 4~gleich große Portionen aufteilen und mit je
	5~Lorbeerblättern abwechselnd auf die Spieße stecken. \\
	Die Grillpfanne bzw. den Grillrost dünn mit Öl einpinseln. Die Spieße
	in die Pfanne bzw. auf den Rost legen und unter häufigem Wenden und
	Bestreichen mit der restlichen Marinade 8--10~Minuten grillen, bis sich
	der Fisch fest anfühlt. \\
	Die Spieße oben mit einem gefalteten Tuch anfassen und Fischwürfel und
	Lorbeerblätter mit einer Gabel vom Spieß ziehen. Die Lorbeerblätter
	entfernen und die Fischwürfel auf einer Platte anrichten. Mit Dressing
	zum Beträufeln servieren. \\
      \end{zubereitung}

    \mynewsection{Thunfischrollen}

      \begin{einleitung}
        Diese rasch gezauberten knusprigen Filo-Teig-Röllchen werden mit
	Thunfisch aus der Dose zubereitet. Sie schmecken am besten heiß, können
	aber auch kalt zu Picknicks mitgenommen werden. \\
      \end{einleitung}

      \begin{zutaten}
	& \myindex{Olivenöl}\index{Oel=Öl>Oliven-} zum Einölen und Bestreichen
	  \\
	12 Blätter & \myindex{Filo-Teig} (ca. 30x23 cm), Tiefkühlware
	             aufgetaut \\
	1 Eßlöffel & \myindex{Sesamsaat}, zum Garnieren \\
	& \myindex{Zitrone}nspalten zum Servieren \\
      \end{zutaten}

      \begin{zutat}{Füllung}
	1 Dose & \myindex{Thunfisch} (200 g) in Öl, abgetropft und 1~Eßlöffel Öl
	         aufgefangen \\
	1 & hartgekochtes \myindex{Ei}, geschält und fein gehackt \\
	2 Eßlöffel & frisch gehackter \myindex{Dill} \\
	1\brev{} Eßlöffel & \myindex{Tomatenmark} \\
	\brev{} Teelöffel & \myindex{Harissa} (nach Geschmack) \\
	25 & dicke, aromatische \myindex{schwarze Oliven}\index{Oliven>schwarz}
	     (in Salzlake eingelegte Oliven abspülen), entsteint und fein
	     gehackt \\
	& \myindex{Salz} \\
	& \myindex{Pfeffer} \\
      \end{zutat}

      \ergibt{8 Stück}

      \begin{zubereitung}
        Backofen auf \grad{180} vorheizen und ein Backblech einölen. Für die
	Füllung den abgetropften Thunfisch und 1~Eßlöffel Thunfischöl mit Ei,
	Dill, Tomatenmark und Harissa in einer Schüssel zu einer gut gemischten
	Masse zerdrücken. Die Oliven unterrühren und mit Salz und Pfeffer nach
	Geschmack würzen. \\
	Ein Blatt Filo-Teig auf eine Arbeitsfläche legen und ganz mit Öl
	bestreichen. Ein zweites Filo-Blatt darauflegen und ebenfalls mit Öl
	bestreichen, dann ein drittes Blatt ebenso drauflegen und bestreichen.
	Den geschichteten Teig in 2~lange Streifen (je 11,5 Zentimeter breit)
	schneiden. Auf diese Weise insgesamt 8 Teigstreifen zubereiten. Einen
	Streifen so legen, daß die kurze Seite nach vorne zeigt. Die übrigen
	Streifen mit feuchtem Küchenpapier abdecken, damit sie nicht
	austrocknen. \\
	Ein Achtel der Füllung in einer Linie oben auf den Teig geben, dabei
	zum oberen Rand und zu den Seitenrändern je 1~Zentimeter frei lassen.
	Den oberen Teigrand über die Füllung falten und den Teig einmal eng um
	die Füllung nach vorne rollen. Die beiden langen Seiten einschlagen,
	sodaß die Ränder zur Mitte zeigen. Den Teig weiter um die Füllung
	aufrollen, bis das Ende des Streifens erreicht ist.  Mit der Naht nach
	unten auf das Backblech legen. Restliche Teigstreifen fertigstellen. \\
	Die Rollen mit Öl bestreichen und dünn mit Sesam bestreuen. Im
	vorgeheizten Ofen 12--15~Minuten knusprig und goldbraun backen. Auf
	einen Teller mit Küchenpapier legen und leicht abkühlen lassen. \\
	Nach Wunsch mit Zitronenspalten servieren. \\
      \end{zubereitung}

    \mynewsection{Bratfisch mit Walnuß-Knoblauch-Soße}

      \begin{einleitung}
        Ein im Ganzen gebratener Fisch, kalt in dieser cremigen Soße
	angerichtet, gehört im Nahen Osten zu den traditionellen
	Lieblingsgerichten, mit denen eine größere Gästeschar bewirtet wird. \\
      \end{einleitung}

      \begin{zutaten}
	4 & weiße \myindex{Fischfilets} mit Haut wie z.B. 
	    \myindex{Barsch}\index{Fisch>Barsch},
	    \myindex{Brasse}\index{Fisch>Brasse},
	    \myindex{Meerbarbe}\index{Fisch>Meerbarbe} oder
	    \myindex{Red Snapper}\index{Fisch>Red Snapper} \'a 125 g \\
	1--2 Eßlöffel & \myindex{Olivenöl}\index{Oel=Öl>Oliven-} zum Braten \\
	& \myindex{Zitrone}nspalten zum Servieren \\
      \end{zutaten}

      \begin{zutat}{Marinade}
	125 ml & \myindex{Olivenöl}\index{Oel=Öl>Oliven-} \\
	2 & frische \myindex{Lorbeer}blätter, halb durchgerissen \\
	1 & \myindex{Knoblauchzehe}, fein gehackt \\
	2 Eßlöffel & frisch gehackte \myindex{Kräuter}, z.B.
	             \myindex{glatte Petersilie}\index{Petersilie>glatt},
	             \myindex{Rosmarin},
	             \myindex{Thymian},
	             \myindex{Majoran} oder
	             \myindex{Schnittlauch} \\
	& \myindex{Salz} \\
	& \myindex{Pfeffer} \\
      \end{zutat}

      \begin{zutat}{Walnuß-Knoblauch-Soße}
	2 Scheiben & \myindex{Weißbrot}\index{Brot>Weiß-} (50 g) vom Vortag
	             ohne Rinde in kleine Stücke gezupft \\
	2 Eßlöffel & \myindex{Milch} \\
	100 g & Walnuß\index{Walnüsse}kerne \\
	3 & \myindex{Knoblauchzehe}n \\
	\breh{} & \myindex{Zitrone}, entsaftet (nach Belieben) oder
	         \myindex{Weißweinessig}\index{Essig>Weißwein-} \\
	200 ml & \myindex{Olivenöl}\index{Oel=Öl>Oliven-} \\
      \end{zutat}

      \personen{4}

      \begin{zubereitung}
        \begin{enumerate}
	  \item In einer Auflaufform alle Zutaten für die Marinade verrühren.
	        Fischfilets hinein und wenden. Abdecken, im Kühlschrank
		2--4~Stunden marinieren.
	  \item Soße bereiten. Brot mit Milch beträufeln, Nüsse und Knoblauch
	        im Mixer sehr fein hacken, Brot, Zitronensaft und 3~Eßlöffel
		Öl zugeben und zu einer dicken klebrigen Paste verarbeiten.
		Bei laufendem Gerät restliches Öl hineinträufeln, sodaß eine
		glatte Soße entsteht. Abschmecken mit Salz, Zitronensaft.
		Abgedeckt in den Kühlschrank stellen.
	  \item Große beschichtete Pfanne auf höchster Stufe erhitzen, bis ein
	        Wassertropfen darin abperlt. 1~Eßlöffel Öl darin erhitzen.
		Fischfilets nebeneinander mit der Hautseite nach oben in die
		Pfanne geben und 5~Minuten goldbraun braten. Mit einem
		Pfannenwender wenden und weitere 1--2~Minuten braten, bis sich
		das Fleisch fest anfühlt. Auf einen Teller mit Küchenpapier
		geben. Falls die Filets nicht alle in die Pfanne gepaßt haben,
		die gebratenen Filets im Ofen warm halten.
	  \item Eine dünne Schicht Soße auf einen Teller verstreichen und die
	        Fischfilets darauf mit der Hautseite nach unten anrichten.
		Sofort servieren, eventuell mit Zitronenspalten.
        \end{enumerate}
      \end{zubereitung}

    \mynewsection{Gegrillte Sardinen mit Chermonla}

      \begin{einleitung}
        Marokkanisches Essen ist für seine kühnen Aromen und seinen
	großzügigen Umgang mit Gewürzen bekannt. Chermonla --- eine
	pikant-scharfe Knoblauch-Kräutersoße mit ordentlicher Schlagkraft ---
	gehört zu den Klassikern der marokkanischen Küche. \\
      \end{einleitung}

      \begin{zutaten}
	12 & \myindex{Sardinen}, ohne Kopf, gesäubert und gespült \\
	& \myindex{Olivenöl}\index{Oel=Öl>Oliven-} zum Einreiben und Einölen \\
	& grobes \myindex{Meersalz}\index{Salz>Meer-} zum Bestreuen \\
	& grob gezupfter \myindex{Koriander} zum Garnieren \\
	& \myindex{Zitrone}nspalten zum Servieren \\
      \end{zutaten}

      \begin{zutat}{Chermonla}
	40 g & frische \myindex{Koriander}blätter \\
	25 g & frische \myindex{glatte Petersilie}nblätter
	               \index{Petersilie>glatt} \\
	1\breh{} Eßlöffel & gemahlener \myindex{Kreuzkümmel} \\
	1 Teelöffel & geräucherte \myindex{Paprika} \\
	\breh{} Teelöffel & \myindex{Cayennepfeffer}\index{Pfeffer>Cayenne-} 
	                    (o. nach Geschmack) \\
	1 & \myindex{Zitrone}, entsaftet \\
	1 Teelöffel & \myindex{Salz} \\
	3 Eßlöffel & \myindex{Olivenöl}\index{Oel=Öl>Oliven-} \\
      \end{zutat}

      \personen{4}

      \begin{zubereitung}
        Für die Chermonla Koriander, Petersilie, Kreuzkümmel, Paprika,
	Cayennepfeffer, Zitronensaft und Salz im Mixer fein hacken. Das Öl
	zugeben und zu einer dickflüssigen Soße verarbeiten (eventuell Öl
	nachgeben). \\
	Je 1~Teelöffel Chermonla in die Bauchöhlen der Sardinen verstreichen.
	Abgedeckt bei Zimmertemperatur 15~Minuten oder im Kühlschrank
	2--4~Stunden marinieren. \\
	Eine gußeiserne Gußpfanne auf höchster Stufe erhitzen oder den Backofen-
	oder Holzkohlegrill vorheizen. Die Sardinen mit Öl einreiben und
	leicht mit Meersalz bestreuen. \\
	Die Grillpfanne bzw. den Grillrost mit Öl bestreichen. Die Sardinen
	in der Pfanne bzw. auf den Rost geben und 3~Minuten grillen. Mit einem
	Pfannenwender wenden, noch einmal mit Salz bestreuen und weitere
	2~Minuten grillen, bis die Fische leicht geschwärzt und gar sind. \\
	Mit Koriander garnieren, sofort mit Zitronenspalten servieren. \\
      \end{zubereitung}

    \mynewsection{Kichererbsen-Curry}

      \begin{zutaten}
	6 Eßlöffel & Öl\index{Oel=Öl} \\
	2 & \myindex{Zwiebel}n, in Ringe geschnitten \\
	1 Teelöffel & frisch gehackter \myindex{Ingwer} \\
	1 Teelöffel & gemahlener \myindex{Kreuzkümmel} \\
	1 Teelöffel & gemahlener \myindex{Koriander} \\
	1 & \myindex{Knoblauchzehe}, zerdrückt \\
	1 Teelöffel & \myindex{Chilipulver} \\
	2 & frische \myindex{grüne Chili}\index{Chili>grün}s \\
	\breh{} Bund & frischer \myindex{Koriander}, gehackt \\
	150 ml & \myindex{Wasser} \\
	1 große & \myindex{Kartoffel}, in Würfel geschnitten \\
	480 g & \myindex{Kichererbsen} aus der Dose abgetropft (oder getrocknete
	        Kichererbsen über Nacht einweichen, 30--60~Minuten kochen) \\
	1 Eßlöffel & \myindex{Zitrone}nsaft \\
	& \myindex{Salz} \\
      \end{zutaten}

      \personen{4}

      \begin{zubereitung}
        Das Öl in einem großen Topf erhitzen. Zwiebelringe zugeben und unter
	Rühren goldbraun braten. Hitze reduzieren und Ingwer, Kreuzkümmel,
	gemahlener Koriander, Knoblauch, Chilipulver, Chilis und frischen
	Koriander zugeben. Unter Rühren 2~Minuten anbraten. Das Wasser zufügen
	und alles gut verrühren. Die Kartoffel schälen und in kleine Würfel
	schneiden. Die Kartoffel mit den Kichererbsen zu der
	Zwiebel-Gewürzmischung in den Topf geben und abgedeckt 5--7~Minuten
	unter gelegentlichem Rühren köcheln. Das Curry mit dem Zitronensaft
	beträufeln und mit Salz abschmecken. Anrichten und servieren. \\
      \end{zubereitung}

    \mynewsection{Kichererbsen-Salat}

      \begin{zutaten}
	480 g & \myindex{Kichererbsen} aus der Dose \\
	4 & \myindex{Möhren} \\
	1 Bund & \myindex{Frühlingszwiebel}n \\
	1 & \myindex{Salatgurke}\index{Gurke>Salat-} \\
	\breh{} Teelöffel & \myindex{Salz} \\
	\breh{} Teelöffel & \myindex{Pfeffer} \\
	3 Eßlöffel & \myindex{Zitrone}nsaft \\
	1 & \myindex{rote Paprika}\index{Paprika>rot} \\
      \end{zutaten}

      \personen{4}

      \begin{zubereitung}
        Abgetropfte Kichererbsen in eine große Schüssel geben. Die Karotten
	schälen und in feine Stifte schneiden. Frühlingszwiebeln putzen und in
	dünne Streifen schneiden. Die Gurke in dicke Scheiben schneiden, dann
	die Scheiben vierteln. Karottenscheiben, Frühlingszwiebeln und Gurken
	in die Salatschüssel geben und mit den Kichererbsen mischen. Mit Salz
	und Pfeffer abschmecken, mit Zitronensaft beträufeln. Gut durchmischen.
	Die Paprika entkernen und in dünne Streifen schneiden. Den Salat mit
	diesen Streifen garnieren. Sofort servieren oder abgedeckt in den
	Kühlschrank. \\
      \end{zubereitung}

    \mynewsection{Kichererbsen-Snack}

      \begin{zutaten}
	480 g & \myindex{Kichererbsen} aus der Dose, abgetropft \\
	1 & \myindex{Zwiebel}, fein gehackt \\
	2 & \myindex{Kartoffeln}, gewürfelt \\
	2 Eßlöffel & \myindex{Tamarindenpaste} \\
	6 Eßlöffel & \myindex{Wasser} \\
	1 Teelöffel & \myindex{Chilipulver} \\
	2 Teelöffel & \myindex{Zucker} \\
	1 Teelöffel & \myindex{Salz} \\
      \end{zutaten}

      \begin{zutat}{Garnierung}
	1 & \myindex{Tomate}, in Spalten geschnitten \\
	2 & frische \myindex{grüne Chili}\index{Chili>grün}s, gehackt \\
	2--3 Eßlöffel & frisch gehackter \myindex{Koriander} \\
      \end{zutat}

      \personen{2--4}

      \begin{zubereitung}
        Kichererbsen in eine Schüssel geben. Zwiebel fein hacken. Kartoffeln
	würfeln und weich kochen. Tamarindenpaste und Wasser in einer kleinen
	Schüssel verrühren. Chilipulver, Zucker und Salz zur Tamarindenmischung
	geben, zu den Kichererbsen geben. Kartoffeln und Zwiebel zufügen. Mit
	Salz abschmecken. In eine Servierschüssel geben und garnieren mit
	Tomaten, Chilis und Koriander. \\
      \end{zubereitung}

    \mynewsection{Auberginen in Joghurtsoße}

      \begin{zutaten}
	2 & \myindex{Aubergine}n \\
	4 Eßlöffel & Öl\index{Oel=Öl} \\
	1 & \myindex{Zwiebel}, in Ringe geschnitten \\
	1 Teelöffel & weiße \myindex{Kreuzkümmelsamen} \\
	1 Teelöffel & \myindex{Chilipulver} \\
	1 Teelöffel & \myindex{Salz} \\
	3 Eßlöffel & \myindex{Naturjoghurt}\index{Joghurt>Natur-} \\
	\breh{} Teelöffel & \myindex{Minzsoße} \\
      \end{zutaten}

      \begin{zutat}{Zum Garnieren}
	& frische gehackte \myindex{Minze} \\
      \end{zutat}

      \personen{4}

      \begin{zubereitung}
        Den Backofen auf \grad{160} vorheizen. Auberginen gewaschen und
	getrocknet in eine Auflaufform geben für 45~Minuten. Abkühlen lassen.
	Mit einem scharfen Messer sie der Länge nach halbieren und das
	Fruchtfleisch mit einem Löffel ausnehmen. Schalen wergwerfen.
	Fruchtfleisch beiseite stellen. \\
	Öl in einen großen Topf erhitzen, Zwiebelringe und Kreuzkümmel unter
	Rühren 1--2~Minuten anbraten. Chilipulver, Salz, Joghurt und Minzsoße
	in den Topf geben und gut verrühren. Auberginen zur
	Zwiebel-Joghurt-Mischung geben, 5--7~Minuten unter Rühren anbraten, bis
	die Flüssigkeit aufgesogen und die Mischung eingedickt ist. In eine
	Schüssel geben und mit den Minzblättern garnieren. \\
      \end{zubereitung}

    \mynewsection{Knoblauchpaste}\label{knoblauchpaste}

      \begin{zutaten}
	120 g & \myindex{Knoblauchzehe}n, halbiert \\
	125 ml & \myindex{Wasser} \\
      \end{zutaten}

      \begin{zubereitung}
        Knoblauchzehen und Wasser in einem Mixer/Pürierstab zu einer Paste
	verarbeiten. In ein Schraubglas füllen. Hält sich im Kühlschrank bis zu
	1~Monat. \\
      \end{zubereitung}

    \mynewsection{Ingwerpaste}\label{ingwerpaste} 

      \begin{zutaten}
	120 g & \myindex{Ingwer}wurzel, grob gehackt \\
	125 ml & \myindex{Wasser} \\
      \end{zutaten}

      \begin{zubereitung}
        Zutaten zu einer Paste verarbeiten im Mixer/Pürierstab. In ein
	Schraubglas gefüllt hält sie sich bis zu 1~Monat im Kühlschrank. \\
      \end{zubereitung}

    \mynewsection{Auberginen-Curry}

      \begin{zutaten}
	40 g & getrocknete \myindex{Tamarinde}, grob gehackt \\
	125 ml & kochendes \myindex{Wasser} \\
	2 große & \myindex{Aubergine}n, in Scheiben geschnitten \\
	& \myindex{Salz} \\
	2 Eßlöffel & \myindex{Ghee} oder Öl\index{Oel=Öl} \\
	3 & \myindex{Zwiebel}n, in Ringe geschnitten \\
	1 Teelöffel & \myindex{Knoblauchpaste} (siehe Seite
	              \pageref{knoblauchpaste}) \\
	1 Teelöffel & \myindex{Ingwerpaste} (siehe Seite
	              \pageref{ingwerpaste}) \\
	4 & \myindex{Curry}blätter \\
	1 & frische \myindex{grüne Chili}\index{Chili>grün}, entkernt und fein
	    gehackt\\
	1 & frische \myindex{rote Chili}\index{Chili>rot}, entkernt und fein
	    gehackt\\
	1 Eßlöffel & gemahlener \myindex{Koriander} \\
	2 Teelöffel & \myindex{Kreuzkümmelsamen} \\
	2 Teelöffel & gelbe \myindex{Senfkörner} \\
	2 Eßlöffel & \myindex{Tomatenmark} \\
	500 ml & \myindex{Kokosmilch} \\
	3 Eßlöffel & frisch gehackter \myindex{Koriander} + etwas zum Garnieren
	             \\
      \end{zutaten}

      \personen{4}

      \begin{zubereitung}
        Tamarinde in eine Schüssel geben, mit kochendem Wasser übergießen und
	30~Minuten einweichen. Auberginenscheiben mit Salz bestreuen und in ein
	Sieb schichten, 30~Minuten ziehen lassen. \\
	Tamarinde in ein feines Sieb geben und die Einweichflüssigkeit in einer
	Schüssel auffangen, dabei das Fruchtfleisch mit einem Löffel ausdrücken.
	Fruchtfleisch wegwerfen. Auberginen kalt abspülen und abtrocknen. \\
	Das Ghee/Öl in einem großen Topf erhitzen und die Zwiebeln bei geringer
	Hitze unter gelegentlichem Rühren 10~Minuten andünsten, bis sie
	goldgelb sind. Knoblauchpaste und Ingwerpaste zugeben und unter
	Rühren 2~Minuten dünsten. Curryblätter, Chilis, Koriander,
	Kreuzkümmel, Senfkörner und Tomatenmark zugeben und unter Rühren
	2~Minuten dünsten, bis die Gewürze zu duften beginnen. \\
	Mit Tamarindenflüssigkeit und Kokosmilch ablöschen und zum Kochen
	bringen. Die Auberginen zugedeckt und abgedeckt 12--15~Minuten köcheln,
	bis die Auberginen gar sind. Deckel abnehmen und weitere 5~Minuten
	köcheln, bis die Soße eingedickt ist. \\
	Abschmecken, Koriander einrühren. Garnieren und Servieren. \\
      \end{zubereitung}

    \mynewsection{Fladenbrot (Pide ekmek)}\label{fladenbrot}

      % Das türkische Kochbuch

      \begin{zutaten}
	500 g & \myindex{Mehl} \\
	40 g & frische \myindex{Hefe} \\
	1 Prise & \myindex{Zucker} \\
	200 ml & lauwarmes \myindex{Wasser} \\
	5 Eßlöffel & \myindex{Olivenöl}\index{Öl>Oliven-} \\
	1 Teelöffel & \myindex{Salz} \\
	1 Teelöffel & \myindex{Schwarzkümmel} \\
      \end{zutaten}

      \personen{4}

      \begin{zubereitung}
        Das Mehl in eine Schüssel geben, eine Mulde hineindrücken, die Hefe
	hineinbröckeln, den Zucker zugeben und mit 150~ml Wasser verrühren.
	Den Vorteig mit etwas Mehl bestäuben, mit einem Tuch abdecken und an
	einem warmen Ort 30~Minuten ruhen lassen. \\
	Danach das restliche Wasser, die Hälfte des Olivenöls sowie das Salz
	dazugeben und das Ganze zu einem geschmeidigen Teig verkneten. Mit Mehl
	bestäuben und nochmals 30~Minuten ruhen lassen. \\
	Den Backofen auf \grad{220} vorheizen. Den Teig nochmals gut verkneten,
	in 2~gleiche Portionen teilen und jeweils einen länglichen Fladen
	formen. Mit den Fingern in die Teigoberfläche längs und quer Rillen
	eindrücken, dabei die Teigränder etwas hochziehen. Den Fladen mit dem
	restlichen Olivenöl bepinseln, mit Schwarzkümmel bestreuen und
	15~Minuten ruhen lassen. Auf ein mit Backpapier belegtes Backblech
	setzen und im Ofen etwa 20~Minuten backen lassen. \\
      \end{zubereitung}

    \mynewsection{Minzsoße}\label{minzsosse}

      \begin{zutaten}
	1 großer Bund & \myindex{Pfefferminze} \\
	1 Teelöffel & \myindex{Zucker} \\
	1 Teelöffel & \myindex{Salz} \\
	2 Eßlöffel & \myindex{Rotweinessig}\index{Essig>Rotwein-} \\
	1 Teelöffel & \myindex{Gemüsebrühe} \\
      \end{zutaten}

      \begin{zubereitung}
        Blätter abzupfen, rollen und schneiden, in den Mörser geben, mit Salz
	und Zucker zerstoßen. Rotweinessig dazugeben, Gemüsebrühe dazu und
	abschmecken. \\
      \end{zubereitung}

    % letztes ##+##

    % \mynewsection{}

      % \begin{einleitung}
      % \end{einleitung}

      % \begin{zutaten}
	% & \myindex{} \\
      % \end{zutaten}

      % \personen{4--6}

      % \begin{zubereitung}
      % \end{zubereitung}

