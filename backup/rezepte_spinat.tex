
% created Montag, 10. Dezember 2012 16:23 (C) 2012 by Leander Jedamus
% modifiziert Dienstag, 17. März 2015 17:20 von Leander Jedamus
% modified Donnerstag, 27. Dezember 2012 14:59 by Leander Jedamus
% modified Montag, 10. Dezember 2012 16:30 by Leander Jedamus

  \mynewchapter{Spinat}

    \mynewsection{Spinat mit Knoblauch}

      \begin{zutaten}
        1--1,5 kg & frischer \myindex{Spinat} \\
        2 & \myindex{Knoblauchzehe}n \\
        2--3 Eßlöffel & \myindex{Olivenöl}\index{Oel=Öl>Oliven-} \\
        & \myindex{Salz}, \myindex{Pfeffer}, \myindex{Muskatnuß} \\
      \end{zutaten}

      \garzeit{10}

      \begin{zubereitung}
        Nur \textbf{frischen} Spinat nehmen, nicht verwertbare Teile entfernen.
	Eventuell auch bei großen Blättern die Rippen entfernen, evtl. auch
	größere Teile der Stengel. In einen großen Topf die tropfnassen Blätter
	geben und zugedeckt kochen lassen, bis der Spinat zusammengefallen und
	weich ist. \\
        Spinat in Durchschlag im Spülbecken abtropfen und etwas abkühlen
	lassen, dann auf großem Brett etwas durchhacken. In einem anderen
	kleineren Topf das Öl erhitzen (nicht zu heiß werden lassen, Olivenöl
	verbrennt sehr schnell und schmeckt dann nicht mehr !!!). Geschälte
	und in Scheiben geschnittene Knoblauchzehen im Öl hellgelb rösten,
	gehackten Spinat dazugeben und etwa 5~Minuten dünsten. Abschmecken mit
	Salz, Pfeffer und wenig geriebener Muskatnuß. \\
        Varianten: man kann statt Spinat auch Mangold verwenden. \\
      \end{zubereitung}

    \mynewsection{Tiefkühlspinat}\glossary{Spinat>Tiefkühl-}

      \begin{zutaten}
        1 Packung & \myindex{Tiefkühlspinat} (450 g) \\
        ca. \breh{} Teelöffel & \myindex{Salz} \\
        1 & kleine \myindex{Zwiebel} \\
        etwas & geriebene \myindex{Muskatnuß} \\
        \brea{} l & \myindex{süße Sahne}\index{Sahne>süß} (etwa) \\
        40 g & \myindex{Butter}/\myindex{Margarine} \\
      \end{zutaten}

      \garzeit{10--15}

      \begin{zubereitung}
        Topf heiß werden lassen, Fett zerlassen. Gewürfelte Zwiebel glasig
	dünsten, Tiefkühlspinat dazu. Spinat ständig drehen, wenden und
	versuchen zu zerteilen, bis Spinat nicht mehr gefrostet ist. Dann Salz
	dazugeben und gardünsten (10~Minuten ?). Mit geriebener Muskatnuß und
	süßer Sahne oder Milch abschmecken. \\
        Dazu am besten Kartoffelbrei und Spiegeleier. \\
      \end{zubereitung}

    \mynewsection{Spinatlasagne mit Lachs}%
      \glossary{Lasagne>Spinat- mit Lachs}
      \glossary{Lachslasagne mit Spinat}

      \begin{zutaten}
        500 g & \myindex{Lachs}\index{Fisch>Lachs}filet (tiefgefroren) \\
        500 g & grüne \myindex{Lasagne}blätter \\
        1 Eßlöffel & \myindex{Olivenöl}\index{Oel=Öl>Oliven-} \\
        & \myindex{Tomatenmark} \\
        & \myindex{Salz} \\
        & \myindex{Pfeffer} \\
        250 g & \myindex{\cremefraiche{}} \\
        200 g & geriebener \myindex{Käse} \\
      \end{zutaten}

      \personen{6}

      \begin{zubereitung}
        Das aufgetaute Lachsfilet in Olivenöl anbraten und aus der Pfanne
	herausnehmen. Den Fond mit Wasser oder Brühe ablöschen, mit
	Tomatenmark andicken und nach Belieben mit Salz und Pfeffer
	abschmecken. Den gebratenen Lachs in kleine Stücke zerteilen und
	hinzugeben. Kurz aufkochen und etwa 10~Minuten köcheln lassen. Die
	Soße sollte nicht zu dick sein, da sonst die Nudelplatten nicht genug
	Flüssigkeit aufnehmen und daher nicht gar werden. Lachssoße,
	Lasagneplatten, \cremefraiche{} und Käse abwechselnd
	übereinanderschichten, bis die Auflaufform gefüllt ist. Die letzte
	Nudelschicht gut mit Soße und Käse bedecken, bei \grad{180}
	40--45~Minuten backen. \\
      \end{zubereitung}

      \bemerkung{Alternativ:}

      \begin{zutaten}
        6--7 & grüne \myindex{Lasagne}blätter \\
        250 g & \myindex{Lachs}\index{Fisch>Lachs}filet \\
        500 g & TK \myindex{Blattspinat} \\
        3 & \myindex{Knoblauchzehe}n \\
        1 Eßlöffel & \myindex{Olivenöl}\index{Oel=Öl>Oliven-} \\
        & \myindex{Tomatenmark} \\
        & \myindex{Salz} \\
        & \myindex{Pfeffer} \\
        1 & \myindex{\cremefraiche{}} \\
        50 g & \myindex{Parmesan}\index{Käse>Parmesan} gerieben \\
      \end{zutaten}

      \personen{2--3}

      \begin{zubereitung}
        wie vor, jedoch Spinat in Olivenöl mit Knoblauch, Salz und Pfeffer kurz
	garen. Dann wechselweise schichten. \\
      \end{zubereitung}

    \mynewsection{Spinat-Panna-Cotta}

      \begin{zutaten}
        1 & \myindex{Zwiebel} \\
        2--3 & \myindex{Knoblauchzehe}n \\
        2 Eßlöffel & \myindex{Olivenöl}\index{Oel=Öl>Oliven-} \\
        300 g & \myindex{Spinat} (Tiefkühlpackung) \\
        & \myindex{Salz} \\
        & \myindex{Pfeffer} \\
        & \myindex{Muskatnuß} \\
        \breh{} Teelöffel & grüne \myindex{Thai-Currypaste} \\
        4--5 Blatt & \myindex{Gelatine} \\
        200 g & \myindex{saure Sahne}\index{Sahne>sauer} \\
        150 g & \myindex{süße Sahne}\index{Sahne>süß} \\
      \end{zutaten}
      \begin{zutat}{Frisches Tomatenragout}
        3--4 & reife, feste \myindex{Fleischtomate}n\index{Tomate>Fleisch-}
	       (oder doppelt so viele
	       \myindex{Cocktailtomate}n\index{Tomate>Cocktail-},
               insgesamt 500 g) \\
        2--3 Eßlöffel & \myindex{Olivenöl}\index{Oel=Öl>Oliven-} \\
        1 & \myindex{Schalotte} oder
	    \myindex{Frühlingszwiebel}\index{Zwiebel>Frühlings-} \\
        2 & junge \myindex{Knoblauchzehe}n \\
        einige & \myindex{Basilikum}blätter \\
        etwas & gutes Salz (Fleur de sel oder Maldonsalt, ein Meersalz aus
	        England) \\
        & \myindex{Pfeffer} \\
        einige Tropfen & \myindex{Balsamico-Essig}\index{Essig>Balsamico-} \\
        ca. 1 Eßlöffel & guter \myindex{Weinessig}\index{Essig>Wein-} oder
	                 \myindex{Apfelessig}\index{Essig>Apfel-} \\
      \end{zutat}

      \personen{4--6}

      \begin{zubereitung}
        Zwiebel und Knoblauch schälen und fein würfeln. Im heißen Öl andünsten.
	Den Tiefkühlspinat hinzufügen und mitdünsten. Mit Salz, Pfeffer, Muskat
	und Currypaste würzen. Den heißen Pfannenspinat im Mixer glatt
	pürieren, dabei die eingeweichte Gelatine mitmixen --- sie löst sich im
	heißen Spinat sofort auf. Die saure und süße Sahne mitmixen. \\
        Nochmals gut, das heißt sehr kräftig, abschmecken. In Becherförmchen
	oder Gläser verteilen und mit Folie abgedeckt kalt stellen. \\
        Zum Servieren die Tomaten für das Ragout mit kochendem Wasser
	überbrühen, dann eiskalt abkühlen, schließlich häuten. Halbieren, die
	Kerne herausstreifen, das Tomatenfleisch würfeln und die gehackte
	Zwiebel sowie gehackten Knoblauch dazugeben. Mit Olivenöl, Salz,
	Pfeffer und den beiden Essigsorten anmachen. Bis zum Servieren in einem
	Sieb abtropfen lassen, damit das Tomatenragout nicht zu viel Saft
	zieht. \\
        Vor dem Anrichten in Streifen geschnittenes Basilikum unterrühren und
	noch einmal mit Balsamico sowie einem Schuß Olivenöl nachwürzen.
	Ringförmig auf Vorspeisentellern verteilen. Die Spinat-PannaCotta
	stürzen und jeweils in die Mitte davon setzen. Mit einem Kringel von
	erstklassigem Olivenöl dekorieren. \\
        Tip: Stürzen gelingt ohne Probleme, wenn man das Kuchenmesser in ein
	Schälchen mit heißem Wasser taucht und die Panna Cotta am Rand vom Glas
	löst und das Glas dann mit dem Boden kurz eintaucht und so den Boden
	löst. \\
        Beilage: Dazu genügt es, frisches, knusprig aufgebackenes Baguette zu
	reichen. \\
        Getränk: Wir finden, dazu paßt besonders gut ein Sauvignon Blanc ---
	ganz nach Geschmack aus der ,,neuen Welt`` (zum Beispiel ein Flagstone
	aus Südafrika) oder aus dem alten Europa (zum Beispiel ein
	,,Grasnitzberg`` aus der Südsteiermark). \\
      \end{zubereitung}

    \mynewsection{Kleine Pasteten mit Spinatfüllung}

      \begin{zutaten}
        1 Paket & \myindex{Tiefkühlblattspinat}\index{Spinat>Blatt->Tiefkühl-}
                  (300 g) \\
	500 g & \myindex{Mehl} \\
	1 Teelöffel & \myindex{Salz} \\
	2 & \myindex{Ei}er \\
	200 ml & \myindex{Olivenöl}\index{Oel=Öl>Oliven-} \\
	100 g & \myindex{Bratenaufschnitt} \\
	100 g & \myindex{Schinkenspeck} \\
	80 g & \myindex{Parmesan}\index{Käse>Parmesan} oder
	       \myindex{Emmentaler}\index{Käse>Emmentaler} oder
	       \myindex{Greyerzer}\index{Käse>Greyerzer} \\
        2 & \myindex{Knoblauchzehe}n \\
	4 Eßlöffel & \myindex{\cremefraiche{}} \\
	5 Teelöffel & \myindex{Brühe} (Instant) \\
	& frisch gemahlener \myindex{Pfeffer} \\
	3 Eßlöffel & \myindex{Weinbrand} (eventuell) \\
	& \myindex{Mehl} zum Ausrollen \\
	1 & \myindex{Eigelb} \\
      \end{zutaten}

      \personen{20}
      \kalorien{250}

      \begin{zubereitung}
        Spinat auf einem Sieb auftauen lassen. Mehl, Salz, Eier, Olivenöl und
	\brea{}~l warmes Wasser mit den Händen nur kurz verkneten.
	Zugedeckt 20~Minuten stehenlassen. Spinat in einem Küchentuch
	ausdrücken und grob hacken. Aufschnitt und Schinkenspeck fein würfeln
	und mit Spinat, geriebenem Käse, zerdrücktem Knoblauch und
	\cremefraiche{} mischen. Mit Instant-Brühe, Pfeffer und Weinbrand
	kräftig abschmecken. Den Teig noch einmal kurz verkneten und auf wenig
	Mehl etwa 3~Millimeter dick ausrollen. 20~Kreise (12~cm~\durchmesser{})
	ausstechen. Jeweils etwa 1~Eßlöffel Füllung in die Mitte setzen und
	die Teigränder mit verquirltem Eigelb bestreichen. Teigkreise
	zusammenklappen und die Ränder fest zusammendrücken. Aufrecht so auf
	2~mit Backtrennpapier ausgelegte Bleche setzen, daß die Teignaht
	oben ist. Mit restlichem Eigelb bestreichen und im Backofen bei
	\grad{200} etwa 40--45~Minuten backen, eventuell mit Backtrennpapier
	bedecken. Das zweite Blech braucht nur 30--35~Minuten. Warm oder kalt
	servieren. \\
      \end{zubereitung}

    \mynewsection{Spinatlasagne}%
              \glossary{Lasagne>Spinat-}

      \begin{einleitung}
        Die leichtere Variante, die man auch ganz und gar vegetarisch halten
        kann, wenn man den gekochten Schinken, den wir hier dazwischen packen,
        einfach wegläßt.
      \end{einleitung}

      \begin{zutaten}
        1 Portion & geriebener \myindex{Käse} \\
	1 Portion & \myindex{Butter}flöckchen \\
      \end{zutaten}
      \begin{zutat}{Nudelteig}
        400 g & \myindex{Mehl} \\
	\breh{} Teelöffel & \myindex{Salz} \\
	4 & \myindex{Ei}er \\
	1 Eßlöffel & \myindex{Olivenöl} \\
	1 Schuß & warmes \myindex{Wasser} (eventuell) \\
      \end{zutat}
      \begin{zutat}{Bechamelsoße}
        2 Eßlöffel & \myindex{Butter} \\
	1 Stück & \myindex{Speck} \\
	1 kleine & \myindex{Zwiebel} \\
	1 Eßlöffel & \myindex{Mehl} \\
	\brdv{} l & \myindex{Milch} \\
	& \myindex{Salz} \\
	& \myindex{Pfeffer} \\
	& \myindex{Petersilie}nstängel \\
	& \myindex{Macis} (Muskatblüte, ersatzweise \myindex{Muskatnuß}) \\
	2 & \myindex{Lorbeer}blätter \\
	& \myindex{Zitrone}nschale \\
      \end{zutat}
      \begin{zutat}{Spinat-Ricotta-Füllung}
        1 kg & \myindex{Spinat} \\
	500 g & \myindex{Ricotta}\index{Käse>Ricotta} oder
	        \myindex{Magerquark} \\
        & \myindex{Salz} \\
	& \myindex{Pfeffer} \\
	& \myindex{Muskatnuß} \\
	& \myindex{Cayennepfeffer}\index{Pfeffer>Cayenne-} \\
	250 g & \myindex{gekochter Schinken}\index{Schinken>gekocht} \\
      \end{zutat}
      
      \personen{6--8}

      \begin{zubereitung}
        Als Erstes den Nudelteig herstellen. Er muß nämlich \breh{}~Stunde
	ruhen, bevor man ihn weiterverarbeitet. Dafür das Mehl auf die
	Arbeitsfläche häufen, mit der Faust eine Vertiefung in die Mitte
	drücken, dorthin Salz und Eier geben. Mit einer Gabel zunächst die Eier
	verquirlen und immer mehr vom Mehlrand einarbeiten. Schließlich den
	Teig von Hand durchkneten und dabei das Olivenöl hinzugeben. Falls er
	nicht ausreichend feucht ist, einen Schuß lauwarmes Wasser einarbeiten.
	\\
	Den Teig tüchtig durchwalken. Schließlich in einen Plastikbeutel
	gehüllt \breh{}~Stunde bei Zimmertemperatur ruhen lassen. Jetzt
	wird der sogenannte Kleber im Mehl zusammen mit der Feuchtigkeit aktiv
	und sorgt dafür, daß der Teig elastisch wird. \\
	Für die Bechamelsoße die Butter schmelzen, die Speckschwarte oder das
	Schinkenstück einlegen, die fein gewürfelte Zwiebel mitdünsten. Erst
	wenn sie richtig blond aussieht, das Mehl hineinrühren und gründlich
	durchschwitzen. Mit Milch ablöschen, Petersilienstängel,
	Lorbeerblätter, Macis (Muskat) und das Stück Zitronenschale dazugeben
	--- die Soße soll nun leise köcheln. Am Ende, nach 20~Minuten, werden
	diese Gewürze wieder herausgefischt, oder es wird die Soße kurzerhand
	durch ein Sieb passiert. Schließlich mit Salz und Pfeffer abschmecken.
	\\
	Für die Spinatfüllung den Spinat verlesen, mehrmals gründlich in immer
	wieder frischem Wasser waschen. Dann in einem großen Topf in Salzwasser
	einmal aufwallen lassen, abgießen und in kaltem Wasser abkühlen. Dieses
	Blanchieren stabilisiert die Vitamine, tötet Keime und hält die schöne
	grüne Farbe frisch. \\
	Den Spinat ausdrücken, grob hacken, mit Ricotta mischen, dabei mit
	Salz, Pfeffer, reichlich Muskat und einer kräftigen Prise
	Cayennepfeffer mutig abschmecken --- Ricotta (oder Magerquark)
	schlucken viel Gewürz! \\
	Der Teig wird mit Hilfe der glatten Walzen der Nudelmaschine zu
	hauchdünnen Bändern ausgerollt. Diese werden auf die Länge und Breite
	der Form zugeschnitten. Man braucht sie nicht vorzukochen, wenn man sie
	tatsächlich durchscheinend dünn auswalzt, denn das Vorkochen ist ein
	Geduldsspiel: Man muß die Teogbänder nacheinander in kochendes Wasser
	tauchen, dann auf feuchten Tüchern nebeneinander abtropfen lassen,
	damit sie nicht zusammenkleben. Einfacher ist es allemal, sie ungekocht
	zu verwenden. \\
	Die Lasagneform mit etwas Bechamel ausstreichen, darüber Nudelblätter
	nicht überlappend legen, diese mit Füllung bedecken. Darauf gekochten
	Schinken in Streifen oder Flecken verteilen. \\
	Wieder Bechamel, Nudeln, Füllung, Schinkenflecken etc. --- bis alles
	aufgebraucht ist. Die oberste Schicht ist Bechamel, die erst nach gut
	5--10~Minuten im Rohr mit geriebenem Käse sowie mit Butterflöckchen
	bedeckt wird. Im Ofen bei \grad{200} 15--20~Minuten backen. \\
	Getränk: Zur leichteren Spinatlasagne mit der säuerlichen Ricotta
	trinkt man lieber einen zarteren Wein, etwa einen Valpolicella oder
	einen Weißwein, und zwar einen Vernacchia di San Gimignano oder einen
	Verdicchio dei Castelli di Jesi. Man könnte aber auch durchaus einen
	herzhaften Silvaner aus Deutschland dazu trinken. \\
      \end{zubereitung}

    \mynewsection{Hühnerfilet auf Spinat mit Joghurtsoße}

      \begin{zutaten}
        1 kg & \myindex{Spinat} \\
	2 Becher & \myindex{Joghurt} (3,5 \%) \\
	6 & \myindex{Knoblauchzehe}n \\
	& \myindex{Salz} \\
	2 Eßlöffel & \myindex{Olivenöl}\index{Oel=Öl>Oliven-} \\
	3 & \myindex{Hühnerfilets} (\'a ca. 300 g) \\
	& frisch gemahlener \myindex{Pfeffer} \\
	3 Eßlöffel & Öl\index{Oel=Öl} \\
	2 Eßlöffel & gehackte \myindex{Mandel}n \\
      \end{zutaten}

      \personen{6}
      \kalorien{300}

      \begin{zubereitung}
        Spinat waschen und die dicken Stiele entfernen. Mit kochendem Wasser
	überbrühen. Joghurt mit Salz und 3~zerdrückten Knoblauchzehen
	verrühren. Olivenöl unterrühren. Hühnerfilets leicht salzen und
	pfeffern und in heißem Öl bei großer Hitze von jeder Seite 8~Minuten
	braten, dabei ab und zu wenden. Restlichen Knoblauch in dünnen
	Scheiben zufügen. Fleisch herausnehmen und warm stellen. Spinat im
	Bratfett andünsten. Mit Salz und Pfeffer abschmecken. Mandeln ohne
	Fett in einer Pfanne hellbraun rösten. Fleisch auf dem Spinat
	anrichten. Mit Mandeln bestreuen. Knoblauchsoße dazu servieren. \\
	Dazu: Reis. \\
      \end{zubereitung}

    \mynewsection{Geflügelleber mit Spinat}

      \begin{zutaten}
        120 g & \myindex{Geflügelleber}\index{Leber>Geflügel-} \\
        & \myindex{Salz} \\
        & \myindex{Pfeffer} \\
        1 Scheibe & \myindex{Parmaschinken} (20 g) \\
        etwas & \myindex{Lauchgrün} \\
	1 & \myindex{Knoblauchzehe} \\
        & \myindex{Salbeiblätter} \\
        1 Teelöffel & Öl\index{Oel=Öl} \\
        3 Eßlöffel & \myindex{Brühe} (Instant) \\
        1 Eßlöffel & \myindex{Zitrone}nsaft \\
        200 g & frischer \myindex{Spinat} \\
        1 Teelöffel & \myindex{Parmesan}\index{Käse>Parmesan}käse \\
      \end{zutaten}

      \kalorien{400}

      \begin{zubereitung}
        Hühner- oder Putenleber putzen und in dünne Scheiben schneiden, mit
	Pfeffer bestreuen. Parmaschinken in Streifen schneiden und vom Fett
	befreien. \\
	Den gewaschenen Spinat tropfnaß zusammen mit dem gehackten Lauchgrün
	in einen Topf geben und bei geschlossenem Deckel bei starker Hitze
	zusammenfallen lassen. Abtröpfeln, dann eine Knoblauchzehe zerdrücken
	und unter den Spinat rühren, leicht salzen. \\
	Den Spinat nicht zu lange kochen lassen, warm stellen. Gleichzeitig
	die Leberscheiben in einem Teelöffel Öl in einer beschichteten Pfanne
	kurz braten, etwa 1--2~Minuten pro Seite. Dann herausnehmen und die
	Schinkenstreifen und ein paar Salbeiblätter in der Pfanne schwenken.
	Alles zusammen mit Spinat und Leber auf einem Teller anrichten. 
	Den Bratensatz mit Brühe und Zitronensaft aus der Pfanne lösen und über
	das Gericht verteilen. Alles mit frischgemahlenem Pfeffer und Parmesan
	überstreuen. \\
      \end{zubereitung}

    \mynewsection{Spinattaschen}

      \begin{einleitung}       
        Dafür nimmt man den dünnen Frühlingsrollenteig, den man im asiatischen
        Lebensmittelgeschäft bekommt, oder Börekteig vom türkischen
        Lebensmittelhändler. Für die Füllung braucht man tiefgekühlten Spinat,
        der natürlich zunächst auftauen muß --- entweder nimmt man die Packung
        rechtzeitig aus dem Eis (wer ohnehin vom Einkauf kommt, läßt ihn
	einfach unverpackt, sodaß der Spinat bereits nahezu aufgetaut ist, bis
	man zu Hause eintrifft). Oder man bemüht die Mikrowelle. \\
      \end{einleitung}       

      \begin{zutaten}
	ca. 400 g & \myindex{Spinat} (tiefgekühlt) \\
        150 g & \myindex{Magerquark}\index{Quark>Mager-} \\
        100 g & \myindex{gekochter Schinken}\index{Schinken>gekocht} \\
	& \myindex{Salz} \\
	& \myindex{Pfeffer} \\
	ca. 10--12 & \myindex{Frühlingsrollenhüllen} oder \\
	2--3 & \myindex{Börekblätter} \\
	\brea{} l & \myindex{Milch} \\
	1 Teelöffel & \myindex{Curry}pulver \\
      \end{zutaten}

      \personen{2--3}

      \begin{zubereitung}
        Den Spinat hacken, in einer Schüssel mit dem Quark verrühren.
	Gewürfelten Schinken untermischen. Die Masse mit Salz und Pfeffer
	würzen. \\
	Jeweils 1--2 gehäufte Eßlöffel auf ein Eck der Teighülle setzen und
	flach streichen, rechts und links die Ecken darüberschlagen, die
	Füllung nun einwickeln --- dabei nicht zu fest rollen, weil sich die
	Füllung beim Backen ausdehnt und die Hülle zerreißen kann. Mit der Naht
	nach unten auf ein mit Backpapier belegtes Blech setzen. Mit Milch
	einpinseln, die mit Currypulver verquirlt wurde. \\
	Bei \grad{200} etwa 10~Minuten backen, bis die Taschen schön gebräunt
	sind. \\
	Beilage: ein grüner oder auch ein Tomatensalat. \\
      \end{zubereitung}

    \mynewsection{Spinatauflauf}

      \begin{zutaten}
	ca. 400 g & \myindex{Spinat} (tiefgekühlt) \\
        150 g & \myindex{Magerquark}\index{Quark>Mager-} \\
        100 g & \myindex{gekochter Schinken}\index{Schinken>gekocht} \\
	& \myindex{Salz} \\
	& \myindex{Pfeffer} \\
	ca. 10--12 & \myindex{Frühlingsrollenhüllen} oder \\
	2--3 & \myindex{Börekblätter} \\
	\brea{} l & \myindex{Milch} \\
	1 Teelöffel & \myindex{Curry}pulver \\
      \end{zutaten}

      \begin{zubereitung}
        Wem die Wickelei der Spinattaschen zu mühsam ist, der schichtet die
	Zutaten einfach in eine feuerfeste Form --- immer wieder ein Teigblatt
	zwischen die Schichten betten (oder nur als oberste Schicht) und dick
	mit Currymilch einpinseln. Oberste Schicht ist ein Teigblatt. \\
	Ebenfalls bei \grad{200} backen, insgesamt 10--15~Minuten, bis die
	Oberfläche gebräunt ist und es in der Form brodelt. \\
	Getränk: Bier, ein würziger Weißwein (zum Beispiel wie zu vielen
	Wokgerichten ein Gelber Muskateller aus der Steiermark oder eine
	trockene Scheurebe aus der Pfalz) oder ein junger, ebenfalls aus dem
	Kühler oder dem Eisschrank servierter Rotwein (zum Beispiel ein
	einfacher Spätburgunder oder ein fruchtiger Cuv\'ee vom Kaiserstuhl,
	ein Beaujolais, Valpolicella oder Bardolino). \\
      \end{zubereitung}

    \mynewsection{Spinatpastete}

      \begin{zutaten}
      \end{zutaten}
      \begin{zutat}{Teig}
        & \myindex{Mehl} \\
	& \myindex{Wasser} \\
	& \myindex{Salz} \\
	& \myindex{Olivenöl}\index{Oel=Öl>Oliven-} \\
      \end{zutat}
      \begin{zutat}{Füllung}
        & \myindex{Spinat} \\
	1 Handvoll & \myindex{Brennnessel} \\
	1 Handvoll & \myindex{Minze} \\
	1 Handvoll & \myindex{Sauerampfer} \\
	1 Bund & \myindex{Frühlingszwiebel}\index{Zwiebel>Frühlings-}n
	         zerteilt \\
        200 g & \myindex{Schafkäse}\index{Käse>Schaf-} \\
	2 & \myindex{Ei}er \\
	& \myindex{Olivenöl}\index{Oel=Öl>Oliven-} \\
	& \myindex{Salz} \\
	& \myindex{Pfeffer} \\
	& \myindex{Muskatnuß} \\
      \end{zutat}

      \begin{zubereitung}
        Teig: Papierdünne Teige (2~Stück) aus Mehl, Wasser, Salz machen und
	Olivenöl auf die erste Teigplatte. \\
	Füllung: Spinat blanchieren, dann in Streifen schneiden. Brennnessel,
	Minze, Sauerampfer und Frühlingszwiebeln in Olivenöl andünsten, Spinat
	dazugeben. \\
	Schafkäse würfeln und mit 2~Eiern an Gemüsemischung geben, würzen und
	vermischen. \\
	Füllung auf die erste Teigplatte geben, dann die zweite Schicht oben
	drauf. Bei \grad{200} 45~Minuten im Backofen. \\
      \end{zubereitung}
