
% created Montag, 10. Dezember 2012 16:22 (C) 2012 by Leander Jedamus
% modifiziert Mittwoch, 11. März 2015 17:15 von Leander Jedamus
% modifiziert Montag, 09. März 2015 14:24 von Leander Jedamus
% modified Donnerstag, 27. Dezember 2012 08:40 by Leander Jedamus
% modified Montag, 10. Dezember 2012 16:30 by Leander Jedamus

  \mynewchapter{Neues}

    \mynewsection{grüne Paprika mit Fetafüllung}\glossary{Paprika>grün}

      \begin{zutaten}
        2 größere & \myindex{grüne Paprika}\index{Paprika>grün}schoten \\
      \end{zutaten}
      \begin{zutat}{Füllung}
        100 g & gewürfelter \myindex{Schafkäse}\index{Käse>Schaf-} /
	                    \myindex{Ziegenkäse}\index{Käse>Ziegen-} /
			    \myindex{Kuhkäse}\index{Käse>Kuh-} \\
        ca. 60 g & gewürfelte \myindex{Brotrinde} (1 große Scheibe) \\
        \breh{} & klein geschnittene \myindex{Paprika}schote \\
        1--2 & klein geschnittene \myindex{Schalotte}n \\
        1--2 & klein geschnittene \myindex{Frühlingszwiebel}n \\
      \end{zutat}
      \begin{zutat}{Würze}
        & \myindex{Cayennepfeffer}\index{Pfeffer>Cayenne-} \\
        & \myindex{edelsüßer Paprika}\index{Paprika>edelsüß} \\
        & \myindex{Salz} \\
        & \myindex{Pfeffer} \\
        & frische \myindex{Oregano}blätter \\
      \end{zutat}

      \garzeit{20}

      \begin{zubereitung}
        Schote halbieren (Deckel drauflassen, auch den Stielansatz), weißes und
	Kerngehäuse entfernen. Paprika leicht vorgaren. Füllung vermengen und
	locker in die Schoten geben. \\
        Auf ein Backblech setzen und 20~Minuten bei \grad{200} garen. \\
      \end{zubereitung}

    \mynewsection{Gekochter Blumenkohl mit würziger Marinade}%
              \glossary{Blumenkohl>gekocht}\glossary{Kohl>Blumen-}

      \begin{zutaten}
        1 & \myindex{Blumenkohl}\index{Kohl>Blumen-}kopf und/oder \\
        1 & \myindex{Romanesco}kopf \\
        & \myindex{Salz} \\
        2--3 Eßlöffel & \myindex{Zitrone}nsaft \\
      \end{zutaten}
      \begin{zutat}{Marinade}
        4 & \myindex{getrocknete Tomate}n\index{Tomate>getrocknet} \\
        50 g & \myindex{Pinienkerne} \\
        1 & \myindex{Schalotte} \\
        2 & \myindex{Knoblauchzehe}n \\
        3--4 Eßlöffel & \myindex{Olivenöl}\index{Oel=Öl>Oliven-} \\
        4--6 & \myindex{Anchovis}\index{Fisch>Anchovis}filets \\
        1 Tasse & \myindex{schwarze Oliven}\index{Oliven>schwarz} \\
        & Saft von 1 \myindex{Zitrone} \\
        & Schale von \breh{} \myindex{Zitrone} \\
        & \myindex{Salz} \\
        & \myindex{Pfeffer} \\
        & \myindex{glatte Petersilie}\index{Petersilie>glatt} oder
	  \myindex{Basilikum} \\
      \end{zutat}

      \personen{4}

      \begin{zubereitung}
        Blumenkohl und Romanesco putzen: Die Blätter entfernen und den Strunk
	am unteren Ende kappen. Den Blumenkohl in eine Schüssel mit gut
	gesalzenem Wasser legen, das mit Zitronensaft gesäuert ist. Achtung:
	Den Romanesco dagegen auf keinen Fall mit Zitrone in Verbindung
	bringen, er färbt sich sonst beim Kochen grau! Die Rosen in wenig
	Salzwasser oder gleich über Dampf insgesamt ca. 10--15~Minuten garens
	(je nach Größe --- der Strunk sollte sich mit einem spitzen Messer
	leicht durchstechen lassen). \\
        Für die \emph{Marinade} die getrockneten Tomaten mit kochendem Wasser
	bedecken und einweichen. Pinienkerne in der trockenen Pfanne anrösten,
	die sehr fein gewürfelte Schalotte sowie den gehackten oder
	durchgepreßten Knoblauch hinzufügen. Mit einem Löffel Olivenöl
	benetzen. Die Anchovis darin zerdrücken, die grob zerkleinerten Oliven
	hinzufügen. Mit Zitronensaft ablöschen, Zitronenschale sowie die
	gewürfelten Tomaten unterrühren und einen guten Schuß Einweichwasser
	angießen. Die Marinade wenige Minuten leise schmurgeln, bis sich alles
	gut verbindet. Mit Salz und Pfeffer abschmecken. \\
        Am Ende die fein gehackte Petersilie (oder Basilikum) und das restliche
	Olivenöl unterrühren. \\
        Blumenkohl mit einer Schaumkelle aus dem Topf in eine Schüssel heben.
	Den Romanesco immer kurz in Eiswasser abschrecken, damit die grüne
	Farbe erhalten bleibt. Mit der Marinade übergießen und durchziehen
	lassen. Dabei immer wieder erneut mit der Marinade beschöpfen. Den
	Blumenkohl lauwarm oder abgekühlt servieren und zum Schluß mit einem
	dünnen Faden Olivenöl überziehen. \\
        \emph{Beilage}: Als Vorspeise servieren, dann genügt frisches Weißbrot,
	mit dem man die Marinade aufwischen kann. Oder als Beilage zu
	gegrilltem Fleisch und Ofenkartoffeln. \\
        \emph{Getränk}: ein herzhafter Weißwein, zum Beispiel aus Sardinien,
	Sizilien oder Apulien. Oder, falls der Blumenkohl Beilage zum Fleisch
	ist, ein Rotwein, der das Fleisch begleitet. \\
      \end{zubereitung}

    \mynewsection{Blumenkohlcremesüppchen}%
              \glossary{Kohl>Blumen-}\glossary{Suppe>Blumenkohl-}

      \begin{zutaten}
        1 & \myindex{Zwiebel} \\
        2 Eßlöffel & \myindex{Butter} \\
        1--2 & \myindex{Knoblauchzehe}n \\
        ca. 300--400 g & \myindex{Blumenkohl}\index{Kohl>Blumen-}
                         (Stiele und Röschen) \\
        & \myindex{Zitrone}nsaft \\
        & \myindex{Salz} \\
        ca. \breh{} l & \myindex{Hühnerbrühe} oder \myindex{Fleischbrühe} \\
        & \myindex{Zitrone}nschale \\
        200 g & \myindex{Sahne} \\
        & \myindex{Muskatnuß} \\
        & \myindex{Cayennepfeffer}\index{Pfeffer>Cayenne-} \\
        1 Spritzer & \myindex{Worcestershiresoße} \\
        2--3 große & \myindex{Blumenkohl}\index{Kohl>Blumen-}röschen \\
        & \myindex{Olivenöl}\index{Oel=Öl>Oliven-} zum Braten \\
      \end{zutaten}
      \begin{zutat}{Petersilienöl}
        1 Händchen voll &
	  \myindex{glatte Petersilie}\index{Petersilie>glatt}nblätter \\
        & \myindex{Salz} \\
        ca. 100 ml & \myindex{Olivenöl}\index{Oel=Öl>Oliven-} \\
      \end{zutat}

      \personen{4--6}

      \garzeit{30}

      \begin{zubereitung}
        Die Zwiebel fein würfeln und in der Butter andünsten, den Knoblauch
	zerquetschen und mitdünsten, schließlich die zerkleinerten
	Blumenkohlteile hinzufügen, mit Brühe auffüllen, salzen, Zitronensaft
	und ein Stück Zitronenschale hinzufügen. Zugedeckt leise etwa
	30~Minuten köcheln, bis der Blumenkohl absolut weich ist. Mit Sahne
	auffüllen und mit dem Pürierstab glatt mixen --- Zitronenschale vorher
	herausfischen. Anschließend noch einmal kurz aufkochen. \\
        Die Suppe mit Muskat, Worcestershiresoße, Cayennepfeffer und
	Zitronensaft frisch und herzhaft abschmecken. Die Blumenkohlröschen
	senkrecht in dünne Scheiben schneiden und in etwas Olivenöl auf beiden
	Seiten anbraten, dabei salzen und pfeffern, mit einem Hauch Muskat
	würzen. \\
        Die Suppe vor dem Servieren nochmals schaumig aufschlagen. In tiefen
	Tellern oder in Suppentassen anrichten. In die Mitte ein, zwei
	Blumenkohlscheiben betten. Mit Klecksen von Petersilienöl garnieren:
	Dafür die Petersilienblätter mit einigen Krümeln Salz und Olivenöl im
	Mixbecher glatt mixen. \\
        Tip: Das Petersilienöl bleibt einige Tage in einem Schraubglas im
	Kühlschrank frisch, wenn man die Blätter vor dem Mixen blanchiert. \\
        Beilage: Hauchdünne Weißbrotscheiben (auf der Aufschnittmaschine dünn
	aufschneiden) auf einem Backblech in \grad{180} heißen Backofen golden
	rösten. \\
        Getränk: Eigentlich trinkt man zur Suppe nichts --- sie ist Flüssigkeit
	genug. Es paßt jedoch sehr gut ein Sherry. Das kann entweder ein
	staubtrockener Fino sein oder sogar ein eher milder gereifter
	Amontillado. Es paßt auch ein (trockener) Marsala Secco! \\
      \end{zubereitung}

    \mynewsection{Zimt-Pfannkuchen mit Blaubeer-Ragout}%
              \glossary{Kuchen>Pfann-}\glossary{Pfannkuchen>Zimt-}

      \begin{zutaten}
      \end{zutaten}
      \begin{zutat}{Pfannkuchen}
        2 große Eßlöffel & \myindex{Weizenmehl} gehäuft \\
        2 & \myindex{Ei}er \\
        \breh{} Eßlöffel & \myindex{Zucker} \\
        \breh{} Eßlöffel & flüssige \myindex{Butter} \\
        180 ml & \myindex{Milch} \\
        etwas & \myindex{Zimt} \\
        kleiner Schuß & \myindex{Rum} \\
        \breh{} Teelöffel & \myindex{Salz} \\
        2 & \myindex{Vanilleeis}kugeln \\
        & \myindex{Mandelblättchen} \\
        & \myindex{Pistazien} \\
        & \myindex{Pfefferminze} \\
        & \myindex{Puderzucker} \\
      \end{zutat}
      \begin{zutat}{Blaubeer-Ragout}
        \brev{} l & \myindex{Rotwein}\index{Wein>rot} \\
        200 g & \myindex{Blaubeeren} \\
        1--2 Eßlöffel & \myindex{Zucker} \\
      \end{zutat}

      \personen{2}

      \begin{zubereitung}
        Rotwein erhitzen, leicht kochen und Zucker einrühren. Gewaschene
	Blaubeeren\footnote{Vor dem Kauf probieren, ob lecker!} darin garziehen
	und abkühlen lassen. \\
        Zuerst das Weizenmehl mit dem Ei klumpenfrei verrühren, dann restliche
	Zutaten dazu. Teig sollte ziemlich flüssig sein! \brdv{}~Schöpfkelle =
	1~Pfannkuchen. Pfannkuchen in Butter bei mäßiger Hitze auf Stufe~1
	ausbacken, falten (\brev{}). 1~Eiskugel Vanilleeis in die Öffnung
	geben. Dazu seitlich das Blaubeer-Ragout legen. Als Garnitur geröstete
	Mandelblättchen und gehackte Pistazien. Obenauf ein Blatt Pfefferminze
	und Puderzucker drüberstreuen. \\
        Schmeckt am besten frisch, kalt nicht mehr so gut. Am besten eine
	beschichtete Pfanne nehmen. \\
      \end{zubereitung}

    \mynewsection{Blumenkohlauflauf}\glossary{Kohl>Blumen-}

      \begin{zutaten}
        1 & \myindex{Blumenkohl}\index{Kohl>Blumen-}kopf \\
        & \myindex{Zitrone}nsaft \\
        & \myindex{glatte Petersilie}\index{Petersilie>glatt} \\
        150--200 g & \myindex{gekochter Schinken}\index{Schinken>gekocht}
	             (in dicken Scheiben) oder
		     \myindex{roher Schinken}\index{Schinken>roh}, zerteilt \\
        150 g & frisch geriebener Käse (nach Belieben:
	        \myindex{Parmesan}\index{Käse>Parmesan} oder
		\myindex{Peccorino}\index{Käse>Peccorino}; aber auch ein
        \myindex{Edamer}\index{Käse>Edamer},
	  \myindex{Gouda}\index{Käse>Gouda} oder
	  \myindex{Bergkäse}\index{Käse>Berg-}) \\
        2--3 & \myindex{Ei}er \\
        \brev{} l & \myindex{Milch} \\
        & \myindex{Salz} \\
        & \myindex{Pfeffer} \\
        & \myindex{Muskatnuß} \\
        1 Spritzer & \myindex{Worcestershiresoße} \\
        1 Prise & \myindex{Cayennepfeffer}\index{Pfeffer>Cayenne-} \\
      \end{zutaten}

      \personen{4}

      \garzeit{30}

      \begin{zubereitung}
        Den Blumenkohl im Ganzen in Salzwasser kochen, allerdings nur ca.
	4--5~Minuten garen. Herausheben, senkrecht in gut fingerdicke
	Scheiben schneiden. Dachziegelartig in eine flache Gratinform
	schichten. Petersilie fein hacken, Schinken klein würfeln und beides
	mit dem Käse mischen. Über den Blumenkohl verteilen. \\
        Eier und Milch sowie einen guten Schuß Blumenkohlwasser verquirlen, mit
	Salz, Pfeffer, Muskat, Cayennepfeffer und Worcestershiresoße würzen und
	über den Blumenkohl gießen. Im auf \grad{200} vorgeheizten Backofen
	eine knappe \breh{}~Stunde backen, bis die Oberfläche goldbraun ist. \\
        Beilage: Es genügt eigentlich Brot, ein herzhaftes Bauernbrot. Man kann
	aber auch kross gebratene Würste dazu reichen oder ein Kotelett. \\
        Tip: Um ihn gehaltvoller zu machen, kann man gewürfelte, gekochte
	Kartoffeln unter den Auflauf mischen oder auch gekochte Pasta (zum
	Beispiel Penne oder in Stücke gebrochene Maccaroni). \\
        Getränk: ein frischer, junger Rotwein, ein Spätburgunder vom
	Kaiserstuhl, ein Chianti oder auch ein Beaujolais. \\
      \end{zubereitung}

    \mynewsection{Lauchgratin}\glossary{Gratin>Lauch-}

      \begin{zutaten}
        250 g & gekochte \myindex{Nudeln} (ob Band-, Röhren- oder
	        Örchennudeln --- nahezu jede Sorte ist geeignet) \\
        2--3 & \myindex{Lauch}stangen \\
        2 Eßlöffel & \myindex{Butter} \\
        & \myindex{Salz} \\
        & \myindex{Pfeffer} \\
        & \myindex{Muskatnuß} \\
        100 g & geriebener \myindex{Bergkäse}\index{Käse>Berg-} \\
        1 Teelöffel & \myindex{rosa Pfeffer}\index{Pfeffer>rosa}beeren \\
        2 & \myindex{Ei}er \\
        200 g & \myindex{Sahne} \\
        & \myindex{Butter}flöckchen \\
      \end{zutaten}

      \personen{2}

      \garzeit{20}

      \begin{zubereitung}
        Die Nudeln sollten nicht zusammenkleben (damit sie das nicht tun, noch
	vor dem Erkalten mit etwas Olivenöl durchmischen). \\
        Den Lauch putzen, schräg in feine Ringe schneiden (je nachdem, wie dick
	die Nudeln sind, entsprechend die Stärke anpassen). In einer großen
	Pfanne in der heißen Butter andünsten, bis sie glänzen und eine
	leuchtende Farbe angenommen haben. Dabei mit Salz, Pfeffer und Muskat
	würzen. Mit den Nudeln mischen, dabei auch den Bergkäse und die rosa
	Pfefferbeeren hinzufügen. In eine feuerfeste flache Auflaufform
	verteilen. \\
        Eier und Sahne verquirlen und über das Gemisch verteilen, einige
	Butterflöckchen aufsetzen und im \grad{200} heißen Backofen (Umluft)
	etwa 15--20~Minuten (je nach Größe und Tiefe der Form) backen, bis
	alles brodelt und die Oberfläche zart gebräunt ist. \\
        Tip: Hübsch ist es, für jeden eine eigene Portion in kleinen Förmchen
	zu backen. \\
        Beilage: Natürlich gehört ein Salat dazu, möglichst aus verschiedenen
	Blättern, zum Beispiel Endivie, Radicchio, Romana und \chicoree{}. Man
	kann auch noch ganz klein geschnittenen, rohen Lauch untermischen. \\
        Getränk: ein herzhafter, frischer Rotwein, etwa ein Gamay aus dem
	Waadtland oder ein Dolcetto aus Piemont. Wir haben dazu eine Beaujolais
	--- der ja aus der Gamay-Rebe gekeltert wird --- getrunken, der mit
	seiner Frucht bestens dazu paßte. \\
      \end{zubereitung}

    \mynewsection{Gemüseauflauf mit Möhren, Kartoffeln und Blumenkohl}%
              \glossary{Auflauf>Gemüse-}

      \begin{zutaten}
        500--600 g & \myindex{Möhre}n \\
        4 große & \myindex{Kartoffel}n \\
        3 & \myindex{Zwiebel}n \\
        1 & \myindex{Blumenkohl}\index{Kohl>Blumen-} \\
        5 Scheiben & \myindex{roher Schinken}\index{Schinken>roh} \\
        2 Becher & \myindex{Sahne} \'a 200 g \\
        3 & \myindex{Ei}er \\
        & \myindex{Salz} \\
        & \myindex{Pfeffer} \\
        & \myindex{Muskatnuß} \\
        & \myindex{Chiliflocken} \\
        60 g & geriebener \myindex{Parmesan}\index{Käse>Parmesan} \\
        & \myindex{Olivenöl}\index{Oel=Öl>Oliven-} \\
      \end{zutaten}

      \garzeit{40}

      \begin{zubereitung}
        Gemüse putzen, Möhren dünn in Scheiben schneiden. Kartoffeln in dickere
	Scheiben, Blumenkohl in ca. 2~cm dicke Scheiben. Möhren in Pfanne mit
	Olivenöl dünsten, Zwiebeln in Scheiben dazu, salzen, pfeffern und
	Chiliflocken drüber geben mit einer Prise Zucker. Kartoffeln in
	Salzwasser angaren, Blumenkohl ebenso. Auflaufform einölen. Kartoffeln
	salzen, pfeffern und Muskatnuß drübergeben. Schinken in Streifen
	schneiden und mit etwas geriebenem Käse über die Kartoffeln geben.
	Darüber Blumenkohl legen. Sahne, Eier salzen, pfeffern, Chiliflocken
	und Muskatnuß dran sowie den Rest Parmesan und verquirlen. Alles über
	das Gemüse geben. Im Backofen bei \grad{200} 40~Minuten garen. \\
      \end{zubereitung}

    \mynewsection{Hirsefrikadellen}%
              \glossary{Frikadellen>Hirse-}\label{hirsefrikadellen}

      \begin{zutaten}
        500 g & \myindex{Hirse} \\
        1 & dicke Scheibe \myindex{Sellerie} \\
        1 große & \myindex{Möhre} (oder 2 kleine) \\
        & \myindex{Lauch} (oder eine große \myindex{Zwiebel}) \\
        2 & \myindex{Ei}er \\
        1 & \myindex{Brötchen} \\
        2 & \myindex{Chilischote}n \\
        & \myindex{Salz} \\
        & \myindex{Pfeffer} \\
        & \myindex{Cayennepfeffer}\index{Pfeffer>Cayenne-} \\
        & \myindex{Chili} getrocknet aus der Mühle \\
        1 kleines Stück & \myindex{Ingwer} \\
      \end{zutaten}

      \personen{4--6} (12--13 Stücke) \\

      \begin{zubereitung}
        Vorab die Hirse mit Salzwasser 15--20~Minuten kochen, abkühlen lassen.
	Brötchen einweichen, dann ausdrücken. Sellerie, Möhren, Lauch und
	Chilischoten putzen, dann sehr klein schneiden. Alles vermischen,
	2~Eier und Gewürze dazugeben. Würzig abschmecken. \\
        Öl in die Pfanne geben und ca. handtellergroße Frikadellen goldbraun
	braten. \\
        Dazu gebratene Zucchinischeiben längs geschnitten in Olivenöl hellgelb
	gebraten, mit Salz, schwarzer Pfeffer, Oregano, Knoblauch gewürzt und
	mit Tomatenscheiben geschichtet. \\
      \end{zubereitung}

    \mynewsection{Zanderfilets mit Reisplätzchen}

      \begin{zutaten}
        4 & \myindex{Zander}\index{Fisch>Zander}filets (teuer: 35 Euro/kg ?) \\
        3 & unbehandelte \myindex{Orange}n \\
        1 sehr dicker & \myindex{Thymian}bund (1 Topf ca.) \\
        & \myindex{Salz} \\
        & \myindex{Pfeffer} \\
        ca. \brev{} l & \myindex{Sahne} \\
        \brev{} l & \myindex{Weißwein}\index{Wein>weiß} \\
        & \myindex{Butter} \\
        2 Tassen & kalter gekochter \myindex{Reis} \\
        3--4 & \myindex{Chilischote}n \\
        1 & \myindex{Eigelb} \\
      \end{zutaten}

      \personen{4}

      \garzeit{35}

      \begin{zubereitung}
        Chilischoten in Scheiben schneiden, den Reis und Eigelb vermischen.
	Stehen lassen und salzen. Zander mit grobem Salz und Pfeffer aus der
	Mühle würzen. Auflaufform ausbuttern. Orangen mit Schale in ca. 1~cm
	dicke Scheiben schneiden und in die Form legen, darauf ein Bett aus
	Thymian. Zanderfilets obenauf legen. Sahne und Wein angießen. Alles in
	den Backofen 30--35~Minuten bei \grad{200}. \\
        Reisplätzchen formen und in der Pfanne braten. \\
      \end{zubereitung}

    \mynewsection{Tortiglioni alla Napoletana}

      \begin{zutaten}
        350 g & \myindex{Tortiglioni} \\
        400 g & \myindex{Tomate}n \\
        80 g & geriebener \myindex{Parmigiano-Reggiano-Käse}
	                  \index{Käse>Parmigiano-Reggiano-} \\
        1 & \myindex{Pfefferschote} (Chili) \\
        1 & \myindex{Knoblauchzehe} \\
        \breh{} & \myindex{Zwiebel} \\
        8 & frische \myindex{Thymian}blätter \\
        4 Eßlöffel & \myindex{Olivenöl} (extra vergine)
	             (2 für Soße, 2 für Thymianöl) \\
        & \myindex{Salz} \\
        & \myindex{Pfeffer} \\
      \end{zutaten}

      \personen{4}

      \garzeit{25}

      \begin{zubereitung}
        Die Zwiebel hacken und 2~Minuten mit 2~Eßlöffel Olivenöl leicht
	andünsten. Den Knoblauch und die Pfefferschote fein hacken, dazugeben
	und 2~Minuten dünsten. Dann die geschälten und gewürfelten Tomaten
	dazufügen, salzen und pfeffern und 7~Minuten weiterkochen. Die
	Thymianblätter zerhacken und mit dem restlichen Olivenöl verrühren. Die
	Tortiglioni in reichlich Salzwasser kochen, abgießen und mit der Soße
	anrichten. Mit dem Thymianöl und dem geriebenen
	Parmigiano-Reggiano-Käse garnieren. \\
      \end{zubereitung}

    \mynewsection{Lauch-Carbonara}

      \begin{zutaten}
        100 g & \myindex{Frühstücksspeck} \\
        1 Stange & \myindex{Lauch} \\
        2 & \myindex{Knoblauchzehe}n \\
        1 Eßlöffel & \myindex{Butter} \\
        2 & \myindex{Eigelb}e \\
        100 g & \myindex{Sahne} \\
        50 g & \myindex{Parmesan}\index{Käse>Parmesan} gerieben \\
        400 g & \myindex{Spaghetti} \\
      \end{zutaten}

      \begin{zubereitung}
        Nudeln nach Anweisung kochen. Speck fein würfeln. Lauch in feine Ringe
	schneiden. Knoblauch fein hacken. Butter in einem Topf erhitzen, Speck
	darin auslassen. Lauch und Knoblauch zufügen und 2~Minuten mitdünsten.
	1~Eigelb und Sahne verrühren. Heiße Nudeln in der Speckmischung wenden
	und die Eiersahne unterheben. Topf sofort vom Herd ziehen, Pasta mit
	Pfeffer würzen und mit Parmesan bestreut servieren. \\
      \end{zubereitung}

    \mynewsection{Tomatensoße}\label{tomatensosse}

      \begin{zutaten}
        1--2 & \myindex{Möhre}n \\
        & \myindex{Olivenöl}\index{Oel=Öl>Oliven-} \\
        1--2 & \myindex{Knoblauchzehe}n \\
        1 & \myindex{Zwiebel} \\
        1 Dose & \myindex{geschälte Tomate}n\index{Tomate>geschält}
	         (kleine Dose) \\
        1 Dose & \myindex{passierte Tomate}n\index{Tomate>passiert}
	         (Pizzatomaten) \\
        & \myindex{Salz} \\
        & \myindex{Pfeffer} \\
        & \myindex{Oregano} \\
        1 kleines & \myindex{Lorbeer}blatt \\
      \end{zutaten}

      \garzeit{20--30}

      \begin{zubereitung}
        Möhren schälen und in kleine Würfel oder dünne Scheiben schneiden.
	Zwiebel in Würfel schneiden. Olivenöl in den Topf geben und fast heiß
	werden lassen, dann Möhren anrösten, dann die Zwiebel glasig dünsten.
	Die Dosen Tomaten dazu, alle Gewürze reingeben, umrühren und mindestens
	20~Minuten köcheln lassen. Danach Lorbeerblatt entfernen und  mit einem
	Pürierstab die Soße glätten. \\
	Für den großen Topf Tomatensoße als Vorrat: \\
      \end{zubereitung}

      \begin{zutaten}
        5 Dosen & \myindex{geschälte Tomate}n\index{Tomate>geschält} \\
        5 Dosen & \myindex{passierte Tomate}n\index{Tomate>passiert} \\
	200 g & \myindex{Gemüse} (Lauch, Sellerie, Möhren) \\
	3 große & \myindex{Zwiebel}n \\
	6--8 & \myindex{Knoblauchzehe}n \\
        3 kleine & \myindex{Lorbeer}blätter \\
	2 Teelöffel & \myindex{Salz} \\
	& \myindex{schwarzer Pfeffer}\index{Pfeffer>schwarz} \\
	3 gehäufte Teelöffel & \myindex{Gemüsebrühe} \\
	2 gehäufte Teelöffel & \myindex{Oregano} \\
      \end{zutaten}

    \mynewsection{Lachs in Lauchsoße mit grünen Nudeln}

      \begin{zutaten}
        1 Stange & \myindex{Lauch} \\
	1 & \myindex{Schalotte}\index{Zwiebel>Schalotte} \\
	100 g & grüne \myindex{Bandnudeln}\index{Nudeln>Band-} \\
	einige & Stiele \myindex{Estragon} \\
	einige & Stiele \myindex{glatte Petersilie}\index{Petersilie>glatt} \\
	1 Eßlöffel & \myindex{Rapsöl}\index{Oel=Öl>Raps-} \\
	1 Teelöffel & \myindex{Mehl} \\
	100 ml & \myindex{Gemüsebrühe} \\
	350 g & \myindex{Lachs}\index{Fisch>Lachs}filet ohne Haut \\
	& \myindex{Salz} \\
	& \myindex{Pfeffer} \\
	25 ml & \myindex{Sahne} \\
      \end{zutaten}

      \personen{2}

      \begin{zubereitung}
        Lauch waschen und in feine Ringe schneiden, Schalotte fein würfeln. \\
	Schalotte in Pfanne mit 1~Eßlöffel Rapsöl anschwitzen, Lauch zugeben
	und kurz mitdünsten. Mit Mehl bestäuben, dann mit Gemüsebrühe
	ablöschen, einige Minuten köcheln lassen. \\
	Nudeln bißfest garen, Kräuter hacken. \\
	Lachs in 3~cm große Stücke schneiden, würzen mit Salz und Pfeffer und
	in der Pfanne kurz anbraten. \\
	Lauchsoße mit Sahne verfeinern, abschmecken. Lachswürfel zugeben und
	auf den Punkt garen. \\
	Nudeln abschütten und mit den Kräutern mischen. \\
	Anrichten und servieren. \\
      \end{zubereitung}

    \mynewsection{Bandnudeln mit Gorgonzolasahne}

      \begin{zutaten}
        & \myindex{Bandnudeln}\index{Nudeln>Band-} \\
	400 ml & \myindex{Sahne} \\
	200 g & \myindex{Gorgonzola}\index{Käse>Gorgonzola} \\
	& \myindex{Salz} \\
	& \myindex{Pfeffer} \\
	& \myindex{Muskatnuß} \\
	3--4 Eßlöffel & geriebener \myindex{Parmesan}\index{Käse>Parmesan}käse
	 \\
      \end{zutaten}

      \personen{4}

      \begin{zubereitung}
        Die Sahne in der Pfanne aufkochen, dann die Hitze reduzieren. \\
	Gorgonzola in Stücke schneiden und in die Sahne geben. Umrühren und das
	Ganze ca. 5~Minuten köcheln lassen, damit die Soße eindickt. Immer
	wieder gut umrühren. Nach Geschmack mit Salz, Pfeffer und Muskat
	würzen. Die Soße über die Bandnudeln geben und etwas geriebenen
	Parmesan darüberstreuen. \\
      \end{zubereitung}

    \mynewsection{Lauch-Schnecken}

      \begin{zutaten}
      \end{zutaten}
      \begin{zutat}{Teig}
        300 g & \myindex{Mehl} \\
	\breh{} Päckchen & \myindex{Trockenhefe}\index{Hefe>Trocken-} \\
	1 Teelöffel & \myindex{Salz} \\
	1 Prise & \myindex{Zucker} \\
	\brea{} l & \myindex{Milch} \\
	30 g & \myindex{Butter} \\
      \end{zutat}
      \begin{zutat}{Rest}
        2 Stangen & \myindex{Lauch} \\
	60 g & \myindex{durchwachsener Speck}\index{Speck>durchwachsen} \\
      \end{zutat}

      \personen{4}

      \begin{zubereitung}
        Aus den oben genannten Zutaten einen Hefeteig schlagen. 20~Minuten
	gehen lassen. \\
	Lauch in feinen Stücken in gewürfeltem Speck goldbraun braten. \\
	Teig zu 2~Rechtecken ausrollen. Lauchmasse darauf verteilen.
	Aufrollen. In 3~cm dicke Scheiben geschnitten in der Springform von
	32~cm~\durchmesser{} 10~Minuten bei \grad{75} vorbacken, in 30~Minuten
	bei \grad{220} backen. Warm mit Käsesoße essen. \\
      \end{zubereitung}

    \mynewsection{Fisch-Lauch-Gratin}\glossary{Gratin>Fisch-Lauch-}

      \begin{zutaten}
        400 g & \myindex{Lauch} \\
	1 & \myindex{Zwiebel} \\
	2 & \myindex{Knoblauchzehe}n \\
	4 & \myindex{Tomate}n \\
	800 g & \myindex{Zander}\index{Fisch>Zander}filet \\
	110 g & \myindex{Butter} \\
	\brez{} l & \myindex{Weißwein}\index{Wein>weiß} (trocken) \\
	\brev{} l & \myindex{Sahne} \\
	\brev{} l & \myindex{Milch} \\
	200 g & \myindex{Weichkäse}\index{Käse>Weich-} mit Blauschimmel \\
	& \myindex{Salz} \\
	& \myindex{weißer Pfeffer}\index{Pfeffer>weiß} \\
	& Saft einer \myindex{Zitrone} \\
      \end{zutaten}

      \personen{4}

      \garzeit{45}

      \begin{zubereitung}
        Den Lauch putzen, der Länge nach halbieren und in etwa 2~cm breite
	Stücke schneiden. In kochendem Salzwasser ca. 30~Sekunden blanchieren,
	abschrecken und abtropfen lassen. \\
	Das Zanderfilet kurz waschen, trocken tupfen und in ca. 5~cm große
	Würfel schneiden. \\
	Zwiebel und Knoblauch schälen und fein würfeln. Tomaten waschen, den
	Stielansatz entfernen und in Scheiben schneiden. \\
	Die Zwiebel- und Knoblauchwürfel mit 80~g Butter in einer tiefen Pfanne
	glasig anschwitzen. Mit dem Weißwein ablöschen und einmal kurz
	aufkochen lassen. Sahne und Milch dazugeben, unter Rühren ebenfalls
	aufkochen und mit Salz, Pfeffer und Zitronensaft abschmecken. Dann den
	gewürfelten Käse hinzufügen. \\
	Eine mit der restlichen Butter gefettete Auflaufform abwechselnd mit
	Lauch, Fisch und Tomatenscheiben auslegen. Darüber zum Schluß
	gleichmäßig die Käsesahnesoße gießen und im vorgeheizten Ofen bei
	\grad{180} 35~Minuten hellbraun backen. \\
      \end{zubereitung}

    \mynewsection{Gefüllte Champignons}

      \begin{zutaten}
        10 große & \myindex{Champignon}\index{Pilze>Champignon}s (etwa 500 g) \\
	2--3 & \myindex{Knoblauchzehe}n \\
	60 g & gehackte \myindex{Mandel}n \\
	1 Bund & \myindex{Petersilie} \\
	1 Glas & \myindex{Sardellen}\index{Fisch>Sardellen}filets
	         (Einwaage 25 g) \\
	2 Eßlöffel & \myindex{Weinbrand} oder
	             verdünnter \myindex{Zitrone}nsaft \\
        \breh{} & \myindex{Zitrone} \\
	1 Eßlöffel & Öl\index{Oel=Öl} \\
	5 Scheiben & \myindex{Frühstücksspeck}\index{Speck>Frühstücks-}
	             (Bacon) \\
      \end{zutaten}

      \personen{5}
      \kalorien{230}

      \begin{zubereitung}
        Champignonstiele herausdrehen und hacken. Mit zerdrücktem Knoblauch,
	Mandeln, gehackter Petersilie, abgespülten, kleingeschnittenen
	Sardellen und Weinbrand mischen. Pilze mit Zitronensaft und Öl
	bestreichen und die Füllung in die Öffnung häufen. Champignons auf ein
	Backblech setzen und mit halbierten Speckscheiben belegen. Unter dem
	Grill oder auf \grad{250} vorgeheizten Backofen etwa 5~Minuten
	überbacken. Heiß oder kalt servieren. \\
      \end{zubereitung}

    \mynewsection{Crostini mit Leber- und Kaperncreme}

      \begin{zutaten}
        2 & \myindex{Baguette}s \\
	100 g & \myindex{Schinkenspeck}\index{Speck>Schinken-} \\
	150 g & \myindex{Geflügelleber}\index{Leber>Geflügel-} \\
	3--6 & \myindex{Salbei}blätter \\
	40 g & \myindex{Butter} \\
	2 Eßlöffel & \myindex{Wein} \\
	1 gestrichener Eßlöffel & \myindex{Mehl} \\
	2 Eßlöffel & frisch geriebener \myindex{Parmesan}käse
	             \index{Käse>Parmesan} \\
	& \myindex{Salz} \\
	& frisch gemahlener \myindex{Pfeffer} \\
	2 Gläschen & \myindex{Kapern} (Einwaage 35 g) \\
	50 g & \myindex{Pinienkerne} \\
	2 Stiele & \myindex{Petersilie} \\
	2 Stiele & \myindex{Oregano} \\
	4 Eßlöffel & Öl\index{Oel=Öl} \\
	1 kleiner Zweig & \myindex{Rosmarin} \\
      \end{zutaten}

      \personen{5}
      \kalorien{490}

      \begin{zubereitung}
        Von den Baguettes 20 schräge Scheiben abschneiden. Für die Lebercreme
	40 g Schinkenspeck, Leber und 3~Salbeiblätter im Blitzhacker oder
	mit dem Schneidstab des Handrührers pürieren. Leberpüree in der Hälfte
	der heißen Butter unter Rühren bei kleiner Hitze andünsten. Erst Wein
	zugießen und untermischen, dann Mehl überstäuben und unter Rühren kurz
	weiterdünsten. Von der Kochstelle nehmen und Parmesan untermischen.
	Salzen, pfeffern und warm stellen. \\
	Für die Kaperncreme abgespülte, abgetropfte Kapern, 40~g Pinienkerne,
	restlichen Schinkenspeck, Petersilien- und Oreganoblätter im
	Blitzhacker oder mit dem Schneidstab des Handrührers zerkleinern.
	In restlicher Butter andünsten. Eventuell pfeffern, warm stellen.
	Restliche Pinienkerne in einer Pfanne ohne Fett bei mittlerer Hitze
	hellbraun rösten. Brotscheiben in heißem Öl bei mittlerer Hitze in
	2~Portionen von beiden Seiten hellbraun rösten. \\
	Leber- und Kaperncreme auf jeweils die Hälfte der Brotscheiben
	verteilen. Auf einer Platte anrichten und mit Pinienkernen, Rosmarin
	und Salbeiblättern bestreuen. \\
      \end{zubereitung}

    \mynewsection{Bohnenpüree mit Oliven}

      \begin{zutaten}
        250 g & \myindex{weiße Bohnen}\index{Bohnen>weiß} getrocknet \\
	\brda{} l & \myindex{Brühe} (Instant) \\
	1 Becher & \myindex{Schlagsahne}\index{Sahne>Schlag-} (250 g) \\
	& \myindex{Salz} \\
	& frisch gemahlener \myindex{Pfeffer} \\
	& \myindex{Muskatnuß} \\
	\breh{} & \myindex{Zitrone} \\
	2 Eßlöffel & \myindex{Olivenöl}\index{Oel=Öl>Oliven-} \\
	200 g & \myindex{grüne Oliven}\index{Oliven>grün} und
	        \myindex{schwarze Oliven}\index{Oliven>schwarz} \\
      \end{zutaten}

      \personen{8}
      \kalorien{300}

      \begin{zubereitung}
        Bohnen über Nacht in reichlich kaltem Wasser einweichen. Abgetropfte
	Bohnen in der Brühe auf kleiner Hitze etwa 1~Stunde kochen. Mit der
	Brühe im Mixer oder mit dem Schneidstab des Handrührgeräts pürieren.
	Mit Sahne verrühren und mit Salz, Pfeffer, Muskat und Zitronensaft
	abschmecken. Mindestens 2~Stunden stehenlassen; das Püree quillt
	in dieser Zeit nach. Mit Olivenöl beträufeln und mit Oliven belegt
	servieren. Dazu Bauernbrot. \\
      \end{zubereitung}

    \mynewsection{Gebackene Senfeier in der Form}

      \begin{zutaten}
        20 g & \myindex{Butter} oder \myindex{Margarine} \\
	80 g & \myindex{mittelalter Gouda}\index{Käse>Gouda>mittelalt}käse \\
	1 Becher & \myindex{Schlagsahne}\index{Sahne>Schlag-} (200 g) \\
	2 Eßlöffel & \myindex{Senfpulver} oder \myindex{Senf} \\
	& \myindex{Salz} \\
	6 & \myindex{Ei}er \\
	2 Stiele & \myindex{Majoran} oder \myindex{glatte Petersilie}
	           \index{Petersilie>glatt} \\
      \end{zutaten}

      \personen{6}
      \kalorien{285}

      \begin{zubereitung}
        Eine ofenfeste Form mit Fett austreichen und mit 2~Eßlöffel
	geriebenem Käse austreuen. Sahne mit Senfpulver und Käse
	(2~Eßlöffel zurücklassen) verschlagen. Salzen und in die Form gießen.
	Die Eier aufschlagen und in die Form gleiten lassen. Mit restlichem
	Käse bestreuen, mit Majoran belegen und in den Ofen schieben. Bei
	\grad{180} etwa 30~Minuten goldgelb backen. \\
      \end{zubereitung}

    \mynewsection{Gebratene Champignons (Funghi in Padella)}

      \begin{zutaten}
        250 g & kleine \myindex{Champignon}\index{Pilze>Champignon}s \\
	125 g & \myindex{Zwiebel}n \\
	5 Eßlöffel & Öl\index{Oel=Öl} \\
	& \myindex{Pfeffer} \\
	& \myindex{Salz} \\
	1 kleiner Bund & \myindex{Petersilie} \\
	einige Blättchen & \myindex{Salbei} \\
	4 Eßlöffel & \myindex{Weißweinessig}\index{Essig>Weißwein-} \\
      \end{zutaten}

      \personen{4}
      \kalorien{170}

      \begin{zubereitung}
        Pilze waschen, eventuell putzen. Zwiebeln schälen, in Ringe schneiden.
	Champignons im heißen Öl ringsherum anbraten. Aus der Pfanne nehmen.
	Zwiebelringe anbraten. Mit Champignons mischen, würzen. Kräuter hacken
	und mit Essig dazugeben. \\
      \end{zubereitung}

    \mynewsection{Gefülltes Gemüse (Verdura Ripiena)}

      \begin{zutaten}
        4 & \myindex{Zwiebel}n (\'a ca. 75 g) \\
	& \myindex{Salz} \\
	1 & \myindex{Aubergine} (ca. 200 g) \\
	1 & \myindex{rote Paprika}\index{Paprika>rot} (150 g) \\
      \end{zutaten}
      \begin{zutat}{Füllung}
        1 Päckchen & \myindex{Tiefkühlblattspinat}
	             \index{Spinat>Blatt->Tiefkühl-} (grob gehackt, 450 g) \\
        3 & \myindex{Eigelb} \\
	250 g & feiner \myindex{Frischkäse}\index{Käse>Frisch-} oder
	        \myindex{Ricotta}\index{Käse>Ricotta} \\
        60 g & geriebener \myindex{Parmesan}\index{Käse>Parmesan} (ca. 3
	       Eßlöffel) \\
        & \myindex{Pfeffer} \\
	& \myindex{Salz} \\
	& \myindex{Oregano} \\
      \end{zutat}
      \begin{zutat}{Außerdem}
        40 g & Öl\index{Oel=Öl} \\
	2 Eßlöffel & \myindex{Paniermehl} (ca. 30 g) \\
      \end{zutat}
      
      \personen{4}
      \kalorien{330}

      \begin{zubereitung}
        Zuerst für die Füllung den Spinat auftauen. \\
	Zwiebeln schälen und in kochendem, gesalzenem Wasser ca. 15~Minuten
	kochen. Aubergine waschen, Stielenden abschneiden, in Scheiben
	schneiden. Die letzten 10~Minuten zu den Zwiebeln geben. Paprika putzen,
	waschen, vierteln. \\
	Für die Füllung aufgetauten Spinat ausdrücken und mit den restlichen
	Füllungszutaten verrühren. Pikant abschmecken. Gemüse abtropfen lassen.
	Zwiebeln längs halbieren und bis zur Mitte einschneiden.
	Zwiebelschichten vorsichtig lösen und immer 2~ineinanderlegen.
	Vorbereitetes Gemüse mit der Füllung füllen und auf ein geöltes
	Backblech geben. Mit Paniermehl bestreuen und im Backofen bei \grad{200}
	ca. 30~Minuten backen. \\
      \end{zubereitung}

    \mynewsection{Zitronenhähnchen mit Rosmarinkartoffeln}%
              \glossary{Hähnchen mit Zitrone und Rosmarinkartoffeln}

      \begin{zutaten}
        4 & \myindex{Hähnchenkeulen} (\'a ca. 200 g) \\
	3 & \myindex{Zitrone}n \\
	100 g & \myindex{Butter} \\
	2 & \myindex{Knoblauchzehe}n \\
	& \myindex{Salz} \\
	& \myindex{Pfeffer} \\
	4 Eßlöffel & \myindex{Speiseöl}\index{Oel=Öl>Speise-} \\
	600 g & kleine \myindex{Kartoffel}n \\
	2 Eßlöffel & gehackter \myindex{Rosmarin} \\
      \end{zutaten}

      \personen{4}
      \garzeit{20+40+5}
      \kalorien{775}

      \begin{zubereitung}
        Elektrobackofen auf \grad{200} vorheizen. Hähnchenkeulen waschen,
	trocken tupfen, halbieren und die Haut einschneiden. 1~Zitrone waschen
	und in Scheiben schneiden, die restlichen Zitronen auspressen. Butter
	in einem Topf schmelzen lassen, mit Zitronensaft verrühren, Knoblauch
	abziehen, zerdrücken und dazugeben, mit Salz und Pfeffer würzen. Die
	Keulen mit der Zitronen-Knoblauchbutter einpinseln. Öl im Schmortopf
	erhitzen, Keulen darin anbraten, Zitronenscheiben darüber verteilen, im
	vorgeheizten Backofen bei \grad{200} ca. 40~Minuten backen.
	Zwischendurch mit der Butter weiter einpinseln, Kartoffeln waschen,
	restliche Zitronen-Knoblauchbutter in der Pfanne erhitzen, Kartoffeln
	darin schwenken, mit Rosmarin und Salz würzen. 25~Minuten vor Ende der
	Backzeit Kartoffeln zu den Keulen geben, mitbacken. Mit
	Rosmarinkartoffeln und Rosmarinzweigen garniert servieren. \\
      \end{zubereitung}

    \mynewsection{Apple-Peach}

      \begin{zutaten}
        1 & weißer \myindex{Pfirsich} \\
	& \myindex{Eiswürfel} \\
	& \myindex{Zitrone}nsaft \\
	2--3 & \myindex{Minzeblättchen} \\
	etwas & \myindex{Zucker} \\
	& naturtrüber \myindex{Apfelsaft} \\
      \end{zutaten}

      \personen{1}

      \begin{zubereitung}
        Einen weißen Pfirsich schälen, in Stücke schneiden und mit einigen
	Eiswürfeln und einem Schuß Zitronensaft mixen. Dabei pro Pfirsich
	2--3~Minzenblättchen mitmixen, nur wenn nötig etwas Zucker
	dazu. Mit einem guten (möglichst naturtrüben) Apfelsaft auffüllen. Mit
	Minzezweig dekorieren und in einem hohen Glas servieren. \\
      \end{zubereitung}

    \mynewsection{Gefüllte Pfirsiche in Folie}%
              \glossary{Pfirsiche gefüllt in Folie}

      \begin{einleitung}
        Eines dieser herrlichen Rezepte, mit denen man seine Gäste verblüffen
        kann: sieht einfach umwerfend aus, schmeckt super, läßt sich prima
        vorbereiten und macht wirklich kaum Arbeit! Vor allem feste Pfirsiche
        sind hier bestens geeignet. \\
      \end{einleitung}

      \begin{zutaten}
      \end{zutaten}
      \begin{zutat}{Pro Person}
        1 & schöner \myindex{Pfirsich} \\
	& \myindex{Butter} für die Folie \\
	2--3 & \myindex{Amaretti} oder \myindex{Cantucci}
	       (\myindex{Mandelkekse}) \\
        etwas & \myindex{Himbeerlikör}\index{Likör>Himbeer-} \\
	1--2 Teelöffel & \myindex{Himbeerkonfitüre}\index{Konfitüre>Himbeer-} \\
      \end{zutat}
      \begin{zutat}{Pistaziensoße für vier Personen}
        50 g & ausgelöste \myindex{Pistazien}kerne \\
	1--2 Eßlöffel & \myindex{Zucker} \\
	200 g & \myindex{Sahne} \\
	etwas & \myindex{Zitrone}nschale (eventuell) \\
      \end{zutat}
      
      \begin{zubereitung}
        Die Füllung ist schnell gemacht: Die Amaretti oder Mandelkekse
	zerbröseln, mit einigen Tropfen Himbeerlikör würzen und mit
	Himbeerkonfitüre vermischen. \\
	Ein ausreichend großes Quadrat Alufolie (um den Pfirsich locker zu
	umhüllen) in der Mitte mit Butter bepinseln. Den Pfirsich halbieren,
	den Kern auslösen, stattdessen die Füllung in die Höhlung geben. \\
	Den Pfirsich wieder zusammensetzen, auf die Alufolie setzen. Die Enden
	der Folie fassen und über dem Pfirsich hochschlagen, überall gut
	verschließen, es muß noch ein Luftraum über dem Pfirsich sein, es soll
	aber nichts auslaufen. \\
	Das Paket in den \grad{200} heißen Ofen stellen, den Pfirsich etwa
	15--20~Minuten garen. Wie lange das genau dauert, hängt von der Größe
	der Pfirsiche sowie von ihrem Reifegrad ab: je unreifer und fester,
	desto länger! Um festzustellen, ob der Pfirsich gar ist, einfach mit
	dem Messer einstechen. Wenn er richtig weich ist, ist er fertig. \\
	Zum Servieren die Pfirsiche auspacken und auf Desserttellern anrichten
	--- nach Belieben jetzt die Haut abziehen. \\
	Dazu gibt es eine Pistaziensoße: Dafür die Pistazien mit dem Zucker
	im Zerhacker ganz fein pulverisieren. Mit der Sahne glatt rühren und
	wenn nötig mit Zitronenschale abschmecken. \\
	Getränk: ein Rosenmuskateller aus Südtirol, ein herbsüßer, nach Rosen
	duftender rosafarbener Wein, oder ein milder, konzentrierter Muscat
	de Beaumes de Venise. \\
      \end{zubereitung}

    \mynewsection{Aprikosenknödel}

      \begin{einleitung}
        Klassisch sind Knödel aus Kartoffelteig, aus einem Topfenteig sind sie
        besonders zart. Seltener findet man die Knödel von einem Brandteig
        umhüllt, dabei sind sie besonders köstlich: weil hier die Teigschicht
        ungewöhnlich dünn ist und die Knödel extrem leicht sind. \\
      \end{einleitung}

      \begin{zutaten}
        12--15 & schöne, reife \myindex{Aprikosen} \\
	12--15 & Stücke \myindex{Würfelzucker}\index{Zucker>Würfel-}
	         (braun oder weiß) \\
        etwas & \myindex{Zitronenlikör}\index{Likör>Zitronen-} \\
      \end{zutaten}
      \begin{zutat}{Brandteig}
        knapp 300 ml & \myindex{Milch} \\
	1 Prise & \myindex{Salz} \\
	30 g & \myindex{Butter} \\
	150 g & \myindex{Mehl} \\
	1 & \myindex{Ei} \\
      \end{zutat}
      \begin{zutat}{Außerdem}
        & \myindex{Butterschmalz} zum Frittieren \\
	& \myindex{Aprikosensoße} \\
      \end{zutat}
      \begin{zutat}{Sowie}
        75 g & geriebene \myindex{Haselnüsse} \\
        75 g & \myindex{Semmelbrösel} \\
	50 g & \myindex{Butter} \\
	2 Eßlöffel & \myindex{Zucker} \\
      \end{zutat}
      
      \personen{4}

      \begin{zubereitung}
        Die Milch mit der Salzprise zum Kochen bringen, die Butter hinzufügen
	und, sobald sie geschmolzen ist, das Mehl mit Schwung hineinkippen.
	Sofort kräftig mit dem Kochlöffel rühren, bis sich die Masse wie ein
	Kloß vom Topfboden löst und dort einen dünnen Film hinterläßt. Vom
	Feuer ziehen, auskühlen lassen, bis man den Teig anfassen kann und erst
	dann das Ei einarbeiten. Jetzt muß der Teig weich sein und glänzen.
	Wenige Minuten stehen lassen, dann wird die Masse so steif, daß man sie
	gut formen kann. \\
	Die Zuckerwürfel mit etwas Zitronenlikör tränken, die Aprikosen
	ensteinen und stattdessen ein Zuckerstück hineingeben. Kleine Portionen
	Teig abzupfen, auf dem Handteller flach drücken und die Aprikosen
	darin einwickeln. \\
	Die so gefüllten Früchte gleichmäßig mit dem Teig umhüllen und außen
	schön glatt streichen. Man kann sie in Salzwasser in circa 8~Minuten
	gar sieden lassen, aber auch in Butterschmalz schwimmend frittieren,
	bis sie golden leuchten. Im letzterem Fall serviert man dazu
	,,Marillenröster`` beziehungsweise ,,Aprikosensoße`` (siehe nächsten
	Abschnitt). \\
	Die gekochten Aprikosenknödel bekommen einen duftenden Überzug aus
	gemahlenen Haselnüssen. Dafür die geriebenen Haselnüsse und Brösel in
	einer Pfanne in der Butter rösten, bis sie duften --- dabei den Zucker
	hinzufügen und alles gut mischen. Die Knödel abtropfen und in diesen
	Bröseln wälzen, bis sie rundum davon dünn überzogen sind. \\
	Getränk: ein Sankt Magdalena aus Südtirol paßt gut dazu, vor allem,
	wenn er kühlschrankkalt ist. \\
      \end{zubereitung}

    \mynewsection{P\^eche Melba}

      \begin{einleitung}
        Aus den \underline{eingemachten Pfirsichen} ist im Handumdrehen dieses
        Dessert angerichtet, eines der berühmtesten der Welt. Der große
        französische Koch Auguste Escoffier hat es einst erfunden, als Hommage
        an die australische Sängerin Helen Porter Mitchell, die sich Nellie
	Melba nannte. Ein Klassiker unter den Desserts und in seiner
	Einfachheit immer noch absolut überzeugend. Wobei natürlich die
	Qualität des Pfirsichs ebenso entscheidend ist wie die des
	Himbeerpürees und des Vanilleeises. Und dann ist alles ganz einfach: \\
      \end{einleitung}

      \begin{zutaten}
        1 & eingemachter \myindex{Pfirsich} \\
	1 Kugel & \myindex{Vanilleeis} \\
	& \myindex{Himbeermark} \\
	1 Blättchen & \myindex{Minze} oder \myindex{Melisse} oder
	              \myindex{Basilikum} \\
      \end{zutaten}

      \begin{zubereitung}
        Je einen eingemachten Pfirsich halbieren, die Haut abziehen, den Stein
	entfernen, stattdessen in die Höhlung eine Kugel Vanilleeis hineingeben.
	Am schönsten sieht es aus, wenn man den so gefüllten Pfirsich mit der
	Wölbung nach unten auf Desserttellern oder, besser noch, in
	Dessertschalen anrichtet. Mit Himbeermark überziehen und mit einem
	Blättchen dekorieren: Klassisch ist Minze oder Melisse, originell und
	passend Basilikum. Schmeckt überwältigend gut! \\
	Getränk: entweder eine fruchtige Riesling Spät- oder Auslese. Es kann
	auch ein eleganter Rotwein sein. Oder ein Glas Champagner --- der paßt
	schließlich immer! Natürlich auch genauso gut ein Sekt. \\
      \end{zubereitung}

    \mynewsection{Kullerpfirsich}%
              \glossary{Pfirsich-Kuller}

      \begin{zubereitung}
        Dafür braucht man zunächst einmal große Gläser: Richtige
	Kullerpfirsichgläser sehen aus wie Sektgläser für Riesen. Geräumige
	Rotweingläser eignen sich weniger, weil die Pfirsiche darin leider
	nicht so schön kullern. Der Pfirsich muß Platz haben, und die Gläser
	sollten sich wie Sekttulpen unten verjüngen. \\
	In jedem Fall sollte man möglichst kleine reife Pfirsiche nehmen, die
	aromatischen weißen oder Weinbergpfirsiche (sie färben den Sekt schön
	rötlich) sind perfekt dafür. Sie werden mit einem Tuch abgerieben und
	dann rundherum mit einer Gabel mehrmals eingestochen. \\
	Im Glas mit gut gekühltem Sekt übergießen. Nach wenigen~Minuten
	beginnen die Pfirsiche, sich zu drehen, zuerst langsam, dann immer
	schneller, bis sie regelrecht kullern und ihrem Namen alle Ehre machen.
	\\
	Natürlich darf man am Ende --- das ist sogar erwünscht --- den Pfirsich
	herausfischen und verspeisen. Er ist jetzt vollgesogen mit Sekt und
	macht ein bißchen beschwipst ... \\
      \end{zubereitung}

    \mynewsection{Mangoldgemüse}

      \begin{einleitung}
        Das grüne Blatt und die Stiele werden getrennt voneinander verarbeitet.
        Sie haben unterschiedliche Struktur, Textur und Biß, deshalb zunächst
	das Grundrezept für Mangoldblätter: \\
        Wie vielseitig sich Mangold einsetzen läßt, kann man an diesem
        Grundrezept sehen. Vier Variationen, die jeweils den erdigen
        Grundgeschmack des Mangolds ganz unterschiedlich zur Geltung bringen.
	Und zwar mit den charakteristischen Zutaten verschiedener Gegenden der
	Welt. \\\\
      \end{einleitung}

      \begin{einleitung}Französisch:\end{einleitung} \begin{zutaten}
        & \myindex{Zwiebel}n \\
	& \myindex{Mangold} \\
	1--2 & \myindex{Thymian}zweiglein \\
	& \myindex{Piment} \\
      \end{zutaten}

      \begin{zubereitung}
        Fein gewürfelte Zwiebeln in Butter andünsten, Mangoldblätter
        in Streifen geschnitten dazugeben und zugedeckt gar dünsten. Ein, zwei
        Thymianzweiglein dazu (getrockneten Thymian vorher mit anbraten),
        pfeffern, salzen und mit zerdrücktem Piment (Allgewürz oder französisch
        ,,quatre epices``) würzen. \\\\
      \end{zubereitung}

      \begin{einleitung}Mediteran oder italienisch:\end{einleitung}
      \begin{zutaten}
        & \myindex{Olivenöl}\index{Oel=Öl>Oliven-} \\
	etwas & \myindex{Knoblauch} \\
	& \myindex{Chili} \\
	1--2 & \myindex{Anchovis}\index{Fisch>Anchovis} \\
	& \myindex{Mangold} \\
	& \myindex{Salz} \\
	& \myindex{Pfeffer} \\
	& \myindex{Zitrone}nschale \\
      \end{zutaten}

      \begin{zubereitung}
        Zunächst in Olivenöl etwas Knoblauch andünsten, nach Belieben Chili
	dazu (frisch mit oder ohne Kerne, auch getrockneter Chili paßt, einfach
	zwischen den Fingerspitzen zerkrümeln). Dann ein, zwei Anchovis
	hinzufügen und mit dem Kochlöffel ein wenig zerdrücken. Zu diesem
	sogenannten ,,Soffritto`` die nur grob zerkleinerten Mangoldblätter
	geben. Langsam, im eigenen Saft möglichst, im offenen Topf schmurgeln
	lassen, bis alles zusammengefallen ist. Nur sparsam salzen (wegen der
	Anchovis), dafür gut pfeffern und eine Spur Zitronenschale
	hineinreiben. Vor dem Servieren kommt noch etwas frisches aromatisches
	Olivenöl darüber. \\
      \end{zubereitung}

      \begin{einleitung}Deutsch:\end{einleitung} \begin{zutaten}
        & \myindex{Speck} \\
	& \myindex{Zwiebel}n \\
	& \myindex{Petersilie} \\
	& \myindex{Mangold} \\
	& \myindex{Salz} \\
	& \myindex{Pfeffer} \\
	& \myindex{Muskatnuß} \\
      \end{zutaten}

      \begin{zubereitung}
        Speck sehr fein würfeln und auslassen, darin gehackte Zwiebeln
        und Petersilie andünsten, Mangoldblätter dazu, etwas salzen und
	pfeffern und mit Muskat abschmecken. \\
      \end{zubereitung}

      \begin{einleitung}Chinesisch:\end{einleitung} \begin{zutaten}
        & \myindex{Knoblauch} \\
	& \myindex{Ingwer} \\
	& \myindex{Chili} \\
	& \myindex{Frühlingszwiebel}\index{Zwiebel>Frühlings-} \\
	& \myindex{Erdnußöl}\index{Oel=Öl>Erdnuß-} \\
	& \myindex{Sesamöl} \\
	& \myindex{Mangold} \\
	& \myindex{Salz} \\
	& \myindex{Zucker} \\
	& \myindex{Sojasoße} \\
	& \myindex{Reiswein} oder
	  \myindex{weißer Burgunder}\index{Burgunder>weiß} \\
      \end{zutaten}
      
      \begin{zubereitung}
        Fein gehackter Knoblauch, Ingwer, Chili und Frühlingszwiebel
        in (neutralem) Erdnußöl und einem kleinen Schuß Sesamöl anbraten,
        Mangoldblätter dazu, dieses Mal richtig braten, am besten im Wok, worin
        man genügend Platz zum Umwenden hat. Salzen und mit Zucker würzen. Mit
        etwas Sojasoße und Reiswein oder weißen Burgunder ablöschen, nochmals
        unter Rühren aufkochen und sofort servieren.\\
        \\
        So kann man Mangold immer wieder anders als Beilage servieren, zum
        gegrillten Kotelett, zur Bratwurst, zum gedämpften Fisch. So paßt er
	auch als Gemüsevorspeise aufs Büfett. \\
        Getränk: ein runder, weicher Weißburgunder, zum Beispiel aus Baden,
	paßt eigentlich zu allen vier Zubereitungen. \\
      \end{zubereitung}

    \mynewsection{Gratin von Mangoldstielen}%
              \glossary{Mangold-Gratin}%
	      \glossary{Gratin>Mangold-}

      \begin{einleitung}
        Die dicken, breiten Stiele sind ein eigenes Gemüse. Unbedingt muß man
	die Stiele sorgsam putzen, dabei alle Fäden abziehen, wie bei
	Rhabarber. Dann schneidet man sie für unser Rezept längs in fingerdicke
	und höchstens fingerlange Streifen. In Zitronenwasser aufbewahren, dann
	werden die Streifen appetitlich weiß. \\
        Als Grundrezept kann man sich merken: Diese Streifen in etwas Olivenöl
        andünsten, möglichst in einem flachen, weiten Topf, in dem sie viel
        Bodenkontakt haben und deshalb schön im Olivenöl schmurgeln können.
        Salzen, pfeffern und mit etwas Brühe besprenkeln, zugedeckt
        10--20~Minuten dünsten --- dann sind die Mangoldstiele zart, und man
        kann sie als eine fabelhafte Beilage servieren. Für unser Gratin
	braucht man jedoch noch mehr. \\
      \end{einleitung}

      \begin{zutaten}
        ca. 800 g & \myindex{Mangold}stiele \\
	1 & \myindex{Zitrone} \\
	2--3 Eßlöffel & \myindex{Olivenöl}\index{Oel=Öl>Oliven-} \\
	ca. \brea{} l & \myindex{Brühe} \\
	& \myindex{Salz} \\
	& \myindex{Pfeffer} \\
	1 & \myindex{weiße Zwiebel}\index{Zwiebel>weiß} oder
	    \myindex{rote Zwiebel}\index{Zwiebel>rot} \\
        100 g & \myindex{Speck} in dünnen Scheiben \\
	ca. 150 g & \myindex{Scamorza}\index{Käse>Scamorza} (geräucherter
	            Mozzarella) oder ein frischer
                    \myindex{Büffelmilchkäse}\index{Käse>Büffelmilch-} \\
	50 g & \myindex{Semmelbrösel} \\
	& abgeriebene \myindex{Zitrone}nschale \\
	3--4 Stängel & \myindex{Petersilie} \\
	3 Eßlöffel & \myindex{Olivenöl}\index{Oel=Öl>Oliven-} \\
      \end{zutaten}

      \personen{4--5}

      \begin{zubereitung}
        Die Mangoldstiele wie oben beschrieben schälen und in Streifen
	schneiden, im Olivenöl andünsten, in dem bereits die sehr fein
	gewürfelte Zwiebel angeschwitzt wurde. Etwas Brühe angießen, salzen und
	pfeffern, zugedeckt knapp 10--20~Minuten dünsten. Dann mit einer
	Schaumkelle herausheben und in eine flache, feuerfeste Gratinform
	verteilen, dabei schön parallel legen --- wie Spargel. \\
	Speck in feine Streifchen schneiden und zwischen den Mangoldstielen
	verteilen, auch den in kleine Würfel geschnittenen Scamorza. \\
	Dann Semmelbrösel, Zitronenschale und fein gehackte Petersilie mischen
	und gleichmäßig auf der Oberfläche verteilen. Alles mit Olivenöl
	beträufeln und im Ofen überbacken, bis der Käse schmilzt und alles
	brodelt. \\
	Beilage: Weißbrot. Für richtige Fans sind aber auch Pellkartoffeln eine
	ideale Begleitung --- damit kann man den Gemüsesaft schön auftunken. \\
	Getränk: Dazu gibt es einen erfrischenden Weißwein. \\
      \end{zubereitung}

    \mynewsection{Gerstenfrikadellen}

      \begin{zutaten}
        200 g & \myindex{Gerstenschrot} \\
	40 g & \myindex{Butter} \\
	400 ml & \myindex{Gemüsebrühe} (Instant) \\
	2 & \myindex{Lauchzwiebel}\index{Zwiebel>Lauch-}n \\
	1 & \myindex{Ei} \\
	2 Eßlöffel & \myindex{Vollkornsemmelbrösel}
	             \index{Semmelbrösel>Vollkorn-} \\
        & \myindex{Salz} \\
        & frischgemahlener \myindex{Pfeffer} \\
	4 Eßlöffel & Öl\index{Oel=Öl} \\
	2 & \myindex{Zwiebel}n \\
	2 & \myindex{rote Paprika}\index{Paprika>rot}schoten \\
	2 & \myindex{gelbe Paprika}\index{Paprika>gelb}schoten \\
	\brea{} l & \myindex{Weißwein}\index{Wein>weiß} 
	            (ersatzweise Brühe) \\
        1 Zweig & \myindex{Basilikum} \\
      \end{zutaten}

      \personen{4}
      \garzeit{45}
      \kalorien{470}

      \begin{zubereitung}
        Gerste in 20~g heißer Butter andünsten. Brühe zugießen und 20~Minuten
	bei kleiner Hitze ausquellen lassen. Dabei ab und zu umrühren. Gerste
	ganz auskühlen lassen. Lauchzwiebelringe in restlicher Butter
	andünsten. Zusammen mit dem Ei und Semmelbrösel unter die Gerste
	rühren. Mit Salz und Pfeffer abschmecken. Aus dem Teig 8~Frikadellen
	formen. In heißem Öl bei mittlerer Hitze von jeder Seite 4~Minuten
	braten. Frikadellen warm stellen. Zwiebelviertel und kleine
	Paprikawürfel im Bratfett andünsten. Wein zugießen und 5~Minuten
	schmoren. Mit Salz und Pfeffer abschmecken. Frikadellen auf dem Gemüse
	anrichten. Mit Basilikum garnieren. \\
      \end{zubereitung}

    \mynewsection{Graupen mit Brokkoli}

      \begin{zutaten}
        150 g & \myindex{Perlgraupen} \\
	2 & \myindex{Zwiebel}n \\
	2 & \myindex{Möhre}n \\
	30 g & \myindex{Butter} \\
	300 ml & \myindex{Gemüsebrühe} (Instant) \\
	1 & \myindex{gelbe Paprika}\index{Paprika>gelb}schote \\
	2 & \myindex{Tomate}n \\
	500 g & \myindex{Brokkoli} \\
	2 Eßlöffel & \myindex{Walnußöl}\index{Oel=Öl>Walnuß-} \\
        & \myindex{Salz} \\
        & frischgemahlener \myindex{Pfeffer} \\
	& \myindex{Muskatnuß} \\
	60 g & \myindex{Greyerzer}\index{Käse>Greyerzer}käse \\
	3 Eßlöffel & \myindex{Schlagsahne}\index{Sahne>Schlag-} \\
      \end{zutaten}

      \personen{3}
      \garzeit{50}
      \kalorien{505}

      \begin{zubereitung}
        Graupen in 1~Liter Wasser aufkochen, auf ein Sieb geben und mit kaltem
	Wasser abspülen. Zwiebel- und Möhrenwürfel in heißer Butter andünsten.
	Graupen und Brühe zugeben und 20~Minuten bei kleiner Hitze kochen.
	Paprikaschote in kleine Würfel schneiden und zu den Graupen geben.
	Weitere 10~Minuten kochen. In der Zwischenzeit Tomaten überbrühen,
	abziehen und in kleine Stücke schneiden. Brokkoli putzen und in heißem
	Öl unter Rühren 10~Minuten dünsten. Tomaten zugeben und mit Salz,
	Pfeffer und etwas Muskat würzen. Käse reiben. Die Hälfte davon mit der
	Sahne unter die Graupen rühren. Graupen und Brokkoli auf
	Portionstellern anrichten und mit dem restlichen Käse bestreuen. \\
      \end{zubereitung}

    \mynewsection{Grünkernbratlinge}

      \begin{zutaten}
        1 & \myindex{Zwiebel} \\
	20 g & \myindex{Butter} oder \myindex{Margarine} \\
	250 g & \myindex{Grünkern}schrot \\
	\breh{} l & \myindex{Gemüsebrühe} (Instant) \\
	250 g & \myindex{Möhre}n \\
	1 & \myindex{Kohlrabi} (250 g) \\
	& \myindex{Meersalz}\index{Salz>Meer-} \\
	1 Bund & \myindex{Petersilie} \\
	2 & \myindex{Ei}er \\
	60 g & \myindex{Vollkornpaniermehl}\index{Paniermehl>Vollkorn-} \\
	& \myindex{Pfeffer} \\
	3 Eßlöffel & \myindex{Sonnenblumenöl}\index{Oel=Öl>Sonnenblumen-} \\
      \end{zutaten}

      \personen{4}
      \kalorien{430}

      \begin{zubereitung}
        Zwiebel schälen, hacken und in der Butter andünsten. Grünkernschrot
	zufügen und andünsten. Mit der Brühe ablöschen, aufkochen und
	ausquellen lassen. Anschließend abkühlen lassen. Inzwischen Möhren
	und Kohlrabi putzen, schälen, in Würfel schneiden und in Salzwasser
	bißfest dünsten. Anschließend gut abtropfen lassen. Petersilie
	waschen, trockenschütteln, hacken und mit den Eiern und dem Paniermehl
	unter den Grünkernteig rühren. Mit Salz und Pfeffer abschmecken. Das
	Gemüse unterheben. Mit nassen Händen Frikadellen formen und diese im
	heißen Öl braten. \\
      \end{zubereitung}

    \mynewsection{Vegetarische Frikadellen}

      \begin{zutaten}
        200 g & \myindex{Grünkern} \\
	\brev{} l & \myindex{Gemüsebrühe} \\
	1 & \myindex{Zwiebel} \\
	\breh{} Bund & \myindex{Petersilie} \\
	\breh{} Bund & \myindex{Schnittlauch} \\
	\breh{} Bund & \myindex{Dill} \\
	100 g & \myindex{Magerquark}\index{Quark>Mager-} \\
	50 g & gemahlene \myindex{Haselnüsse} \\
	1 & \myindex{Eigelb} \\
	& \myindex{Meersalz}\index{Salz>Meer-} \\
	& \myindex{weißer Pfeffer}\index{Pfeffer>weiß} \\
	2 Eßlöffel & \myindex{Paniermehl} \\
	25 g & \myindex{Butter} oder \myindex{Margarine} \\
	125 g & \myindex{Kirschtomaten}\index{Tomate>Kirsch-} \\
	200 g & \myindex{Brunnenkresse} \\
	1 Becher & \myindex{Schlagsahne}\index{Sahne>Schlag-} \\
	2 & \myindex{Ei}er \\
      \end{zutaten}

      \personen{4}
      \garzeit{45}
      \kalorien{570}

      \begin{zubereitung}
        Grünkern abspülen und über Nacht in \brda{} l Wasser einweichen. Brühe
	angießen. Alles zum Kochen bringen und bei mittlerer Hitze ca.
	1~\breh{}~Stunden garen, bis die Flüssigkeit aufgesogen ist. Grünkern
	abkühlen lassen. Inzwischen Zwiebel schälen und fein hacken. Kräuter
	waschen, trocken schütteln. Bis auf einige Stiele zum Garnieren fein
	hacken. Grünkern, Zwiebelwürfel und Kräuter in eine Schüssel geben. Den
	abgetropften Quark, Haselnüsse und Eigelb zufügen. Alles gut
	miteinander verkneten und mit Salz und Pfeffer abschmecken. Aus der
	Grünkernmasse 4~große Bällchen formen, flachdrücken und in Pamiermehl
	wenden. Fett in einer großen Pfanne erhitzen und die Frikadellen darin
	bei mittlerer Hitze von jeder Seite ca. 7~Minuten goldbraun braten.
	Kirschtomaten waschen, je nach Größe halbieren und kurz miterhitzen.
	Kresse putzen, waschen. Mit der Hälfte der Sahne pürieren. Eier im
	heißen Wasserbad cremig aufschlagen. Kressemus zufügen und so lange
	weiterschlagen, bis die Masse dicklich ist, abschmecken. Restliche
	Sahne steif schlagen, unterheben. Frikadellen, Kirschtomaten und
	Kresseschaum anrichten. \\
      \end{zubereitung}

    \mynewsection{Bruschetta}

      \begin{zubereitung}
        Bruschette sind nichts weiter als geröstete Brotscheiben, die mit
	Knoblauch abgerieben und mit Olivenöl beträufelt werden. Salz, ein
	bisschen Pfeffer --- fertig ist ein wundervoller Leckerbissen. \\
	Man kann das beträufelte Brot noch garnieren und anreichern mit: \\
	\begin{itemize}
	  \item Tomaten, gehäutet, entkernt und gewürfelt --- oder einfach aus
	        der Schale auf das Brot gerieben
          \item getrockneten Tomaten, eingeweicht und in Olivenöl mariniert
	  \item Kräutern, zum Beispiel Basilikum, Rauke, Schnittlauch
	  \item Oliven, entkernt und gehackt, eventuell mit Kapern, Petersilie
	        und Sardellen vermischt
	  \item Ziegenfrischkäse
	  \item Puffbohnen, ausgebrochen und aus der dünnen inneren Haut
	        gelöst
	  \item Tapenade (Olivenpaste)
	\end{itemize}
      \end{zubereitung}

    \mynewsection{Mittelmeergemüse im Folienpaket}%
              \glossary{Gemüse im Folienpaket aus dem Mittelmeer}
      
      \begin{einleitung}
        Wer sich darüber ärgert, daß die Gemüsescheiben, die man grillen will,
        immer durch den Rost fallen, der packe das Grillgemüse doch einfach in
        ein Päckchen. Das ist nicht nur pfiffig, sondern auch praktisch. Man
	kann sie fix und fertig vorbereiten, sogar bequem zum Picknick
	mitnehmen und legt sie dann nur noch auf den Rost. \\
      \end{einleitung}

      \begin{zutaten}
        4--6 & \myindex{getrocknete Tomate}\index{Tomate>getrocknet}n \\
	1 & \myindex{weiße Zwiebel}\index{Zwiebel>weiß} \\
	2 mittelgroße & \myindex{Zucchini} \\
	1 & \myindex{Aubergine} \\
	2 & feste \myindex{Tomate}n \\
	1 großer & \myindex{Basilikum}strauß \\
	3--4 & \myindex{Knoblauchzehe}n \\
	50 g & \myindex{schwarze Oliven}\index{Oliven>schwarz} \\
	& \myindex{Salz} \\
	& \myindex{Pfeffer} \\
	4--5 Eßlöffel & \myindex{Olivenöl}\index{Oel=Öl>Oliven-} \\
	2 & \myindex{Thymian}zweige \\
	1 & \myindex{Rosmarin}zweig \\
	200 g & ausgelöste rohe \myindex{Garnelen} \\
      \end{zutaten}

      \personen{6}

      \begin{zubereitung}
        Die getrockneten Tomaten mit kochend heißem Wasser bedecken und
	\breh{}~Stunde einweichen. Zwiebel, Zucchini und Aubergine in
	zentimeterkleine Würfel schneiden. Tomaten häuten, entkernen, das
	Fleisch ebenfalls würfeln. Alles in einer Schüssel mischen, die
	Basilikumblätter zerzupfen, Knoblauch feinst würfeln oder durch die
	Presse drücken. Oliven nach Belieben entsteinen, wenn sie klein sind,
	ruhig ganz lassen. Alles behutsam mischen, auch die getrockneten,
	inzwischen weichen Tomaten fein gehackt hinzufügen. Alles mit Salz,
	Pfeffer und Olivenöl würzen. \\
	Die Kräuterblättchen abstreifen. Das Garnelenfleisch zentimetergroß
	würfeln und untermischen, dabei ebenfalls mit Salz und Pfeffer sowie
	mit einem guten Schuß Olivenöl würzen. \\
	Pro Person ein ausreichend großes Quadrat Alufolie doppelt auslegen
	--- die Kantenlänge sollte der Rollenbreite entsprechen. In der Mitte
	mit Öl einpinseln, darauf die vorbereitete Mischung häufen, auch die
	Garnelenwürfel dazwischen verteilen. \\
	Die Folie hochnehmen und locker über der Füllung verschließen, so, daß
	nichts an der Seite herauslaufen kann --- es wird sich beim Grillen ja
	Saft in dem Päckchen bilden! \\
	Die Päckchen auf den Rost über der Glut setzen und etwa 20~Minuten
	grillen, gegen Ende der Garzeit kann man die Päckchen oben ein wenig
	öffnen, damit Flüssigkeit verdampfen kann. \\
      \end{zubereitung}

    \mynewsection{Gepfefferte Bandnudeln mit Steinpilzen und Kräutern}%
              \glossary{Bandnudeln gefeffert mit Steinpilzen und Kräutern}

      \begin{zutaten}
        500 g & \myindex{Bandnudeln}\index{Nudeln>Band-} \\
	\breh{} & \myindex{rote Pfefferschote}\index{Pefferschote>rot}
	          (Peperoncino) \\
        4 & \myindex{Knoblauchzehe}n in Scheibchen \\
	100 ml & \myindex{Olivenöl}\index{Oel=Öl>Oliven-} \\
	1 Eßlöffel & frische, fein gewiegte \myindex{Petersilie} \\
	& \myindex{Salz} \\
      \end{zutaten}

      \personen{4}

      \begin{zubereitung}
        Den Knoblauch in Olivenöl leicht anrösten. Die Pfefferschote in kleine
	Ringchen schneiden und zum Knoblauch ins heiße Olivenöl geben.
	Aufpassen, daß nichts zu braun wird. Die Nudeln im Salzwasser al dente
	kochen, absieben und heiß in vorgewärmte tiefe Teller geben. Man kann
	die Nudeln auch zum Knoblauch in die Pfanne geben, mischen und dann
	anrichten. Der Peperoncino gibt dem Gericht eine pikante Note, aber wer
	nicht gern scharf ißt, kann ihn natürlich auch weglassen. \\
      \end{zubereitung}

    \mynewsection{Hühnerpastete mit Kapern}

      \begin{zutaten}
        350 g & gegartes Hühnerfleisch\index{Huhn}%
	        \footnote{z.B. Fleisch von der Hühnercremesuppe oder von
		          fertigen Grillhähnchen} \\
        2 kleine & \myindex{Zwiebel}n \\
	\breh{} Becher & \myindex{\cremefraiche{}} (100 g) \\
	1 Teelöffel & \myindex{Senf} \\
	1 Eßlöffel & abgetropfte \myindex{Kapern} \\
	\breh{} Bund & \myindex{Majoran} (ersatzweise 1 Teelöffel getrockneter)
	               \\
        & frisch gemahlener \myindex{Pfeffer} \\
	& \myindex{Salz} \\
      \end{zutaten}

      \personen{6}
      \kalorien{145}

      \begin{zubereitung}
        Fleisch im Blitzhacker oder im Mixer fein zerkleinern. Kleine
	Zwiebelwürfel, \cremefraiche{}, Senf, Kapern, und Majoranblättchen
	unterrühren. Die Creme mit Salz und Pfeffer abschmecken. \\
	Dazu: Bauernbrot. \\
      \end{zubereitung}

    \mynewsection{Hähnchenfilets in Petersilienhülle}

      \begin{zutaten}
        2 & \myindex{Hähnchenbrustfilet}s (500 g) \\
	& \myindex{Salz} \\
	& frisch gemahlener \myindex{Pfeffer} \\
	1 & \myindex{Zitrone} \\
	1 Bund & \myindex{Petersilie} \\
	2 & \myindex{Ei}er \\
	3 Eßlöffel & \myindex{Mehl} \\
	2 Eßlöffel & Öl\index{Oel=Öl} \\
	30 g & \myindex{Butter} oder \myindex{Margarine} \\
	150 g & \myindex{Spinat} \\
	3 Eßlöffel & \myindex{Estragonessig}\index{Essig>Estragon-} \\
	1 Prise & \myindex{Zucker} \\
	\breh{} Bund & \myindex{Radieschen} \\
	5 Eßlöffel & \myindex{Traubenkernöl}\index{Oel=Öl>Traubenkern-} \\
      \end{zutaten}

      \personen{4}
      \kalorien{495}

      \begin{zubereitung}
        Hähnchenfilets in mundgerechte Stücke schneiden. Mit Salz, Pfeffer und
	Zitronensaft würzen. Gehackte Petersilie mit Eiern vermischen.
	Hähnchenfilets erst in Mehl, dann in der Eiermasse wenden. Öl und
	Butter erhitzen. Hähnchenfilets bei kleiner Hitze etwa 8~Minuten
	goldgelb braten. Auf Küchenkrepp abtropfen lassen. Spinat putzen,
	waschen und trockentupfen. Essig mit Salz, Pfeffer, Zucker und
	feingehackten Radieschen verrühren. Öl unterschlagen. Spinat auf
	Portionsteller verteilen. Mit der Marinade beträufeln. Hähnchenfilets
	darauf anrichten. \\
      \end{zubereitung}

    \mynewsection{Grüne Lasagne}\glossary{Lasagne grün}

      \begin{zutaten}
      \end{zutaten}
      \begin{zutat}{Teig}
        400 g & \myindex{Mehl} \\
	4 & \myindex{Ei}er \\
	1 Teelöffel & \myindex{Salz} \\
	4 Teelöffel & \myindex{Olivenöl}\index{Oel=Öl>Oliven-} \\
      \end{zutat}
      \begin{zutat}{Füllung}
        500 g & \myindex{Spinat} \\
	500 g & \myindex{Ricotta}\index{Käse>Ricotta}käse (ersatzweise
	        \myindex{Schichtkäse}\index{Käse>Schicht-}) \\
        1 & \myindex{Mozzarella}\index{Käse>Mozzarella}käse \\
	200 g & \myindex{Champignon}\index{Pilze>Champignon}s \\
	2 & \myindex{Knoblauchzehe}n \\
	2 Bund & \myindex{Basilikum} \\
	2 Bund & \myindex{glatte Petersilie}\index{Petersilie>glatt} \\
	\breh{} & \myindex{Zitrone} \\
	1--2 Eßlöffel & \myindex{Instant-Brühe} \\
	& \myindex{Salz} \\
	& frisch gemahlener \myindex{Pfeffer} \\
	1 Becher & \myindex{Schlagsahne}\index{Sahne>Schlag-} (200 g) \\
	& \myindex{Mehl} zum Ausrollen \\
	& \myindex{Fett} für die Form \\
	1 Päckchen & \myindex{Helle Soße Instant} \\
	\brev{} l & \myindex{Weißwein} (ersatzweise Wasser) \\
	100 g & \myindex{Cheddar}\index{Käse>Cheddar}käse \\
      \end{zutat}
      
      \personen{8}
      \kalorien{765}

      \begin{zubereitung}
        Für den Teig Mehl mit Eiern, Salz und Öl vermischen. Mit den Händen
	mindestens 10~Minuten zu einem elastischen Teig verkneten. In
	Klarsichtfolie verpackt etwa 1~Stunde bei Zimmertemperatur ruhenlassen.
	Inzwischen für die Füllung tropfnassen Spinat bei großer Hitze
	zusammenfallen lassen. Einmal aufkochen. Auf einem Sieb abtropfen
	lassen und gut ausdrücken. Grob hacken und mit Ricottakäse,
	Mozzarellawürfeln, Champignonscheiben, zerdrücktem Knoblauch und
	zerpflückten Kräutern mischen. Mit Zitronensaft, Instant-Brühe, Salz
	und Pfeffer abschmecken. Sahne unterrühren. Teig auf wenig Mehl oder
	mit der Nudelmaschine zu dünnen Teigplatten ausrollen. Eine ofenfeste
	Form fetten und Boden und Rand der Form mit Teig auslegen.
	Restlichen Teig und die Spinatfüllung abwechselnd einschichten. Die
	letzte Schicht bildet eine Nudelplatte. Instant-Soße und Wein
	verrühren. Unter Rühren aufkochen lassen. Über die Lasagne gießen und
	mit geriebenem Käse bestreuen. In den Backofen schieben, auf \grad{200}
	schalten und 1~Stunde backen. \\
      \end{zubereitung}

    \mynewsection{Grüne Bandnudeln mit Lachs und Weinbrandsoße}%
              \glossary{Bandnudeln (grün) mit Lachs und Weinbrandsoße}%
	      \glossary{Nudeln>Band-}

      \begin{zutaten}
        250 g & grüne \myindex{Bandnudeln}\index{Nudeln>Band-} \\
	& \myindex{Salz} \\
	1 Eßlöffel & Öl\index{Oel=Öl} \\
	250 g & \myindex{geräucherter Lachs}\index{Lachs>geräuchert}
	        \index{Fisch>Lachs>geräuchert} oder
	        \myindex{Graved Lachs}\index{Lachs>Graved}
		\index{Fisch>Lachs>Graved} \\
        1\breh{} Becher & \myindex{Cr\`eme double} (etwa 200 g) \\
	4 Eßlöffel & \myindex{Weinbrand} \\
	1 Bund & \myindex{Dill} \\
	1--1\breh{} Eßlöffel & \myindex{Soßenbinder} (Instant) \\
	1 Gläschen & \myindex{Keta-Kaviar}\index{Kaviar>Keta-} (40 g) \\
      \end{zutaten}

      \personen{4}
      \kalorien{610}

      \begin{zubereitung}
        Bandnudeln in reichlich sprudelnd kochendem Salzwasser mit Öl 8~Minuten
	garen. Inzwischen Lachs in Streifen schneiden. Cr\`eme double und
	Weinbrand in einer Pfanne aufkochen. Feingeschnittenen Dill
	unterrühren. Mit Soßenbinder andicken. Abgetropfte Nudeln mit Lachs,
	Soße und Kaviar auf Tellern anrichten. Kurz vor dem Essen mischen. \\
      \end{zubereitung}

    \mynewsection{Nudeln selber machen}
        
      \begin{einleitung}
        Falls Sie's zum ersten Mal probieren wollen, hier ein ganz simples und
        absolut sicheres Grundrezept:
      \end{einleitung}

      \begin{zutaten}
        100 g & \myindex{Mehl} (Type 405) \\
	1 & \myindex{Ei} \\
	1 Teelöffel & Öl\index{Oel=Öl} \\
	etwas & \myindex{Salz} \\
      \end{zutaten}

      \begin{zubereitung}
        Alles erst mit den Knethaken des Handrührers, dann 10~Minuten mit
	den Händen kräftig kneten, bis der Teig glänzt und geschmeidig ist. Zu
	trockener Teig wird mit ein paar Tropfen Öl elastischer, unter
	klebrig-feuchten noch etwas Mehl kneten. Mindestens 1~Stunde in
	Klarsichtfolie ruhenlassen. Noch mal durchkneten, auf wenig Mehl dünn
	ausrollen und zuschneiden. Dieses Rezept ergibt 150~g, also
	1--2~Portionen, und kann vervielfacht werden. Kochzeit für frische
	Nudeln: je nach Dicke höchstens 5~Minuten! \\
      \end{zubereitung}

    \mynewsection{Pesto}
      
      \begin{einleitung}
        Basilikumsoße aus Ligurien: ,,Pesto alla genovese`` wird kalt über die
        heißen Nudeln gegeben. \\
      \end{einleitung}

      \begin{zutaten}
        5 Bund & \myindex{Basilikum} (etwa 100 g) \\
	50 g & \myindex{Pinienkerne} \\
	30 g & \myindex{Parmesan}\index{Käse>Parmesan}käse \\
	30 g & \myindex{Peccorino}\index{Käse>Peccorino}- oder
	       \myindex{Schafkäse}\index{Käse>Schaf-} \\
        3 & \myindex{Knoblauchzehe}n \\
	& \myindex{Salz} \\
	150 ml & \myindex{Olivenöl}\index{Oel=Öl>Oliven-} \\
      \end{zutaten}

      \personen{6}
      \kalorien{335}

      \begin{zubereitung}
        Basilikum waschen und trockentupfen. Blätter grob hacken. Pinienkerne
	und Käse im Blitzhacker oder im Mixer fein zerhacken. Abgezogene
	Knoblauchzehen, Basilikum, Salz und Öl im Mixer oder im Blitzhacker
	pürieren. Pinienkerne und Käse zugeben und 3~Sekunden durchmixen. \\
      \end{zubereitung}

    \mynewsection{Rohe Tomatensoße}\glossary{Tomatensoße roh}

      \begin{einleitung}
        Leicht und sommerlich: Für diese rohe Soße brauchen Sie nicht mal den
        Herd einzuschalten. \\
      \end{einleitung}

      \begin{zutaten}
        500 g & \myindex{Tomate}n \\
	60 g & \myindex{schwarze Oliven}\index{Oliven>schwarz} \\
	1 Bund & \myindex{Basilikum} \\
	4 & \myindex{Sardellen}\index{Fisch>Sardellen}filets \\
	2 & \myindex{Knoblauchzehe}n \\
	50 ml & \myindex{Olivenöl}\index{Oel=Öl>Oliven-} \\
	& \myindex{Salz} \\
	& grober \myindex{Pfeffer} aus dem Mörser \\
      \end{zutaten}

      \personen{4}
      \kalorien{265}

      \begin{zubereitung}
        Tomaten mit kochendem Wasser überbrühen, abziehen und halbieren. Kerne
	herausdrücken. Oliven entsteinen. Tomaten, Oliven, Basilikumblätter
	und Sardellenfilets fein hacken. Mit zerdrücktem Knoblauch mischen. Öl
	unterrühren und mit Salz und Pfeffer abschmecken. \\
      \end{zubereitung}

    \mynewsection{Kalte Walnuß-Soße}\glossary{Walnuß-Soße kalt}

      \begin{einleitung}
        In 5~Minuten fertig: eine kalte Soße aus pürierten Walnußkernen,
        Weißbrot, Majoran, Weinbrand. \\
      \end{einleitung}

      \begin{zutaten}
        3 Scheiben & \myindex{Weißbrot}\index{Brot>Weiß-} oder
	             \myindex{Toastbrot}\index{Brot>Toast-} ohne Rinde \\
        150 ml & \myindex{Brühe} (Instant) \\
	100 g & \myindex{Walnußkerne} \\
	1 Bund & \myindex{Majoran} \\
	1 Becher & \myindex{Schlagsahne}\index{Sahne>Schlag-} \\
	2 Eßlöffel & \myindex{Weinbrand} (ersatzweise \myindex{Traubensaft}) \\
	& \myindex{Salz} \\
	& \myindex{Pfeffer} \\
      \end{zutaten}

      \personen{6}
      \kalorien{445}

      \begin{zubereitung}
        Weißbrot würfeln und in Brühe einweichen. Walnußkerne im Blitzhacker
	fein mahlen. Majoranblätter abzupfen. Alle Zutaten im Mixer pürieren.
	Mit Salz und Pfeffer abschmecken. \\
      \end{zubereitung}

    \mynewsection{Rohkostplatte}

      \begin{zutaten}
        1 & weißer \myindex{Rettich} \\
	1 & \myindex{Kohlrabi} \\
	1 Stück & \myindex{Knollensellerie}\index{Sellerie>Knollen-}
	          (ca. 250 g)\\
	2 & \myindex{Möhre}n \\
	2 & \myindex{Zucchini} \\
	2 & \myindex{rote Paprika}\index{Paprika>rot}schoten \\
	1 Bund & \myindex{Radieschen} \\
	1 Bund & \myindex{Frühlingszwiebel}\index{Zwiebel>Frühlings-}n \\
	2 Becher & \myindex{Naturjoghurt}\index{Joghurt>Natur-} (\'a 150 g) \\
	2 Eßlöffel & Öl\index{Oel=Öl} \\
	& \myindex{Salz} \\
	& \myindex{Pfeffer} aus der Mühle \\
	& \myindex{Zitrone}nsaft \\
	1 Bund & \myindex{gemischte Kräuter}\index{Kräuter>gemischt} \\
	1 Portion & \myindex{Vinaigrette} (siehe Seite \pageref{vinaigrette})
	            z.B. mit \myindex{Distelöl} und
                    \myindex{Weißweinessig}\index{Essig>Weißwein-} \\
      \end{zutaten}

    \mynewsection{Vinaigrette}\label{vinaigrette}

      \begin{zutaten}
        4 Eßlöffel & \myindex{Weinessig} \\
	1 Prise & \myindex{Salz} \\
	1 Prise & \myindex{Zucker} \\
	& frisch gemahlener \myindex {Pfeffer} \\
	6--8 Eßlöffel & gutes Öl\index{Oel=Öl} \\
      \end{zutaten}
      
      \personen{4}

      \begin{zubereitung}
        Weinessig mit Salz, Zucker und Pfeffer so lange verrühren, bis sich
	Salz und Zucker vollkommen aufgelöst haben. \\
	Gutes Öl (die Menge richtet sich nach Geschmack und Säure des Essigs)
	langsam, in dünnem Strahl zugießen. Wichtig: Dabei immerzu
	weiterrühren. \\
	Die fertige Vinaigrette sollte ganz cremig sein. Erst dann weitere
	Zutaten unterrühren. Größere Mengen am besten im Mixer oder mit dem
	Pürierstab zubereiten. Die Zutaten reichen zum Anmachen einer großen
	Schüssel Salat für 4~Personen. \\
      \end{zubereitung}

    \mynewsection{Teigtaschen mit Tomatensoße (Pansoti con Salsa di Pomodori)}

      \begin{zutaten}
      \end{zutaten}
      \begin{zutat}{Teig}
        200 g & \myindex{Mehl} \\
	100 ml & \myindex{Weißwein}\index{Wein>weiß} \\
	50 ml & \myindex{Wasser} \\
      \end{zutat}
      \begin{zutat}{Füllung}
        1 Paket & tiefgefrorener \myindex{Blattspinat}\index{Spinat>Blatt-}
	          (300 g) \\
	250 g & feiner \myindex{Frischkäse}\index{Käse>Frisch-} oder
	        \myindex{Ricotta}\index{Käse>Ricotta} \\
	1 & \myindex{Ei} \\
	75 g & geriebener \myindex{Parmesan}\index{Käse>Parmesan} \\
	& \myindex{Salz} \\
	& \myindex{Pfeffer} \\
      \end{zutat}
      \begin{zutat}{Soße}
        1 Dose & \myindex{Tomate}n (500 g Abtropfgewicht) \\
	1 & \myindex{Knoblauchzehe} \\
	1 & \myindex{Zwiebel} (50 g) \\
	1 Bund & \myindex{Petersilie} \\
	1 Zweig & \myindex{Rosmarin} \\
	2 Eßlöffel & Öl\index{Oel=Öl} \\
	& \myindex{Salz} \\
	& \myindex{Pfeffer} \\
	& \myindex{Zucker} \\
      \end{zutat}
      \begin{zutat}{Außerdem}
        1 Bund & \myindex{Thymian} (eventuell) \\
	50 g & \myindex{Parmesan}\index{Käse>Parmesan} \\
      \end{zutat}
      
      \personen{4}
      \kalorien{680}

      \begin{zubereitung}
        Zuerst für die Füllung den Spinat auftauen. Für den Nudelteig alle
	Zutaten verkneten, bis ein glatter Teig entsteht. Kalt stellen. Den
	aufgetauten Spinat kleinschneiden, ausdrücken und mit restlichen
	Zutaten mischen. Für die Soße Tomaten grob zerkleinern. Knoblauchzehe
	und Zwiebel schälen und würfeln. Petersilie und Rosmarin feinhacken.
	Öl erhitzen, Zwiebel- und Knoblauchwürfel andünsten. Tomaten und
	restliche Zutaten zugeben, ca. 15~Minuten einkochen. Abschmecken. Teig
	in 6 Teile teilen, nacheinander zu Streifen ausrollen (8~cm Breite und
	48~cm Länge). Die Streifen in 6~Quadrate schneiden, in jedes Quadrat
	1 Teelöffel Füllung geben, den Teig zu Dreiecken übereinanderklappen.
	Zusammendrücken. Teigtaschen in leicht gesalzenem Wasser mit 1~Eßlöffel
	Öl ca 3--4~Minuten ziehen lassen. Mit der Tomatensoße anrichten.
	Eventuell mit Thymianblättchen und Parmesan servieren. \\
      \end{zubereitung}

    \mynewsection{Zahnbrasse mit Rosmarin (Dentice al Rosmarino)}

      \begin{zutaten}
        2 & \myindex{Zahnbrassen}\index{Fisch>Zahnbrasse} oder \\
	1 & \myindex{Rotbarsch}\index{Fisch>Rotbarsch} (ca. 1 kg) \\
	& \myindex{Essig} \\
	& \myindex{Salz} \\
	1 kleiner Zweig & \myindex{Lorbeer}blätter \\
	1 Zweig & \myindex{Rosmarin} \\
	1 & \myindex{Zitrone} \\
	5 Eßlöffel & \myindex{Olivenöl}\index{Oel=Öl>Oliven-} \\
	3--4 Teelöffel & \myindex{Balsamico-Essig}\index{Essig>Balsamico-}
	                 (Aceto-Balsamico) oder
	                 \myindex{Rotweinessig}\index{Essig>Rotwein-} \\
      \end{zutaten}

      \personen{4}
      \kalorien{270}

      \begin{zubereitung}
        Fisch waschen und säuern. Salzen. In eine feuerfeste Form geben.
	Gewürzzutaten, Zitronenscheiben und Olivenöl darübergeben. Mit Alufolie
	abdecken und im Backofen bei \grad{180} ca. 25~Minuten dünsten.
	Tranchieren und mit Balsam-Essig beträufeln. \\
	Mit Salzkartoffeln und gemischtem Salat servieren. \\
      \end{zubereitung}

    \mynewsection{Eierspeise mit Brokkoli}\glossary{Brokkoli mit Eierspeise}

      \begin{zutaten}
        500 g & \myindex{Brokkoli} \\
	6 & \myindex{Ei}er \\
	1 Bund & \myindex{Petersilie} \\
	50 g & \myindex{Reibekäse}\index{Käse>Reibe-} \\
	& \myindex{Muskatnuß} \\
	& \myindex{Salz} \\
	& \myindex{Pfeffer} \\
	& \myindex{Zucker} \\
	& \myindex{edelsüßer Paprika}\index{Paprika>edelsüß} \\
	50 g & \myindex{Margarine} \\
      \end{zutaten}

      \begin{zubereitung}
        Wenn man die Brokkoliköpfe gewaschen hat, schneidet man die Stiele in
	dünne Scheiben und zerteilt die Röschen. Man gart das Gemüse in wenig
	Salzwasser (das dauert knapp eine Viertelstunde) und läßt es dann
	abtropfen. Inzwischen verquirlt man die Eier, mischt die fein gehackte
	Petersilie und den Reibekäse drunter und würzt die Masse kräftig mit
	Muskat, Salz, Pfeffer, Zucker und Paprika. Dann läßt man in der Pfanne
	die Margarine heiß werden, verteilt das Gemüse darin und gießt die
	Eiermasse drüber. Das Ganze wird wie ein Omelett gebacken und
	zusammengeklappt. Wer mag, kann Bratkartoffeln oder Kartoffelbrei dazu
	servieren. Nötig ist das aber nicht. Denn so ein Omelett sättigt sehr
	gut. \\
      \end{zubereitung}

    \mynewsection{Brokkoli in Hollandaise}

      \begin{zutaten}
        500 g & \myindex{Brokkoli} \\
	40 g & \myindex{Margarine} \\
	40 g & \myindex{Mehl} \\
	1 & \myindex{Eigelb} \\
	3 Eßlöffel & \myindex{süße Sahne}\index{Sahne>süß} \\
	& \myindex{Salz} \\
	& \myindex{Zucker} \\
	& \myindex{Pfeffer} \\
	& \myindex{Muskatnuß} \\
      \end{zutaten}

      \begin{zubereitung}
        In \breh{} l Salzwasser läßt man die gewaschenen und zerkleinerten
	Brokkoli etwa 15~Minuten garen. Während sie abtropfen, läßt man die
	Margarine heiß werden und schwitzt darin das Mehl an. Zum Ablöschen
	(ständig dabei rühren!) nimmt man ein wenig Brokkoliwasser. Zum Schluß
	legiert man die Soße mit Eigelb und Sahne und gießt sie über das
	Gemüse. Lecker schmecken dazu gekochter Fisch und Salzkartoffeln. \\
      \end{zubereitung}

    \mynewsection{Brokkoli auf Italienisch}

      \begin{zutaten}
        500 g & \myindex{Brokkoli} \\
	50 g & \myindex{Margarine} \\
	\brea{} l & \myindex{Fleischbrühe} (Würfel) \\
	\brea{} l & \myindex{süße Sahne}\index{Sahne>süß} \\
	& \myindex{Salz} \\
	& \myindex{Zucker} \\
	& \myindex{Pfeffer} \\
	75 g & \myindex{Mandelblättchen} \\
      \end{zutaten}

      \begin{zubereitung}
        Man verliest und wäscht die Brokkoli und zerteilt Stiele und Röschen.
	Im heißen Fett brät man sie kurz an, gießt Brühe und Sahne dazu und
	läßt alles 15~Minuten garen. Das Gemüse wird mit Salz, Zucker und
	Pfeffer abgeschmeckt und vorm Servieren mit den Mandelblättchen
	bestreut, die man ohne Fett in der Pfanne bei ständigem Rühren
	hellbraun geröstet hat. Schmeckt zu Kotelett oder Kalbsschnitzel und
	Kartoffeln. \\
      \end{zubereitung}

    \mynewsection{Hähnchenbrust mit grüner Soße}

      \begin{zutaten}
        1 & \myindex{Hähnchenbrustfilet} (150 g) \\
        1 Tasse & \myindex{Brühe} (Instant) \\
        1 Tasse & \myindex{Wasser} \\
	& \myindex{Zitrone}nsaft \\
	& \myindex{Lorbeer}blatt \\
        1 & \myindex{Lauchzwiebel}\index{Zwiebel>Lauch-} \\
        1 & \myindex{Möhre} (100 g) \\
        1--2 Stangen & \myindex{Staudensellerie}\index{Sellerie>Stauden-}
	               (50 g) \\
        1 mittelgroße & \myindex{Kartoffel} \\
        & \myindex{Salz} \\
        & \myindex{Cayennepfeffer}\index{Pfeffer>Cayenne-} \\
      \end{zutaten}
      \begin{zutat}{grüne Soße}
        1 Teelöffel & \myindex{Sardellenpaste} \\
        \breh{} Bund & \myindex{Petersilie} \\
        & \myindex{Schnittlauch} \\
        3 Teelöffel & \myindex{Kapern} \\
        \breh{} Scheibe & \myindex{Flachknäckebrot}\index{Brot>Flachknäcke-} \\
        1\breh{} Teelöffel & \myindex{Olivenöl}\index{Oel=Öl>Oliven-} \\
      \end{zutat}
      
      \kalorien{400}

      \begin{zubereitung}
        Brühe, Wasser, Zitronensaft und Lorbeerblatt langsam zum Kochen
	bringen. Das Gemüse putzen und in fingerlange Stücke schneiden.
	Hähnchenbrustfilet mit Salz und Cayennepfeffer einreiben und mit den
	Kartoffelstücken in die kochende Brühe geben. Nach 10~Minuten das
	restliche Gemüse hinzufügen und weitere 15~Minuten lang im
	geschlossenen Topf auf niedriger Wärmestufe garen. Alles in einem
	tiefen Teller anrichten. Die Brühe etwas einkochen und über Fleisch
	und Gemüse schöpfen. Die Hähnchenbrust mit der grünen Soße bestreichen:
	Alle Zutaten mit einem Wiegemesser oder im Blitzhacker zerkleinern und
	mit dem Olivenöl verrühren. (Wer den leichten Fischgeschmack der
	Sardellenpaste nicht mag, salzt etwas stärker.) \\
      \end{zubereitung}

    \mynewsection{Minestrone}

      \begin{zutaten}
        70 g & \myindex{Zucchini} (1 Stück) \\
        50 g & \myindex{Champignon}\index{Pilze>Champignon}s \\
        1 & \myindex{Möhre} (35 g) \\
        1 Stange & \myindex{Staudensellerie}\index{Sellerie>Stauden-} (30 g) \\
        50 g & \myindex{weiße Bohnen}\index{Bohnen>weiß} (aus der Dose) \\
        30 g & \myindex{Nudeln} \\
        \brda{} l & klare \myindex{Gemüsebrühe} \\
        2 & \myindex{Tomate}n (100 g) \\
        1 Teelöffel & geriebener \myindex{Parmesan}\index{Käse>Parmesan}käse \\
      \end{zutaten}

      \kalorien{400}

      \begin{zubereitung}
        Das Gemüse putzen, waschen und kleinschneiden. Champignons in Scheiben
	schneiden. Die Tomaten überbrühen und abziehen. Grob zerkleinern und
	die Kerne entfernen. Die Gemüsebrühe erhitzen. Die Nudeln (Spaghetti
	in Stücke brechen), Möhren, Zucchini und Staudensellerie 5~Minuten
	aufkochen. Dann die abgetropften Bohnen, Champignons und Tomatenstücke
	hinzufügen. Alles eine Minute lang weiterköcheln, mit Salz und Pfeffer
	abschmecken. Parmesankäse darüberstreuen. \\
      \end{zubereitung}

    \mynewsection{Käse-Tomaten}

      \begin{zutaten}
        3 & \myindex{Tomate}n (300 g) \\
        1 & \myindex{Lauchzwiebel}\index{Zwiebel>Lauch-} \\
        50 g & \myindex{Schafkäse}\index{Käse>Schaf-} \\
        1 & \myindex{Knoblauchzehe} \\
        & frisches \myindex{Basilikum} \\
        1 Teelöffel & \myindex{Kapern} \\
        & \myindex{Pfeffer} aus der Mühle \\
        2 Scheiben & \myindex{Meterbrot}\index{Brot>Meter-} \\
      \end{zutaten}

      \kalorien{200}

      \begin{zubereitung}
        Von den Tomaten einen kleinen Deckel abschneiden. Mit einem Teelöffel
	die Kerne herauskratzen und die Tomaten umgedreht abtropfen lassen.
	Für die Füllung die Lauchzwiebel in feine Ringe schneiden und den
	Schafkäse fein zerbröckeln. Mit durchgepreßter Knoblauchzehe,
	feingehacktem Basilikum, Kapern und Pfeffer aus der Mühle vermischen.
	Die Tomaten füllen und nebeneinander in eine kleine Gratinform setzen.
	Im Backofen bei \grad{200} 20~Minuten backen. Dazu gibt's 2~Scheiben
	Meterbrot. \\
      \end{zubereitung}

    \mynewsection{Gebratene Seelachs-Filets mit Kartoffel-Pilz-Auflauf}

      \begin{zutaten}
        500 g & \myindex{Kartoffel}n \\
        25 g & \myindex{Butter} \\
        750 g & \myindex{Champignon}\index{Pilze>Champignon}s \\
        200 g & \myindex{Schlagsahne}\index{Sahne>Schlag-} \\
        50 g & geriebener \myindex{Hartkäse}\index{Käse>Hart-} aus Italien \\
        & \myindex{Jodsalz}\index{Salz>Jod-} \\
        & \myindex{Pfeffer} \\
        2 & \myindex{Knoblauchzehe}n \\
        8 & \myindex{Alaska-Seelachs-Filets}\index{Fisch>Seelachs} \\
        & \myindex{Paprikapulver} scharf \\
        6 Eßlöffel & \myindex{Olivenöl}\index{Oel=Öl>Oliven-}
	             mit Gewürzeinlage Oregano \\
      \end{zutaten}

      \personen{4}
      \garzeit{55}

      \begin{zubereitung}
        Backofen auf \grad{180} vorheizen. Kartoffeln waschen, schälen und in
	dünne Scheiben hobeln. Eine ofenfeste Form buttern, Kartoffel- und
	Champignonscheiben hineinschichten. Sahne und Hartkäse verrühren,
	salzen, pfeffern, über den Auflauf gießen. 30--40~Minuten im Ofen
	backen. \\
	Knoblauch schälen und fein hacken. Die aufgetauten
	Alaska-Seelachs-Filets trockentupfen, dann mit Knoblauch, Salz, Pfeffer
	und Paprikapulver würzen. Eine beschichtete Pfanne mit Oregano-Olivenöl
	erhitzen und die Seelachs-Filets etwa 3~Minuten auf jeder Seite bei
	mittlerer Hitze braten. Vorsichtig wenden. \\
	Die Fischfilets mit dem Kartoffel-Pilz-Auflauf anrichten und mit dem
	restlichen heißen Oregano-Olivenöl beträufeln. \\
      \end{zubereitung}
