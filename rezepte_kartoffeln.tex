
% created Montag, 10. Dezember 2012 16:20 (C) 2012 by Leander Jedamus
% modifiziert Mittwoch, 11. März 2015 17:17 von Leander Jedamus
% modifiziert Montag, 09. März 2015 14:20 von Leander Jedamus
% modified Montag, 10. Dezember 2012 16:30 by Leander Jedamus

  \mynewchapter{Kartoffeln}

    \mynewsection{Kartoffelbrei}

      \begin{zutaten}
        1kg & \myindex{Kartoffel}n, möglichst mehlig kochende \\
        1 Teelöffel & \myindex{Salz} \\
        1 Tasse & \myindex{Wasser} \\
        \brev{} l & heiße \myindex{Milch} \\
        40 g & \myindex{Butter} \\
        etwas & geriebene \myindex{Muskatnuß} \\
      \end{zutaten}

      \garzeit{20--25}

      \begin{zubereitung}
        Kartoffeln schälen und vierteln wie Salzkartoffeln. Mit Wasser und Salz
	aufsetzen und etwa 20~Minuten garen. Abgießen und in Kartoffelquetsche
	geben und pressen. Inzwischen Milch erhitzen. Milch solange nach und
	nach unter Rühren mit dem Schneebesen über die gepreßten Kartoffeln
	geben, bis Breibeschaffenheit erreicht ist. Mit Salz und Muskatnuß
	abschmecken und Butter unterrühren. \\
      \end{zubereitung}

    \mynewsection{Kartoffelpuffer}

      \begin{zutaten}
        ca. 1 kg & \myindex{Kartoffel}n \\
        1 & \myindex{Zwiebel} \\
        2 & \myindex{Ei}er \\
        1 Teelöffel & \myindex{Salz} \\
        2--3 Eßlöffel & \myindex{Mehl} (20-30g) \\
        80--100 g & \myindex{Fett} zum Braten \\
      \end{zutaten}

      \begin{zubereitung}
        Rohe Kartoffeln schälen und reiben. Salz und geriebene Zwiebeln dazu,
	Mehl und Eier unterrühren. Dann 1~Stunde das Mehl ausquellen lassen.
	Schürze anziehen und Herdumgebung vor Spritzern schützen, am besten
	auch den Boden vor dem Herd mit alten Zeitungen schützen (man rutscht
	sonst evtl. auf Ölfilm aus). Für Abzug der Fettschwaden sorgen. \\
        Zum Ausbraten nimmt man Öl, Palmin oder Mischfett.
	Pfanne\footnote{keine beschichtete Pfanne! Diese verträgt die Hitze
	nicht.} recht heiß werden lassen, Fett hinein. In das sehr heiße Fett
	(bis es fast raucht) gibt man löffelweise den Teig. Teig flachdrücken.
	Puffer auf beiden Seiten braun und knusprig braten und immer schwimmen
	lassen im Fett. Dabei sollten nicht mehr als 3--4~Puffer in einer
	Pfanne sein und diese sollten etwa handtellergroß sein. Der
	Pfannenwender sollte öfter gereinigt werden, damit man noch unter die
	Puffer kommt. Fertig gebratene Puffer auf großen Teller, der mit
	Küchenkrepp belegt ist, ablegen. Puffer probieren, eventuell Teig
	nachwürzen. Bratkrümel aus der Pfanne entfernen, frisches Fett dazu und
	heiß werden lassen und weitere Puffer braten. \\
        Dazu gibt man Scheiben Schwarzbrot und Apfelkompott, oder anderen
	Obstkompott oder grünen Salat. Man kann die Puffer auch sehr gut kalt
	essen oder kalt als Brotbelag verwenden. \\
      \end{zubereitung}

    \mynewsection{Bratkartoffeln}\glossary{Kartoffeln>Brat-}

      \begin{zutaten}
        ca. 1 kg & \myindex{Pellkartoffel}\index{Kartoffel>Pell-}n oder
	           \myindex{Salzkartoffel}\index{Kartoffel>Salz-}n vom Vortag
		   \\
        1 & \myindex{Zwiebel} \\
        & \myindex{Salz} \\
        40 g & \myindex{Fett} zum Braten \\
      \end{zutaten}

      \begin{zubereitung}
        Am besten festkochende Sorte Kartoffeln nehmen und auf die gleiche
	kleine Größe achten, dann wird es leichter, die Scheiben zu braten. Am
	besten läßt man die großen Kartoffeln für Salzkartoffeln und die
	kleineren Kartoffeln nimmt man für Pellkartoffeln. Gekochte Kartoffeln
	pellen und in Scheiben schneiden. Sind die gekochten Kartoffeln zu naß,
	kann man etwas Mehl über die Scheiben stäuben. Wenn man ein Brettchen
	voll geschnitten hat, die Kartoffeln in heiße Pfanne ins heiße Fett
	(aber keine volle Hitze nehmen) geben. Dabei die Scheiben auf den Boden
	verteilen. Nach und nach weitere Scheiben zugeben und bräunen lassen.
	Salz über die Kartoffeln streuen. Zwiebel schälen und würfeln. Wenn die
	meisten Kartoffeln Farbe angenommen haben, die Zwiebel dazugeben und
	ebenfalls bräunen lassen. Man kann auch Speckwürfel im Fett auslassen,
	wenn die Bratkartoffeln etwas herzhafter schmecken sollen. \\
      \end{zubereitung}

    \mynewsection{Bauernomelett}\glossary{Omelett>Bauern-}

      \begin{zutaten}
        1 Portion & \myindex{Bratkartoffel}\index{Kartoffel>Brat-}n \\
        2--3 & \myindex{Ei}er \\
        etwas & \myindex{Salz}, \myindex{Pfeffer} \\
        1 Eßlöffel & \myindex{Sahne} oder \myindex{Milch} \\
        2 Eßlöffel & \myindex{Butter} \\
        1 & \myindex{Gewürzgurke}\index{Gurke>Gewürz-} \\
        3 & \myindex{Salat}blätter \\
        1--2 & \myindex{Tomate}n \\
        \brev{} & \myindex{Schlangengurke}\index{Gurke>Schlangen-} \\
      \end{zutaten}

      \begin{zubereitung}
        Kartoffeln braten und warm stellen. Die Pfanne für das Omelett sollte
	einen schweren Boden haben, gut läufig sein, d.h. Fett wurde nur mit
	Hilfe von Salz entfernt und die Pfanne ist nicht vom Spülen entfettet
	(backt dann an). Eier mit Salz, Pfeffer, Milch schaumig schlagen.
	Butter in die recht heiße Pfanne geben, Pfanne schwenken, damit sich
	das Fett gut in der Pfanne verteilen kann. Wenn die Butter geschmolzen
	ist, schäumt und gerade anfängt, sich haselnußbraun zu färben, Hitze
	zurücknehmen und die und Eier hineingeben, dabei die Pfanne leicht hin
	und her kippen, damit sich die Flüssigkeit auf die gesamte Fläche
	verteilt. Die Eier beginnen zu stocken. Jetzt wird der Omelettrand
	rundherum mit dem Pfannenmesser (Spachtel) von der Pfanne gelöst,
	Pfanne ein wenig schütteln, Omelettrand hier und da anheben und bei
	schräg gehaltener Pfanne die noch flüssige Eimasse von der Oberfläche
	herab unter das Omelett laufen lassen. Nach weniger als 1~Minute sind
	die Eier gestockt. Sobald die Eier stocken, unbedingt aufpassen, sonst
	wird das Omelett nicht locker und zart sondern zäh und gummiartig, weil
	zu lange gebacken. Pfanne erneut schwenken, damit das Omelett nicht
	festbackt. Seine Oberfläche sollte nicht mehr flüssig, sondern nur noch
	feucht sein. \\
        Vorbereitete Füllung (hier Bratkartoffeln) daraufgeben und eine
	Omeletthälfte mit dem Pfannenmesser vorsichtig darüber klappen. \\
        Dazu Gewürzgurke und Dekoration aus Salatblättern, Tomate, Gurke
	(mit Salatsoße beträufelt). \\
        Macht ziemlich satt und zufrieden. \\
      \end{zubereitung}

    \mynewsection{Schweizer Käsekartoffeln (im Backofen)}%
              \glossary{Kartoffeln>Käse-}

      \begin{zutaten}
        1 kg & gekochte \myindex{Kartoffel}n (festkochend) \\
        1 Teelöffel & \myindex{Kümmel} \\
        1 Teelöffel & weiche \myindex{Butter} \\
        1 & \myindex{Zwiebel} \\
        1 Zehe & \myindex{Knoblauch} \\
        ca. 3 Zweiglein & frischer \myindex{Thymian} \\
        200 g & geriebener \myindex{Greyerzer}\index{Käse>Greyerzer} Käse \\
        ersatzweise & \myindex{Emmentaler}\index{Käse>Emmentaler} \\
        10 & kleine \myindex{Salami}scheiben \\
        1 Becher & \myindex{saure Sahne}\index{Sahne>sauer} \\
        1 Becher & \myindex{süße Sahne}\index{Sahne>süß} \\
        & \myindex{Salz}, \myindex{Muskatnuß} \\
      \end{zutaten}

      \garzeit{25}

      \begin{zubereitung}
        Rohe, etwa mittelgroße unbeschädigte festkochende Kartoffeln mit
	1~Teelöffel Kümmel und Wasser aufsetzen und garen (ca. 20--25~Minuten).
	Kartoffeln pellen und in Scheiben schneiden. Feuerfeste Form mit Butter
	ausstreichen (mit dem Pinsel, sonst Finger nehmen). Form mit
	feingehackter Zwiebel, gehacktem Knoblauch und gehacktem Thymian
	ausstreuen. Die Hälfte der Kartoffeln in die Form schuppenartig
	schichten. Jetzt \brdv{} des geriebenen Käses darüber streuen.
	Restliche Kartoffelscheiben und die Salami darauf schichten. Süße Sahne
	und saure Sahne mit Muskatnuß verrühren und über den Auflauf gießen. \\
        Backofen vorheizen (dauert etwas!!) auf \grad{200}. Auflauf einschieben
	auf mittlere oder etwas darunter liegende Schiene und etwa 25~Minuten
	backen. Wenn die Bräunung zu stark wird, Alufolie zum Abdecken nehmen.
	Mit Pfeffer übermahlen, mit Thymianzweiglein garnieren. \\
        Dazu paßt gut gebratener Fisch, feines Gemüse (Brokkoli) oder grüner
	Salat. \\
      \end{zubereitung}

    \mynewsection{Kartoffeltaler}

      \begin{zutaten}
        & gekochte \myindex{Pellkartoffel}\index{Kartoffel>Pell-}n \\
        & \myindex{Salz} \\
        1 & \myindex{Ei} \\
        etwas & \myindex{Mehl} \\
        40 g & \myindex{Fett} zum Braten \\
      \end{zutaten}

      \begin{zubereitung}
        Wenn man Reste von gekochten Kartoffeln hat, kann man o. g. Gericht
	daraus bereiten. Oder man kocht die Kartoffeln frisch. Geht auch mit
	Salzkartoffeln. Die gekochten Kartoffeln pellt man und preßt sie durch
	die Kartoffelquetsche. Dazu gibt man 1 Ei und Mehl. Daraus knetet man
	einen Teig. Pfanne erhitzen, Fett zergehen lassen und etwa 5--6~cm
	große Taler formen und flachdrücken. Von beiden Seiten braun braten. \\
        Dazu kann man Obstkompott oder frischen Salat essen oder die Taler zu
	Fleischgerichten reichen. \\
      \end{zubereitung}

    \mynewsection{Döppekoche}

      \begin{zutaten}
        2 \breh{} kg & weich kochende \myindex{Kartoffel}n \\
        \breh{} l & \myindex{Milch} \\
        2 & \myindex{Brötchen} \\
        1--2 & \myindex{Ei}er \\
        150 g & gutes \myindex{Dörrfleisch}, gewürfelt \\
        1 \breh{} & \myindex{Zwiebel}n gewürfelt \\
        \breh{} & Stange \myindex{Lauch} in Ringe geschnitten \\
        1 \breh{} Teelöffel & \myindex{Salz} \\
        & \myindex{Pfeffer} \\
        & \myindex{Muskatnuß} \\
        & \myindex{Petersilie} frisch gehackt \\
        2 Teelöffel & \myindex{Kräuter der Provence} \\
      \end{zutaten}

      \garzeit{120}

      \begin{zubereitung}
        Kartoffeln schälen, waschen und fein reiben. In eine große Schüssel
	geben. Milch erhitzen, Brötchen darin einweichen und zerstampfen.
	Restliche Zutaten mit den Kartoffeln gut durcharbeiten. Gußbräter
	sorgfältig mit viel Sonnenblumenöl ausstreichen und im Backofen
	vorheizen. Kartoffelmasse einfüllen, ohne Deckel in den Backofen
	stellen 30~Minuten bei \grad{250} und 90~Minuten bei \grad{180}. Dazu
	Apfelmus geben. \\
      \end{zubereitung}

    \mynewsection{Kartoffeln gratiniert}\glossary{Gratin>Kartoffel-}

      \begin{zutaten}
        1 kg & rohe \myindex{Kartoffel}n in hauchdünne Scheiben gehobelt \\
        5 & große \myindex{Zwiebel}n in hauchdünne Scheiben gehobelt \\
        \brev{} l & \myindex{süße Sahne}\index{Sahne>süß} \\
        \brev{} l & \myindex{Milch} \\
        ca. 150 g & \myindex{Butter} \\
        & \myindex{Salz} \\
        & \myindex{Pfeffer} frisch gemahlen \\
      \end{zutaten}

      \garzeit{70}

      \begin{zubereitung}
        Form ausbuttern, Kartoffeln und Zwiebeln schichtweise mit Salz und
	Pfeffer bestreuen. Sahne und Milch verquirlen und über die Kartoffeln
	gießen. Reichlich Butterflöckchen darauf, in den Ofen bei \grad{200}
	gut 70~Minuten garen lassen. \\
      \end{zubereitung}

    \mynewsection{Kartoffelklöße gefüllt \'a la Martha Tante}

      \begin{zutaten}
        2--3 kg & rohe \myindex{Kartoffel}n \\
        6--7 große & \myindex{Pellkartoffel}\index{Kartoffel>Pell-}n \\
        & \myindex{Salz} \\
        1--2 & \myindex{Ei}er \\
      \end{zutaten}
      \begin{zutat}{Füllung}
        2--3 mittlere & Stangen \myindex{Lauch} \\
        viel & \myindex{Schnittlauch} \\
        viel & \myindex{Petersilie} \\
        2 & \myindex{Brötchen} eingeweicht \\
        & \myindex{Pfeffer} \\
        & \myindex{Jodsalz}\index{Salz>Jod-} \\
      \end{zutat}
      \begin{zutat}{Buttersoße}
        250 g & \myindex{Butter} \\
        \brev{} l & \myindex{Sahne} \\
        1 Teelöffel & klare \myindex{Brühe} \\
      \end{zutat}

      \garzeit{20--30}

      \begin{zubereitung}
        Rohe Kartoffeln in etwas Wasser reiben, in einem Tuch kräftig
	auspressen, Wasser auffangen und stehen lassen (Stärke setzt sich ab).
	Pellkartoffeln durchdrehen, beide Kartoffelmassen in Schüssel geben,
	vorsichtig abgießen. Stärke an die Kartoffeln geben, salzen und gut
	durchkneten. \\
        Füllung: Lauch putzen, in Stückchen schneiden, waschen, abtropfen
	lassen. Fett in Pfanne erhitzen, Lauch ohne Wasser 15~Minuten darin
	dünsten, danach ausgedrückte Brötchen zerpflücken und dazugeben, 5
	Minuten mitrösten, ebenso die Kräuter. Kräftig würzen. Abkühlen lassen.
	\\
        Jeweils 1--2~Eßlöffel Füllung 1~cm dick mit Kartoffelmasse umhüllen,
	Klöße werden etwa faustgroß. In siedendes Wasser geben und
        20--30~Minuten ziehen lassen (keinesfalls kochen lassen!), bis sie
	hochkommen. \\
        Für die Soße Butter zerlassen, Sahne dazu und mit klarer Brühe würzen.
	\\
      \end{zubereitung}

    \mynewsection{Käsekartoffeln}\glossary{Kartoffeln>Käse-}%
              \label{kaesekartoffeln}

      \begin{zutaten}
        15--20 & \myindex{Kartoffel}n gleicher Größe \'a 80 g ca. \\
        2 Eßlöffel & \myindex{Butter} \\
      \end{zutaten}
      \begin{zutat}{Gewürzmischung}
        1 Teelöffel & \myindex{Pfeffer}körner \\
        1 Teelöffel & \myindex{Kümmel} \\
        1 Eßlöffel & \myindex{Majoran} \\
        1--2 & getrocknete \myindex{Chilischote}n \\
        1 Teelöffel & \myindex{Salz} \\
        & \myindex{Muskatnuß} \\
        250 g & Scheiben \myindex{Käse} \\
        ca. 150 g & durchwachsener \myindex{Speck} dünn geschnitten \\
        200 g & \myindex{süße Sahne}\index{Sahne>süß} \\
      \end{zutat}

      \personen{8--10}

      \garzeit{20--25}

      \begin{zubereitung}
        Kartoffeln in der Schale kochen, eventuell bereits am Morgen. Pellen
	und längs halbieren. Mit der Schnittfläche nach oben auf ein tiefes
	Backblech oder in eine große Auflaufform/Reine dicht an dicht setzen.
	Gewürze im Mörser zerstoßen, die Kartoffeln mit der Hälfte davon
	bestreuen. Käsescheiben passend zuschneiden und auf jede Kartoffel ein
	Stück legen. Mit dem Rest der Gewürzmischung bestreuen und alles mit
	Speckscheiben belegen. Mit Sahne angießen (unsere Kartoffeln waren
	damals viel zu trocken). In den Backofen für 15--20~bzw. 20--25~Minuten
	bei \grad{180} (Nachsehen), bis alles brodelt. \\
        Dazu: Frischer Krautsalat mit Chinakohl (siehe Seite
	\pageref{krautsalat}). \\
        Wein: Beaujolais, Dornfelder oder Valpolicella. Gut auch ein Weißwein
	--- Weißburgunder aus Baden, Südtirol oder ein Silvaner aus Rheinhessen
	oder Franken und zwar ein Qualitätswein, aus 2003 darf es auch ein
	Kabinett sein. \\
      \end{zubereitung}

    \mynewsection{Käsepfanne mit Pellkartoffeln}%
              \glossary{Kartoffeln>Käsepfanne mit Pell-}

      \begin{zutaten}
        1 kg & \myindex{Pellkartoffel}\index{Kartoffel>Pell-}n geraffelt \\
        50 g & \myindex{Butter} \\
        100--150 g & durchwachsener \myindex{Speck} in Streifen \\
        100--200 g & junger \myindex{Gouda}\index{Käse>Gouda} in Streifen \\
      \end{zutaten}

      \garzeit{30}

      \begin{zubereitung}
        Speck in der Butter anbraten, Kartoffeln darüber geben und fest
	andrücken. Bei leichter Hitze braten bis sich ein goldener Boden
	bildet. Wenden, den Käse dazugeben und gut untermischen. Pfanne
	zudecken und bei leichter Hitze ca. 15~Minuten weiterbraten. Auf
	vorgewärmte Platte stürzen. Dazu: Grüner Salat. \\
      \end{zubereitung}

    \mynewsection{Zerzauste Nocken (aus rohen Kartoffeln)}%
              \glossary{Nocken>zerzaust}

      \begin{zutaten}
        300 g & rohe \myindex{Kartoffel}n, gerieben \\
        200 g & \myindex{Mehl}, gesiebt \\
        1 kleines Glas & \myindex{Milch} \\
        1 & \myindex{Ei} \\
        1 Teelöffel & \myindex{Salz} \\
        150 g & geräucherter \myindex{Speck} in Würfeln \\
        2 Eßlöffel & \myindex{Schmalz} zum Braten \\
      \end{zutaten}
      \begin{zutat}{Sauerkraut}
        500 g & \myindex{Sauerkraut} \\
	1 & \myindex{Zwiebel} \\
	1 Teelöffel & \myindex{Zucker} \\
	& \myindex{Salz} (eventuell) \\
	30--50 g & \myindex{Schweineschmalz}\index{Schmalz>Schweine-} \\
	3--4 & \myindex{Wacholderbeeren} \\
	3--4 & \myindex{Pfeffer}körner \\
	1--2 & \myindex{Lorbeer}blätter
      \end{zutat}

      \begin{zubereitung}
        Kartoffeln, Mehl, Ei und Salz mischen. Den Teig lange
	(1--2~Stunden) wegen des Mehls quellen lassen, am Schluß vorsichtig
	Milch einrühren. Der Teig sollte so fest sein, daß man mit dem Eßlöffel
	Nocken abstechen kann. Dann in reichlich Salzwasser 5--6~Minuten
	kochen lassen, mit dem Schaumlöffel rausheben und in eine große
	Schüssel geben. \\
	Sauerkraut: Schmalz auslassen, gewürfelte Zwiebel glasig dünsten,
	Sauerkraut und Gewürze dazu, eventuell etwas Wasser, und 30--40~Minuten
	auf kleiner Flamme garen. Ab und zu umrühren (brennt sonst an) und am
	Schluß abschmecken. \\
	Die halbe Menge Sauerkraut gibt man heiß zu den frisch gekochten
	Nocken. \\
	Speck: In 1--2~Eßlöffel Schmalz braten und in die Schüssel zu
	Sauerkraut und Nocken geben. Alles mischen und heiß servieren. \\
      \end{zubereitung}

    \mynewsection{Kartoffeln im Backofen}

      \begin{zutaten}
        6--8 kleine & \myindex{Kartoffel}n \\
        6--8 & \myindex{Lorbeer}blätter \\
        2--3 Eßlöffel & \myindex{Olivenöl}\index{Oel=Öl>Oliven-} \\
      \end{zutaten}

      \personen{2}

      \garzeit{60}

      \begin{zubereitung}
        Kartoffeln sauber bürsten, seitlich einschneiden und je 1~Lorbeerblatt
	einstecken. In eine Form mit Öl geben. Im Backofen 60~Minuten bei
	\grad{200} garen. Dazu schwarze Oliven geben. \\
        Dazu kurzgebratenes Fleisch und Salat. \\
      \end{zubereitung}

    \mynewsection{Kartoffelauflauf mit Anchovis}

      \begin{zutaten}
	500 g & \myindex{Tomate}n \\
        1--2 & \myindex{Gemüsezwiebel}\index{Zwiebel>Gemüse-}n \\
	1--2 & \myindex{Knoblauchzehe}n \\
	1--2 & \myindex{Anchovis}\index{Fisch>Anchovis} \\
	\breh{} Teelöffel & gehacktes \myindex{Basilikum} \\
	\breh{} Teelöffel & gehackter \myindex{Thymian} \\
        5 Eßlöffel & \myindex{Olivenöl}\index{Oel=Öl>Oliven-} \\
	700 g & \myindex{Kartoffel}n \\
	& \myindex{Salz} \\
	& Öl\index{Oel=Öl} zum Fetten der Form \\
      \end{zutaten}

      \personen{4}

      \begin{zubereitung}
        Tomaten überbrühen, häuten, entkernen und hacken. Zwiebeln und
	Knoblauch fein hacken. \\
	Paste bereiten aus Knoblauch, Basilikum, Thymian, Anchovis und
	2~Eßlöffel Olivenöl. \\
	Kartoffeln schälen, waschen und in dünne Scheiben schneiden. Restliches
	Öl in einer Pfanne erhitzen und die Zwiebeln darin weichdünsten.
	Tomaten dazugeben, salzen und 5--7~Minuten offen schmoren. \\
	Auflaufform fetten. Ein Drittel Tomatensoße, die Hälfte der
	Kartoffelscheiben und darauf die Hälfte der Anchovispaste geben. Alles
	wiederholen. Obenauf mit Tomatensoße abschließen. Backofen auf
	\grad{200} vorheizen und 40~Minuten backen. \\
      \end{zubereitung}

    \mynewsection{Eifeler Döppekooche}\glossary{Döppekooche aus der Eifel}

      \begin{zutaten}
        1 kg & \myindex{Kartoffel}n \\
	2 & \myindex{Ei}er \\
	1 & \myindex{Zwiebel} \\
	250 g & durchwachsenen \myindex{Speck} \\
	1 Teelöffel & \myindex{Salz} \\
	etwas & \myindex{Muskatnuß} \\
	& \myindex{Fett} \\
      \end{zutaten}

      \begin{zubereitung}
        Kartoffeln schälen und reiben. Wasser von der Kartoffelmasse entfernen,
	Eier, Salz, Muskat und die geriebene Zwiebel zufügen und alles gut
	vermengen. Topf gut einfetten, Boden mit durchwachsenen Speckscheiben
	belegen und abwechselnd mit der Kartoffelmasse und den Speckscheiben
	füllen. Obenauf soll Kartoffelmasse sein. Im Backofen muß der
	Döppekooche etwa 2~Stunden bei \grad{200} backen, bis er eine schöne
	braune Kruste hat. \\
	Dazu gibt es mit Butter bestrichenes Schwarzbrot und Apfelmus. \\
      \end{zubereitung}

    \mynewsection{Syr\'es Rheinischer Debbekoche}%
              \glossary{Debbekoche rheinisch}

      \begin{zutaten}
        800 g & rohe, geschälte, mehlig kochende \myindex{Kartoffel}n 
	        besser 1200 g!\footnote{Cilena war gut!} \\
	3 & \myindex{Zwiebel}n \\
	200 g & geräucherter \myindex{Bauchspeck} \\
	4 & grobe Mettwürste\index{Mettwurst} \\
	10 g & \myindex{Walnußkerne} (eventuell) \\
	100 ml & \myindex{Olivenöl}\index{Oel=Öl>Oliven-} \\
	1 & \myindex{Ei} \\
	1 & altbackenes \myindex{Brötchen} \\
	200 ml & \myindex{Milch} \\
	& \myindex{Salz} \\
	& \myindex{Pfeffer} \\
	& \myindex{Muskatnuß} \\
      \end{zutaten}

      \begin{zubereitung}
        Rohe Kartoffeln und Zwiebeln fein reiben, vermischen. Brötchen
	entrinden, in Scheiben schneiden und mit der erhitzten Milch
	übergießen. Durchziehen lassen, verrühren, mit Ei, Salz, Pfeffer,
	Muskat zu den Kartoffeln geben, gut durchrühren. In einem gußeisernen
	Bräter Öl erhitzen, die Hälfte des in Scheiben geschnittenen
	Bauchspecks anbraten. Kartoffelmasse dazufüllen, Mettwürstchen in die
	Masse eindrücken, die restlichen Speckstreifen und die Walnußkerne
	obenauf verteilen. Im Backofen etwa 1~\breh{}--2~Stunden bei
	\grad{200} backen, bis eine braune Kruste entsteht. Mit einem Pieker
	wird die Garprobe gemacht. \\
	Dazu paßt Apfelkompott und als Getränk ein rheinisches Bier oder ein
	Grauburgunder aus Leutesdorf. \\
      \end{zubereitung}

    \mynewsection{Döppekuchen Alfredissimo}

      \begin{zutaten}
        5 kg & \myindex{Kartoffel}n \\
	8 & \myindex{Zwiebel}n \\
	8 & Mettwürste\index{Mettwurst} \\
	250 g & durchwachsener \myindex{Speck} \\
	5 & \myindex{Ei}er \\
	3--4 Eßlöffel & \myindex{Mehl} \\
	3 Eßlöffel & \myindex{Salz} \\
	& \myindex{Pfeffer} frisch gemahlen \\
	\breh{} Teelöffel & \myindex{Muskatnuß} \\
	4 Eßlöffel & \myindex{Olivenöl}\index{Oel=Öl>Oliven-} \\
      \end{zutaten}

      \begin{zubereitung}
        Kartoffeln schälen, waschen, reiben. Ausdrücken und ausgetretenes
	Kartoffelwasser wegschütten. Zwiebeln fein würfeln, Speck würfeln,
	Wurst in Scheiben schneiden. Backofen auf \grad{250} vorheizen.
        Kartoffeln, Zwiebeln, Eier, Mehl, Salz, Pfeffer und Muskatnuß
	gründlich mischen, dann Würstchen und Speck untermengen. Den Boden
	eines Bräters mit Öl bedecken, auf dem Herd erhitzen und den Bräter
	so schwenken, daß auch die Seiten eingefettet sind. Kartoffelteig
	in den Bräter geben, kurz anbraten lassen und dann ohne Deckel im
	vorgeheizten Backofen ca. 2\breh{}~Stunden braten. Sollte die Kruste
	zu schnell braun werden, Deckel auflegen und/oder die Hitze auf
	\grad{200} reduzieren. \\
      \end{zubereitung}

    \mynewsection{Kartoffelgratin mit zweierlei Bohnen}%
              \glossary{Gratin>Kartoffel-}

      \begin{zutaten}
        1 kg & vorwiegend fest kochende \myindex{Kartoffel}n
	       (Quarta, Christa, Granola oder die rosa Laura) \\
        4 & \myindex{rote Zwiebel}\index{Zwiebel>rot}n \\
	2--3 Eßlöffel & \myindex{Olivenöl}\index{Oel=Öl>Oliven-} \\
	8--10 & \myindex{Anchovis}\index{Fisch>Anchovis}filets \\
	250 g & blanchierte \myindex{grüne Bohnen}\index{Bohnen>grün} \\
	2 Tassen & gekochte \myindex{weiße Bohnen}\index{Bohnen>weiß}kerne \\
	1 Glas & \myindex{Brühe} \\
	1 Bund & \myindex{Petersilie} \\
	6--8 & \myindex{Knoblauchzehe}n \\
	3 gehäufte Eßlöffel & \myindex{Semmelbrösel} \\
	& \myindex{Olivenöl}\index{Oel=Öl>Oliven-} \\
	& \myindex{Salz} \\
      \end{zutaten}

      \personen{4}

      \begin{zubereitung}
        Die gekochten Kartoffeln pellen und in Scheiben schneiden. Die Zwiebeln
	und den Knoblauch in dünne Scheiben hobeln und in einer Pfanne in
	heißem Olivenöl andünsten, zum Schluß die gehackten Anchovis
	untermischen. \\
	In einer flachen Gratinform zunächst eine Schicht Kartoffeln
	ausbreiten, darauf gedünstete Zwiebeln, dann grüne sowie weiße Bohnen,
	wieder Kartoffeln etc. Zwischendurch immer wieder salzen. Alles mit
	Brühe tränken. Petersilie fein hacken, mit den Semmelbröseln mischen
	und auf der Oberfläche verteilen. Mit Olivenöl beträufeln und in den
	\grad{200} heißen Ofen schieben. Etwa 20~Minuten backen, bis die Brösel
	gebräunt sind und alles brodelt. \\
	Beilage: Schön ist dazu ein Salat, Endivie zum Beispiel, nicht zu fein
	geschnitten, oder Zuckerhut, mit einer Marinade aus zerdrücktem
	Knoblauch und Anchovis. \\
	Getränk: ein fruchtiger, leichter Rotwein, ein Spätburgunder vom
	Kaiserstuhl oder auch ein frischer Ros\'e aus Südfrankreich. Oder ein
	fruchtiger Marzemino aus dem Trentino, Mozarts Lieblingswein. \\
	Schmeckt auch kalt gut. \\
      \end{zubereitung}

    \mynewsection{Kartoffel-Lauch-Küchlein}

      \begin{zutaten}
        200 g & vorwiegend festkochende \myindex{Kartoffel}n \\
	150 g & \myindex{Lauch} \\
	2 & \myindex{Ei}er \\
	100 g & \myindex{Mehl} \\
	 & \myindex{Salz} \\
	 & \myindex{Pfeffer} \\
	 & \myindex{Muskatnuß} \\
	 & \myindex{Butterschmalz} \\
      \end{zutaten}

      \begin{zubereitung}
        Kartoffeln schälen, reiben und kräftig auspressen (in einem sauberen
	Tuch). Lauch in feine Ringe schneiden und in Butterschmalz dünsten.
	Etwas salzen und zu den Kartoffeln geben, Eier, Mehl und Gewürze
	dazugeben und vermischen. \\
	Kleine Küchlein in die Pfanne mit Butterschmalz geben und goldgelb
	ca. 4~Minuten backen bei kleiner bis mittlerer Hitze (sonst bleiben
	die Kartoffeln roh). \\
      \end{zubereitung}

    \mynewsection{Ofenkartoffeln französisch}

      \begin{zutaten}
        & längliche \myindex{Kartoffel}n \\
	& \myindex{Knoblauch} \\
	& \myindex{\cremefraiche{}} \\
	& \myindex{Salz} \\
	& \myindex{Pfeffer} \\
      \end{zutaten}

      \begin{zubereitung}
        Kartoffeln halbieren (der Länge nach). In der Mitte einschneiden und
	Knoblauch-Scheibe einklemmen. \\
	Schnittfläche mit \cremefraiche{} bestreichen, salzen und pfeffern.
	Kartoffelhälften zusammenfügen und in Alufolie einwickeln. \\
	Für etwa 50--90~Minuten bei \grad{200} im Backofen garen. \\
      \end{zubereitung}

    % \mynewsection{}

      % \begin{zutaten}
      % \end{zutaten}

      % \begin{zubereitung}
      % \end{zubereitung}

