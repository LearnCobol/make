
% created Montag, 10. Dezember 2012 16:23 (C) 2012 by Leander Jedamus
% modifiziert Mittwoch, 22. April 2015 17:56 von Leander Jedamus
% modifiziert Montag, 09. März 2015 14:27 von Leander Jedamus
% modified Montag, 10. Dezember 2012 16:30 by Leander Jedamus

  \mynewchapter{Suppen}

    \mynewsection{Rhabarbersuppe}\glossary{Suppe>Rhabarber-}

      \begin{zutaten}
	500 g & \myindex{Rhabarber} \\
	1 l & \myindex{Wasser} \\
	100 g & \myindex{Zucker} \\
	1 Prise & \myindex{Salz} \\
        1 Päckchen & \myindex{Vanillezucker}\index{Zucker>Vanille-}
	             oder \myindex{Zitrone}nschale \\
	20 g & \myindex{Stärkemehl}\index{Mehl>Stärke-} (2~Eßlöffel) oder \\
	30 g & \myindex{Grieß} \\
      \end{zutaten}

      \garzeit{15--20}

      \begin{zubereitung}
        Rhabarberstange mit Schale waschen, Schale abziehen, Stangen in etwa
	3--4~cm lange Stücke schneiden. Obst mit Wasser und Gewürzen aufsetzen
	und bei kleiner Flamme gar kochen (nicht ganz zerfallen lassen). Die
	Suppe wird mit kalt angerührtem Stärkemehl oder trocken eingestreutem
	Grieß verdickt und mit Zucker abgeschmeckt. Als Einlage kann man
	Grießklößchen nehmen. Statt Rhabarber kann man auch anderes Obst nehmen,
	auch Holunderbeeren. Und man kann auch einen Schlag Sahne auf die
	abgekühlte oder kalte Suppe geben oder vorsichtig ,,marmorieren``. \\
      \end{zubereitung}

    \mynewsection{Gemüseeintopf}

      \begin{zutaten}
        1 Eßlöffel & \myindex{Margarine} \\
        1 kg & \myindex{Wirsing} \\
        500 g & \myindex{Kartoffel}n \\
        250 g & \myindex{Gehacktes} (Rind + Schwein) \\
        1 & \myindex{Brötchen} \\
        1 & \myindex{Zwiebel} \\
        1 & \myindex{Ei} \\
        & \myindex{Salz} \\
        & \myindex{Paprika} \\
        & \myindex{Paniermehl} \\
        & \myindex{Fondorwürfel} \\
        \brea{}--\brev{} l & \myindex{Wasser} \\
      \end{zutaten}

      \garzeit{30--45}

      \begin{zubereitung}
        Aus dem Gehackten, dem eingeweichten und gut ausgedrückten Brötchen,
	dem Ei und Salz, der Zwiebel und dem Paprika bereitet man Fleischklöße.
	Man zerläßt die Margarine, legt Kohl, Kartoffeln und Klöße ein, gibt
	dazwischen etwas Salz und Fondor und gießt das kochende Wasser darüber.
	Man läßt das Gericht bei kleiner Flamme gar kochen. \\
      \end{zubereitung}

    \mynewsection{Selleriecremesuppe}\glossary{Suppe>Selleriecreme-}

      \begin{zutaten}
        ca. 500 g & \myindex{Knollensellerie}\index{Sellerie>Knollen-} \\
        1 & \myindex{Zwiebel} \\
        2--3 & \myindex{Knoblauchzehe}n \\
        1 & kleine \myindex{Chilischote} (frisch oder getrocknet) \\
        2 Eßlöffel & \myindex{Butter} \\
        1--2 & \myindex{Kartoffel}n ca. 200 g \\
        & \myindex{Salz} \\
        & \myindex{Pfeffer} \\
        \brdv{} l & \myindex{Wasser} oder \myindex{Gemüsebrühe} \\
        100 g & \myindex{Sahne} \\
        & \myindex{Zitrone}nsaft \\
        & \myindex{Worcestershiresoße} \\
        & \myindex{Muskatnuß} \\
        1 Messerspitze & \myindex{Cayennepfeffer}\index{Pfeffer>Cayenne-} \\
      \end{zutaten}
      \begin{zutat}{Füllung}
        8--12 & \myindex{Wan-Tan-Hüllen} \\
        1 & \myindex{Eiweiß} \\
        100 g & \myindex{Blutwurst} (aus dem Glas oder angeräuchert am Stück) \\
        & \myindex{Petersilie} \\
      \end{zutat}
      \begin{zutat}{Paprikaöl}
        1 Eßlöffel & milder \myindex{Delikatesspaprika} \\
        3--4 Eßlöffel & \myindex{Olivenöl}\index{Oel=Öl>Oliven-} \\
      \end{zutat}

      \personen{4--6}

      \garzeit{25}

      \begin{zubereitung}
        Eine Knolle Sellerie schälen, in Scheiben und dann in Würfel schneiden.
	Zwiebel und Kartoffeln würfeln. Die Sellieriewürfel in Butter andünsten
	und Zwiebeln, Knoblauch sowie Chilischote hinzugeben. Danach die grob
	gewürfelten Kartoffeln dazugeben. Mit Wasser oder Brühe auffüllen,
	salzen und pfeffern. Zugedeckt ca. 25~Minuten ganz weich kochen. Die
	Masse passieren oder mit dem Passierstab bearbeiten, Sahne angießen und
	leise köcheln lassen. Abschmecken mit Zitronensaft und einem Spritzer
	Worcestershiresoße, Muskat, Cayenne. \\
        Für die Einlage Wan-Tan-Hüllen vorsichtig lösen, vollflächig mit Eiweiß
	einpinseln und mit gehackter Blutwurst und Petersilie füllen.
	Teighüllen zum Dreieck zusammenklappen, ringsum die Ränder gut
	zusammendrücken. Die Teigtaschen auf ein Glasbrett legen und mit Folie
	abdecken, in den Kühlschrank legen. Vor dem Servieren in Salzwasser
	1--2~Minuten garziehen lassen, dann in die Suppe geben. Man kann die
	Einlage auch weglassen. \\
        Für das Paprikaöl das Pulver in heißem Olivenöl kurz ziehen lassen.
	Kann man im Kühlschrank im Schraubglas aufbewahren. \\
        Das Paprikaöl gibt man auf jeden einzelnen Suppenteller (Kleckse,
	Linien). Kann man auch weglassen. \\
        Restlichen Sellerie kann man auch fertig gewürfelt einfrieren und bei
	Bedarf entnehmen. \\
        gekocht Weihnachten 2005: Wan-Tan-Taschen kann man gut weglassen. \\
      \end{zubereitung}

    \mynewsection{Lauchcremesuppe Excelsior}

      \begin{zutaten}
        1 & \myindex{Zwiebel} \\
	20 g & \myindex{Butter} \\
	500 g & \myindex{Lauch} \\
	große & \myindex{Kartoffel}n \\
	& \myindex{Salz} \\
	\brda{} l & \myindex{Hühnerbrühe} \\
	\brev{} l & \myindex{Sahne} \\
	\brev{} l & \myindex{Milch} \\
	& \myindex{schwarzer Pfeffer}\index{Pfeffer>schwarz} \\
	& \myindex{Tabasco} \\
	\brea{} l & \myindex{Sahne} \\
	& \myindex{Schnittlauch} \\
      \end{zutaten}

      \personen{4}

      \begin{zubereitung}
        Eine gehackte Zwiebel in Butter gelb braten. Lauch ohne dunkles Grün
	in Streifen und große Kartoffeln in Würfeln zugeben. Salzen.
	Hühnerbrühe zugießen. 35~Minuten kochen. Im Mixer pürieren. Mit Sahne
	und Milch erhitzen. Durchs Sieb passieren. Mit schwarzem Pfeffer und
	Tabasco abschmecken. Ausgekühlt mit halbsteif geschlagener Sahne und
	Schnittlauch servieren. \\
      \end{zubereitung}

    \mynewsection{Französische Basilikumsuppe (Soupe au pistou)}

      \begin{zutaten}
        600 g & \myindex{grüne Bohnen}\index{Bohnen>grün} \\
        300 g & frische \myindex{weiße Bohnen}\index{Bohnen>weiß}kerne \\
	6 große & \myindex{Kartoffel}n \\
	8 & \myindex{Tomate}n \\
	1 & \myindex{Zwiebel} \\
        6 Eßlöffel & \myindex{Olivenöl}\index{Oel=Öl>Oliven-} \\
	2\breh{} l & \myindex{Rindfleischbrühe} (Würfel) \\
	& \myindex{Salz} \\
	& frisch gemahlener \myindex{Pfeffer} \\
        300 g & \myindex{rote Bohnen}\index{Bohnen>rot} (Dose) \\
	1 Bund & frisches \myindex{Basilikum} (oder 2 Teelöffel getrocknetes)
	         \\
	4 & \myindex{Knoblauchzehe}n \\
      \end{zutaten}

      \personen{4}
      \kalorien{430}

      \begin{zubereitung}
        Die grünen Bohnen putzen und einmal durchbrechen. Die weißen
	Bohnenkerne abspülen, Kartoffeln schälen und in Würfel schneiden.
	Tomaten mit kochendem Wasser überbrühen und die Haut abziehen. Das
	Fruchtfleisch würfeln. Die Zwiebel fein würfeln und in 4~Eßlöffel
	Olivenöl andünsten. Grüne Bohnen, Bohnenkerne, Kartoffeln und Tomaten
	kurz darin andünsten. Die Fleischbrühe zugießen, Mit Salz und Pfeffer
	würzen und die Bohnensuppe bei schwacher Hitze 35~Minuten kochen.
	Dann die abgetropften roten Bohnen zugeben. Noch einmal würzen und
	weitere 10~Minuten ziehen lassen. In der Zwischenzeit das Basilikum
	grob hacken und die Knoblauchzehen zerdrücken. Beides mit dem
	restlichen Olivenöl mischen und auf die Suppe geben. \\
      \end{zubereitung}

    \mynewsection{Italienische Gemüsesuppe (Minestrone di verdura)}

      \begin{zutaten}
        200 g & frische \myindex{weiße Bohnen}\index{Bohnen>weiß}kerne \\
        1 & \myindex{Staudensellerie}\index{Sellerie>Stauden-} \\
	250 g & \myindex{Möhre}n \\
	3 & \myindex{Zucchini} \\
	\breh{} Kopf & \myindex{Wirsing}kohl (350~g) \\
        300 g & \myindex{grüne Bohnen}\index{Bohnen>grün} \\
	500 g & \myindex{Erbsen} in der Schote \\
	1 kleine & \myindex{Fenchel}knolle \\
	400 g & \myindex{Tomate}n \\
	150 g & durchwachsener \myindex{Speck} \\
        4 Eßlöffel & \myindex{Olivenöl}\index{Oel=Öl>Oliven-} \\
	2 & \myindex{Zwiebel}n \\
	2 & \myindex{Knoblauchzehe}n \\
	1 Zweig & frisches \myindex{Basilikum} \\
	1 Zweig & frischer \myindex{Salbei} (oder je \breh{} Teelöffel
	          getrocknete Kräuter) \\
	3--3\breh{} l & \myindex{Hühnerbrühe} (Würfel) \\
	200 g & \myindex{Spaghetti} \\
	& \myindex{Salz} \\
	& frisch gemahlener \myindex{Pfeffer} \\
        1 Bund & \myindex{glatte Petersilie}\index{Petersilie>glatt} \\
      \end{zutaten}

      \personen{6}
      \kalorien{585}

      \begin{zubereitung}
        Die Bohnenkerne waschen, Sellerie putzen, waschen, die Stengel in
	Stücke schneiden. Möhren schälen und in Scheiben schneiden. Zucchini
	in Scheiben schneiden. Kohl putzen, waschen und in grobe Stücke
	schneiden. Bohnen putzen, waschen und zweimal durchbrechen. Die Erbsen
	auspalen und die geputzte Fenchelknolle in Würfel schneiden. Die
	Tomaten mit kochendem Wasser überbrühen, die Haut abziehen und die
	Tomaten in Achtel schneiden. Den Speck fein würfeln und im Olivenöl
	ausbraten. Zwiebelwürfel, zerdrückte Knoblauchzehen, gehacktes
	Basilikum und gehackten Salbei darin andünsten. Dann weiße Bohnen,
	Sellerie, Möhren, Zucchini, Wirsingkohl, grüne Bohnen und Fenchel
	zugeben. Die Hühnerbrühe zugießen (das Gemüse muß gut bedeckt sein).
	30~Minuten kochen. Dann die Erbsen zugeben und noch weitere 25~Minuten
	bei schwacher Hitze garen. In den letzten 10~Minuten die in kleine
	Stücke gebrochenen Spaghetti zugeben. Kräftig mit Salz und Pfeffer
	würzen. Die Tomaten in den letzten 5~Minuten mit erhitzen. Die Suppe
	mit gehackter Petersilie bestreut servieren. \\
	Dazu: frisch geriebener Parmesankäse. \\
      \end{zubereitung}

    \mynewsection{Griechische Fischsuppe (Kakavia)}

      \begin{zutaten}
	500 g & frische \myindex{Sardinen}\index{Fisch>Sardinen} (oder kleine
	        \myindex{grüne Heringe}\index{Hering>grün}%
		\index{Fisch>Hering>grün}) \\
	& \myindex{Salz} \\
	2 & \myindex{Möhre}n \\
        1 & \myindex{Staudensellerie}\index{Sellerie>Stauden-} \\
	1 & \myindex{Fenchel}knolle \\
        2 & \myindex{Gewürznelken}\index{Nelke>Gewürz-} \\
	3 & \myindex{Zwiebel}n \\
	2 & \myindex{Lorbeer}blätter \\
	250 g & \myindex{Tomate}n \\
	2 Tütchen & \myindex{Safranfäden} \\
	6 Eßlöffel & Öl\index{Oel=Öl} \\
	2 & \myindex{Knoblauchzehe}n \\
        \brea{} l & herber \myindex{Weißwein}\index{Wein>weiß} \\
	1 kg & \myindex{Kabeljau}\index{Fisch>Kabeljau}koteletts \\
	6 & \myindex{Langusten}\index{Fisch>Langusten}schwänze \\
	250 g & \myindex{Garnelen}\index{Fisch>Garnelen} \\
	& frisch gemahlener \myindex{Pfeffer} \\
        & \myindex{Cayennepfeffer}\index{Pfeffer>Cayenne-} \\
	\breh{} & \myindex{Zitrone} ausgepreßt \\
      \end{zutaten}

      \personen{6}
      \kalorien{410}

      \begin{zubereitung}
        Die Sardinen oder Heringe putzen, waschen und mit Salz einreiben. Die
	Möhren schälen und in Scheiben schneiden. Den Sellerie putzen und in
	grobe Stücke, die geputzte Fenchelknolle in Würfel schneiden. Sardinen
	oder Heringe mit Gemüse und einer mit Nelken gespickten Zwiebel und
	den Lorbeerblättern 45~Minuten in 1\breh{}~l leicht gesalzenem Wasser
	garen. Dann die Brühe mit Fisch und Gemüse durch ein feines Sieb
	rühren. Die Tomaten mit kochendem Wasser überbrühen und die Haut
	abziehen. Die Tomaten halbieren, entkernen und das Fruchtfleisch
	würfeln. Safranfäden in einer halben Tasse Wasser aufkochen und durch
	ein Sieb gießen. Die restlichen Zwiebeln fein würfeln. Öl erhitzen.
	Zwiebeln, zerdrückten Knoblauch und Tomaten darin so lange dünsten, bis
	eine dickliche Soße enstanden ist. Dann Wein, Fisch- und Safranbrühe
	zugießen. Aufkochen und den gewaschenen, abgetrockneten Kabeljau darin
	25~Minuten gar ziehen lassen. Nach 10~Minuten die Langustenschwänze
	und nach weiteren 10~Minuten die Garnelen zugeben. Die Fischsuppe mit
	Salz, Pfeffer, Cayennepfeffer und Zitronensaft abschmecken. \\
      \end{zubereitung}

    \mynewsection{Kartoffelsuppe}\glossary{Suppe>Kartoffel-}

      \begin{zutaten}
        1 Bund & \myindex{Suppengrün} (Möhre, Porree, Sellerie, Petersilie,
	                               Zwiebel) \\
        750 g & \myindex{Kartoffel}n \\
        1 & \myindex{Zwiebel} \\
        1 Eßlöffel & \myindex{Butter} \\
        1 l & \myindex{Brühe} vom Schinkenknochen oder Schwarte gekocht \\
        & \myindex{Pfeffer}, \myindex{Salz}, \myindex{Majoran},
	  \myindex{Petersilie} \\
      \end{zutaten}

      \garzeit{30}

      \begin{zubereitung}
        Suppengrün waschen und putzen, würfeln, in Fett anrösten. Geschälte
	Kartoffeln in Stücken dazugeben und mitrösten. Mit heißer Brühe
	auffüllen und weich kochen, umrühren. Würzen mit Salz, Pfeffer, Majoran
	und Petersilie. Eventuell einen Schuß Sahne dazugeben. Servieren mit
	Brühwürstchen. \\
      \end{zubereitung}

    \mynewsection{Kartoffel-Lauch-Suppe}\glossary{Suppe>Kartoffel-Lauch-}

      \begin{zutaten}
        500 g & \myindex{Lauch} \\
        500 g & \myindex{Kartoffel}n \\
        100 g & \myindex{Speck} \\
        \brea{} l & \myindex{Weißwein}\index{Wein>weiß} \\
        & \myindex{Instantsuppe} \\
        & \myindex{Muskatnuß} \\
        & \myindex{Oregano} \\
        & \myindex{Pfeffer} \\
        & \myindex{\cremefraiche{}} \\
      \end{zutaten}

      \begin{zubereitung}
        Lauch fein schneiden, Kartoffeln in kleine Würfel schneiden, Speck fein
	hacken. Speck in ca. 3~Eßlöffel Öl gut anbraten und herausnehmen. Lauch
	und Kartoffeln in das Fett geben und gut durchbraten, mit ca.
	\brdv{}~Liter Brühe aufgießen und kochen, bis Lauch und Kartoffeln
	weich sind. Etwa \bred{}~Liter der Kartoffel-Lauch-Masse pürieren und
	wieder zum Rest geben. Mit Muskat, Pfeffer und Oregano abschmecken,
	Speckwürfel wiederzugeben. Weißwein mit etwas \cremefraiche{}
	verrühren und kurz vor dem Servieren zugeben. Mit Baguette servieren.
	\\
      \end{zubereitung}

    \mynewsection{\chicoree{}-Cremesuppe}\glossary{Suppe>\chicoree{}-}

      \begin{zutaten}
        500 g & \myindex{\chicoree{}} \\
        30 g & \myindex{Butter} \\
        50 g & \myindex{Speck}würfel \\
        1 & \myindex{Zwiebel} gewürfelt \\
        1 mittlere & \myindex{Kartoffel} fein gewürfelt \\
        1\breh{} l & \myindex{Brühe} \\
        1 & \myindex{Eigelb} \\
        50 ml & \myindex{süße Sahne}\index{Sahne>süß} \\
        & \myindex{Muskatnuß} \\
        & \myindex{Petersilie} \\
        & \myindex{Jodsalz}\index{Salz>Jod-} \\
        & \myindex{weißer Pfeffer}\index{Pfeffer>weiß}
      \end{zutaten}

      \garzeit{20--30}

      \begin{zubereitung}
        \chicoree{} in feine Streifen schneiden, davon eine Handvoll
	zurücklassen. Butter zerlassen und Speck- und Zwiebelwürfel darin
	anbräunen, \chicoree{} und fein gewürfelte Kartoffeln hinzugeben. Mit
	Brühe auffüllen, 20--30~Minuten leicht kochen lassen. Abschmecken mit
	Salz, Pfeffer und Muskat. Legieren mit Eigelb. Bestreuen mit
	zurückgelassenem \chicoree{} und gehackter Petersilie. \\
        Dazu: Getoastetes Weißbrot oder gebackenes Baguette. \\
      \end{zubereitung}

    \mynewsection{Zucchinisuppe}\glossary{Suppe>Zucchini-}

      \begin{zutaten}
        500 g & \myindex{Zucchini} \\
        1 Bund & \myindex{Petersilie} \\
        1 Bund & \myindex{Basilikum} \\
        2 Eßlöffel & \myindex{Schweineschmalz} \\
        1 Eßlöffel & \myindex{Olivenöl}\index{Oel=Öl>Oliven-} \\
        1 l & heißes \myindex{Wasser} \\
        & \myindex{Salz} \\
        & \myindex{Pfeffer} \\
        4 & \myindex{Weißbrot}scheiben \\
        1 Eßlöffel & \myindex{Butter} \\
        2 & \myindex{Ei}er \\
        3 Eßlöffel & \myindex{Parmesan}\index{Käse>Parmesan}käse, gerieben \\
      \end{zutaten}

      \garzeit{30}

      \begin{zubereitung}
        Zucchini waschen, Enden abschneiden und würfeln. Kräuter fein hacken.
	In einem Topf Schmalz und öl erhitzen, das Gemüse 10~Minuten bei milder
	Hitze anbraten. Mit heißem Wasser aufgießen. Gut abschmecken mit Salz
	und Pfeffer und zugedeckt langsam 20~Minuten kochen lassen. \\
        Weißbrotscheiben in kleine Würfel schneiden und in Butter goldgelb
	rösten. Eier mit Kräutern und Käse gut verquirlen, kurz vor dem
	Anrichten unter die Suppe rühren. Brotwürfel auf vorgewärmte Teller
	geben und heiße Suppe darüber geben. \\
      \end{zubereitung}

    \mynewsection{Suppe mit Nudeln und Bohnen (Pasta e Fagioli)}

      \begin{zutaten}
        125 g kleine & getrocknete \myindex{rote Bohnen}\index{Bohnen>rot} \\
        125 g kleine & getrocknete \myindex{weiße Bohnen}\index{Bohnen>weiß} \\
	2 l & \myindex{Wasser} \\
	1 kleine & \myindex{Sellerie}knolle mit Grün (ca. 200 g) \\
	125 g & \myindex{Möhre}n \\
	1 & \myindex{Knoblauchzehe} \\
	2 Teelöffel & \myindex{Salz} \\
	& \myindex{Pfeffer} \\
      \end{zutaten}
      \begin{zutat}{Außerdem}
        125 g & \myindex{Nudeln} \\
	150 g & geräucherter durchwachsener \myindex{Speck} \\
	1 Bund & \myindex{glatte Petersilie}\index{Petersilie>glatt} \\
      \end{zutat}
      
      \personen{4}
      \kalorien{500}

      \begin{zubereitung}
        Bohnen über Nacht im Wasser einweichen. Wasser abgießen und die Bohnen
	mit 2~l Wasser zum Kochen bringen. Zirka 1~Stunde kochen.
	Sellerieknolle schälen und waschen. Das Grün feinschneiden. Möhren
	schälen, in Scheiben schneiden. Knoblauchzehe schälen und durchpressen.
	Alles nach 1~Stunde zufügen, pfeffern und eine weitere Stunde garen.
	Salzen. Nudeln zufügen und nach Packungsanweisung mitkochen. Speck in
	Streifen schneiden, anbraten und mit gehackter Petersilie zur Suppe
	geben. \\
      \end{zubereitung}

    \mynewsection{Italienische Gemüsesuppe II (Minestrone di verdura)}

      % Italienische Küche -mit Pfiff! S. 24-25

      \begin{zutaten}
        150 g & weiße oder rote \myindex{Trockenbohnen} \\
	100 g & durchwachsener \myindex{Räucherspeck} oder frischer
	        \myindex{Bauchspeck} \\
	2 Stengel & \myindex{Bleichsellerie} \\
	2 & \myindex{Möhre}n \\
	2 kleine & \myindex{Zucchini} \\
	3 & mehlige \myindex{Kartoffel}n \\
	1 mittelgroße & \myindex{Zwiebel} \\
	250 g & \myindex{Tomate}n \\
	5 Stengel & \myindex{Petersilie} \\
	1 & \myindex{Knoblauch}zehe \\
	\brev{} Kopf & \myindex{Wirsing}kohl \\
	einige Blättchen & \myindex{Basilikum} \\
	einige Blättchen & \myindex{Salbei} \\
	1 Handvoll & kleine zarte \myindex{Brechbohnen} \\
	300 g & frische junge Erbsen oder \\
	1 & \myindex{Fenchel}knolle \\
	4 Eßlöffel & \myindex{Olivenöl}\index{Oel=Öl>Oliven-} \\
	& \myindex{Salz} \\
	150 g & \myindex{Spaghetti}, \myindex{Hörnchennudeln} oder
	        \myindex{Reis} \\
	& \myindex{Parmesan}\index{Käse>Parmesan}käse zum Bestreuen \\
      \end{zutaten}

      \begin{zubereitung}
        In Italien werden in den Sommermonaten frische weiße Bohnenkerne
	verwendet. Von diesen frischen Bohnenkernen braucht man 400~Gramm. Sie
	werden, im Gegensatz zu den Trockenbohnen, nicht eingeweicht und nicht
	vorgekocht. Die Trockenbohnen werden am Vorabend in kaltem Wasser
	eingeweicht und am nächsten Tag in leicht gesalzenem Wasser gargekocht.
	Der Speck wird in Würfel geschnitten, Bleichsellerie und geputzte
	Möhren in Scheibchen. Die Zucchini, die geschälten Kartoffeln und die
	Zwiebel schneidet man in Würfelchen. Die Tomaten werden gebrüht,
	abgezogen und ohne den harten Kern in Stückchen geschnitten. Petersilie
	und Knoblauchzehe hackt man fein, der Wirsingkohl wird in Streifen
	geschnitten, große Basilikumblätter ebenfalls. Die Brechbohnen werden
	ein- bis zweimal gebrochen, Erbsen enthülst, Fenchel feinstreifig
	aufgeschnitten. \\
	Das Olivenöl mit den Speckwürfeln, Zwiebeln, gehacktem Knoblauch und
	Petersilie erhitzen, Basilikum und Salbei hinzufügen und 5~Minuten
	rösten lassen. Dann das vorbereitete Gemüse (ohne Tomaten, grüne Bohnen
	und Erbsen) in das Öl geben, mit 2\breh{}~Liter Wasser aufgießen und
	salzen. Die gekochten Bohnen hinzufügen und zugedeckt 1\breh{}~Stunden
	kochen lassen. Die Tomatenstückchen und die Brechbohnen hinzufügen und
	eine weitere \breh{}~Stunde kochen lassen. Dann erst Nudeln oder Reis
	an die Suppe geben sowie die jungen Erbschen. Nach 20~Minuten Kochzeit
	mit Salz abschmecken. Wenn die Suppe in die Teller gefüllt ist, mit
	geriebenem Parmesankäse bestreuen. \\
	Je nach Region wird statt Speck oder zusätzlich zum Speck noch Schwarte
	oder ein Stückchen Schinken mitgekocht. Die Gemüse wechseln nach der
	Jahreszeit. Oft gehört auch ein scharfer Peperoncino --- die kleine
	scharfe rote Pfefferschote --- an den Minestrone. Die Einlage besteht
	im Norden Italiens häufig aus Reis, im Trentino und Friaul aus Graupen,
	im übrigen Titalien meistens aus Hörnchennudeln oder aus Spaghetti. \\
	Der Minestrone wird in Italien häufig noch wesentlich länger gekocht,
	die hier angegebene Kochzeit reicht aber völlig aus, um den Geschmack
	von Gemüse und Kräutern voll zur Entfaltung zu bringen. \\
      \end{zubereitung}

    \mynewsection{gelbe Paprikasuppe}

      % 15.07.2010 gekocht. Lecker.

      \begin{zutaten}
        1 kg & gelbe \myindex{Paprika} \\
	5--7 & \myindex{Schalotte}n \\
	etwa 120 g & \myindex{Butter} \\
	3 & gepreßte \myindex{Knoblauchzehe}n \\
	5--8 mittlere & \myindex{Kartoffel}n \\
	& \myindex{Salz} \\
	& \myindex{Pfeffer} aus der Mühle \\
	& \myindex{Gemüsebrühe} \\
	1 l & \myindex{Wasser} (für die Paprika) \\
      \end{zutaten}

      \bemerkung{ergibt ca. 8--10~Teller}

      \begin{zubereitung}
        Paprika waschen, Weißes entfernen, in Würfel schneiden. Schalotten in
	feine Ringe schneiden. In Butter zunächst die Schalotten weich dünsten,
	dann die Paprika dazu. Öfter wenden, bis die Paprika nicht mehr roh
	aussehen. Gewürze, Knoblauch und Wasser darauf. Etwa 20~Minuten
	köcheln lassen. \\
	Kartoffeln schälen und in kleine Würfel schneiden. In einem gesonderten
	Topf in reichlich Salzwasser weich kochen. Kartoffeln zu den Paprika
	geben und pürieren. Abschmecken. \\
	Kalt servieren. \\
      \end{zubereitung}

    \mynewsection{Kürbissuppe}

      % Kaffee oder Tee SWR 13.09.2010

      \begin{zutaten}
        1 kleiner & Hokkaido-\myindex{Kürbis} \\
	1 & \myindex{Zwiebel} gewürfelt \\
	1 Stück & \myindex{Ingwer} \\
	1--2 Zehen & \myindex{Knoblauch} \\
	& Öl\index{Oel=Öl} zum Dünsten \\
	& \myindex{Salz} \\
	& \myindex{Pfeffer} aus der Mühle \\
	& \myindex{Chiliflocken} aus der Mühle (oder 1~frische Schote) \\
	etwas & \myindex{Zitrone}nabrieb \\
	& \myindex{Sojasoße} \\
	& \myindex{Fischsoße} \\
	1 Prise & \myindex{Zucker} \\
	etwas & \myindex{Balsamico-Essig}\index{Essig>Balsamico-} \\
      \end{zutaten}

      \begin{zutat}{geschäumte Milch}
        1 Tasse & \myindex{Milch} \\
	1 Schuß & \myindex{Sahne} \\
	& geriebener \myindex{Ingwer} (wenig) \\
      \end{zutat}

      \begin{zubereitung}
        Kürbis mit Schale durchschneiden. Hokkaido hat sehr festes Fleisch.
	Kerne mit einem Eßlöffel ausräumen, Stiel und Krutzen abschneiden.
	Dann in Stücke schneiden. \\
	Öl erhitzen, gewürfelte Zwiebel, Ingwer (zerkleinert) und zerkleinerten
	Knoblauch angehen lassen, aber nicht bräunen. Kürbisstücke dazugeben
	und knapp mit Wasser bedecken. Würzen mit Salz, Pfeffer, Chiliflocken.
	Will man eine frische Chilischote nehmen, kocht man sie
	\underline{nicht} mit, sondern gibt sie kurz vor Garende dazu. \\
	20--25~Minuten weichkochen lassen. Pürieren und mit wenig
	Zitronenabrieb, Sojasoße, Fischsoße, Prise Zucker, etwas Balsamico-Essig
	abschmecken. Eventuell einen Schuß Sahne dazugeben, wenn man die
	geschäumte Milch weglassen will. \\
	Geschäumte Milch mit Schuß Sahne und geriebenem Ingwer aufkochen, dann
	mit Pürierstab nur knapp unter die Oberfläche gehen, um dem Schaum zu
	erzeugen. Dann löffelweise auf die Kürbissuppe geben. \\
      \end{zubereitung}

    \mynewsection{Cremesüppchen (mit Brokkoli-Blanchierwasser)}

      \begin{zutaten}
        2 & \myindex{Zwiebel}n, gewürfelt \\
	40 g & \myindex{Butter} \\
	40 g & \myindex{Mehl} \\
	30 g & geriebener \myindex{Gouda}\index{Käse>Gouda} \\
	ca. \breh{} l & Blanchierwasser vom Brokkoli \\
	& \myindex{schwarzer Pfeffer}\index{Pfeffer>schwarz} (aus der Mühle) \\
	& \myindex{Muskatnuß}, gerieben \\
	& \myindex{Cayennepfeffer}\index{Pfeffer>Cayenne-} \\
        & \myindex{Worcestershiresoße} \\
      \end{zutaten}

      \begin{zubereitung}
        Butter langsam zerlassen, Zwiebeln glasig dünsten, Mehl dazu. Alles gut
	vermischen, so daß Butter und Mehl zusammen sind. Dann aufgießen mit
	Blanchierwasser und ca. 20~Minuten köcheln lassen, Käse reingeben.
	Öfter mit dem Schneebesen rühren, eventuell Milch oder Wasser
	nachgießen, würzen. Salz braucht man meist nicht, da das
	Blanchierwasser salzig ist. Abschmecken. \\
      \end{zubereitung}

    \mynewsection{Kartoffelsuppe mit Würstchen}

      \begin{zutaten}
        ca. 1 kg & mehlig kochende \myindex{Kartoffel}n \\
	2 mittlere & \myindex{Möhren} in Würfeln \\
	\breh{} Stange & \myindex{Lauch} in Ringen \\
	1 & TK-Packung gefrorenes \myindex{Suppengemüse} (Möhren, Lauch,
	    Sellerie) \\
	& \myindex{Salz} \\
	& \myindex{weißer Pfeffer}\index{Pfeffer>weiß} \\
	& \myindex{Gemüsebrühe} \\
	1 Teelöffel & getrockneter \myindex{Majoran} \\
	3--5 Zweige & frischer \myindex{Majoran} \\
	\breh{} Bund & \myindex{Petersilie} \\
	1 Eßlöffel & \myindex{Schweineschmalz}\index{Schmalz>Schweine-} \\
	1 dickere Scheibe & geräucherter, durchwachsener \myindex{Speck} in
	  feinen Streifen \\
      \end{zutaten}

      \begin{zubereitung}
        Kartoffeln schälen, würfeln. Topf erhitzen, Schmalz rein und auf kleiner
	Stufe den Speck auslassen. Rausnehmen und Möhren und Lauch anrösten.
	Dann TK-Suppengemüse (in Scheiben getrennt) rein, andünsten. Speck
	und Schwarte wieder rein. \\
	Kartoffelwürfel, 1 Teelöffel Gemüsebrühe, Salz, Pfeffer und getrockneter	Majoran rein. 20~Minuten kochen, bis die Kartoffeln weich sind. 1~Kelle
	Kartoffeln rausheben, Schwarte rausnehmen, einen Pürierstab nehmen und
	im Topf pürieren. Abschmecken. \\
	Geschnittene Würstchen rein und die Kelle gewürfelter Kartoffeln auf
	kleiner Flamme ziehen lassen für ca. 10~Minuten. \\
	Frische Kräuter abspülen, trocken schütteln und hacken. Majoran rein,
	die Petersilie erst kurz vor dem Servieren. Abschmecken mit Zucker,
	Salz, Gemüsebrühe und Worcestershiresoße. \\
	Wenn sehr viele Personen satt werden müssen, gibt es noch Pfannkuchen
	danach. Eventuell mit Obstkompott oder Apfelmus. \\
      \end{zubereitung}

    \mynewsection{Kürbiscremesuppe}
      % WDR 16.12.2011 Servicezeit Weihnachtsmenü Martina & Moritz

      \begin{zutaten}
        800 g & \myindex{Kürbis}fleisch (ohne Schale und Kerne gewogen) \\
        1 & gehackte \myindex{Zwiebel} \\
        2 & durchgepreßte \myindex{Knoblauch}zehen \\
        2 Eßlöffel & \myindex{Butter} \\
        300 ml & \myindex{Brühe} \\
        \breh{} Teelöffel & \myindex{Kümmel} \\
        1 Teelöffel & getrockneter \myindex{Majoran} \\
        & \myindex{Salz} \\
        & \myindex{Pfeffer} \\
        200 g & \myindex{Sahne} \\
        1 Spritzer & \myindex{Essig} und/oder \myindex{Zitrone}nsaft \\
      \end{zutaten}

      \begin{zutat}{Außerdem}
        & \myindex{Kürbiskernöl}\index{Oel=Öl>Kürbiskern-} \\
	& \myindex{Aceto Balsamico}, beides für Kleckse \\
      \end{zutat}

      \personen{6}

      \begin{zubereitung}
	Kürbisfleisch würfeln, mit der Zwiebel und dem Knoblauch in heißer
	Butter andünsten. Mit Brühe auffüllen. Kümmel, Majoran, Salz und
	Pfeffer hinzufügen. \\
	Zugedeckt eine \breh{}~Stunde köcheln lassen. Sahne angießen, alles
	mit dem Pürierstab zerkleinern, noch 2~Minuten leise köcheln lassen und
	erneut abschmecken. Vor dem Servieren kurz aufmixen, damit die Suppe
	luftig wird und mit Essig/Zitrone abschmecken. \\
	Dazu frisches Baguette. \\
      \end{zubereitung}

    \mynewsection{Rote-Beete-Cremesuppe}
      % WDR 16.12.2011 Servicezeit Weihnachtsmenü Martina & Moritz

      \begin{zutaten}
        500 g & \myindex{Rote Beete}, geschält und gewürfelt \\
        2 & \myindex{Schalotte}n, gehackt \\
        1 kleine & \myindex{Chilischote} \\
        \breh{} l & \myindex{Gemüsebrühe} \\
        1 kleine & \myindex{Kartoffel} zum Binden, zerkleinert \\
        & \myindex{Salz} \\
        & \myindex{Pfeffer} \\
        2 & zerdrückte \myindex{Piment}körner \\
        \breh{} Teelöffel & \myindex{Senfsamen} \\
        1--2 Eßlöffel & \myindex{Essig} zum Abschmecken \\
	150 ml & \myindex{Sahne} \\
      \end{zutaten}

      \begin{zutat}{Außerdem}
        & \myindex{Meerrettich} zum Garnieren \\
	& \myindex{Aceto Balsamico} \\
	& \myindex{Olivenöl}\index{Oel=Öl>Oliven-}, beides für Kleckse \\
      \end{zutat}

      \personen{6}

      \begin{zubereitung}
        Rote Beete mit den Schalotten, den Kartoffelstücken und den
	Gewürzen (Chilischote, Salz, Pfeffer, Pimentkörner, Senfsamen) in einem
	Teebeutel in der Brühe 30~Minuten weich kochen.  Mit dem Pürierstab
	glatt mixen, dabei die Sahne angießen. \\
	Sehr kräftig abschmecken. \\
      \end{zubereitung}

    % \mynewsection{Text}

      % \begin{zutaten}
      % \end{zutaten}

      % \begin{zubereitung}
      % \end{zubereitung}
